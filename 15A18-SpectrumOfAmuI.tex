\documentclass[12pt]{article}
\usepackage{pmmeta}
\pmcanonicalname{SpectrumOfAmuI}
\pmcreated{2013-03-22 15:32:49}
\pmmodified{2013-03-22 15:32:49}
\pmowner{PrimeFan}{13766}
\pmmodifier{PrimeFan}{13766}
\pmtitle{spectrum of $A-\mu I$}
\pmrecord{9}{37444}
\pmprivacy{1}
\pmauthor{PrimeFan}{13766}
\pmtype{Theorem}
\pmcomment{trigger rebuild}
\pmclassification{msc}{15A18}
\pmrelated{SpectralValuesClassification}

\endmetadata

% this is the default PlanetMath preamble.  as your knowledge
% of TeX increases, you will probably want to edit this, but
% it should be fine as is for beginners.

% almost certainly you want these
\usepackage{amssymb}
\usepackage{amsmath}
\usepackage{amsfonts}
\usepackage{amsthm}

\usepackage{mathrsfs}

% used for TeXing text within eps files
%\usepackage{psfrag}
% need this for including graphics (\includegraphics)
%\usepackage{graphicx}
% for neatly defining theorems and propositions
%
% making logically defined graphics
%%%\usepackage{xypic}

% there are many more packages, add them here as you need them

% define commands here

\newcommand{\sR}[0]{\mathbb{R}}
\newcommand{\sC}[0]{\mathbb{C}}
\newcommand{\sN}[0]{\mathbb{N}}
\newcommand{\sZ}[0]{\mathbb{Z}}

 \usepackage{bbm}
 \newcommand{\Z}{\mathbbmss{Z}}
 \newcommand{\C}{\mathbbmss{C}}
 \newcommand{\F}{\mathbbmss{F}}
 \newcommand{\R}{\mathbbmss{R}}
 \newcommand{\Q}{\mathbbmss{Q}}



\newcommand*{\norm}[1]{\lVert #1 \rVert}
\newcommand*{\abs}[1]{| #1 |}



\newtheorem{thm}{Theorem}
\newtheorem{defn}[thm]{Definition}
\newtheorem{prop}[thm]{Proposition}
\newtheorem{lemma}[thm]{Lemma}
\newtheorem{cor}[thm]{Corollary}
\begin{document}
Let $A$ be an endomorphism of the vector space 
$V$ over a field $k$.  Denote by $\sigma(A)$ the spectrum 
of $A$.  Then we have:

\begin{thm}
\label{thm:spec}
\[
    \sigma(A-\mu I)=\{\lambda-\mu\colon \lambda
    \in\sigma(A)\}
\]
\end{thm}

Theorem~\ref{thm:spec} is equivalent to:

\begin{thm}
\label{thm:sp}
$\lambda$ is a spectral value of $A$ if and only if 
$\lambda-\mu$ is a spectral value of $A-\mu I$.
\end{thm}
\begin{proof}[Proof of Theorem~\ref{thm:sp}]
Note that
\[
    A-\lambda I=(A-\mu I)-(\lambda I-\mu I)
    =(A-\mu I)-(\lambda-\mu)I
\]
and thus $A-\lambda I$ is invertible if and only if 
$(A-\mu I)-(\lambda-\mu)I$ is invertible.  Equivalently, 
$\lambda$ is a spectral value of $A$ iff $\lambda-\mu$ is a 
spectral value of $(A-\mu I)$, as desired.
\end{proof}
%%%%%
%%%%%
\end{document}
