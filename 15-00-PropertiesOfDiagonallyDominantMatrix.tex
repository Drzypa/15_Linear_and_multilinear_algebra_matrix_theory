\documentclass[12pt]{article}
\usepackage{pmmeta}
\pmcanonicalname{PropertiesOfDiagonallyDominantMatrix}
\pmcreated{2013-03-22 15:34:32}
\pmmodified{2013-03-22 15:34:32}
\pmowner{Andrea Ambrosio}{7332}
\pmmodifier{Andrea Ambrosio}{7332}
\pmtitle{properties of diagonally dominant matrix}
\pmrecord{15}{37483}
\pmprivacy{1}
\pmauthor{Andrea Ambrosio}{7332}
\pmtype{Result}
\pmcomment{trigger rebuild}
\pmclassification{msc}{15-00}

% this is the default PlanetMath preamble.  as your knowledge
% of TeX increases, you will probably want to edit this, but
% it should be fine as is for beginners.

% almost certainly you want these
\usepackage{amssymb}
\usepackage{amsmath}
\usepackage{amsfonts}

% used for TeXing text within eps files
%\usepackage{psfrag}
% need this for including graphics (\includegraphics)
%\usepackage{graphicx}
% for neatly defining theorems and propositions
\usepackage{amsthm}
% making logically defined graphics
%%%\usepackage{xypic}

% there are many more packages, add them here as you need them

% define commands here
\begin{document}
1)(Levy-Desplanques theorem) A strictly diagonally dominant matrix is non-singular.

\begin{proof}
Let $A$ be a strictly diagonally dominant matrix and let's assume $A$ is singular, that is, $\lambda=0\in\sigma(A)$. Then, by Gershgorin's circle theorem, an index $i$ exists such that:
\[
\sum_{j\ne i}\left|a_{ij}\right|\geq\left|\lambda-a_{ii}\right|=\left|a_{ii}\right|,
\]
which is in contrast with strictly diagonally dominance definition.
\end{proof}

2)(\PMlinkescapetext{Ostrowski theorem}) $\left\vert\det(A)\right\vert\geq\prod_{i=1}^{n}\left(\left\vert a_{ii}\right\vert -\sum_{j=1,j\neq i}\left\vert
a_{ij}\right\vert\right)$ (See \PMlinkname{here}{ProofOfDeterminantLowerBoundOfAStrictDiagonallyDominantMatrix} for a proof.)

3) A Hermitian diagonally dominant matrix with real nonnegative diagonal entries is positive semidefinite.

\begin{proof}
Let $A$ be a Hermitian diagonally dominant matrix with real nonnegative diagonal entries; then its eigenvalues are real and, by Gershgorin's circle theorem, for each eigenvalue an index $i$ exists such that:
\[
\lambda\in[a_{ii}-\sum_{j\ne i}\left|a_{ij}\right|, a_{ii}+\sum_{i\ne j}\left|a_{ij}\right| ],
\]
which implies, by definition of diagonally dominance,$\lambda\geq 0.$
\end{proof}
%%%%%
%%%%%
\end{document}
