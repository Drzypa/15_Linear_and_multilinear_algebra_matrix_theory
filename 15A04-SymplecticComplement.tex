\documentclass[12pt]{article}
\usepackage{pmmeta}
\pmcanonicalname{SymplecticComplement}
\pmcreated{2013-03-22 13:32:25}
\pmmodified{2013-03-22 13:32:25}
\pmowner{matte}{1858}
\pmmodifier{matte}{1858}
\pmtitle{symplectic complement}
\pmrecord{8}{34139}
\pmprivacy{1}
\pmauthor{matte}{1858}
\pmtype{Definition}
\pmcomment{trigger rebuild}
\pmclassification{msc}{15A04}
\pmdefines{symplectic complement}
\pmdefines{isotropic subspace}
\pmdefines{coisotropic subspace}
\pmdefines{symplectic subspace}
\pmdefines{Lagrangian subspace}

\endmetadata

% this is the default PlanetMath preamble.  as your knowledge
% of TeX increases, you will probably want to edit this, but
% it should be fine as is for beginners.

% almost certainly you want these
\usepackage{amssymb}
\usepackage{amsmath}
\usepackage{amsfonts}

% used for TeXing text within eps files
%\usepackage{psfrag}
% need this for including graphics (\includegraphics)
%\usepackage{graphicx}
% for neatly defining theorems and propositions
%\usepackage{amsthm}
% making logically defined graphics
%%%\usepackage{xypic}

% there are many more packages, add them here as you need them

% define commands here

\newcommand{\sR}[0]{\mathbb{R}}
\newcommand{\sC}[0]{\mathbb{C}}
\newcommand{\sN}[0]{\mathbb{N}}
\newcommand{\sZ}[0]{\mathbb{Z}}
\begin{document}
{\bf Definition} \cite{mcduff, abraham} 
Let $(V,\omega)$ be a  symplectic vector space and let $W$ be a
vector subspace of $V$. Then the \emph{symplectic complement} of $W$
is
$$W^\omega = \{x\in V\, | \, \omega(x,y)=0\,\, \mbox{for all}\,\, y\in W\}.$$

It is easy to see that $W^\omega$  is also a vector subspace of $V$.
Depending on the relation between $W$ and $W^\omega$, 
$W$ is given different names.
\begin{enumerate}
\item If $W\subset W^\omega$, then $W$ is an \emph{isotropic subspace} (of $V$).
\item If $W^\omega \subset W$, then $W$ is an \emph{coisotropic subspace}.
\item If $W \cap W^\omega=\{0\}$, then $W$ is an \emph{symplectic subspace}.
\item If $W = W^\omega$, then $W$ is an \emph{Lagrangian subspace}.
\end{enumerate}

For the symplectic complement, we have the
following dimension theorem.

{\bf Theorem} \cite{mcduff, abraham} Let $(V,\omega)$ be a symplectic vector
space, and let $W$ be a vector subspace of $V$. Then
$$\dim V = \dim W^\omega + \dim W.$$

\begin{thebibliography}{9}
 \bibitem {mcduff} D. McDuff, D. Salamon,
 \emph{Introduction to Symplectic Topology},
 Clarendon Press, 1997.
\bibitem{abraham} R. Abraham, J.E. Marsden, \emph{Foundations of Mechanics},
2nd ed., Perseus Books, 1978.
 \end{thebibliography}
%%%%%
%%%%%
\end{document}
