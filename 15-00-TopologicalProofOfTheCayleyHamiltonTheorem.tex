\documentclass[12pt]{article}
\usepackage{pmmeta}
\pmcanonicalname{TopologicalProofOfTheCayleyHamiltonTheorem}
\pmcreated{2013-03-22 12:33:22}
\pmmodified{2013-03-22 12:33:22}
\pmowner{rmilson}{146}
\pmmodifier{rmilson}{146}
\pmtitle{topological proof of the Cayley-Hamilton theorem}
\pmrecord{7}{32802}
\pmprivacy{1}
\pmauthor{rmilson}{146}
\pmtype{Proof}
\pmcomment{trigger rebuild}
\pmclassification{msc}{15-00}

\endmetadata

\usepackage{amsmath}
\usepackage{amsfonts}
\usepackage{amssymb}

\newcommand{\bv}{\mathbf{v}}
\newcommand{\End}{\mathrm{End}}
\newcommand{\trace}{\mathrm{tr}}
\newcommand{\reals}{\mathbb{R}}
\newcommand{\natnums}{\mathbb{N}}
\newcommand{\cnums}{\mathbb{C}}
\newcommand{\znums}{\mathbb{Z}}

\newcommand{\lp}{\left(}
\newcommand{\rp}{\right)}
\newcommand{\lb}{\left[}
\newcommand{\rb}{\right]}

\newcommand{\supth}{^{\text{th}}}


\newtheorem{proposition}{Proposition}
\begin{document}
We begin by showing that the theorem is true if the characteristic
polynomial does not have repeated roots,  and then prove the general case.

Suppose then that the discriminant of the characteristic polynomial is
non-zero, and hence that $T:V\rightarrow V$ has $n=\dim V$ distinct
eigenvalues once we extend\footnote{Technically, this means that we must
work with the vector space $\bar{V}=V\otimes\bar{k}$, where $\bar{k}$ is the
algebraic closure of the original field of scalars, and with
$\bar{T}:\bar{V}\to \bar{V}$ the extended automorphism with action
$$\bar{T}(v\otimes a) \to T(V)\otimes a,\quad v\in V,\; a\in \bar{k}.$$} to the algebraic closure of the ground
field.
We can therefore choose a basis of eigenvectors, call them
$\bv_1,\ldots,\bv_n$, with $\lambda_1,\ldots,\lambda_n$ the
corresponding eigenvalues.  From the definition of characteristic
polynomial we have that
$$c_T(x) = \prod_{i=1}^n (x-\lambda_i).$$
The factors on the right commute, and hence
$$c_T(T)\bv_i = 0$$
for all $i=1,\ldots,n$.  Since $c_T(T)$ annihilates a basis, it must, in
fact, be zero.

To prove the general case, let $\delta(p)$ denote the discriminant of
a polynomial $p$, and let us remark that the discriminant mapping
$$
T\mapsto \delta(c_T),\quad T\in\End(V)$$
is polynomial on
$\End(V)$. Hence the set of $T$ with distinct eigenvalues is a dense
open subset of $\End(V)$ relative to the Zariski
topology.  Now the characteristic polynomial map
$$T\mapsto c_T(T),\quad T\in\End(V)$$
is a polynomial map on the vector
space $\End(V)$.  Since it vanishes on a dense
open subset, it must vanish identically. Q.E.D.
%%%%%
%%%%%
\end{document}
