\documentclass[12pt]{article}
\usepackage{pmmeta}
\pmcanonicalname{PauliMatrices}
\pmcreated{2013-03-22 17:57:01}
\pmmodified{2013-03-22 17:57:01}
\pmowner{invisiblerhino}{19637}
\pmmodifier{invisiblerhino}{19637}
\pmtitle{Pauli matrices}
\pmrecord{9}{40449}
\pmprivacy{1}
\pmauthor{invisiblerhino}{19637}
\pmtype{Definition}
\pmcomment{trigger rebuild}
\pmclassification{msc}{15A57}
\pmsynonym{sigma matrices}{PauliMatrices}
\pmrelated{Spinor}
\pmrelated{SchrodingersWaveEquation}
\pmrelated{UnitaryGroup}
\pmrelated{HermitianMatrix}
\pmrelated{DiracMatrices}
\pmrelated{DiracEquation}

\endmetadata

% this is the default PlanetMath preamble.  as your knowledge
% of TeX increases, you will probably want to edit this, but
% it should be fine as is for beginners.

% almost certainly you want these
\usepackage{amssymb}
\usepackage{amsmath}
\usepackage{amsfonts}

% used for TeXing text within eps files
%\usepackage{psfrag}
% need this for including graphics (\includegraphics)
%\usepackage{graphicx}
% for neatly defining theorems and propositions
%\usepackage{amsthm}
% making logically defined graphics
%%%\usepackage{xypic}

% there are many more packages, add them here as you need them

% define commands here

\begin{document}
The Pauli matrices are a set of three Hermitian, unitary matrices used by Wolfgang Pauli in his theory of quantum-mechanical spin. They are given by:
\begin{align*}
\sigma_1 &= \begin{pmatrix} 0 & 1\\
                            1 & 0
            \end{pmatrix}\\
\sigma_2 &= \begin{pmatrix} 0 & -i\\
                            i & 0
            \end{pmatrix}\\
\sigma_3 &= \begin{pmatrix} 1 & 0\\
                            0 & -1
            \end{pmatrix}\\
\end{align*}

They satisfy the following commutation and anticommutation identities:
\begin{align*}
\left[ \sigma_i, \sigma_j \right] &= 2i\epsilon_{ijk} \sigma_k\text{where $\epsilon_{ijk}$ is the Levi-Civita symbol}\\
\lbrace \sigma_i, \sigma_j \rbrace &=2 \mathbf{I} \delta_{ij} \text{where $\mathbf{I}$ is the identity matrix and $\delta_{ij}$ is the Kronecker delta}
\end{align*}

\subsection{Delta notation}
With the identity matrix $\textbf{I}$, the Pauli matrices form a group. When combined in this way, they are often given the symbols $\delta_i$, as follows:

\begin{align*}
\delta_0 &= \begin{pmatrix} 1 & 0\\
                            0 & 1
            \end{pmatrix}\\
\delta_1 &= \begin{pmatrix} 0 & 1\\
                            1 & 0
            \end{pmatrix}\\
\delta_2 &= \begin{pmatrix} 0 & -i\\
                            i & 0
            \end{pmatrix}\\
\delta_3 &= \begin{pmatrix} 1 & 0\\
                            0 & -1
            \end{pmatrix}\\
\end{align*}
This choice is useful when writing the Dirac matrices.
%%%%%
%%%%%
\end{document}
