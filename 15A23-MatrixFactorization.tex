\documentclass[12pt]{article}
\usepackage{pmmeta}
\pmcanonicalname{MatrixFactorization}
\pmcreated{2013-03-22 14:15:07}
\pmmodified{2013-03-22 14:15:07}
\pmowner{mathcam}{2727}
\pmmodifier{mathcam}{2727}
\pmtitle{matrix factorization}
\pmrecord{10}{35699}
\pmprivacy{1}
\pmauthor{mathcam}{2727}
\pmtype{Definition}
\pmcomment{trigger rebuild}
\pmclassification{msc}{15A23}
\pmsynonym{matrix decomposition}{MatrixFactorization}
\pmrelated{IsawasaDecomposition}
\pmdefines{factor matrix}

\endmetadata

\usepackage{amssymb}
\usepackage{amsmath}
\usepackage{amsfonts}

%\usepackage{psfrag}
%\usepackage{graphicx}
%%%\usepackage{xypic}
\begin{document}
\subsubsection*{Matrix Factorization}

A \emph{matrix factorization} (or \emph{matrix decomposition}) is the right-hand-side product in

$$ A = F_1 F_2 \ldots F_k $$

for ``input'' matrix $A$.  The number of factor matrices $k$ depends on the situation.  Most often, $k = 2$ or $k = 3$.

Note that the process of \emph{producing} a factorization/decomposition is also called ``factorization'' or ``decomposition''.

\subsubsection*{Examples}

Some common factorizations and related devices are:

\begin{itemize}
\item LU-decomposition: $A = LU$, where $L$ is lower triangular, and $U$ is upper triangular
\item QR-decomposition: $A = QR$, where $Q$ is orthogonal, and $R$ is right triangular.
\item Singular value decomposition (SVD): $A = USV^T$, where $U$ and $V$ are orthogonal, and $S$ is a partially diagonal matrix.
\item The Cholesky Decomposition.
\item For a positive definite matrix, we can decompose it into its \PMlinkname{square root}{SquareRootOfPositiveDefiniteMatrix} squared.
\item Polar decomposition
\item Jordan canonical form
\item Iwasawa decomposition
\end{itemize}

See the entries for these and other matrix factorizations for details on the contents of the factor matrices, where to apply them, and how to best calculate them.

\subsubsection*{Simultaneous matrix factorization}
A related problem is to diagonalize or tridiagonalize many matrices using
the same matrix. Some results in this direction are listed below:
\begin{itemize}
\item commuting matrices are simultanenously triangularizable
\item commuting normal matrices are simultanenously diagonalizable
\end{itemize}
%%%%%
%%%%%
\end{document}
