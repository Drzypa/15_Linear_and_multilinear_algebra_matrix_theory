\documentclass[12pt]{article}
\usepackage{pmmeta}
\pmcanonicalname{ExampleOfPermutationMatrix}
\pmcreated{2013-03-22 15:03:14}
\pmmodified{2013-03-22 15:03:14}
\pmowner{Wkbj79}{1863}
\pmmodifier{Wkbj79}{1863}
\pmtitle{example of permutation matrix}
\pmrecord{8}{36771}
\pmprivacy{1}
\pmauthor{Wkbj79}{1863}
\pmtype{Example}
\pmcomment{trigger rebuild}
\pmclassification{msc}{15A36}

\endmetadata

\usepackage{graphicx}
%%%\usepackage{xypic} 
\usepackage{bbm}
\newcommand{\Z}{\mathbbmss{Z}}
\newcommand{\C}{\mathbbmss{C}}
\newcommand{\R}{\mathbbmss{R}}
\newcommand{\Q}{\mathbbmss{Q}}
\newcommand{\mathbb}[1]{\mathbbmss{#1}}
\newcommand{\figura}[1]{\begin{center}\includegraphics{#1}\end{center}}
\newcommand{\figuraex}[2]{\begin{center}\includegraphics[#2]{#1}\end{center}}
\newtheorem{dfn}{Definition}
\usepackage{amsmath}
\begin{document}
\PMlinkescapeword{column}
\PMlinkescapeword{columns}
\PMlinkescapeword{row}
\PMlinkescapeword{rows}

Consider the matrix
\[
P=\begin{pmatrix}
0 & 1 & 0 & 0 \\
0 & 0 & 0 & 1 \\
1 & 0 & 0 & 0 \\
0 & 0 & 1 & 0
\end{pmatrix}
\]
that corresponds to permuting the columns of the identity matrix under the permutation $(1 2 4 3)$ (\PMlinkname{i.e.}{Ie}, the first column of the identity matrix is the second column of $P$, the second column of the identity matrix is the fourth column of $P$, etc.).  Then $P$ is a permutation matrix.

We will consider what happens when we multiply a $4\times 4$ matrix by $P$.  For example, let $A$ be the matrix
\[
A=\begin{pmatrix}
4 & 2 & 6 & 8 \\
1 & 3 & 5 & 7 \\
1 & 0 & 1 & 0 \\
-1& -2& -3& -4
\end{pmatrix}.
\]
Then
\[
PA=\begin{pmatrix}
0 & 1 & 0 & 0 \\
0 & 0 & 0 & 1 \\
1 & 0 & 0 & 0 \\
0 & 0 & 1 & 0
\end{pmatrix}
\begin{pmatrix}
4 & 2 & 6 & 8 \\
1 & 3 & 5 & 7 \\
1 & 0 & 1 & 0 \\
-1& -2& -3&-4
\end{pmatrix}
= 
\begin{pmatrix}
1 & 3 & 5 & 7 \\
-1& -2& -3&-4 \\
4 & 2 & 6 & 8 \\
1 & 0 & 1 & 0
\end{pmatrix}.
\]

We notice that $PA$ has the same rows as $A$.  Moreover, the rows of $A$ are the rows of $PA$ permuted under $(1 2 4 3)$: The first row of $PA$ is the second row of $A$, the second row of $PA$ is the fourth row of $A$, etc.
%%%%%
%%%%%
\end{document}
