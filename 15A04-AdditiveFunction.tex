\documentclass[12pt]{article}
\usepackage{pmmeta}
\pmcanonicalname{AdditiveFunction}
\pmcreated{2013-03-22 16:17:31}
\pmmodified{2013-03-22 16:17:31}
\pmowner{paolini}{1187}
\pmmodifier{paolini}{1187}
\pmtitle{additive function}
\pmrecord{9}{38409}
\pmprivacy{1}
\pmauthor{paolini}{1187}
\pmtype{Definition}
\pmcomment{trigger rebuild}
\pmclassification{msc}{15A04}
\pmrelated{LinearFunctional}

\endmetadata

% this is the default PlanetMath preamble.  as your knowledge
% of TeX increases, you will probably want to edit this, but
% it should be fine as is for beginners.

% almost certainly you want these
\usepackage{amssymb}
\usepackage{amsmath}
\usepackage{amsfonts}

% used for TeXing text within eps files
%\usepackage{psfrag}
% need this for including graphics (\includegraphics)
%\usepackage{graphicx}
% for neatly defining theorems and propositions
\usepackage{amsthm}
% making logically defined graphics
%%%\usepackage{xypic}

% there are many more packages, add them here as you need them

% define commands here
\newcommand{\R}{\mathbb R}
\newcommand{\N}{\mathbb N}
\newcommand{\Z}{\mathbb Z}
\newcommand{\Q}{\mathbb Q}
\newtheorem{theorem}{Theorem}
\newtheorem{definition}{Definition}
\theoremstyle{remark}
\newtheorem{example}{Example}
\begin{document}
\begin{definition}
Let $f\colon V\to \R$ be a function on a real vector space $V$ (more generally we can consider a vector space $V$ over a field $F$). 
We say that $f$ is \emph{additive} if 
\[
  f(x+y)= f(x) + f(y)
\]
for all $x,y \in V$.
\end{definition}

If $f$ is additive, we find that 
\begin{enumerate}
\item
$f(0)=0$. In fact $f(0)=f(0+0)=f(0)+f(0)=2f(0)$.

\item
$f(nx) = nf(x)$ for $n\in \N$. In fact $f(nx)=f(x)+\cdots +f(x) = nf(x)$.

\item
$f(nx) = nf(x)$ for $n\in \Z$. In fact $0=f(0)=f(x+(-x)) =f(x)+f(-x)$ so that $f(-x)=-f(x)$ and hence $f(-nx)=-f(nx)=-nf(x)$.

\item
$f(qx) = q f(x)$ for $q\in \Q$. In fact $q f(px/q) = f(q(px/q))=f(px) = p f(x)$ so that $f(px/q) = pf(x)/q$.
\end{enumerate}

This means that $f$ is $\Q$ linear. 
Quite surprisingly it is possible to show that there exist additive functions which are not linear (for example when $V$ is a vector space over the field $\R$).

%%%%%
%%%%%
\end{document}
