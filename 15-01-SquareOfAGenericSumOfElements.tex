\documentclass[12pt]{article}
\usepackage{pmmeta}
\pmcanonicalname{SquareOfAGenericSumOfElements}
\pmcreated{2013-03-22 19:34:48}
\pmmodified{2013-03-22 19:34:48}
\pmowner{mat}{27197}
\pmmodifier{mat}{27197}
\pmtitle{Square of a generic sum of elements}
\pmrecord{8}{42567}
\pmprivacy{1}
\pmauthor{mat}{27197}
\pmtype{Definition}
\pmcomment{trigger rebuild}
\pmclassification{msc}{15-01}
%\pmkeywords{square}
%\pmkeywords{sum}
%\pmkeywords{squared}
%\pmkeywords{squared sum}
%\pmkeywords{square of a sum}


\begin{document}
The formula for computing the square of a generic sum of terms is as follows:

$$\left(\sum_{k=1}^{K}a_{k}\right)^{2}=\sum_{k=1}^{K}a_{k}^{2}+2\sum_{j=1}^{K}\sum_{i<j}a_{i}a_{j}$$

We can prove this property by induction, considering that it holds for K=2, since \[(a+b)^2 = a^2 + b^2 + 2ab\] and that if the property holds for a generic K, then it holds also for K+1, as is proven in the following passages:

\begin{equation}
\begin{aligned}\left(\sum_{k=1}^{K+1}a_{k}\right)^{2} & =\left(\sum_{k=1}^{K}a_{k}+a_{K+1}\right)^{2}=\left(\sum_{k=1}^{K+1}a_{k}\right)^{2}+a_{K+1}^{2}+2\sum_{k=1}^{K}a_{k}a_{K+1}\\
& =\sum_{k=1}^{K}a_{k}^{2}+2\sum_{j=1}^{K}\sum_{i<j}a_{i}a_{j}+a_{K+1}^{2}+2\sum_{k=1}^{K}a_{k}a_{K+1}=\\
& =\sum_{k=1}^{K+1}a_{k}^{2}+2\sum_{j=1}^{K+1}\sum_{i<j}a_{i}a_{j}
\end{aligned}
\end{equation}
%%%%%
%%%%%
\end{document}
