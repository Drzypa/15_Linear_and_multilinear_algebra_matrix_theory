\documentclass[12pt]{article}
\usepackage{pmmeta}
\pmcanonicalname{InvariantSubspace}
\pmcreated{2013-03-22 12:19:55}
\pmmodified{2013-03-22 12:19:55}
\pmowner{rmilson}{146}
\pmmodifier{rmilson}{146}
\pmtitle{invariant subspace}
\pmrecord{9}{31962}
\pmprivacy{1}
\pmauthor{rmilson}{146}
\pmtype{Definition}
\pmcomment{trigger rebuild}
\pmclassification{msc}{15-00}
\pmrelated{LinearTransformation}
\pmrelated{Invariant}

\usepackage{amsmath}
\usepackage{amsfonts}
\usepackage{amssymb}


\newtheorem{proposition}{Proposition}
\begin{document}
Let $T: V\rightarrow V$ be a linear transformation of a vector space $V$.   A subspace $U\subset V$ is
called a {\em $T$-invariant subspace} if $T(u)\in U$ for all $u\in U$.

If $U$ is an invariant subspace, then the restriction of $T$ to $U$
gives a well defined linear transformation of $U$.  Furthermore,
suppose that $V$ is $n$-dimensional and that $v_1,\ldots, v_n$ is a
basis of $V$ with the first $m$ vectors giving a basis of $U$.  Then,
the representing matrix of the transformation $T$ relative to this
basis takes the form
$$
\begin{pmatrix}
  A & B \\
  0 & C
\end{pmatrix}$$
where $A$ is an $m\times m$ matrix representing the restriction
transformation $T\big|_U:U\to U$ relative to the basis $v_1,\ldots, v_m$.
%%%%%
%%%%%
\end{document}
