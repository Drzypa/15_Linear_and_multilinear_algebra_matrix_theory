\documentclass[12pt]{article}
\usepackage{pmmeta}
\pmcanonicalname{CharacteristicValuesAndVectorsofAMatrix}
\pmcreated{2013-03-22 17:43:58}
\pmmodified{2013-03-22 17:43:58}
\pmowner{perucho}{2192}
\pmmodifier{perucho}{2192}
\pmtitle{characteristic values and vectors (of a matrix)}
\pmrecord{6}{40182}
\pmprivacy{1}
\pmauthor{perucho}{2192}
\pmtype{Topic}
\pmcomment{trigger rebuild}
\pmclassification{msc}{15A18}
\pmsynonym{eigenvalues}{CharacteristicValuesAndVectorsofAMatrix}
\pmsynonym{eigenvectors}{CharacteristicValuesAndVectorsofAMatrix}

\endmetadata

% this is the default PlanetMath preamble.  as your knowledge
% of TeX increases, you will probably want to edit this, but
% it should be fine as is for beginners.

% almost certainly you want these
\usepackage{amssymb}
\usepackage{amsmath}
\usepackage{amsfonts}
\usepackage{amsthm}

% used for TeXing text within eps files
%\usepackage{psfrag}
% need this for including graphics (\includegraphics)
%\usepackage{graphicx}
% for neatly defining theorems and propositions
%\usepackage{amsthm}
% making logically defined graphics
%%%\usepackage{xypic}

% there are many more packages, add them here as you need them

% define commands here
\newtheorem{theorem}{Theorem}
\newtheorem{defn}{Definition}
\newtheorem{prop}{Proposition}
\newtheorem{lemma}{Lemma}
\newtheorem{cor}{Corollary}
\newtheorem{theorem*}{Theorem}
\begin{document}
Over the spectrum $\sigma(A)$ of a matrix $A$, its eigenvalues $\lambda_1, \lambda_2, \ldots, \lambda_s$ possess multiplicities $n_1, n_2, \ldots, n_s$, respectively, with $\sum_{k=1}^s n_k=n$. Its associated characteristic polynomial is then factored as  \begin{equation}
\Delta(\lambda)\equiv |\lambda I-A|=\Pi_{k=1}^s (\lambda-\lambda_k)^{n_k}.
\end{equation}        

Let us set $\mathrm{mult}(\lambda_k)=n_k$ for multiplicity of $\lambda_k$($k=1,\ldots,s$). We will now prove the following theorem.
\begin{theorem*}
If $\sigma(A)=\{\lambda_k\}_{k=1}^s$, $\mathrm{mult}(\lambda_k)=n_k$, and $g(\mu)$ is a scalar polynomial, then $\sigma(g(A))=\{g(\lambda_k)\}_{k=1}^s$, $\mathrm{mult}(g(\lambda_k))=n_k$.
\end{theorem*}
\begin{proof}
Let $g(\mu)$ be an arbitrary scalar polynomial. We want to find the characteristic values of $g(A)$. For this purpose we split $g(\mu)$ into linear factors
\begin{equation}
g(\mu)=a_0\Pi_{i=1}^t(\mu-\mu_i)^{l_i}, \qquad a_0\neq 0, \qquad \sum_{i=1}^t l_i=l.
\end{equation}
On substitution $\mu \mapsto A$, we have
\begin{equation}
g(A)=a_0\Pi_{i=1}^t(A-\mu_i I)^{l_i}, 
\end{equation}
being $I$ the identity matrix. Let us compute the determinant of $g(A)$. (Coefficient $a_0$ will be powered to $n$, the order of the square matrix $A$).
\begin{align*}
|g(A)| &= a_0^n\Pi_{i=1}^t|(-1)(\mu_iI-A)|^{l_i}=a_0^n\Pi_{i=1}^t(-1)^{nl_i}|\mu_iI-A|^{l_i} \\
       &= a_0^n(-1)^{n\sum_{i=1}^t l_i}\Pi_{i=1}^t|\mu_iI-A|^{l_i}=a_0^n(-1)^{nl}\Pi_{i=1}^t\Delta(\mu_i)^{l_i} \\
       &= a_0^n(-1)^{nl}\Pi_{i=1}^t[\Pi_{k=1}^s(\mu_i-\lambda_k)^{n_k}]^{l_i},
\end{align*} 
because on substitution $\lambda \mapsto \mu_i$ in (1). Next we commute the binomial by introducing $(-1)^{nl}$ into the product signs and also we note that $a_0^n=a_0^{\sum_{k=1}^s n_k}=\Pi_{k=1}^s a_0^{n_k}$, so that
\begin{equation*}
|g(A)|=\Pi_{k=1}^s[a_0\Pi_{i=1}^t(\lambda_k-\mu_i)^{l_i}]^{n_k},
\end{equation*}
and we may use (2) for $\mu=\lambda_k$ to obtain
\begin{equation}
|g(A)|=\Pi_{k=1}^s g(\lambda_k)^{n_k}.
\end{equation}
Finally we substitute the polynomial $g(\mu)$ by $\lambda-g(\mu)$, where $\lambda$ is an arbitrary parameter, getting for (4)
\begin{equation}
\Delta(g(A))\equiv|\lambda I-g(A)|=\Pi_{k=1}^s[\lambda-g(\lambda_k)]^{n_k}.
\end{equation}
This proves the theorem.
\end{proof}
As an important particular case we have: $\sigma(A^m)=\{\lambda_k^m\}_{k=1}^s$, ($m=0, 1, \cdots $), $\mathrm{mult}(\lambda_k)=n_k$.

\textbf{Connection between the characteristic polynomial $\Delta(\lambda)$ and the adjugate matrix $B(\lambda)$ of $A$.} \\
As it is well known, the adjugate matrix $B$ of a matrix $A$ there corresponds to the algebraic complement or cofactor matrix of the transpose of $A$. From this definition we have 
\begin{equation}
B(\lambda)(\lambda I-A)=\Delta(\lambda)I\qquad\mathrm{and}\qquad (\lambda I-A)B(\lambda)=\Delta(\lambda)I.
\end{equation}
Let us suppose $\Delta(\lambda)$ is given by
\begin{equation}
\Delta(\lambda)=\lambda^n-\sum_{k=1}^n c_k\lambda^{n-k}.
\end{equation}
It is clear that the difference $\Delta(\lambda)-\Delta(\mu)$ is divisible by $\lambda-\mu$ without remainder, hence
\begin{equation}
\delta(\lambda,\mu)\equiv \frac{\Delta(\lambda)-\Delta(\mu)}{\lambda-\mu}=\lambda^{n-1}+(\mu-c_1)\lambda^{n-2}+(\mu^2-c_1\mu-c_2)\lambda^{n-3}+\cdots
\end{equation}
is a polynomial in $\lambda, \mu$. If we replace in (8) $(\lambda, \mu)$ by the permutable matrices $(\lambda I, A)$ and recalling that from Cayley-Hamilton theorem $\Delta(A)=0$, then
\begin{equation}
\delta(\lambda I,A)(\lambda I-A)= \Delta(\lambda)I,
\end{equation}
which by comparing it with (6)${}_1$ we conclude that
\begin{equation}
B(\lambda)=\delta(\lambda I, A)
\end{equation}
is the desired formula by virtue of the uniqueness of the quotient. Therefore (10) and (8) let to write the adjugate $B(\lambda)$ as the matrix polynomial
\begin{equation}
B(\lambda)=I\lambda^{n-1}+\sum_{k=1}^{n-1}B_k\lambda^{n-k-1},
\end{equation}
where ($\mu\mapsto A$ in (8))
\begin{equation}
B_k=A^k-\sum_{i=1}^k c_iA^{k-i}, \qquad (k=1,\ldots, n-1),
\end{equation}
which can also be obtained from the recurrence equation
\begin{equation}
B_k=AB_{k-1}-c_kI, \qquad (k=1, \ldots, n-1; \quad B_0=I).
\end{equation}
What is more,
\begin{equation}
AB_{n-1}-c_nI=0\equiv B_n.
\end{equation}
(13) as well as (14) follow inmediately from (6)${}_2$ if we equate the coefficients of equal powers of $\lambda$ on both sides. Also, if we substitute $B_{n-1}$ from (12), into (14), we get $\Delta(A)=0$ (Cayley-Hamilton), an implicit consequence of generalized B\'ezout theorem. On the other hand, by setting $\lambda=0$ in (7) we obtain $c_n=\Delta(0)/(-1)=|-A|/(-1)=(-1)^{n-1}|A|\neq 0$, whenever $A$ be non- singular. From this and from (14) follow that
\begin{equation}
A^{-1}=\frac{1}{c_n}B_{n-1}.
\end{equation}
Let now $\lambda_c$ be a characteristic value of $A$, then $\Delta(\lambda_c)=0$ and (6)${}_2$ becomes
\begin{equation}
(\lambda_c I-A)B(\lambda_c)=0.
\end{equation}
Let us assume that $B(\lambda_c)\neq 0$ and denote by $\mathbf{b}$ an arbitrary non-zero column of this matrix. From (16) we have $(\lambda_cI-A)\mathbf{b}=\mathbf{0}$. That is,
\begin{equation}
A\mathbf{b}=\lambda_c\mathbf{b}.
\end{equation}
Therefore every non-zero column of $B(\lambda_c)$ determines a characteristic vector corresponding to the characteristic value $\lambda_c$. Moreover, if to the characteristic value $\lambda_c$ there correspond $l$ linearly independent characteristic vectors, $n-l$ will be the rank of $\lambda_cI-A$ and so the rank of $B(\lambda_c)$ does not exceed $l$. In particular, if only one characteristic vector there corresponds to $\lambda_c$, then in $B(\lambda_c)$ the elements of any two columns will be proportional (In such a case $l=1$, hence the rank of $\lambda_cI-A$ will be $n-1$). \\
In conclusion: {\em If the coefficients of the characteristic polynomial are known, then the adjugate matrix may be found by (10). In addition, if the given matrix $A$ is non-singular, then the inverse matrix $A^{-1}$ can be found from (15). Also if $\lambda_c$ is a characteristic value of $A$, the non-zero columns of $B(\lambda_c)$ are characteristc vectors of A for $\lambda=\lambda_c$. \\

\textbf{Example.}\, We find out the characteristic values and vectors from the matrix
\begin{equation*}
A=
\begin{bmatrix}
3 & -3 & 2 \\
-1 & 5 & -2 \\
-1 & 3 & 0
\end{bmatrix}
.
\end{equation*}
From (1),
\begin{equation*}
\Delta(\lambda)=|\lambda I-A|=
\begin{vmatrix}
\lambda-3 & 3 & -2 \\
1 & \lambda-5 & 2 \\
1 & -3 & \lambda
\end{vmatrix}
=\lambda^3-8\lambda^2+20\lambda-16.
\end{equation*}
Comparing with (7), we have
\begin{equation*}
c_1=8, \qquad c_2=-20, \qquad c_3=16.
\end{equation*}
Next we use (8),
\begin{equation*}
\delta(\lambda,\mu)=\frac{\Delta(\lambda)-\Delta(\mu)}{\lambda-\mu}=\lambda^2+(\mu-8)\lambda+\mu^2-8\mu+20,
\end{equation*}
so that from (11)
\begin{equation*}
B(\lambda)=\delta(\lambda I,A)=\lambda^2 I+(\underbrace{A-8I}_{B_1})\lambda+\underbrace{A^2-8A+20I}_{B_2}.
\end{equation*}
We will now evaluate $B_1$ and $B_2$ by using (12) and (13), respectively.
\begin{equation*}
B_1=A-8I=
\begin{bmatrix}
-5 & -3 & 2 \\
-1 & -3 & -2 \\
-1 & 3 & -8
\end{bmatrix}
, \qquad
B_2=AB_1+20I=
\begin{bmatrix}
6 & 6 & -4 \\
2 & 2 & 4 \\
2 & -6 & 12
\end{bmatrix}
,
\end{equation*}
thus $B(\lambda)$ is
\begin{equation*}
B(\lambda)=
\begin{bmatrix}
\lambda^2-5\lambda+6 & -3\lambda+6 & 2\lambda-4 \\
-\lambda+2 & \lambda^2-3\lambda+2 & -2\lambda+4 \\
-\lambda+2 & 3\lambda-6 & \lambda^2-8\lambda+12
\end{bmatrix}
.
\end{equation*}
Also $|A|=16$ and $A^{-1}$ is obtained from (15), i.e.
\begin{equation*}
A^{-1}=\frac{1}{16}B_2=\frac{1}{8}
\begin{bmatrix}
3 & 3 & -2 \\
1 &1 & 2 \\
1 & -3 & 6
\end{bmatrix}
.
\end{equation*}
Furthermore,
\begin{equation*}
\Delta(\lambda)=(\lambda-2)^2(\lambda-4).
\end{equation*}
We notice the eigenvalue $\lambda=2$ possesses multiplicity $2$ and also that all the entries of the adjugate 
$B(\lambda)$ are divisible by the binomial $\lambda-2$ ($|B(2)|=0$, i.e. $\lambda=2$ annihilates it), therefore it can be reduced which makes instructive this problem. Thus,
\begin{equation*}
C(\lambda)=
\begin{bmatrix}
\lambda-3 & -3 & 2 \\
-1 & \lambda-1 & -2 \\
-1 & 3 & \lambda-6
\end{bmatrix}
,
\end{equation*}
which for $\lambda=2$ it becomes
\begin{equation*}
C(2)=
\begin{bmatrix}
-1 & -3 & 2 \\
-1 & 1 & -2 \\
-1 & 3 & -4
\end{bmatrix}
.
\end{equation*}
From this we get the charactreristic vectors $(1,1,1)$ by multiplying the first colum by $-1$, and also $(-3,1,3)$, both correponding to $\lambda=2$. Third column is a linear combination of the first two (subtract it). Likewise we find for the another characteristic value $\lambda=4$
\begin{equation*}
C(4)=
\begin{bmatrix}
1 & -3 & 2 \\
-1 & 3 & -2 \\
-1 & 3 & -2
\end{bmatrix}
,
\end{equation*}
whence we get the eigenvector $(1,-1,-1)$, being the remaining two columns clearly proportional to the first one.





%%%%%
%%%%%
\end{document}
