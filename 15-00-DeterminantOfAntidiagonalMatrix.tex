\documentclass[12pt]{article}
\usepackage{pmmeta}
\pmcanonicalname{DeterminantOfAntidiagonalMatrix}
\pmcreated{2013-03-22 15:50:25}
\pmmodified{2013-03-22 15:50:25}
\pmowner{cvalente}{11260}
\pmmodifier{cvalente}{11260}
\pmtitle{determinant of anti-diagonal matrix}
\pmrecord{6}{37817}
\pmprivacy{1}
\pmauthor{cvalente}{11260}
\pmtype{Result}
\pmcomment{trigger rebuild}
\pmclassification{msc}{15-00}

\endmetadata

% this is the default PlanetMath preamble.  as your knowledge
% of TeX increases, you will probably want to edit this, but
% it should be fine as is for beginners.

% almost certainly you want these
\usepackage{amssymb}
\usepackage{amsmath}
\usepackage{amsfonts}

% used for TeXing text within eps files
%\usepackage{psfrag}
% need this for including graphics (\includegraphics)
%\usepackage{graphicx}
% for neatly defining theorems and propositions
%\usepackage{amsthm}
% making logically defined graphics
%%%\usepackage{xypic}

% there are many more packages, add them here as you need them

% define commands here
\begin{document}
Let $A=\operatorname{adiag}(a_1, \ldots, a_n)$ be an anti-diagonal matrix. Using the sum over all permutations formula for the determinant of a matrix and since all but possibly the anti-diagonal elements are null we get directly at the result

$$\operatorname{det} A =\operatorname{sgn}(n,n-1,\ldots,1) \prod_{i=1}^n a_i $$

so all that remains is to calculate the sign of the permutation.
This can be done directly.

To bring the last element to the beginning $n-1$ permutations are needed so

$$\operatorname{sgn}(n,n-1,\ldots,1) = (-1)^{n-1} \operatorname{sgn}(1,n,n-1,\cdots,2)$$

Now bring the last element to the second position.
To do this $n-2$ permutations are needed.
Repeat this procedure $n-1$ times to get the permutation $(1,\ldots,n)$ which has positive sign.

Summing every permutation, it takes

$$\sum_{k=1}^{n-1} k = \frac{n(n-1)}{2}$$

permutations to get to the desired permutation.

So we get the final result that 

$$\operatorname{det} \operatorname{adiag}(a_1, \ldots, a_n) = (-1)^{\frac{n(n-1)}{2}}  \prod_{i=1}^n a_i $$

Notice that the sign is positive if either $n$ or $n-1$ is a multiple of $4$ and negative otherwise.
%%%%%
%%%%%
\end{document}
