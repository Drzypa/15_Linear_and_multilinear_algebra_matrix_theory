\documentclass[12pt]{article}
\usepackage{pmmeta}
\pmcanonicalname{CauchyMatrix}
\pmcreated{2013-03-22 14:30:43}
\pmmodified{2013-03-22 14:30:43}
\pmowner{kshum}{5987}
\pmmodifier{kshum}{5987}
\pmtitle{Cauchy matrix}
\pmrecord{9}{36050}
\pmprivacy{1}
\pmauthor{kshum}{5987}
\pmtype{Definition}
\pmcomment{trigger rebuild}
\pmclassification{msc}{15A57}
\pmdefines{Cauchy matrices}

% this is the default PlanetMath preamble.  as your knowledge
% of TeX increases, you will probably want to edit this, but
% it should be fine as is for beginners.

% almost certainly you want these
\usepackage{amssymb}
\usepackage{amsmath}
\usepackage{amsfonts}

% used for TeXing text within eps files
%\usepackage{psfrag}
% need this for including graphics (\includegraphics)
%\usepackage{graphicx}
% for neatly defining theorems and propositions
%\usepackage{amsthm}
% making logically defined graphics
%%%\usepackage{xypic}

% there are many more packages, add them here as you need them

% define commands here
\begin{document}
Let $x_1$, $x_2,\ldots, x_m$, and $y_1$, $y_2 \ldots, y_n$ be elements in a field $F$, satisfying the \PMlinkescapetext{properties} that

\begin{enumerate}
\item $x_1, \ldots, x_m$ are distinct,

\item $y_1, \ldots, y_n$ are distinct, and

\item $x_i+y_j\neq 0$ for $1\leq i \leq m$, $1\leq j \leq n$.
\end{enumerate}

The matrix
\[
\begin{bmatrix}
\frac{1}{x_1+y_1} & \frac{1}{x_1+y_2} & \cdots &\frac{1}{x_1+y_n} \\
\frac{1}{x_2+y_1} & \frac{1}{x_2+y_2} & \cdots &\frac{1}{x_2+y_n} \\
\vdots & \vdots & \ddots & \vdots \\
\frac{1}{x_m+y_1} & \frac{1}{x_m+y_2} & \cdots &\frac{1}{x_m+y_n}
\end{bmatrix}
\]
is called a {\em Cauchy matrix} over $F$.

\bigskip

The determinant of a square Cauchy matrix is
\[
\frac{ \prod_{i<j} (x_i-x_j)(y_i-y_j) } {\prod_{ij} (x_i+y_j)}
\]

Since $x_i$'s are distinct and $y_j$'s are distinct by definition, a square Cauchy matrix is non-singular. Any submatrix of a rectangular Cauchy matrix has full rank.
%%%%%
%%%%%
\end{document}
