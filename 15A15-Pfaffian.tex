\documentclass[12pt]{article}
\usepackage{pmmeta}
\pmcanonicalname{Pfaffian}
\pmcreated{2013-03-22 14:22:13}
\pmmodified{2013-03-22 14:22:13}
\pmowner{PrimeFan}{13766}
\pmmodifier{PrimeFan}{13766}
\pmtitle{Pfaffian}
\pmrecord{26}{35857}
\pmprivacy{1}
\pmauthor{PrimeFan}{13766}
\pmtype{Definition}
\pmcomment{trigger rebuild}
\pmclassification{msc}{15A15}
%\pmkeywords{antisymmetric matrix}

% this is the default PlanetMath preamble.  as your knowledge
% of TeX increases, you will probably want to edit this, but
% it should be fine as is for beginners.

% almost certainly you want these
\usepackage{amssymb}
\usepackage{amsmath}
\usepackage{amsfonts}

% used for TeXing text within eps files
%\usepackage{psfrag}
% need this for including graphics (\includegraphics)
%\usepackage{graphicx}
% for neatly defining theorems and propositions
%\usepackage{amsthm}
% making logically defined graphics
%%%\usepackage{xypic}

% there are many more packages, add them here as you need them

% define commands here
\begin{document}
The {\em Pfaffian} is an analog of the determinant that is defined only for a $2n\times 2n$ antisymmetric matrix. It is a polynomial of the polynomial ring in elements of the matrix, such that its square is equal to the determinant of the matrix. 

The Pfaffian is applied in the generalized Gauss-Bonnet theorem.

{\bf Examples}

$Pf\begin{bmatrix} 0 & a \\ -a & 0 \end{bmatrix}=a,$

$Pf\begin{bmatrix} 0 & a & b & c \\ -a & 0 & d & e \\ -b & -d & 0& f \\-c & -e & -f & 0 \end{bmatrix}=af-be+dc.$

{\bf Standard definition}

Let 

$$A=\begin{bmatrix} 0 & a_{1,2} & \ldots & a_{1,2n} \\ -a_{1,2} & 0 & \ldots & a_{2,2n} \\ \vdots & \vdots & \vdots & \vdots \\-a_{2n,1} & -a_{2n,2} & \ldots & 0 \end{bmatrix}.$$

Let $\Pi^{}_{}$ be the set of all partition of $\{1,2, \ldots ,2n\}$ into pairs of elements $\alpha\in \Pi^{}_{}$, can be represented as 
$$\alpha^{}_{}=\{(i_1,j_1),(i_2,j_2), \ldots ,(i_n,j_n)\} $$ 
with $i_k<j_k$ and $i_1 < i_2 < \cdots < i_n$, let
$$\pi=\begin{bmatrix} 1 & 2 & 3 & 4 & \ldots & 2n \\ i_1 & j_1 & i_2 & j_2 & \ldots & j_{n} \end{bmatrix}$$ 
be a corresponding permutation and let us define 
$sgn(\alpha)$ to be the signature of a permutation $\pi^{}_{}$; clearly it depends only on the partition $\alpha$ and not on the particular choice of $\pi^{}_{}$.
Given a partition $\alpha^{}_{}$ as above let us set
$a_\alpha =a_{i_1,j_1}a_{i_2,j_2} \ldots a_{i_n,j_n},$
then we can define the \emph{Pfaffian} of $A$ as 
$$Pf(A)=\sum_{\alpha\in \Pi} sgn(\alpha)a_\alpha.$$

{\bf Alternative definition}

One can associate to any antisymmetric $2n\times 2n$ matrix $A=\{a_{ij}\}$ 
a bivector 
:$\omega=\sum_{i<j} a_{ij} e_i\wedge e_j$ 
in a basis
$\{e_1,e_2, \ldots ,e_{2n}\}$ of $\mathbb{R}^{2n}$, then 
$$\omega^n= n!Pf(A)e_1\wedge e_2\wedge \cdots \wedge e_{2n},$$
where $\omega^n_{}$ denotes exterior product of $n$ copies of $\omega^{}_{}$.

{\bf Identities}

For any antisymmetric $2n\times 2n$ matrix $A$' and any $2n\times 2n$ matrix $B$

$$Pf(A)^2 = \det(A)$$
$$Pf(BAB^T)= \det(B)Pf(A)$$
%%%%%
%%%%%
\end{document}
