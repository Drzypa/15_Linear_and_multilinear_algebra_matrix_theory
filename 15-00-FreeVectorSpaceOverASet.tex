\documentclass[12pt]{article}
\usepackage{pmmeta}
\pmcanonicalname{FreeVectorSpaceOverASet}
\pmcreated{2013-03-22 13:34:34}
\pmmodified{2013-03-22 13:34:34}
\pmowner{mathcam}{2727}
\pmmodifier{mathcam}{2727}
\pmtitle{free vector space over a set}
\pmrecord{8}{34196}
\pmprivacy{1}
\pmauthor{mathcam}{2727}
\pmtype{Definition}
\pmcomment{trigger rebuild}
\pmclassification{msc}{15-00}
\pmsynonym{vector space generated by a set}{FreeVectorSpaceOverASet}
\pmrelated{TensorProductBasis}

\endmetadata

% this is the default PlanetMath preamble.  as your knowledge
% of TeX increases, you will probably want to edit this, but
% it should be fine as is for beginners.

% almost certainly you want these
\usepackage{amssymb}
\usepackage{amsmath}
\usepackage{amsfonts}

% used for TeXing text within eps files
%\usepackage{psfrag}
% need this for including graphics (\includegraphics)
%\usepackage{graphicx}
% for neatly defining theorems and propositions
%\usepackage{amsthm}
% making logically defined graphics
%%\usepackage{xypic}

% there are many more packages, add them here as you need them

% define commands here
\begin{document}
\newcommand{\sK}[0]{\mathbb{K}}
\newcommand{\lsp}[0]{\mathop{\mathrm{span}}}

In this entry we construct
the \emph{free vector space over a set}, or
the \emph{vector space generated by a set} \cite{greub}. 
For a set $X$, we shall denote this vector space by $C(X)$. 
One application of this construction is given in \cite{madsen},
where the free  vector space is used to define the tensor product for 
modules. 

To define the vector space $C(X)$, let us first define $C(X)$ as a
set. For a set $X$  and a field $\sK$, we define
\begin{eqnarray*}
C(X) &=& \{ f:X\to \sK\,\, |\,\, f^{-1}(\sK\backslash\{0\}) \, \mbox{is finite} \}.
\end{eqnarray*}
In other words, $C(X)$ consists of functions $f:X\to \sK$ 
that are non-zero only
at finitely many points in $X$. 
Here, we denote the identity element in $\sK$ by $1$, and 
the zero element by $0$. 
The vector space structure for $C(X)$
is defined as follows. If $f$ and $g$ are
functions in $C(X)$, then $f+g$ is the mapping $x\mapsto f(x) + g(x)$. 
Similarly, if $f\in C(X)$ and $\alpha \in \sK$, then 
$\alpha f$ is the mapping $x\mapsto \alpha f(x)$. It is not difficult to
see that these operations are well defined, i.e., both $f+g$ and 
$\alpha f$ are again functions in $C(X)$. 

\subsubsection{Basis for $C(X)$} 
If $a\in X$, 
let us define the function $\Delta_a \in C(X)$ by 
\begin{eqnarray*}
\Delta_a(x)&=& \left\{ \begin {array}{ll} 
  1 & \mbox{when} \, x=a, \\
  0 & \mbox{otherwise.} \\
  \end{array} \right.
\end{eqnarray*}
These functions form a linearly independent basis for $C(X)$, i.e.,
\begin{eqnarray}
\label{basiseq}
C(X) &=& \lsp\{ \Delta_a\}_{a\in X}.
\end{eqnarray}
Here, the space $\lsp\{ \Delta_a\}_{a\in X}$ consists of all
finite linear combinations of elements in  $\{ \Delta_a\}_{a\in X}$.
It is clear that any element in $\lsp\{ \Delta_a\}_{a\in X}$
is a member in $C(X)$. 
Let us check the other direction. Suppose $f$ is a member in $C(X)$. 
Then, let
$\xi_1, \ldots, \xi_N$ be the distinct points in $X$ where $f$ is non-zero. 
We then have
\begin{eqnarray*}
f&=&\sum_{i=1}^Nf(\xi_i) \Delta_{\xi_i},
\end{eqnarray*}
and we have established equality in equation \ref{basiseq}.

To see that the set $\{ \Delta_a\}_{a\in X}$ is linearly independent, 
we need to show that its any finite subset is linearly independent. 
Let $\{ \Delta_{\xi_1}, \ldots, \Delta_{\xi_N} \}$ be such
a finite subset, and  
suppose $\sum_{i=1}^N \alpha_i \Delta_{\xi_i }=0$ for some 
$\alpha_i \in \sK$. Since the points $\xi_i$ are pairwise distinct, it 
follows that $\alpha_i=0$ for all $i$. This shows that the set
 $\{ \Delta_a\}_{a\in X}$ is linearly independent. 

Let us define the mapping $\iota:X\to C(X)$, $x\mapsto \Delta_x$.
This mapping gives a bijection between $X$ and the basis
vectors $\{ \Delta_a\}_{a\in X}$. We can thus identify these
spaces. Then  $X$ becomes a linearly independent basis for $C(X)$.

\subsubsection{Universal property of $\iota:X\to C(X)$} 
The mapping $\iota:X\to C(X)$ is universal in
the following sense. If $\phi$ is an arbitrary mapping from $X$ to a
vector space $V$, then there exists a unique mapping $\bar{\phi}$
such that the below diagram commutes:
$$
\xymatrix{
X \ar[r]^\phi\ar[d]_\iota & V \\
C(X) \ar[ur]_{\bar{\phi}} &
}
$$
\emph{Proof.} We define $\bar{\phi}$ as the linear mapping that
maps the basis elements of $C(X)$ as $\bar{\phi}(\Delta_x) = \phi(x)$.
Then, by definition, $\bar{\phi}$ is linear. For uniqueness,
suppose that there are linear mappings
$\bar{\phi},\bar{\sigma}:C(X)\to V$
such that $\phi=\bar{\phi}\circ \iota =\bar{\sigma}\circ \iota$.
For all $x\in X$, we then have $\bar{\phi}(\Delta_x)=\bar{\sigma}(\Delta_x)$.
Thus $\bar{\phi}=\bar{\sigma}$ since  both mappings are linear and
the coincide on the basis elements.$\Box$ 

\begin{thebibliography}{9}
\bibitem{greub}
W. Greub,
\emph{Linear Algebra},
Springer-Verlag, Fourth edition, 1975.
\bibitem {madsen} 
I. Madsen, J. Tornehave,
\emph{From Calculus to Cohomology},
Cambridge University press,
1997.
\end{thebibliography}
%%%%%
%%%%%
\end{document}
