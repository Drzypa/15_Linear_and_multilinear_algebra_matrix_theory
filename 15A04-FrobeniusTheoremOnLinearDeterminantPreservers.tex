\documentclass[12pt]{article}
\usepackage{pmmeta}
\pmcanonicalname{FrobeniusTheoremOnLinearDeterminantPreservers}
\pmcreated{2013-03-22 19:19:52}
\pmmodified{2013-03-22 19:19:52}
\pmowner{kammerer}{26336}
\pmmodifier{kammerer}{26336}
\pmtitle{Frobenius theorem on linear determinant preservers}
\pmrecord{7}{42277}
\pmprivacy{1}
\pmauthor{kammerer}{26336}
\pmtype{Theorem}
\pmcomment{trigger rebuild}
\pmclassification{msc}{15A04}
\pmclassification{msc}{15A15}
%\pmkeywords{linear preserver}
%\pmkeywords{determinant}
\pmrelated{DieudonneTheoremOnLinearPreserversOfTheSingularMatrices}

\endmetadata

% this is the default PlanetMath preamble.  as your knowledge
% of TeX increases, you will probably want to edit this, but
% it should be fine as is for beginners.

% almost certainly you want these
\usepackage{amssymb}
\usepackage{amsmath}
\usepackage{amsfonts}

% used for TeXing text within eps files
%\usepackage{psfrag}
% need this for including graphics (\includegraphics)
%\usepackage{graphicx}
% for neatly defining theorems and propositions
\usepackage{amsthm}
% making logically defined graphics
%%%\usepackage{xypic}

% there are many more packages, add them here as you need them

% define commands here
\newtheorem{thm}{Theorem}
\newtheorem{defi}[thm]{Definition}
\begin{document}
Let $\mathbb{F}$ be an arbitrary field. Consider $\mathcal{M}_n (\mathbb{F})$, the vector space of all $n \times n$ matrices over $\mathbb{F}$. Let $\mathcal{GL}_n (\mathbb{F})$ be the set of all nonsingular matrices $P \in \mathcal{M}_n (\mathbb{F})$. 
\begin{defi}
A linear endomorphism $\varphi : \mathcal{M}_n (\mathbb{F}) \longrightarrow \mathcal{M}_n (\mathbb{F})$ is said to be in standard form, if either $\exists\, P, Q \in \mathcal{GL}_n (\mathbb{F})\, \forall\, A \in \mathcal{M}_n (\mathbb{F}) :\, \varphi (A) = P A Q$ or $\exists\, P, Q \in \mathcal{GL}_n (\mathbb{F})\, \forall\, A \in \mathcal{M}_n (\mathbb{F}) :\, \varphi (A) = P A^\top Q$.
\end{defi}

The classical \PMlinkescapetext{Frobenius theorem} on linear preservers of the determinant function \cite{frob} reads as follows.
\begin{thm}
If $\varphi : \mathcal{M}_n (\mathbb{C}) \longrightarrow \mathcal{M}_n (\mathbb{C})$ is a linear automorphism such that $\det (\varphi (A)) = \det (A)$ for all $A \in \mathcal{M}_n (\mathbb{C})$, then $\varphi$ is in standard form with
\newline
$\det (P Q) = 1$. 
\end{thm}

It is well known that the \PMlinkescapetext{Frobenius theorem} can be strengthened.
\begin{thm}
Let $\mathbb{F}$ be an arbitrary field and let $\varphi : \mathcal{M}_n (\mathbb{F}) \longrightarrow \mathcal{M}_n (\mathbb{F})$ be a linear endomorphism. Then the following conditions are equivalent:\\
\begin{tabular}{cl}
(i)&$\det (\varphi (A)) = \det (A)$ for all $A \in \mathcal{M}_n (\mathbb{F})$,\\
(ii)&$\varphi$ is in standard form with $\det (P Q) = 1$.\\
\end{tabular}
\end{thm}

The above strengthened version of the \PMlinkescapetext{Frobenius theorem} can be derived from the Dieudonn\'e theorem on linear preservers of the singular matrices.


\begin{thebibliography}{99}
\bibitem[GF]{frob}
G. Frobenius, \emph{\"Uber die Darstellung der endlichen Gruppen durch lineare Substitutionen}, Sitzungsber., Preuss. Akad. Wiss., Berlin, 1897 (994--1015).
\end{thebibliography}
%%%%%
%%%%%
\end{document}
