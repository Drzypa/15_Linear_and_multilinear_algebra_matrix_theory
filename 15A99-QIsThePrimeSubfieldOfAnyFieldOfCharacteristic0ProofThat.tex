\documentclass[12pt]{article}
\usepackage{pmmeta}
\pmcanonicalname{QIsThePrimeSubfieldOfAnyFieldOfCharacteristic0ProofThat}
\pmcreated{2013-03-22 15:39:57}
\pmmodified{2013-03-22 15:39:57}
\pmowner{CWoo}{3771}
\pmmodifier{CWoo}{3771}
\pmtitle{Q is the prime subfield of any field of characteristic 0, proof that}
\pmrecord{16}{37600}
\pmprivacy{1}
\pmauthor{CWoo}{3771}
\pmtype{Proof}
\pmcomment{trigger rebuild}
\pmclassification{msc}{15A99}
\pmclassification{msc}{12F99}
\pmclassification{msc}{12E99}
\pmclassification{msc}{12E20}
\pmrelated{RationalNumbersAreRealNumbers}
\pmdefines{prime field}

\endmetadata

\usepackage{amssymb,amscd}
\usepackage{amsmath}
\usepackage{amsfonts}

% used for TeXing text within eps files
%\usepackage{psfrag}
% need this for including graphics (\includegraphics)
%\usepackage{graphicx}
% for neatly defining theorems and propositions
\usepackage{amsthm}
% making logically defined graphics
%%%\usepackage{xypic}

% define commands here
\newcommand{\fp}{\mathbb{F}_p}
\newcommand{\rat}{\mathbb{Q}}
\begin{document}
The following two propositions show that $\rat$ can be embedded in any field of characteristic $0$, while $\fp$ can be embedded in any field of characteristic $p$.

\textbf{Proposition.}\, $\rat$ is the prime subfield of any field of characteristic 0.  
\begin{proof}  Let $F$ be a field of characteristic $0$.\, We want to find a one-to-one field homomorphism $\phi:\rat\to F$.\, For\, $\frac{m}{n}\in\mathbb{Q}$\, with $m,\,n$ coprime, define the mapping $\phi$ that takes $\frac{m}{n}$ into $\frac{m1_F}{n1_F}\in F$.\, It is easy to check that $\phi$ is a well-defined function.\, Furthermore, it is elementary to show   
\begin{enumerate}
\item additive: for $p,q\in\rat$, $\phi(p+q)=\phi(p)+\phi(q)$;
\item multiplicative: for $p,q\in\rat$, $\phi(pq)=\phi(p)\phi(q)$;
\item $\phi(1)=1_F$, and
\item $\phi(0)=0_F$.
\end{enumerate}
This shows that $\phi$ is a field homomorphism.  Finally, if $\phi(p)=0$ and $p\ne 0$, then $1=\phi(1)=\phi(pp^{-1})=\phi(p)\phi(p^{-1})=0\cdot\phi(p^{-1})=0$, a contradiction.
\end{proof}

\textbf{Proposition.}  $\fp$ ($\cong\mathbb{Z}/p\mathbb{Z}$) is the prime subfield of any field of characteristic $p$.
\begin{proof}  Let $F$ be a field of characteristic $p$.  The idea again is to find an injective field homomorphism, this time, from $\fp$ into $F$.  Take $\phi$ to be the function that maps $m\in \fp$ to $m\cdot 1_F$.  It is well-defined, for if $m=n$ in $\fp$, then $p\mid (m-n)$, meaning $(m-n)1_F=0$, or that $m\cdot 1_F=n\cdot 1_F$, (showing that one element in $\fp$ does not get ``mapped'' to more than one element in $F$).  Since the above argument is reversible, we see that $\phi$ is one-to-one.

To complete the proof, we next show that $\phi$ is a field homomorphism.  That $\phi(1)=1_F$ and $\phi(0)=0_F$ are clear from the definition of $\phi$.  Additivity and multiplicativity of $\phi$ are readily verified, as follows: 
\begin{itemize}
\item $\phi(m+n)=(m+n)\cdot 1_F=m\cdot 1_F + n\cdot 1_F=\phi(m)+\phi(n)$; 
\item $\phi(mn)=mn\cdot 1_F=mn\cdot 1_F\cdot 1_F=(m\cdot 1_F)(n\cdot 1_F)=\phi(m)\phi(n)$.  
\end{itemize}
This shows that $\phi$ is a field homomorphism.
\end{proof}
%%%%%
%%%%%
\end{document}
