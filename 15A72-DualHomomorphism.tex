\documentclass[12pt]{article}
\usepackage{pmmeta}
\pmcanonicalname{DualHomomorphism}
\pmcreated{2013-03-22 12:29:33}
\pmmodified{2013-03-22 12:29:33}
\pmowner{rmilson}{146}
\pmmodifier{rmilson}{146}
\pmtitle{dual homomorphism}
\pmrecord{10}{32717}
\pmprivacy{1}
\pmauthor{rmilson}{146}
\pmtype{Definition}
\pmcomment{trigger rebuild}
\pmclassification{msc}{15A72}
\pmclassification{msc}{15A04}
\pmsynonym{adjoint homomorphism}{DualHomomorphism}
\pmsynonym{adjoint}{DualHomomorphism}
\pmrelated{LinearTransformation}
\pmrelated{DualSpace}
\pmrelated{DoubleDualEmbedding}

\usepackage{amsmath}
\usepackage{amsfonts}
\usepackage{amssymb}

\newcommand{\Hom}{\mathop{\mathrm{Hom}}\nolimits}
\newcommand{\Mat}{\mathop{\mathrm{Mat}}\nolimits}
\newcommand{\kfield}{\mathbb{K}}
\newcommand{\supt}{^t}
\newcommand{\dual}{^*}
\newcommand{\adj}{^{\displaystyle \star}}

\newcommand{\cP}[1]{\mathcal{P}_{#1}}
\newcommand{\ev}[1]{\mathrm{Ev}^{(#1)}}
\newcommand{\D}[1]{\mathrm{D}_{#1}}


\newcommand{\reals}{\mathbb{R}}
\newcommand{\natnums}{\mathbb{N}}
\newcommand{\cnums}{\mathbb{C}}

\newcommand{\lp}{\left(}
\newcommand{\rp}{\right)}
\newcommand{\lb}{\left[}
\newcommand{\rb}{\right]}

\newcommand{\supth}{^{\text{th}}}


\newtheorem{proposition}{Proposition}
\begin{document}
\paragraph{Definition.}  Let $U,V$ be vector spaces over a field
$\kfield$, and  $T:U\rightarrow V$ be a homomorphism (a
linear map) between them.
Letting $U\dual, V\dual$  denote the corresponding dual
spaces, we define the {\em dual homomorphism} 
$T\dual:V\dual\rightarrow U\dual$,
to be the linear mapping with action
$$\alpha \to \alpha\circ T,\quad \alpha\in V^*.$$

We can also characterize  $T^*$ as the adjoint of $T$ relative
to the natural evaluation bracket between linear forms and vectors:
$$\left<-,-\right>_U: U\dual\times U\rightarrow \kfield,\qquad
\left<\alpha,u\right> = \alpha(u),\quad \alpha\in U\dual,\; u\in U.$$
To be more precise $T\dual$ is characterized by the condition 
$$\left<T^*\alpha,u\right>_U = \left< \alpha,Tu\right>_V ,\quad \alpha\in
V\dual,\; u\in U.$$
% In other words, for $\alpha\in V\dual$ we define $T\dual\alpha$
% to be the element of $U\dual$ with action
% $$(T\dual\alpha):u\mapsto \alpha(Tu),\quad u\in U,\quad \alpha\in
% V\dual.$$

If $U$ and $V$ are finite dimensional, we can also characterize the
dualizing operation as the composition of the following canonical
isomorphisms:
$$\Hom(U,V)
\stackrel{\simeq}{\longrightarrow}
U\dual\otimes V
\stackrel{\simeq}{\longrightarrow}
(V^*)^*\otimes U^*
\stackrel{\simeq}{\longrightarrow}
\Hom(V^*,U^*).
$$

\paragraph{Category theory perspective.} The dualizing operation behaves
contravariantly with respect to composition, i.e.
$$(S\circ T)^* = T\dual \circ S\dual,$$
for all vector space homomorphisms $S,
T$ with suitably matched domains.  Furthermore, the dual of the
identity homomorphism is the identity homomorphism of the dual space.
Thus, using the language of category theory, the dualizing operation
can be characterized as the homomorphism action of the contravariant,
dual-space functor.

\paragraph{Relation to the matrix transpose.} The above properties closely
mirror the algebraic properties of the matrix transpose operation.
Indeed, $T^*$ is sometimes referred to as the transpose of $T$,
because at the level of matrices the dual homomorphism is calculated
by taking the transpose.  

To be more precise, suppose that $U$ and $V$ are finite-dimensional,
and let $M\in \Mat_{n,m}(\kfield)$ be the matrix of $T$ relative to
some fixed bases of $U$ and $V$. Then, the dual homomorphism $T\dual$
is represented as the transposed matrix $M\supt\in\Mat_{m,n}(\kfield)$
relative to the corresponding dual bases of $U\dual, V\dual$.


%%%%%
%%%%%
\end{document}
