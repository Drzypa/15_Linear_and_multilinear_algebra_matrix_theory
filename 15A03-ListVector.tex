\documentclass[12pt]{article}
\usepackage{pmmeta}
\pmcanonicalname{ListVector}
\pmcreated{2013-03-22 12:51:50}
\pmmodified{2013-03-22 12:51:50}
\pmowner{rmilson}{146}
\pmmodifier{rmilson}{146}
\pmtitle{list vector}
\pmrecord{5}{33200}
\pmprivacy{1}
\pmauthor{rmilson}{146}
\pmtype{Definition}
\pmcomment{trigger rebuild}
\pmclassification{msc}{15A03}
\pmclassification{msc}{15A90}
\pmdefines{column vector}
\pmdefines{row vector}

\newcommand{\cU}{\mathcal{U}}
\newcommand{\bF}{\mathbf{F}}
\newcommand{\bG}{\mathbf{G}}
\newcommand{\hbF}{\hat{\bF}}
\newcommand{\va}{a}
\newcommand{\vu}{u}
\newcommand{\vv}{v}
\newcommand{\bzero}{\mathbf{0}}
\newcommand{\supt}{^{\scriptscriptstyle\mathrm{T}}}
\newcommand{\kfield}{\mathbb{K}}
\usepackage{amsmath}
\usepackage{amsfonts}
\usepackage{amssymb}
\newcommand{\reals}{\mathbb{R}}
\newcommand{\natnums}{\mathbb{N}}
\newcommand{\cnums}{\mathbb{C}}
\newcommand{\znums}{\mathbb{Z}}
\newcommand{\lp}{\left(}
\newcommand{\rp}{\right)}
\newcommand{\lb}{\left[}
\newcommand{\rb}{\right]}
\newcommand{\supth}{^{\text{th}}}
\newtheorem{proposition}{Proposition}
\newtheorem{definition}[proposition]{Definition}
\newcommand{\nl}[1]{{\PMlinkescapetext{{#1}}}}
\newcommand{\pln}[2]{{\PMlinkname{{#1}}{#2}}}
\begin{document}
Let $\kfield$ be a field and $n$ a positive natural number.  We define
$\kfield^n$ to be the set of all mappings from the index list
$(1,2,\ldots,n)$ to $\kfield$.  Such a mapping $\va\in \kfield^n$ is
just a formal way of speaking of a list of field elements
$\va^1,\ldots, \va^n\in\kfield$.

The above description is somewhat restrictive. A more flexible
definition of a list vector is the following.  Let $I$ be a finite
list of indices\footnote{Distinct index sets are often used when
  working with multiple frames of reference.}, $I=(1,\ldots,n)$ is one
such possibility, and let $\kfield^I$ denote the set of all mappings
from $I$ to $\kfield$.  A list vector, an element of $\kfield^I$, is
just such a mapping.  Conventionally, superscripts are used to denote
the values of a list vector, i.e. for $\vu\in \kfield^I$ and $i\in I$,
we write $\vu^i$ instead of $\vu(i)$.

We add and scale list vectors point-wise, i.e. for $\vu, \vv \in
\kfield^I$ and $k\in \kfield$, we define $\vu+\vv\in \kfield^I$ and $k\vu\in
\kfield^I$, respectively by
\begin{align*}
(\vu+\vv)^i &= \vu^i+\vv^i,\quad i\in I,\\
(k\vu)^i &= k \vu^i,\quad i\in I.
\end{align*}
We also have the zero vector $\bzero\in \kfield^I$, namely the constant mapping
$$\bzero^i = 0,\quad i\in I.$$
The above operations give $\kfield^I$ the
structure of an (abstract) vector space over $\kfield$.

Long-standing traditions of linear algebra hold that elements of
$\kfield^I$ be regarded as column vectors.  For example, we write
$\va\in \kfield^n$ as
$$\va = 
\begin{pmatrix}
  \va^1 \\ \va^2 \\ \vdots \\ \va^n
\end{pmatrix}.$$
Row vectors are usually taken to represents linear forms on
$\kfield^I$.  In other words, row vectors are elements of the dual
space $\lp\kfield^I\rp^*$.  The components of a row vector are
customarily written with subscripts, rather than superscripts.  Thus,
we express a row vector $\alpha\in\lp\kfield^n\rp^*$ as 
$$\alpha = (\alpha_1,\ldots,\alpha_n).$$
%%%%%
%%%%%
\end{document}
