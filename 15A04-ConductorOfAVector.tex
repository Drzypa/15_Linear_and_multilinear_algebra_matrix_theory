\documentclass[12pt]{article}
\usepackage{pmmeta}
\pmcanonicalname{ConductorOfAVector}
\pmcreated{2013-03-22 14:05:19}
\pmmodified{2013-03-22 14:05:19}
\pmowner{CWoo}{3771}
\pmmodifier{CWoo}{3771}
\pmtitle{conductor of a vector}
\pmrecord{5}{35455}
\pmprivacy{1}
\pmauthor{CWoo}{3771}
\pmtype{Definition}
\pmcomment{trigger rebuild}
\pmclassification{msc}{15A04}
\pmsynonym{T-conductor}{ConductorOfAVector}
\pmsynonym{conductor}{ConductorOfAVector}
\pmsynonym{annihilator}{ConductorOfAVector}
\pmsynonym{annihilator polynomial}{ConductorOfAVector}
\pmsynonym{conductor polynomial}{ConductorOfAVector}

\endmetadata

% this is the default PlanetMath preamble.  as your knowledge
% of TeX increases, you will probably want to edit this, but
% it should be fine as is for beginners.

% almost certainly you want these
\usepackage{amssymb}
\usepackage{amsmath}
\usepackage{amsfonts}

% used for TeXing text within eps files
%\usepackage{psfrag}
% need this for including graphics (\includegraphics)
%\usepackage{graphicx}
% for neatly defining theorems and propositions
%\usepackage{amsthm}
% making logically defined graphics
%%%\usepackage{xypic}

% there are many more packages, add them here as you need them

% define commands here
\begin{document}
Let $k$ be a field, $V$ a vector space, $T:V\to V$ a linear transformation, and $W$ a $T$-invariant subspace of $V$. Let $x \in V$. The \emph{$T$-conductor} of $x$ \emph{in} $W$ is the set $S_T(x, W)$ containing all polynomials $g \in k[X]$ such that $g(T)x \in W$. It happens to be that this set is an ideal of the polynomial ring. We also use the term $T$-conductor of $x$ in $W$ to refer to the generator of such ideal.

In the special case $W=\{0\}$, the $T$-conductor is called \emph{$T$-annihilator} of $x$.
Another way to define the $T$-conductor of $x$ in $W$ is by saying that it is a monic polynomial $p$ of lowest degree such that $p(T)x \in W$. Of course this polynomial happens to be unique. So the $T$-annihilator of $x$ is the monic polynomial $m_x$ of lowest degree such that $m_x(T)x = 0$.
%%%%%
%%%%%
\end{document}
