\documentclass[12pt]{article}
\usepackage{pmmeta}
\pmcanonicalname{GeometricDerivationOfAdditionFormulasForSineAndCosine}
\pmcreated{2013-03-22 17:10:16}
\pmmodified{2013-03-22 17:10:16}
\pmowner{rm50}{10146}
\pmmodifier{rm50}{10146}
\pmtitle{geometric derivation of addition formulas for sine and cosine}
\pmrecord{7}{39483}
\pmprivacy{1}
\pmauthor{rm50}{10146}
\pmtype{Derivation}
\pmcomment{trigger rebuild}
\pmclassification{msc}{15-00}
\pmclassification{msc}{26A09}

\endmetadata

% this is the default PlanetMath preamble.  as your knowledge
% of TeX increases, you will probably want to edit this, but
% it should be fine as is for beginners.

% almost certainly you want these
\usepackage{amssymb}
\usepackage{amsmath}
\usepackage{amsfonts}

% used for TeXing text within eps files
%\usepackage{psfrag}
% need this for including graphics (\includegraphics)
%\usepackage{graphicx}
% for neatly defining theorems and propositions
%\usepackage{amsthm}
% making logically defined graphics
%%%\usepackage{xypic}
\usepackage{pst-plot}
\usepackage{psfrag}

% there are many more packages, add them here as you need them

% define commands here

\begin{document}
Here is a geometric derivation of the addition laws for sines (and cosines)


\begin{align*}
\sin(\alpha+\beta)&=\sin(\alpha)\cos(\beta)+\cos(\alpha)\sin(\beta)\\
\cos(\alpha+\beta)&=\cos(\alpha)\cos(\beta)-\sin(\alpha)\sin(\beta)
\end{align*}

First note that, by symmetry, it is clear that
\begin{align*}
\sin\left(x+\frac{\pi}{2}\right)&=\cos x\\
\cos\left(x+\frac{\pi}{2}\right)&=-\sin x\\
\sin(-x)&=-\sin x\\
\cos(-x)&=\cos x
\end{align*}
and so we can reduce proving the addition law to proving it in the case where $\alpha$, $\beta$, and $\alpha+\beta$ are all in the first quadrant.

We then have the situation pictured below:

\begin{center}
\begin{pspicture}(-.4,-.4)(5.4,5.4)
\psline[linewidth=1pt]{->}(-.4,0)(5.4,0)
\psline[linewidth=1pt]{->}(0,-.4)(0,5.4)
\psarc[linewidth=2pt](0,0){5}{0}{90}
\psline[linewidth=2pt, linecolor=red]{->}(0,0)(4.33,2.5)
\psline[linewidth=2pt, linecolor=blue]{->}(0,0)(2.5,4.33)
\psarc[linewidth=2pt, linecolor=red](0,0){1.2}{0}{30}
\rput(0.8,0.2){$\alpha$}
\psarc[linewidth=2pt, linecolor=blue](0,0){1.4}{30}{60}
\rput(1.3,1.2){$\beta$}
\psline[linewidth=2pt, linecolor=red](2.5,4.33)(3.88,2.24)
\psarc[linewidth=2pt,linecolor=red](3.88,2.24){0.7}{90}{124}
\psline[linewidth=2pt, linecolor=blue](3.88,0)(3.88,4.33)(2.5,4.33)
\psline[linewidth=0.5pt](3.68,0)(3.68,0.2)(3.88,0.2)
\psline[linewidth=0.5pt](3.68,4.33)(3.68,4.13)(3.88,4.13)
\psline[linewidth=0.5pt](3.708,2.14)(3.607,2.313)(3.78,2.413)
\rput(0,5.3){.}
\rput(5.3,0){.}
\rput(-.3,-.3){O}
\rput(2.5,4.63){A}
\rput(3.88,4.63){B}
\rput(4.08,2.04){\PMlinkescapetext{C}}
\rput(3.88,-.3){D}
\end{pspicture}
\end{center}

Now, from the definitions of $\sin$ and $\cos$, and assuming that $OA=1$, we see that
\begin{align*}
AC&=\sin(\beta)\\
OC&=\cos(\beta)
\end{align*}

Now,
\begin{align*}
\sin(\alpha)&=\sin(\angle DOC)=\frac{CD}{OC}=\frac{CD}{\cos(\beta)}\\
\cos(\alpha)&=\cos(\angle DOC)=\frac{OD}{OC}=\frac{OD}{\cos(\beta)}
\end{align*}
so we have that
\begin{align*}
CD&=\sin(\alpha)\cos(\beta)\\
OD&=\cos(\alpha)\cos(\beta)
\end{align*}

But also, it is clear that $\angle BCA=\angle DOC=\alpha$, and therefore we have similarly
\begin{align*}
\sin(\alpha)&=\sin(\angle BCA)=\frac{AB}{AC}=\frac{AB}{\sin(\beta)}\\
\cos(\alpha)&=\cos(\angle BCA)=\frac{BC}{AC}=\frac{BC}{\sin\beta}
\end{align*}
so that
\begin{align*}
AB&=\sin(\alpha)\sin(\beta)\\
BC&=\cos(\alpha)\sin(\beta)
\end{align*}
Thus $\sin(\alpha+\beta)$ is the $y$-coordinate of $A$, which is the $y$-coordinate of $B$, so
\[\sin(\alpha+\beta)=CD+BC=\sin(\alpha)\cos(\beta)+\cos(\alpha)\sin(\beta)\]
and $\cos(\alpha+\beta)$ is the $x$-coordinate of $A$, which is the $x$-coordinate of $B$ less the difference in $x$-coordinates between $B$ and $A$, so
\[\cos(\alpha+\beta)=OD-AB=\cos(\alpha)\cos(\beta)-\sin(\alpha)\sin(\beta)\]
%%%%%
%%%%%
\end{document}
