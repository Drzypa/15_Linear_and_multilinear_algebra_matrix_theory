\documentclass[12pt]{article}
\usepackage{pmmeta}
\pmcanonicalname{PartitionedMatrix}
\pmcreated{2013-03-22 13:32:55}
\pmmodified{2013-03-22 13:32:55}
\pmowner{mathcam}{2727}
\pmmodifier{mathcam}{2727}
\pmtitle{partitioned matrix}
\pmrecord{11}{34150}
\pmprivacy{1}
\pmauthor{mathcam}{2727}
\pmtype{Definition}
\pmcomment{trigger rebuild}
\pmclassification{msc}{15-00}
%\pmkeywords{Jordan canonical form}
%\pmkeywords{rational canonical form}
%\pmkeywords{smith normal form}
\pmrelated{JordanCanonicalForm}
\pmrelated{JordanCanonicalFormTheorem}
\pmdefines{block matrix}
\pmdefines{sub-matrix}
\pmdefines{submatrix}

% this is the default PlanetMath preamble.  as your knowledge
% of TeX increases, you will probably want to edit this, but
% it should be fine as is for beginners.

% almost certainly you want these
\usepackage{amssymb}
\usepackage{amsmath}
\usepackage{amsfonts}

% used for TeXing text within eps files
%\usepackage{psfrag}
% need this for including graphics (\includegraphics)
%\usepackage{graphicx}
% for neatly defining theorems and propositions
%\usepackage{amsthm}
% making logically defined graphics
%%%\usepackage{xypic}

% there are many more packages, add them here as you need them

% define commands here
\begin{document}
A \emph{partitioned matrix}, or a \emph{block matrix}, is
 a matrix $M$ that has been constructed from other smaller matrices.
 These smaller matrices are called \emph{blocks} or \emph{sub-matrices}
 of $M$.
 
 For instance, if we partition the below $5\times 5$ matrix
 as follows
 \begin{eqnarray*}
 L&=&\left( \begin{array}{cc|ccc}
 1 & 0 & 1 & 2 & 3 \\
 0 & 1 & 1 & 2 & 3 \\
 \hline
 2 & 3 & 9 & 9 & 9 \\
 2 & 3 & 9 & 9 & 9 \\
 2 & 3 & 9 & 9 & 9 \\
 \end{array} \right),
 \end{eqnarray*}
 then we can define the matrices

\begin{equation*}
A=\left( \begin{array}{cc}
 1 & 0 \\
 0 & 1
 \end{array} \right),
 B=\left( \begin{array}{ccc}
 1 & 2 & 3\\
 1 & 2 & 3
 \end{array} \right),
 C=\left( \begin{array}{cc}
 2 & 3 \\
 2 & 3 \\
 2 & 3
 \end{array} \right),
 D=\left( \begin{array}{ccc}
 9 & 9 & 9 \\
 9 & 9 & 9 \\
 9 & 9 & 9 \\
 \end{array} \right)
\end{equation*}
and write $L$ as

\begin{equation*}
 L=\left( \begin{array}{cc}
 A & B \\
 C & D
 \end{array} \right),\, \mbox{or\,\,}
 L=\left( \begin{array}{c|c}
 A & B \\
\hline
 C & D
 \end{array} \right).
\end{equation*}

If $A_1,\ldots, A_n$ are square matrices (of possibly
different sizes), then we define the \emph{direct sum} of
the matrices $A_1,\ldots, A_n$
as the partitioned matrix
$$\operatorname{diag}(A_1,\ldots, A_n) =\left( \begin{array}{c|c|c}
 A_1 &  &  \\
\hline
  & \ddots &  \\
\hline
  &  & A_n \\
 \end{array} \right),$$
where the off-diagonal blocks are zero.

If $A$ and $B$ are matrices of the same size partitioned into blocks of the same size, the partition of the sum is the sum of the partitions.

If $A$ and $B$ are $m\times n$ and $n\times k$ matrices, respectively, then if the blocks of $A$ and $B$ are of the correct size to be multiplied, then the blocks of the product are the products of the blocks.
%%%%%
%%%%%
\end{document}
