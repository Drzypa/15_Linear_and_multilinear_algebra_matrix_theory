\documentclass[12pt]{article}
\usepackage{pmmeta}
\pmcanonicalname{ReducibleMatrix}
\pmcreated{2013-03-22 13:18:20}
\pmmodified{2013-03-22 13:18:20}
\pmowner{Mathprof}{13753}
\pmmodifier{Mathprof}{13753}
\pmtitle{reducible matrix}
\pmrecord{11}{33810}
\pmprivacy{1}
\pmauthor{Mathprof}{13753}
\pmtype{Definition}
\pmcomment{trigger rebuild}
\pmclassification{msc}{15A48}
\pmdefines{irreducible matrix}

\endmetadata

% this is the default PlanetMath preamble.  as your knowledge
% of TeX increases, you will probably want to edit this, but
% it should be fine as is for beginners.

% almost certainly you want these
\usepackage{amssymb}
\usepackage{amsmath}
\usepackage{amsfonts}

% used for TeXing text within eps files
%\usepackage{psfrag}
% need this for including graphics (\includegraphics)
%\usepackage{graphicx}
% for neatly defining theorems and propositions
%\usepackage{amsthm}
% making logically defined graphics
%%%\usepackage{xypic}

% there are many more packages, add them here as you need them

% define commands here
\begin{document}
An $n\times n$ matrix $A$ is said to be a {\it reducible matrix} if and only if for some permutation matrix $P$, the matrix $P^TAP$ is block upper triangular.
If a square matrix is not reducible, it is said to be an {\it irreducible matrix}.  

The following conditions on an $n\times n$ matrix $A$ are equivalent.
\begin{enumerate}
\item
$A$ is an irreducible matrix.

\item
The digraph associated to $A$ is strongly connected.

\item
For each $i$ and $j$, there exists some $k$ such that $(A^k)_{ij}>0$.

\item
For any partition $J\sqcup K$ of the index set $\{1,2,\dots,n\}$, there exist $j\in J$ and $k\in K$ 
such that $a_{jk}\ne 0$.
\end{enumerate}

For certain applications, irreducible matrices are more useful than reducible matrices.  In particular, the Perron-Frobenius theorem gives more information about the spectra of irreducible matrices than of reducible matrices.
%%%%%
%%%%%
\end{document}
