\documentclass[12pt]{article}
\usepackage{pmmeta}
\pmcanonicalname{Determinant}
\pmcreated{2013-03-22 12:33:07}
\pmmodified{2013-03-22 12:33:07}
\pmowner{rmilson}{146}
\pmmodifier{rmilson}{146}
\pmtitle{determinant}
\pmrecord{24}{32797}
\pmprivacy{1}
\pmauthor{rmilson}{146}
\pmtype{Definition}
\pmcomment{trigger rebuild}
\pmclassification{msc}{15A15}
\pmrelated{LaplaceExpansion}
\pmrelated{Permanent}
\pmrelated{GeneralizedRuizsIdentity}

\usepackage{amsmath}
\usepackage{amsfonts}
\usepackage{amssymb}

\newcommand{\mM}{\mathbf{M}}
\begin{document}
\section*{Overview}

The \emph{determinant} is an algebraic operation that transforms a
square matrix $M$ into a scalar. This operation has many useful and
important properties.  For example, the determinant is zero if and
only the matrix $M$ is singular (no inverse exists).  The determinant
also has an important geometric interpretation as the area of a
parallelogram, and more generally as the volume of a
higher-dimensional parallelepiped.

The notion of determinant predates matrices and linear
transformations. Originally, the determinant was a number associated
to a system of $n$ linear equations in $n$ variables. This number
``determined'' whether the system possessed a unique solution.  In
this sense, two-by-two determinants were considered by Cardano at the
end of the 16th century and ones of arbitrary size (see the definition
below) by Leibniz about 100 years later.

\section*{Definition}

Let $M$ be an $n\times n$ matrix with entries $M_{ij}$ that are
elements of a given field\footnote{Most scientific and geometric
applications deal with matrices made up of real or complex numbers.
However, the determinant of a matrix over any field is well defined
sense and has all the properties of the more conventional
determinant. Indeed, many properties of the determinant remain
valid for matrices with entries in a commutative ring.}. The
determinant of $M$, or $\det M$ for short, is the scalar quantity
\begin{equation}
\label{eq:detdef}
\det M = \left|
\begin{array}{cccc}
M_{11} & M_{12} &\ldots & M_{1n} \\
M_{21} & M_{22} & \ldots & M_{2n}\\
\vdots & \vdots & \ddots & \vdots \\
M_{n1} & M_{n2} & \ldots & M_{nn}
\end{array}\right| = \sum_{\pi\in S_n} \mathrm{sgn}(\pi) M_{1\pi_1}
M_{2\pi_2}\cdots
M_{n\pi_n}.
\end{equation}
The index $\pi$ in the above sum varies over all the permutations of
$\{1,\ldots,n\}$ (i.e., the elements of the symmetric group $S_n$.)
Hence, there are $n!$ terms in the defining sum of the determinant.
The symbol $\operatorname{sgn}(\pi)$ denotes the parity of the
permutation; it is $\pm 1$ according to whether $\pi$ is an even or
odd permutation.  Using the Einstein summation convention one can also
express the above definition as
\begin{equation}
  \det M = \epsilon_{\pi_1\pi_2\dots \pi_n} M^{\pi_1}{}_1  M^{\pi_2}{}_2
  \cdots  M^{\pi_n}{}_{n},
\end{equation}
where we've raised the first index so that $M^i{}_j = M_{ij}$, and
where 
\[\epsilon_{\pi_1\dots \pi_n} = \operatorname{sgn}(\pi)\]
is known as the Levi-Civita permutation symbol.

By way of example, the determinant of a $2\times2$ matrix is given by
$$\left| \begin{matrix} M_{11} & M_{12} \\ M_{21}& M_{22}
\end{matrix}\right| = M_{11} M_{22}-M_{12}M_{21},$$
There are six permutations of the numbers $1,2,3$, namely
$$1\overset{+}23,\;2\overset{+}31,\;3\overset{+}12,\;1\overset{-}32,\;
3\overset{-}21,\;2\overset{-}13;$$
the overset sign indicates the permutation's signature. Accordingly,
the $3\times 3$ deterimant is a sum of the following $6$ terms:
$$ \left| \begin{matrix}
M_{11} & M_{12} & M_{13} \\
M_{21} & M_{22} & M_{23} \\
M_{31} & M_{32} & M_{33}
\end{matrix}\right| =
\begin{array}{rr}
\\
M_{11} M_{22} M_{33} + M_{12} M_{23} M_{31} + M_{13}M_{21} M_{32} \\
-M_{11}M_{23}M_{32}-M_{13}M_{22} M_{31} - M_{12}M_{21}M_{33}
\end{array}
$$

\section*{Remarks and important properties}

\begin{enumerate}
\item
\label{pd:mult} The determinant operation converts matrix multiplication into
scalar multiplication;
$$
\det(AB)=\det(A) \det(B),
$$
where $A, B$ are square matrices of the same size.
\item The determinant operation is multi-linear, and anti-symmetric
with respect to the matrix's rows and columns. See the
multi-linearity attachment for more details.
\item The determinant of a lower triangular, or an upper triangular
matrix is the product of the diagonal entries, since all the other summands
in (1) are zero.
\item \PMlinkname{Similar matrices}{SimilarMatrix}
have the same determinant. To be more precise,
let $A$ and $X$ be square matrices with $X$ invertible. Then,
$$
\det(XAX^{-1})= \det(A).
$$
In particular, if we let $X$ be the matrix representing a change of
basis, this shows that the determinant is independent of the basis.
The same is true of the trace of a matrix. In fact, the whole
characteristic polynomial of an endomorphism is definable without
using a basis or a matrix, and it turns out that the determinant and
trace are two of its coefficients.
\item The determinant of a matrix $A$ is zero if and only if $A$ is
singular; that is, if there exists a non-trivial solution to the
homogeneous equation
$$
A\mathbf{x} = \mathbf{0}.
$$
\item The transpose
operation does not change the determinant:
$$
\det A^{\scriptscriptstyle\mathrm{T}} = \det A.
$$
\item
The determinant of a diagonalizable transformation is equal to the product
of its eigenvalues, counted with multiplicities.
\item The determinant is homogeneous of degree $n$. This means that
$$
\det(k M ) = k^n \det M,\quad k\;\text{is a scalar.}
$$
\end{enumerate}
%%%%%
%%%%%
\end{document}
