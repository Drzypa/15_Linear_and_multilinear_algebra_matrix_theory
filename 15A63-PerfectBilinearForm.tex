\documentclass[12pt]{article}
\usepackage{pmmeta}
\pmcanonicalname{PerfectBilinearForm}
\pmcreated{2013-03-22 15:35:04}
\pmmodified{2013-03-22 15:35:04}
\pmowner{matsuura}{2984}
\pmmodifier{matsuura}{2984}
\pmtitle{perfect bilinear form}
\pmrecord{5}{37494}
\pmprivacy{1}
\pmauthor{matsuura}{2984}
\pmtype{Definition}
\pmcomment{trigger rebuild}
\pmclassification{msc}{15A63}
\pmclassification{msc}{11E39}

\endmetadata

% this is the default PlanetMath preamble.  as your knowledge
% of TeX increases, you will probably want to edit this, but
% it should be fine as is for beginners.

% almost certainly you want these
\usepackage{amssymb}
\usepackage{amsmath}
\usepackage{amsfonts}

% used for TeXing text within eps files
%\usepackage{psfrag}
% need this for including graphics (\includegraphics)
%\usepackage{graphicx}
% for neatly defining theorems and propositions
%\usepackage{amsthm}
% making logically defined graphics
%%%\usepackage{xypic}

% there are many more packages, add them here as you need them

% define commands here
\begin{document}
Let $A$, $B$, and $C$ be abelian groups. A bilinear form

$$
\varphi: A\times B \rightarrow C
$$
is said to be \emph{\PMlinkescapetext{perfect}} if the associated group homomorphisms

\begin{align*}
A &\rightarrow \operatorname{Hom}(B,C)\\
a &\mapsto \varphi(a,\cdot)
\end{align*}
and
\begin{align*}
B &\rightarrow \operatorname{Hom}(A,C)\\
b &\mapsto \varphi(\cdot,b)
\end{align*}
are injective. 

In particular, if $C$ is finite then the finiteness of either $A$ or $B$ implies the finiteness of the other.
%%%%%
%%%%%
\end{document}
