\documentclass[12pt]{article}
\usepackage{pmmeta}
\pmcanonicalname{PropertiesOfSpanningSets}
\pmcreated{2013-03-22 18:05:40}
\pmmodified{2013-03-22 18:05:40}
\pmowner{CWoo}{3771}
\pmmodifier{CWoo}{3771}
\pmtitle{properties of spanning sets}
\pmrecord{7}{40634}
\pmprivacy{1}
\pmauthor{CWoo}{3771}
\pmtype{Result}
\pmcomment{trigger rebuild}
\pmclassification{msc}{15A03}
\pmclassification{msc}{16D10}

\endmetadata

\usepackage{amssymb,amscd}
\usepackage{amsmath}
\usepackage{amsfonts}
\usepackage{mathrsfs}

% used for TeXing text within eps files
%\usepackage{psfrag}
% need this for including graphics (\includegraphics)
%\usepackage{graphicx}
% for neatly defining theorems and propositions
\usepackage{amsthm}
% making logically defined graphics
%%\usepackage{xypic}
\usepackage{pst-plot}

% define commands here
\newcommand*{\abs}[1]{\left\lvert #1\right\rvert}
\newtheorem{prop}{Proposition}
\newtheorem{thm}{Theorem}
\newtheorem{ex}{Example}
\newcommand{\real}{\mathbb{R}}
\newcommand{\pdiff}[2]{\frac{\partial #1}{\partial #2}}
\newcommand{\mpdiff}[3]{\frac{\partial^#1 #2}{\partial #3^#1}}
\def\Sp{\operatorname{Sp}}
\begin{document}
Let $V$ be a vector space over a field $k$.  Let $S$ be a subset of $V$.  We denote $\Sp(S)$ the span of the set $S$.  Below are some basic properties of spanning sets.

\begin{enumerate}
\item If $S\subseteq T$, then $\Sp(S)\subseteq \Sp(T)$.  In particular, if $\Sp(S)=V$, every superset of $S$ spans (generates) $V$.
\begin{proof} If $v \in \Sp(S)$, then $v=r_1v_1+\cdots +r_nv_n$ for $v_i\in S$.  But $v_i\in T$ by assumption.  So $v\in \Sp(T)$ as well.  If $\Sp(S)=V$, and $S\subseteq T$, then $V=\Sp(S)\subseteq \Sp(T) \subseteq V$.  \end{proof}
\item If $S$ contains $0$, then $\Sp(S-\lbrace 0\rbrace)=\Sp(S)$.
\begin{proof} Let $T=S-\lbrace 0\rbrace$.  So $\Sp(T)\subseteq \Sp(S)$ by 1 above.  If $v\in \Sp(S)$, then $v=r_1v_1+\cdots + r_nv_n$.  If one of the $v_i$'s, say $v_i$, is $0$, then $v=r_2v_2+\cdots +r_nv_n\in \Sp(T)$.  \end{proof}
\item It is not true that if $S_1\supseteq S_2\supseteq \cdots$ is a chain of subsets, each spanning the same subspace $W$ of $V$, so does their intersection.
\begin{proof}  Take $V=\mathbb{R}^n$, the Euclidean space in $n$ dimensions.  For each $i=1,2,\ldots$, let $S_i$ be the closed ball centered at the origin, with radius $1/i$.  Then $\Sp(S_i)=V$.  But the intersection of these $S_i$'s is just the origin, whose span is itself, not $V$.
\end{proof}
\item $S$ is a basis for $V$ iff $S$ is a minimal spanning set of $V$.  Here, minimal means that any deletion of an element of $S$ is no longer a spanning set of $V$.
\begin{proof}  If $S$ is a basis for $V$, then $S$ spans $V$ and $S$ is linearly independent.  Let $T$ be the set obtained from $S$ with $v\in S$ deleted.  If $T$ spans $V$, then $v$ can be written as a linear combination of elements in $T$.  But then $S=T\cup \lbrace v\rbrace$ would no longer be linearly independent, contradiction the assumption.  Therefore, $S$ is minimal.

Conversely, suppose $S$ is a minimal spanning set for $V$.  Furthermore, suppose that $S$ is linearly dependent.  Let $0=r_1v_1+\cdots r_nv_n$, with $r_1\ne 0$.  Then 
\begin{equation}
v_1=s_2v_2+\cdots +s_nv_n,
\end{equation} 
where $s_i=-r_i/r_1$.  So any linear combination of elements in $S$ involving $v_1$ can be replaced by a linear combination not involving $v_1$ through equation (1).  Therefore $\Sp(S)=\Sp(S-\lbrace v\rbrace)$.  But this means that $S$ is not minimal, contrary to our assumption.  Therefore, $S$ must be linearly independent.
\end{proof}
\end{enumerate}

\textbf{Remark}.  All of the properties above can be generalized to modules over rings, except the last one, where the implication is only one-sided: basis implying minimal spanning set.
%%%%%
%%%%%
\end{document}
