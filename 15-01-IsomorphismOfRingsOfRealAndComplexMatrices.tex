\documentclass[12pt]{article}
\usepackage{pmmeta}
\pmcanonicalname{IsomorphismOfRingsOfRealAndComplexMatrices}
\pmcreated{2013-03-22 16:17:15}
\pmmodified{2013-03-22 16:17:15}
\pmowner{Wkbj79}{1863}
\pmmodifier{Wkbj79}{1863}
\pmtitle{isomorphism of rings of real and complex matrices}
\pmrecord{10}{38403}
\pmprivacy{1}
\pmauthor{Wkbj79}{1863}
\pmtype{Theorem}
\pmcomment{trigger rebuild}
\pmclassification{msc}{15-01}
\pmclassification{msc}{15A33}
\pmclassification{msc}{15A21}

\endmetadata

\usepackage{amssymb}
\usepackage{amsmath}
\usepackage{amsfonts}

\usepackage{psfrag}
\usepackage{graphicx}
\usepackage{amsthm}
%%\usepackage{xypic}

\newtheorem*{thm*}{Theorem}

\begin{document}
Note that \PMlinkname{submatrix notation}{Submatrix} will be used within this entry.  Also, for any positive integer $n$, $M_{n \times n}(R)$ will be used to denote the ring of $n \times n$ matrices with entries from the ring $R$, and $R_n$ will be used to denote the following subring of $M_{2n \times 2n}(\mathbb{R})$:

$$R_n=\left\{ P \in M_{2n \times 2n}(\mathbb{R}) : P=\left( \begin{array}{cc}
A & B \\
-B & A \end{array}
\right)  \text{ for some } A,B \in M_{n \times n}(\mathbb{R}) \right\}$$

\begin{thm*}
For any positive integer $n$, $R_n \cong M_{n \times n}(\mathbb{C})$.
\end{thm*}

\begin{proof}
Define $\varphi \colon R_n \to M_{n \times n}(\mathbb{C})$ by $\displaystyle \varphi \left( \left( \begin{array}{cc}
A & B \\
-B & A \end{array}
\right) \right) = A+iB$ for $A,B \in M_{n \times n}(\mathbb{R})$.

Let $A,B,C,D \in M_{n \times n}(\mathbb{R})$ such that $\displaystyle \varphi \left( \left( \begin{array}{cc}
A & B \\
-B & A \end{array}
\right) \right) =\varphi \left( \left( \begin{array}{cc}
C & D \\
-D & C \end{array}
\right) \right)$.  Then $A+iB=C+iD$.  Therefore, $A=C$ and $B=D$.  Hence, $\displaystyle \left( \begin{array}{cc}
A & B \\
-B & A \end{array}
\right) = \left( \begin{array}{cc}
C & D \\
-D & C \end{array}
\right)$.  It follows that $\varphi$ is injective.

Let $Z \in M_{n \times n}(\mathbb{C})$.  Then there exist $X,Y \in M_{n \times n}(\mathbb{R})$ such that $X+iY=Z$.  Since $\varphi \left( \left( \begin{array}{cc}
X & Y \\
-Y & X \end{array}
\right) \right)=X+iY=Z$, it follows that $\varphi$ is surjective.

Let $A,B,C,D \in M_{n \times n}(\mathbb{R})$.  Then

\begin{center}
$\begin{array}{rl}
\displaystyle \varphi \left( \left( \begin{array}{cc}
A & B \\
-B & A \end{array}
\right) + \left( \begin{array}{cc}
C & D \\
-D & C \end{array}
\right) \right) & \displaystyle =\varphi \left( \left( \begin{array}{cc}
A+C & B+D \\
-B-D & A+C \end{array}
\right) \right) \\
\\
& =A+C+i(B+D) \\
\\
& =A+iB+C+iD \\
\\
& \displaystyle =\varphi \left( \left( \begin{array}{cc}
A & B \\
-B & A \end{array}
\right) \right) + \varphi \left( \left( \begin{array}{cc}
C & D \\
-D & C \end{array}
\right) \right) \end{array}$
\end{center}

and

\begin{center}
$\begin{array}{rl}
\displaystyle \varphi \left( \left( \begin{array}{cc}
A & B \\
-B & A \end{array}
\right) \left( \begin{array}{cc}
C & D \\
-D & C \end{array}
\right) \right) & \displaystyle =\varphi \left( \left( \begin{array}{cc}
AC-BD & AD+BC \\
-AD-BC & AC-BD \end{array}
\right) \right) \\
\\
& =AC-BD+i(AD+BC) \\
\\
& =(A+iB)(C+iD) \\
\\
& \displaystyle =\varphi \left( \left( \begin{array}{cc}
A & B \\
-B & A \end{array}
\right) \right) \varphi \left( \left( \begin{array}{cc}
C & D \\
-D & C \end{array}
\right) \right) . \end{array}$
\end{center}

It follows that $\varphi$ is an \PMlinkname{isomorphism}{RingIsomorphism}.
\end{proof}
%%%%%
%%%%%
\end{document}
