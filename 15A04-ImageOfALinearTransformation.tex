\documentclass[12pt]{article}
\usepackage{pmmeta}
\pmcanonicalname{ImageOfALinearTransformation}
\pmcreated{2013-03-22 13:48:32}
\pmmodified{2013-03-22 13:48:32}
\pmowner{Koro}{127}
\pmmodifier{Koro}{127}
\pmtitle{image of a linear transformation}
\pmrecord{8}{34530}
\pmprivacy{1}
\pmauthor{Koro}{127}
\pmtype{Definition}
\pmcomment{trigger rebuild}
\pmclassification{msc}{15A04}
\pmrelated{RankNullityTheorem}
\pmrelated{KernelOfALinearTransformation}

\endmetadata

% this is the default PlanetMath preamble.  as your knowledge
% of TeX increases, you will probably want to edit this, but
% it should be fine as is for beginners.

% almost certainly you want these
\usepackage{amssymb}
\usepackage{amsmath}
\usepackage{amsfonts}

% used for TeXing text within eps files
%\usepackage{psfrag}
% need this for including graphics (\includegraphics)
%\usepackage{graphicx}
% for neatly defining theorems and propositions
%\usepackage{amsthm}
% making logically defined graphics
%%%\usepackage{xypic}

% there are many more packages, add them here as you need them

% define commands here

\newcommand{\sR}[0]{\mathbb{R}}
\newcommand{\sC}[0]{\mathbb{C}}
\newcommand{\sN}[0]{\mathbb{N}}
\newcommand{\sZ}[0]{\mathbb{Z}}
\begin{document}
\PMlinkescapeword{image}
{\bf Definition} 
Let $T:V\to W$ be a linear transformation. Then the {\bf image} of
$T$ is the set
$$ \operatorname{Im} (T) = \{ w\in W \mid w=T(v) \,\mbox{for some}\, v\in V\} = T(V).$$

\subsubsection{Properties}
%Let $T$ be as above.
\begin{enumerate}
\item The dimension of $\operatorname{Im}(T)$ is called the rank of $T$;
\item $T$ is a surjection, if and only if  $\operatorname{Im}(T)=W$;
\item $\operatorname{Im}(T)$ is a vector subspace of $W$;
\item If $L\colon W\to U$ is a linear transformation, then $\operatorname{Im}(LT) =L(\operatorname{Im}(T))$;
\end{enumerate}
%%%%%
%%%%%
\end{document}
