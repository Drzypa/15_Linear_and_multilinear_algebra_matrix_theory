\documentclass[12pt]{article}
\usepackage{pmmeta}
\pmcanonicalname{SimultaneousTriangularisationOfCommutingMatricesOverAnyField}
\pmcreated{2013-03-22 15:29:38}
\pmmodified{2013-03-22 15:29:38}
\pmowner{lars_h}{9802}
\pmmodifier{lars_h}{9802}
\pmtitle{simultaneous triangularisation of commuting matrices over any field}
\pmrecord{4}{37352}
\pmprivacy{1}
\pmauthor{lars_h}{9802}
\pmtype{Theorem}
\pmcomment{trigger rebuild}
\pmclassification{msc}{15A21}
\pmrelated{CommutingMatrices}

\endmetadata

\usepackage{amsmath,amsfonts,amssymb,amsthm}

\newtheorem*{theorem}{Theorem}
\newtheorem{lemma}{Lemma}


\newcommand{\mc}{\mathcal}
\newcommand{\vek}{\mathbf}

\newcommand{\Mat}{\mathrm{M}}
\newcommand{\GL}{\mathrm{GL}}
\newcommand{\Trans}[1]{#1^{\mathrm{T}}\!}
\begin{document}
Let $\vek{e}_i$ denote the (column) vector whose $i$th position is $1$ 
and where all other positions are $0$. Denote by $[n]$ the set 
$\{1,\dotsc,n\}$. Denote by $\Mat_n(\mc{K})$ the set of all $n \times 
n$ matrices over $\mc{K}$, and by $\GL_n(\mc{K})$ the set of all 
invertible elements of $\Mat_n(\mc{K})$. Let $d_i$ be the function 
which extracts the $i$th diagonal element of a matrix, i.e., \(d_i(A) 
= \Trans{\vek{e}_i} A \vek{e}_i\).


\begin{theorem}
  Let $\mc{K}$ be a field, let \(A_1,\dotsc,A_r \in \Mat_n(\mc{K})\) 
  be pairwise commuting matrices, and let $\mc{L}$ be a field extension 
  of $\mc{K}$ in which the characteristic polynomials of all $A_k$ 
  \PMlinkname{split}{SplittingField}. Then there exists some 
  \(P \in \GL_n(\mc{L})\) such that
  \begin{enumerate}
    \item
      \(P^{-1} A_k P\) is upper triangular for all \(k=1,\dotsc,r\),
      and
    \item
      if \(i,j,l \in [n]\) are such that \(i \leqslant l \leqslant j\) 
      and \(d_i(P^{-1} A_k P) = d_j(P^{-1} A_k P)\) for all 
      \(k=1,\dotsc,r\), then \(d_l(P^{-1} A_k P) = d_j(P^{-1} A_k P)\) 
      for all \(k=1,\dotsc,r\) as well.
  \end{enumerate}
\end{theorem}

The proof relies on two lemmas.

\begin{lemma} \label{L1}
  Let $\mc{K}$ be a field, let \(A_1,\dotsc,A_r \in \Mat_n(\mc{K})\) 
  be pairwise commuting matrices, and let $\mc{L}$ be a field extension 
  of $\mc{K}$ in which the characteristic polynomials of all $A_k$ 
  split. Then there exists some nonzero \(\vek{u} \in \mc{L}^n\) which 
  is an eigenvector of $A_k$ for all \(k=1,\dotsc,r\).
\end{lemma}

\begin{lemma} \label{L2}
  For any sequence \(R_1,\dotsc,R_r \in \Mat_n(\mc{L})\) of upper 
  triangular pairwise commuting matrices and every row index 
  \(i \in [n]\), there exists \(\vek{v} \in \mc{L}^n \setminus \{0\}\) 
  such that
  \begin{equation*}
    R_k \vek{v} = d_i(R_k) \vek{v}
    \quad\text{for all \(k \in [r]\).}
  \end{equation*}
\end{lemma}


\begin{proof}
  This is by induction on $n$. The induction hypothesis is that given 
  pairwise commuting matrices \(A_1,\dotsc,A_r \in \Mat_n(\mc{L})\), 
  whose characteristic polynomials all split in $\mc{L}$, and a 
  sequence of arbitrary scalars \(\mu_1,\dotsc,\mu_r \in \mc{L}\), 
  there exists some \(P \in \GL_n(\mc{L})\) such that:
  \begin{enumerate}
    \item \label{Cond:Triangular}
      \(P^{-1} A_k P\) is upper triangular for all \(k=1,\dotsc,r\).
    \item \label{Cond:Block}
      If some \(i,j \in [n]\) are such that \(i<j\) and 
      \(d_j(P^{-1} A_k P) = d_i(P^{-1} A_k P)\) for all \(k \in [r]\), 
      then \(d_{i+1}(P^{-1} A_k P) = d_i(P^{-1} A_k P)\).
    \item \label{Cond:Forst}
      If some \(j \in [n]\) is such that \(d_j(P^{-1} A_k P) = \mu_k\) 
      for all \(k \in [r]\), then \(d_1(P^{-1} A_k P) = \mu_k\) 
      for all \(k \in [r]\).
  \end{enumerate}
  For \(n=1\) this hypothesis is trivially fulfilled (all $1 \times 1$ 
  matrices are upper triangular). Assume that it holds for \(n=m\) and 
  consider the case \(n=m+1\).
  
  It is easy to see that condition~\ref{Cond:Triangular} implies 
  that $P\vek{e}_1$ must be an eigenvector that is common to all the 
  matrices. If there exists a nonzero vector \(\vek{u}_1 \in \mc{L}^n\) 
  such that \(A_k \vek{u}_1 = \mu_k \vek{u}_1\) for all \(k=1,\dotsc,r\) 
  then this is such a common eigenvector, and in that case let 
  \(\lambda_k=\mu_k\) for all \(k=1,\dotsc,r\). Otherwise there by 
  Lemma~\ref{L1} exists a vector
  \(\vek{u}_1 \in \mc{L}^n \setminus \{\vek{0}\}\) such that 
  \(A_k \vek{u}_1 = \lambda_k \vek{u}_1\) for some 
  \(\{\lambda_k\}_{k=1}^r \subseteq \mc{L}\). Either way, one gets a 
  suitable candidate $\vek{u}_1$ for $P\vek{e}_1$ and eigenvalues 
  \(\lambda_1,\dotsc,\lambda_r\) that incidentally will satisfy 
  \(d_1(P^{-1} A_k P) = \lambda_k\) for all \(k \in [r]\).
  
  Let \(\vek{u}_2,\dotsc,\vek{u}_n \in \mc{L}^n\) be arbitrary 
  vectors such that \(\{\vek{u}_i\}_{i=1}^n\) is a basis of 
  $\mc{L}^n$. Let $U$ be the $n \times n$ matrix whose $i$th column 
  is $\vek{u}_i$ for \(1 \leqslant i \leqslant n\).\footnote{By 
  imposing extra conditions on the choice of the basis 
  \(\{\vek{u}_i\}_{i=1}^n\) (such as for example requesting that 
  it is orthonormal) at this point, one can often prove a stronger 
  claim where the choice of $P$ is restricted to some smaller 
  group of matrices (for example the group of orthogonal 
  matrices), but this requires assuming additional things about 
  the fields $\mc{K}$ and $\mc{L}$.}
  Then $U$ is invertible and for each $k$ the first column of 
  \(B_k = U^{-1} A_k U\) is
  \begin{equation*}
    U^{-1} A_k U \vek{e}_1 =
    U^{-1} A_k \vek{u}_1 =
    \lambda_k U^{-1} \vek{u}_1 =
    \lambda_k \vek{e}_1
    \text{.}
  \end{equation*}
  Furthermore
  \begin{multline*}
    B_j B_k = U^{-1} A_j U U^{-1} A_k U =
    U^{-1} A_j A_k U = \\ = 
    U^{-1} A_k A_j U = 
    U^{-1} A_k U U^{-1} A_j U = B_k B_j
  \end{multline*}
  for all $j$ and $k$.
  
  \PMlinkescapeword{side}
  \PMlinkescapeword{contains}
  
  Now let $A_k'$ be the matrix formed from rows and columns $2$ though 
  $n$ of $B_k$. Since \(\det(A_k -\nobreak xI) = 
  \det( B_k -\nobreak xI ) = (\lambda_k -\nobreak x) 
  \det( A_k' -\nobreak xI )\) by 
  \PMlinkname{expansion}{LaplaceExpansion} along the first column, 
  it follows that the characteristic polynomial of $A_k'$ splits in 
  $\mc{L}$. Furthermore all the $A_k'$ have side \(m = n-1\) and 
  commute pairwise with each other, whence by the induction hypothesis 
  there exists some \(P' \in \GL_{n-1}(\mc{L})\) such that every 
  \(P'^{-1} A_k' P'\) is upper triangular. Let \(P = 
  U \left(\begin{smallmatrix} 1& 0 \\ 0& P' \end{smallmatrix}\right) 
  \). Then the submatrix consisting of rows and columns $2$ through 
  $n$ of $P^{-1} A_k P$ is equal to \(P'^{-1} A_k' P'\) and hence 
  contains no nonzero subdiagonal elements. Furthermore the first 
  column of $P^{-1} A_k P$ is equal to the first column of $B_k$ and 
  thus the $P^{-1} A_k P$ are all upper triangular, as claimed.
  
  It also follows from the induction hypothesis that $P$ can be chosen 
  such that \(d_2(P^{-1} A_k P) = d_1(P'^{-1} A_k' P') = \lambda_k = 
  d_1(P^{-1} A_k P)\) for all \(k \in [r]\) if there is any \(j 
  \geqslant 2\) for which \(d_j(P^{-1} A_k P) = d_{j-1}(P'^{-1} A_k' 
  P') = \lambda_k = d_1(P^{-1} A_k P)\) for all \(k \in [r]\) and 
  more generally if \(2 \leqslant i < j\) are such that 
  \(d_j(P^{-1} A_k P) = d_i(P^{-1} A_k P)\) for all \(k \in [r]\) 
  then similarly \(d_{i+1}(P^{-1} A_k P) = d_i(P^{-1} A_k P)\) 
  for all \(k \in [r]\). This has verified 
  condition~\ref{Cond:Block} of the induction hypothesis. 
  For the remaining condition~\ref{Cond:Forst}, one may first observe 
  that if there is some \(i \in [n]\) such that \(d_i(P^{-1} A_k P) = 
  \mu_k\) for all \(k \in [r]\) then by Lemma~\ref{L2} 
  there exists a nonzero \(\vek{v} \in \mc{L}^n\) such that 
  \(P^{-1} A_k P \vek{v} = \mu_k \vek{v}\) for all \(k \in [r]\). This 
  means $P \vek{v}$ will fulfill the condition for choice of 
  $\vek{u}_1$, and hence \(d_1(P^{-1} A_k P) = \lambda_k = \mu_k\) as 
  claimed.
  
  The theorem now follows from the principle of induction.
\end{proof}
%%%%%
%%%%%
\end{document}
