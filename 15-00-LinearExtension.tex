\documentclass[12pt]{article}
\usepackage{pmmeta}
\pmcanonicalname{LinearExtension}
\pmcreated{2013-03-22 15:24:06}
\pmmodified{2013-03-22 15:24:06}
\pmowner{GrafZahl}{9234}
\pmmodifier{GrafZahl}{9234}
\pmtitle{linear extension}
\pmrecord{7}{37240}
\pmprivacy{1}
\pmauthor{GrafZahl}{9234}
\pmtype{Definition}
\pmcomment{trigger rebuild}
\pmclassification{msc}{15-00}
\pmrelated{basis}
\pmrelated{Basis}
\pmdefines{bilinear extension}
\pmdefines{multilinear extension}
\pmdefines{$n$-linear extension}

% this is the default PlanetMath preamble.  as your knowledge
% of TeX increases, you will probably want to edit this, but
% it should be fine as is for beginners.

% almost certainly you want these
\usepackage{amssymb}
\usepackage{amsmath}
\usepackage{amsfonts}

% used for TeXing text within eps files
%\usepackage{psfrag}
% need this for including graphics (\includegraphics)
%\usepackage{graphicx}
% for neatly defining theorems and propositions
\usepackage{amsthm}
% making logically defined graphics
%%%\usepackage{xypic}

% there are many more packages, add them here as you need them

% define commands here
\newcommand{\<}{\langle}
\renewcommand{\>}{\rangle}
\newcommand{\Bigcup}{\bigcup\limits}
\newcommand{\DirectSum}{\bigoplus\limits}
\newcommand{\Prod}{\prod\limits}
\newcommand{\Sum}{\sum\limits}
\newcommand{\h}{\widehat}
\newcommand{\mbb}{\mathbb}
\newcommand{\mbf}{\mathbf}
\newcommand{\mc}{\mathcal}
\newcommand{\mmm}[9]{\left(\begin{array}{rrr}#1&#2&#3\\#4&#5&#6\\#7&#8&#9\end{array}\right)}
\newcommand{\mf}{\mathfrak}
\newcommand{\ol}{\overline}

% Math Operators/functions
\DeclareMathOperator{\Aut}{Aut}
\DeclareMathOperator{\End}{End}
\DeclareMathOperator{\Frob}{Frob}
\DeclareMathOperator{\cwe}{cwe}
\DeclareMathOperator{\id}{id}
\DeclareMathOperator{\mult}{mult}
\DeclareMathOperator{\we}{we}
\DeclareMathOperator{\wt}{wt}

\newcommand{\Set}{\mathsf{Set}}
\newcommand{\RMod}{\mathsf{RMod}}
\begin{document}
\PMlinkescapeword{extension}
\PMlinkescapeword{representation}
Let $R$ be a commutative ring, $M$ a free $R$-module, $B$ a basis of
$M$, and $N$ a further $R$-module. Each element $m\in M$ then has a
unique representation
\begin{equation*}
m=\Sum_{b\in B}m_b b,
\end{equation*}
where $m_b\in R$ for all $b\in B$, and only finitely many $m_b$ are
non-zero. Given a set map $f_1\colon B\to N$ we may therefore define the $R$-module homomorphism $\varphi_1 \colon M\to N$, called the \emph{linear extension} of $f_1$, such that
\[
 m \mapsto\Sum_{b\in B}m_bf_1(b).
\]
The map $\varphi_1$ is the unique homomorphism from $M$ to $N$ whose restriction to $B$ is $f_1$.   

The above observation has a convenient reformulation in terms of category theory.  Let $\RMod$ denote the category of $R$-modules, and $\Set$ the category of sets. Consider the adjoint functors $U\colon\RMod\to \Set$, the forgetful functor that maps an $R$-module to its underlying set, and $F \colon \Set \to \RMod$,
the free module functor that maps a set to the free $R$-module generated by that set.  To say that $U$ is right-adjoint to $F$ is the same as saying that every set map from $B$ to $U(N)$, the set underlying $N$, corresponds naturally and bijectively to an $R$-module homomorphism from $M=F(B)$ to $N$.

Similarly, given a map $f_2\colon B^2\to N$, we may define the
\emph{bilinear extension}
\begin{align*}
\varphi_2\colon&M^2\to N&(m,n)&\mapsto\Sum_{b\in B}\Sum_{c\in B}m_bn_cf_2(b,c),
\end{align*}
which is the unique bilinear map from $M^2$ to $N$ whose restriction
to $B^2$ is $f_2$.

Generally, for any positive integer $n$ and a map $f_n\colon B^n\to
N$, we may define the \emph{$n$-linear extension}
\begin{align*}
\varphi_n\colon&M^n\to N&m&\mapsto\Sum_{b\in B^n}m_bf_n(b)
\end{align*}
quite compactly using multi-index notation: $m_b=\Prod_{k=1}^nm_{k,b_k}$.

\subsection*{Usage}

The notion of linear extension is typically used as a
\emph{manner-of-speaking}. Thus, when a multilinear map is defined
explicitly in a mathematical text, the images of the basis elements
are given accompanied by the phrase ``by multilinear extension'' or
similar.
%%%%%
%%%%%
\end{document}
