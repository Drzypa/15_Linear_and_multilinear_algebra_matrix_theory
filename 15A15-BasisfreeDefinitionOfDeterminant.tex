\documentclass[12pt]{article}
\usepackage{pmmeta}
\pmcanonicalname{BasisfreeDefinitionOfDeterminant}
\pmcreated{2013-03-22 16:51:39}
\pmmodified{2013-03-22 16:51:39}
\pmowner{mps}{409}
\pmmodifier{mps}{409}
\pmtitle{basis-free definition of determinant}
\pmrecord{9}{39109}
\pmprivacy{1}
\pmauthor{mps}{409}
\pmtype{Definition}
\pmcomment{trigger rebuild}
\pmclassification{msc}{15A15}

\endmetadata

% this is the default PlanetMath preamble.  as your knowledge
% of TeX increases, you will probably want to edit this, but
% it should be fine as is for beginners.

% almost certainly you want these
\usepackage{amssymb}
\usepackage{amsmath}
\usepackage{amsfonts}

% used for TeXing text within eps files
%\usepackage{psfrag}
% need this for including graphics (\includegraphics)
%\usepackage{graphicx}
% for neatly defining theorems and propositions
\usepackage{amsthm}
% making logically defined graphics
%%%\usepackage{xypic}

% there are many more packages, add them here as you need them

% define commands here
\DeclareMathOperator{\sgn}{sgn}
\newtheorem*{proposition*}{Theorem}
\newtheorem*{definition*}{Definition}

\begin{document}
The definition of determinant as a multilinear mapping on rows can be
modified to provide a basis-free definition of determinant.  In order
to make it clear that we are not using bases. we shall speak in terms
of an endomorphism of a vector space over $k$ rather than speaking of
a matrix whose entries belong to $k$.  We start by recalling some
preliminary facts.

Suppose $V$ is a finite-dimensional vector space of dimension $n$ over
a field $k$.  Recall that a multilinear map $f\colon V^n\to k$ is
alternating if $f(x) = 0$ whenever there exist distinct indices
$i,j\in[n]=\{1,\dots,n\}$ such that $x_i = x_j$.  Every alternating
map $f\colon V^n\to k$ is skew-symmetric, that is, for each
permutation $\pi\in\mathfrak{S}_n$, we have that $f(x) =
\sgn(\pi)f(x^{\pi})$, where $x^{\pi}$ denotes
$(x_{\pi(i)})_{i\in[n]}$, the result of $\pi$ permuting the entries of
$x$.

Since the trivial map $0\colon V^n\to k$ is alternating and any linear
combination of alternating maps is alternating, it follows that
alternating maps form a subspace of the space of multilinear maps.  In
the following proposition we show that this subspace is
one-dimensional.  

\begin{proposition*}
Suppose $V$ is a finite-dimensional vector space of dimension $n$ over
a field $k$.  Then the space of alternating maps from $V^n$ to $k$ is
one-dimensional.
\end{proposition*}

\begin{proof}
We use a basis here, but we will throw it away later.  We need the
basis here because each map we will consider has exactly as many
elements as a basis of $V$.  So let $B = \{b_i\colon i\in[n]\}$ be a
basis of $V$.

Suppose $f$ and $g$ are nontrivial alternating maps from $V^n$ to $k$.
We claim that $f$ and $g$ are linearly dependent.  Let $x\in V^n$.  We
may assume that the entries of $x$ are basis vectors, that is, that $X
= \{x_i\colon i\in[n]\}\subset\{b_i\colon i\in[n]\}$.  
If $X\subsetneq B$, then there exist distinct indices $i,j\in[n]$ such
that $x_i = x_j$.  Since $f$ and $g$ are alternating, it follows that
$f(x) = g(x) = 0$, which implies that $f(b)g(x) = g(b)f(x)$.
On the other hand, if $X = B$, then there is a permutation
$\pi\in\mathfrak{S}_n$ such that $x = b^{\pi}$.  Since $f$ and $g$ are
skew-symmetric, it follows that
\[
f(b)g(x) = \sgn(\pi)f(b)g(b) = g(b)f(x).
\]
In either case we find that $f(b) g(x) = g(b) f(x)$.  Since $f(b)$ and
$g(b)$ are fixed scalars, it follows that $f$ and $g$ are linearly
dependent.

So far we have shown only that the dimension of the space of
alternating maps is less than or equal to one.  In order to show that
the space is one-dimensional we simply need to find a nontrivial
alternating form.  To do this, let $\{b^*_i\colon i\in[n]\}$ be the
natural basis of $V^*$, so that $b^*_i(b_j)$ is the Kronecker delta of
$i$ and $j$ for any $i,j\in[n]$.  Define a map $f\colon V^n\to k$ by
\[
f(x) = \sum_{\pi\in\mathfrak{S}_n} \sgn(\pi)\prod_{i\in[n]} b^*_i(x_{\pi(i)}).
\]
One can check that $f$ is multilinear and alternating.  Moreover,
$f(b) = 1$, so it is nontrivial.  Hence the space of alternating maps
is one-dimensional.
\end{proof}

For an alternate view of the above results, we could look instead at
linear maps from the exterior product $\bigwedge^n V$ into $k$.  The
proposition above can be viewed as saying that the dimension of
$\bigwedge^n V$ is $\displaystyle\binom{n}{n} = 1$.

We define the determinant of an endomorphism in terms of the action of
the endomorphism on alternating maps.  Recall that if $M\colon V\to V$
is an endomorphism, its pullback $M^*$ is the unique operator such
that
\[
(M^*f)(x_i)_{i\in [n]} = f(M(x_i))_{i\in [n]}.
\]
Since the space of alternating maps is one-dimensional and
endomorphisms of a one-dimensional space reduce to scalar
multiplication, it follows that $M^*f$ is a scalar multiple of $f$.
We call this scalar the determinant.  It is well-defined because the
scalar depends on $M$ but not on $f$.

\begin{definition*}
Suppose $V$ is a finite-dimensional vector space of dimension $n$ over
a field $k$, and let $M\colon V\to V$ be an endomorphism.  Then the
\emph{determinant} of $M$ is the unique scalar $\det(M)$ such that 
\[
M^* f = \det(M) f
\]
for all alternating maps $f\colon V^n\to k$.
\end{definition*}

% bleargh
\PMlinkescapeword{clear}
\PMlinkescapeword{order}
\PMlinkescapeword{proposition}
\PMlinkescapeword{rows}
\PMlinkescapeword{terms}
%%%%%
%%%%%
\end{document}
