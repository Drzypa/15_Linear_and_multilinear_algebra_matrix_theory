\documentclass[12pt]{article}
\usepackage{pmmeta}
\pmcanonicalname{Blade}
\pmcreated{2013-03-22 15:58:40}
\pmmodified{2013-03-22 15:58:40}
\pmowner{PhysBrain}{974}
\pmmodifier{PhysBrain}{974}
\pmtitle{blade}
\pmrecord{5}{37994}
\pmprivacy{1}
\pmauthor{PhysBrain}{974}
\pmtype{Definition}
\pmcomment{trigger rebuild}
\pmclassification{msc}{15A03}
\pmclassification{msc}{15A75}
\pmclassification{msc}{15A66}
\pmrelated{Basis}
\pmrelated{UnitVector}

\endmetadata

% this is the default PlanetMath preamble.  as your knowledge
% of TeX increases, you will probably want to edit this, but
% it should be fine as is for beginners.

% almost certainly you want these
\usepackage{amssymb}
\usepackage{amsmath}
\usepackage{amsfonts}

% used for TeXing text within eps files
%\usepackage{psfrag}
% need this for including graphics (\includegraphics)
%\usepackage{graphicx}
% for neatly defining theorems and propositions
%\usepackage{amsthm}
% making logically defined graphics
%%%\usepackage{xypic}

% there are many more packages, add them here as you need them

% define commands here

\begin{document}
A blade is a term often used to describe a basis entity in the space defined by a geometric algebra.  Since a geometric algebra is a multi-graded space, the basis entities also have multiple grades.  To distinguish the various graded entities, the blades are often prefixed by their grade.  For example a grade-$k$ basis entity would be called a $k$-blade.

The number of linearly independent $k$-blades in a particular geometric algebra is dependent on the number of dimensions of the manifold on which the algebra is defined.  For an $n$-dimensional manifold, the number of $k$-blades is given by the binomial coefficient.
\[
N_k = \left( \begin{array}{c}
n \\
k
\end{array} \right)
\]
The total number of basis blades of all grades in a geometric algebra defined on an $n$-manifold is then:
\[
N = \sum_{k=0}^n N_k = 2^n
\]

%%%%%
%%%%%
\end{document}
