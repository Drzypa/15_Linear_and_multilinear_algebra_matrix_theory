\documentclass[12pt]{article}
\usepackage{pmmeta}
\pmcanonicalname{OrthogonalVectors}
\pmcreated{2013-03-22 12:07:33}
\pmmodified{2013-03-22 12:07:33}
\pmowner{akrowne}{2}
\pmmodifier{akrowne}{2}
\pmtitle{orthogonal vectors}
\pmrecord{8}{31285}
\pmprivacy{1}
\pmauthor{akrowne}{2}
\pmtype{Definition}
\pmcomment{trigger rebuild}
\pmclassification{msc}{15-00}
\pmrelated{GramSchmidtOrthogonalization}

\usepackage{amssymb}
\usepackage{amsmath}
\usepackage{amsfonts}
\usepackage{graphicx}
%%%\usepackage{xypic}
\begin{document}
Two vectors, $v_1$ and $v_2$, are orthogonal if and only if their inner product $\left<x,y\right>$is 0.  In two dimensions, orthogonal vectors are perpendicular (or in $n$ dimensions in the plane defined by the two vectors.)

A set of vectors is orthogonal when, taken pairwise, any two vectors in the set are orthogonal.
%%%%%
%%%%%
%%%%%
\end{document}
