\documentclass[12pt]{article}
\usepackage{pmmeta}
\pmcanonicalname{SubmatrixNotation}
\pmcreated{2013-03-22 16:13:36}
\pmmodified{2013-03-22 16:13:36}
\pmowner{Mathprof}{13753}
\pmmodifier{Mathprof}{13753}
\pmtitle{submatrix notation}
\pmrecord{5}{38325}
\pmprivacy{1}
\pmauthor{Mathprof}{13753}
\pmtype{Definition}
\pmcomment{trigger rebuild}
\pmclassification{msc}{15-00}

% this is the default PlanetMath preamble.  as your knowledge
% of TeX increases, you will probably want to edit this, but
% it should be fine as is for beginners.

% almost certainly you want these
\usepackage{amssymb}
\usepackage{amsmath}
\usepackage{amsfonts}

% used for TeXing text within eps files
%\usepackage{psfrag}
% need this for including graphics (\includegraphics)
%\usepackage{graphicx}
% for neatly defining theorems and propositions
%\usepackage{amsthm}
% making logically defined graphics
%%%\usepackage{xypic}

% there are many more packages, add them here as you need them

% define commands here

\begin{document}
Let $n$ and $k$ be integers with $1 \leq k \leq n$. Denote by
$Q_{k,n}$ the totality  of all sequences of $k$ integers, where the elements of
the sequence are strictly increasing and choosen from $\{1, \ldots , n\}$.


Let $A=(a_{ij})$ be an $m \times n$ matrix with elements from some set, usually
taken to be a field for ring. Let  $k$ and $r$ be positive integers with
$1 \leq k \leq m$, $1 \leq r \leq n$, $ \alpha \in Q_{k,m}$ and
$ \beta \in Q_{r,n}$. 
We let $ \alpha = (i_1, \ldots , i_k)$ and $ \beta =(j_1, \ldots , j_r)$

The submatrix $A[\alpha, \beta]$ has $(s,t)$ entry equal to
$a_{i_sj_t}$ and has $k$ rows and $r$ columns. 

We denote by  $A(\alpha, \beta)$ the submatrix of $A$ whose rows  and columns
are complementary to $\alpha $ and $\beta$, respectively.
 
We can also define similarly the notations $A[\alpha, \beta)$ and $A(\alpha, \beta]$.

%%%%%
%%%%%
\end{document}
