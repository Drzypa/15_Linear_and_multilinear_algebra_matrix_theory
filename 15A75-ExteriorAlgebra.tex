\documentclass[12pt]{article}
\usepackage{pmmeta}
\pmcanonicalname{ExteriorAlgebra}
\pmcreated{2013-03-22 12:34:14}
\pmmodified{2013-03-22 12:34:14}
\pmowner{rmilson}{146}
\pmmodifier{rmilson}{146}
\pmtitle{exterior algebra}
\pmrecord{35}{32819}
\pmprivacy{1}
\pmauthor{rmilson}{146}
\pmtype{Definition}
\pmcomment{trigger rebuild}
\pmclassification{msc}{15A75}
\pmsynonym{Grassmann algebra}{ExteriorAlgebra}
\pmrelated{AntiSymmetric}
\pmdefines{exterior product}
\pmdefines{wedge product}
\pmdefines{multivector}
\pmdefines{exterior power}

\usepackage{amsmath}
\usepackage{amsfonts}
\usepackage{amssymb}
\usepackage{amsthm}
\usepackage{graphicx}

%%%\usepackage{xypic}
%\usepackage{pstricks}
%\usepackage{pst-node}


\newcommand{\Zset}{\mathbb{Z}}
\newcommand{\Sset}{\mathbb{S}}
\newcommand{\Aset}{\mathbb{A}}
\newcommand{\Vset}{\mathbb{V}}

\newcommand{\Hom}{\operatorname{Hom}}
\newcommand{\sgn}{\operatorname{sgn}}

\newcommand{\lp}{\left(}
\newcommand{\rp}{\right)}
\newcommand{\lb}{\left[}
\newcommand{\rb}{\right]}

\newcommand{\supth}{^{\text{th}}}


\newtheorem{proposition}{Proposition}
\begin{document}
\PMlinkescapeword{simple}
\PMlinkescapeword{closed}
\PMlinkescapeword{section}
\PMlinkescapeword{difference}
\PMlinkescapeword{equivalent}
\PMlinkescapeword{structure}
\PMlinkescapeword{characteristic}

\section{Introductory remarks.}
We begin with some informal remarks to motivate the formal definitions
found in the next section. Throughout, $V$ is a vector space over a
field $K$.  Many of the concepts and constructions discussed below
apply verbatim to modules over commutative rings, but we will stick to
vector spaces to keep things simple.

The exterior product, commonly denoted by the wedge symbol $\wedge$
and also known as the wedge product, is an antisymmetric variant of
the tensor product.  The former, like the latter is an associative,
bilinear operation. Thus, for all $u,v\in V$ and $a,b,c,d\in K$,
we have
\begin{align}
  \label{eq:aubvwedge}
  (au+bv) \wedge (cu + dv ) &= ac\, (u\wedge u) + ad\, (u\wedge v) + bc\,   (v\wedge u)
  + bd\, (v\wedge v)\\
  \label{eq:aubvotimes}
  (au+bv) \otimes (cu + dv ) &= ac\, (u\otimes u) + ad\, (u\otimes v) + bc\,
  (v\otimes u)   + bd\, (v\otimes v).
\end{align}
The essential difference between the two operations is that all
squares formed using the exterior product vanish, by definition.
Thus,
\begin{equation}
  \label{eq:vvzero}
  v\wedge v =0,  
\end{equation}
whereas $v\otimes v\neq 0$. Hence, the expressions in
\eqref{eq:aubvwedge} are equal to $(ad-bc) u\wedge v$, but there is no
way to simplify further the right-hand side of \eqref{eq:aubvotimes}.

A polarization argument shows
that for $u,v\in V$ we have
\begin{align*}
  0 &= (u+v)\wedge (u+v) \\& = u\wedge u + u\wedge v + v\wedge u +
  v\wedge
  v\\
  &= u\wedge v + v\wedge u.
\end{align*}
Therefore, if the \PMlinkname{characteristic}{characteristic} of the underlying field $K$ is not equal to $2$, that is if $1+1\neq 0$, then the key postulate \eqref{eq:vvzero} is
logically equivalent to the antisymmetry condition
\begin{equation}
  \label{eq:antisym}
  u\wedge v = -v\wedge u,\quad u,v\in V.
\end{equation}
However, if the characteristic is 2, that is if $K$ is a field where
$1=-1$, then \eqref{eq:vvzero} does not, necessarily, follow from
\eqref{eq:antisym}.  Therefore, to keep things as general as possible,
we must use \eqref{eq:vvzero} to formulate the essential identity
satisfied by the exterior product.



So far so good, but we have not yet given a meaning to the symbol
$u\wedge v$.  The geometric interpretation of $u\wedge v$ is that of
an oriented area element in the plane spanned by $u$ and $v$.  Without
additional structure, there is no way to assign a area measurement to
a parallelogram in a vector space.  However, parallelograms that lie
in the same plane are commensurate. If we adopt the parallelogram
spanned by $u$ and $v$ as the standard area, we can say that the
oriented area of another parallelogram, say one that is spanned by
$au+bv$ and $cu+dv$, has an area that is $ad-bc$ times the area of the
first parallelogram.  The exterior product allows us to express this
algebraically. To wit,
\[ (au+bv)\wedge(cu+dv) = (ad-bc) (u\wedge v),\quad u,v\in V,\;
a,b,c,d\in K. \] The analogous interpretation for vectors is that of
an oriented length element on a line.  For this reason, the object
$u\wedge v$ is referred to as a bivector.


From a more algebraic point of view, a bivector $u\wedge v$ can be
considered as a formal antisymmetric product of vectors $u$ and $v$,
in much the same way that $u\otimes v$ can be regarded as a formal
non-commutative product of two vectors.  Such descriptions can hardly
serve as rigorous definitions, but an explicit construction is not
really the way to go here.  

Take the case of the tensor product.  Formal sums of formal products
$u\otimes v$, where $u,v\in V$, form a certain vector space, which we
denote as $V\otimes V$.  However, rather than saying that $V\otimes V$
\textbf{is} such and such a thing, it is better to state a certain
universal property that describes $V\otimes V$ up to vector space
isomorphism.  The property in question is that every \textbf{bilinear}
map $f:V\times V\to W$ determines a unique \textbf{linear} map from
$g:V\otimes V\to W$ such that
\[ f(u,v) = g(u\otimes v),\quad u,v \in V.\] Similarly, formal sums of
bivectors constitute a vector space $\Lambda^2(V)$, called the second
exterior power of $V$.  This vector space is defined, up to
isomorphism, by the condition that every antisymmetric, bilinear map
$f:V\times V\to W$ determines a unique linear map $g:\Lambda^2(V) \to
W$ with
\[ f(u,v) = g(u\wedge v).\] Thus, in the same way that the tensor
product replaces bilinear maps with a certain kind of linear map,
the exterior product replaces bilinear, antisymmetric maps with linear
maps from $\Lambda^2 V$.

More generally, $k$-multivectors are $k$-fold products
$v_1\wedge\cdots\wedge v_k$, and the $k\supth$ exterior power,
$\Lambda^k(V)$, is the vector space of formal sums of
$k$-multivectors.  The product of a $k$-multivector and an
$\ell$-multivector is a $(k+\ell)$-multivector.  So, the direct sum
$\bigoplus_k \Lambda^k(V)$ forms an associative algebra, which is
closed with respect to the wedge product.  This algebra, commonly
denoted by $\Lambda(V)$, is called the exterior algebra of $V$.

Again, the analogy with the tensor product is useful.  The tensor
algebra $T(V)$ can be characterized as the associative,
non-commutative algebra freely generated by $V$.  If the
characteristic of $k$ is not 2, then the wedge product satisfies the
supercommutativity relations
\[ \alpha\wedge \beta = (-1)^{k+\ell} \beta\wedge\alpha,\quad
\alpha\in\Lambda^k(V),\; \beta\in\Lambda^\ell(V).\] Thus, $\Lambda(V)$
can be characterized as the supercommutative algebra which is freely
generated by $V$.

\section{Formal definitions.}
\paragraph{Supercommutative algebras.}
For the purposes of this discussion, we define a supercommutative
algebra to  be an associative, unital $K$-algebra $A$ with an
$\Zset_2$-grading, $A=A^+\oplus A^-$, such that for all odd 
$a\in A^-$ we have
\[ a^2 = 0, \]
and such that for all even $b\in A^+$ and all $a\in A$, we have
\[ ab = ba.\] Using a polarization argument we see that the first
condition implies that for all odd $a,b\in A^-$ we have
\[ ab = -ba.\]
If the characteristic of $K$ is different from $2$, then the converse
is true, and we recover the usual definition of supercommutativity,
namely that
\[ ab = \pm ba,\quad a\in A^{\pm},\; b\in A^{\pm},\]
with the minus sign employed if both $a$ and $b$ or odd, and with $+$
employed otherwise.

\paragraph{Exterior algebra.}
Let $E$ be a supercommutative algebra and $\iota:V\to E^-$ a linear
map.  We will say that $(E,\iota)$ is a model for the exterior algebra
of $V$, if every linear map $f:V\to A^-$, where $A$ a supercommutative
algebra, \PMlinkescapetext{lifts} to a unique algebra homomorphism $g:E\to A$, where
``lifts'' means that $f = g\circ\iota$.
Diagrammatically:
 \begin{center}
   \includegraphics[width=4cm]{extalg}
\end{center}
The above condition on $E$ is a universal property; this implies that
all models are isomorphic as algebras.  Thus, when we speak of
$\Lambda(V)$, the exterior algebra of $V$, we are referring to the
isomorphism class of all such models.  It is also common to identify
$V$ with its image $\iota(V)$, and to write $v$ rather than
$\iota(v)$.

\paragraph{Exterior powers.}
For the purposes of the present entry, we define an antisymmetric map
to be a $k$-multilinear map $f:V^{\times k}\to W$ such that $f(\dots,
v,v,\dots)=0$ for all $v\in V$. A polarization argument then implies the
usual antisymmetry condition, namely that for every permutation $\pi$
of $\{ 1,2,\dots, k\}$ we have
\[f(v_{\pi_1},\dots, v_{\pi_k}) = \sgn(\pi) f(v_1,\dots, v_k),\quad
v_1,\dots,v_k \in V.\]
As usual, if the characteristic of $K$ is different from $2$, the two
assertions are equivalent.  However if $1=-1$, then the first
assertion is stronger, and that is why we adopt it as the definition
of antisymmetry.

We now define a model of the $k\supth$ exterior power of $V$ to be a
vector space $E^k$ and an antisymmetric map $\wedge: V^{\times k}\to E^k$ such that every antisymmetric map $f:V^{\times
  k}\to W$ lifts to a unique linear map $g:E^k\to W$, where ``lifts''
means that \[f(v_1,\dots,v_k) = g(v_1\wedge\cdots\wedge v_k),\quad
v_1,\dots,v_k\in V.\]
As above, all models are isomorphic as vector spaces, and we use
$\Lambda^k(V)$ to denote the isomorphism class of all such.


\paragraph{The standard model.}  
A model of the exterior algebra
$\Lambda(V)$, and the exterior powers $\Lambda^k(V)$ can be easily
constructed as the antisymmetrized quotients of the tensor algebra
\[T(V)=\bigoplus_{k=0}^\infty V^{\otimes k},\qquad V^{\otimes k} =
V\otimes\cdots\otimes V \text{ (k times)}.\] To that end, let $S(V)$
denote the two sided ideal of $T(V)$ generated by
elements of the form $v\otimes v,\; v\in V$. Then

\[S(V)=\bigoplus_{k=0}^\infty S^k(V),\qquad  \textrm{where} \quad S^k(V)=S(V)\cap
V^{\otimes k},\]  and let 
\[ E(V)=T(V)/S(V),\qquad E^k(V)=V^{\otimes k}/S^k(V)\] denote the
indicated quotients, with $a:T(V)\to E(V)$ and $a_k: V^{\otimes k} \to
E^k(V)$ denoting the corresponding antisymmetrization surjections.  It
is easy to see that $S^1(V)$ is the trivial vector space, and hence
that $E^1(V) \cong V$.  We leave it as an exercise for the reader to
show that $E^k(V),\; k\geq 2$ is a model of the $k\supth$ exterior
power, while $E(V)$ together with the map $V\to E^1(V)$ is a model of
the full exterior algebra.

\paragraph{The canonical grading.}
An inspection of the above
construction reveals that
\[ E(V) = \bigoplus_{k=0}^\infty E^k(V).\] Indeed, every model of
exterior algebra carries a canonical grading. Let $E$ be a particular
model of the exterior algebra of $V$. For $k=1,2,\dots$, we will call
$\alpha\in E$, a $k$-primitive element if $\alpha =
v_1\wedge\cdots\wedge v_k,$ for some $v_i\in V$.  We now let
$E^k\subset E$ denote the vector space spanned by all $k$-primitive
elements, and let $E^0=K$.
\begin{proposition}
  The subspace $E^k$ is a model for the $k\supth$ exterior power of
  $V$.  Furthermore, 
  \[ E = \bigoplus_{k=0}^\infty E^k.\]
\end{proposition} 



\paragraph{Categorical formulation.}
The above definition of exterior product has a very appealing
categorical formulation.  Let $\Sset$ denote the category of
supercommutative $K$-algebras, let $\Vset$ denote category of vector
spaces over $K$, and let $(\cdot)^-:\Sset\to\Vset$ denote the
forgetful functor $A\mapsto A^-$.  We may now say that the
exterior algebra function $\Lambda:\Vset\to\Sset$ is the left
adjoint of $(\cdot)^-$.  In other words,
\[ \Hom_{\Vset}(V,A^-) \cong \Hom_{\Sset}(\Lambda(V), A),\quad V\in\Vset,\; A\in \Sset,\] with the
isomorphism natural in $V$ and $A$.

It is useful to compare the above definition to the categorical
definition of the tensor algebra.  Let $\Aset$ denote the category of
associative, unital $K$-algebras, and let $F:\Aset\to\Vset$ be the
forgetful functor that gives the underlying vector space structure of
a $K$-algebra.  We can then define the tensor algebra $T(V)$ of a
vector space $V$ by saying that $T:\Vset\to\Aset$ is the left-adjoint
of $F:\Aset\to\Vset$.  Thus, whereas $T(V)$ as the \textbf{associative
  algebra} freely generated by $V$, the exterior algebra $\Lambda(V)$
is the \textbf{supercommutative algebra} freely generated by $V$.  The
antisymmetrization 
quotient map $a_V\colon T(V)\to\Lambda(V)$ is a natural transformation between
these two functors.



\section{Finite dimensional models.}
\paragraph{Basis models.}
If $V$ is an $n$-dimensional vector space, there are some
down-to-earth constructions of $\Lambda(V)$ that go a long way to
illuminate the nature of the exterior product.  Suppose then, that $V$
is $n$-dimensional, and let $e_1,\ldots,e_n$ be a basis of $V$. For
every ascending sequence \[0\leq i_1<i_2<\cdots<i_k\leq n\] let us
introduce the symbol $e_I=e_{i_1\dots i_k}$ to represent the primitive
$k$-multivector $e_{i_1}\wedge\ldots\wedge e_{i_k}$.  If $I$ is the
empty sequence, we let $e_I$ denote the unit element of the field
$K$.
\begin{proposition}
  The $\binom{n}{k}$-dimensional vector space spanned by $e_{i_1\dots
    i_k}$ is a model of $\Lambda^k(V)$.  
\end{proposition}
Note that $\Lambda^1(V)$ is just the $n$-dimensional space spanned by
the basis symbols $e_1,\dots,e_n$.  As such, $\Lambda^1(V)$ is
naturally isomorphic to $V$.  For disjoint sequences $I$ and $J$, let
us define
\[ e_I \wedge e_J = \sgn(IJ) e_{[IJ]},\] where $[IJ]$ denotes the
ascending sequence composed of the union of $I$ and $J$, and where
$\sgn(IJ)=\pm 1$ denotes the parity of the permutation that takes the
sorted list $[IJ]$ to the unsorted concatenation $IJ$. If $I$ and $J$
have one or more elements in common, we define
\[ e_I\wedge e_J = 0.\]
Here are some examples:
\begin{align*}
  &e_{3} \wedge e_{12} = e_{123},\\
  &e_{2}\wedge e_{14} = - e_{124},\\
  &e_{14} \wedge e_{23} = e_{1234},\\
  &e_{24} \wedge e_{13} = -e_{1234},\\
  &e_{24} \wedge e_{14} =0.
\end{align*}
\begin{proposition}
  The $2^n$ dimensional vector spanned by the symbols $e_I$, together
  with the above product and the linear isomorphism from $V$ to
  $\Lambda^1(V)$ is a model of the exterior algebra $\Lambda(V)$.
\end{proposition}

Evidently, any list of numbers between $1$ and $n$ with length greater
than $n$ will contain duplicates.  Thus, an immediate consequence of
this construction is that $\Lambda^k(V)=0$ for $k>n$, and hence that
\[ \Lambda(V) = \bigoplus_{k=0}^n \Lambda^n(V).\]

\paragraph{Alternating forms.}
If $V$ is finite-dimensional, we have the natural isomorphism between
$V$ and the double-dual $V^{**}$.  We can exploit this natural
isomorphism to construct the following model of exterior algebra.  Let
$A^k(V)$ denote the vector space of $k$-multilinear mappings from
$(V^*)\times\cdots\times (V^*)$ ($k$ times) to $K$.  Such an mapping
is known as an alternating $k$-form.  Using the above duality we can
prove that $A^k(V)$ is a model for the $k\supth$ exterior power of
$V$.  

Given alternating forms $u\in A^k(V)$ and $v\in A^\ell(V)$,
let us define $u\wedge v\in A^{k+\ell}(V)$ according to
\[ (u\wedge v)(a_1,\dots,a_{k+\ell})= \sum_{\pi}\operatorname{sgn}(\pi)
u(a_{\pi_1},\dots,a_{\pi_k})
v(a_{\pi_{k+1}},\dots,a_{\pi_{k+\ell}}),\]
where $a_1,\dots,a_{k+\ell}\in V^*,$ and  where the sum is taken over
all permutations $\pi$ of $\{1,2,\dots, k+\ell\}$ such that $\pi_1<
\pi_2 < \cdots < \pi_k$ and $\pi_{k+1} < \cdots < \pi_{k+\ell}$, and
where $\operatorname{sgn} \pi=\pm 1$ according to whether $\pi$ is an
even or odd permutation.  With this definition, we can show that
\[ A(V) = \bigoplus_{k=0}^n A^k(V) \]
together with the above product, and the linear isomorphism $V\to
A^1(V)\cong V^{**}$ is a model for the exterior algebra $\Lambda(V)$.


% Another way to do this construction is to introduce symbols
% $e_I$ for every sequences, sorted or unsorted, $0\leq
% i_1,\dots,i_k\leq n$, and to make the identifications
% \[ e_I = 0 \]
% if $I$ contains duplicate numbers and
% \[ e_I = \pm e_J \]
% if $I$ and $J$ differ by a permutation 
%  of The upshot of all this is that for finite-dimensional vector spaces we
% have another way to construct a model of the exterior algebra.
% Namely, we choose a basis $e_1,\ldots,e_n$ an define formal bivector
% symbols $e_i\wedge e_j$ subject to the anti-symmetric relations
% $$e_i\wedge e_i = 0,\quad\mbox{ and} \quad e_i\wedge e_j = - e_j\wedge
% e_i.$$
% We then define trivector symbols, and more generally
% $k$-vector symbols subject to the obvious $k$-place anti-symmetric
% relations.  The exterior algebra $\Lambda(V)$ is then defined to be
% the vector space of all possible linear combinations of the $k$-vector
% symbols, and the algebra product is defined by linearly extending the
% wedge product to all of $\Lambda(V)$.  Also note that for $k>n$ all
% $k$-vector symbols are identified with zero, and hence that
% $\Lambda^k(V)=\{0\}$ for all $k>n$.

\section{Historical Notes.} The exterior algebra is also known as the Grassmann
algebra after its inventor \PMlinkexternal{Hermann Grassmann}{http://www-groups.dcs.st-and.ac.uk/~history/Mathematicians/Grassmann.html}
who created it \PMlinkescapetext{in order} to give algebraic treatment of linear
geometry.  Grassmann was also one of the first people to talk about
the geometry of an $n$-dimensional space with $n$ an arbitrary natural
number.  The axiomatics of the exterior product are needed to define
differential forms and therefore play an essential role in the theory
of integration on manifolds. Exterior algebra is also an essential
prerequisite to understanding de Rham's theory of differential
cohomology.
%%%%%
%%%%%
\end{document}
