\documentclass[12pt]{article}
\usepackage{pmmeta}
\pmcanonicalname{Vector}
\pmcreated{2013-03-22 12:38:40}
\pmmodified{2013-03-22 12:38:40}
\pmowner{rmilson}{146}
\pmmodifier{rmilson}{146}
\pmtitle{vector}
\pmrecord{13}{32909}
\pmprivacy{1}
\pmauthor{rmilson}{146}
\pmtype{Definition}
\pmcomment{trigger rebuild}
\pmclassification{msc}{15A03}
\pmclassification{msc}{15A90}
\pmsynonym{abstract vector}{Vector}
\pmrelated{VectorSpace}
\pmrelated{Frame}
\pmrelated{TensorArray}
\pmrelated{Vector}

\usepackage{amsmath}
\usepackage{amsfonts}
\usepackage{amssymb}
\newcommand{\reals}{\mathbb{R}}
\newcommand{\natnums}{\mathbb{N}}
\newcommand{\cnums}{\mathbb{C}}
\newcommand{\znums}{\mathbb{Z}}
\newcommand{\lp}{\left(}
\newcommand{\rp}{\right)}
\newcommand{\lb}{\left[}
\newcommand{\rb}{\right]}
\newcommand{\supth}{^{\text{th}}}
\newtheorem{proposition}{Proposition}
\newtheorem{definition}[proposition]{Definition}
\newcommand{\nl}[1]{{\PMlinkescapetext{{#1}}}}
\newcommand{\pln}[2]{{\PMlinkname{{#1}}{#2}}}
\newcommand{\bu}{\mathbf{u}}
\newcommand{\bv}{\mathbf{v}}
\newcommand{\bzero}{\mathbf{0}}
\newcommand{\supt}{^{\scriptscriptstyle\mathrm{T}}}
\newcommand{\kfield}{\mathbb{K}}
\begin{document}
\paragraph{Overview.}
The word \emph{vector} has several distinct, but interrelated
meanings.  The present entry is an overview and discussion of these
concepts, with links at the end to more detailed definitions.

\begin{itemize}
\item A \emph{list vector} (follow the link to the formal definition)
  is a finite list of numbers\footnote{Infinite vectors arise in areas
    such as functional analysis and quantum mechanics, but require a
    much more complicated and sophisticated theory.}.  Most commonly,
  the vector is composed of real numbers, in which case a list vector
  is just an element of $\reals^n$.  Complex numbers are also quite
  common, and then we speak of a complex vector, an element of
  $\cnums^n$.  Lists of ones and zeroes are also utilized, and are
  referred to as binary vectors.  More generally, one can use any
  field $\kfield$, in which case a list vector is just an element of
  $\kfield^n$.
\item A \emph{physical vector} (follow the link to a formal definition
  and in-depth discussion) is a geometric quantity that correspond to
  a linear displacement. Indeed, it is customary to depict a physical
  vector as an arrow.  By choosing a system of coordinates a physical
  vector $\bv$, can be represented by a list vector
  $(v^1,\ldots,v^n)\supt$.  Physically, no single system of
  measurement cannot be preferred to any other, and therefore such a
  representation is not canonical.  A linear change of coordinates
  induces a corresponding linear transformation of the representing
  list vector.

  
  In most physical applications vectors have a magnitude as well as a
  direction, and then we speak of a Euclidean vector.  When lengths
  and angles can be measured, it is most convenient to utilize an
  orthogonal system of coordinates.  In this case, the magnitude of a
  Euclidean vector $\bv$ is given by the usual Euclidean norm of the
  corresponding list vector,
  $$\Vert \bv \Vert = \sqrt{{\textstyle \sum_i} (v^i)^2}\;.$$
  This
  definition is independent of the choice of orthogonal coordinates.
  
\item An \emph{abstract vector} is an element of a vector space.  An
  abstract Euclidean vector is an element of an inner product space.
  The connection
  between list vectors and the more general abstract vectors is fully
  described in the entry on \PMlinkname{frames}{Frame}.  
  
  Essentially, given a finite dimensional abstract vector space, a
  choice of a coordinate frame (which is really the same thing as a
  basis) sets up a linear bijection between the abstract vectors and
  list vectors, and makes it possible to represent the one in terms of
  the other.  The representation is not canonical, but depends on the
  choice of frame.  A change of frame changes the representing list
  vectors by a matrix multiplication.
  
  We also note that the axioms of a vector space make no mention of
  lengths and angles.  The vector space formalism can be enriched to
  include these notions.  The result is the axiom system for inner
  products.
  
  Why do we bother with the ``bare-bones'' formalism of length-less
  vectors? The reason is that some applications involve velocity-like
  quantities, but lack a meaningful notion of speed.  As an example,
  consider a multi-particle system.  The state of the system is
  represented as a point in some manifold, and the evolution of the
  system is represented by velocity vectors that live in that
  manifold's tangent space.  We can superimpose and scale these
  velocities, but it is meaningless to speak of a speed of the
  evolution.

\end{itemize}

\paragraph{Discussion.}
What is a vector?  This simple question is surprisingly difficult to
answer.  Vectors are an essential scientific concept, indispensable
for both the physicist and the mathematicians.  It is strange then,
that despite the obvious importance, there is no clear, universally
accepted definition of this term.

The difficulty is one of semantics. The term {\em vector} is
ambiguous, but its various meanings are interrelated.  The different
usages of {\em vector} call for different formal definitions, which
are similarly interrelated.  List vectors are the most elementary and
familiar kind of vectors.  They are easy to define, and are
mathematically precise.  However, saying that a vector is just a list
of numbers leads to conceptual difficulties.

A physicist needs to be able to say that velocities, forces, fluxes
are vectors.  A geometer, and for that matter a pilot, will think of a
vector as a kind of spatial displacement.  Everyone would agree that a
choice of a vector involves multiple degrees of freedom, and that
vectors can linearly superimposed.  This description of ``vector''
evokes useful and intuitive understanding, but is difficult to
formalize.  

The synthesis of these conflicting viewpoints is the modern
mathematical notion of a vector space.  The key innovation of modern,
formal mathematics is the pursuit of generality by means of
abstraction.  To that end, we do not give an answer to ``What is a
vector?'', but rather give a list of properties enjoyed by all objects
that one may reasonably term a ``vector''.  These properties are just
the axioms of an abstract vector space, or as Forrest Gump\cite{gump}
might have put it, ``A vector is as a vector does.''  

The axiomatic approach afforded by vector space theory gives us
maximum flexibility.  We can carry out an analysis of various physical
vector spaces by employing propositions based on vector space axioms,
or we can choose a basis and perform the same analysis using list
vectors.  This flexibility is obtained by means of abstraction.  We
are not obliged to say what a vector {\em is}; all we have to do is
say that these abstract vectors enjoy certain properties, and make the
appropriate deductions.  This is similar to the idea of an abstract
class in object-oriented programming.


Surprisingly, the idea that a vector is an element of an abstract
vector space has not made great inroads in the physical sciences and
engineering.  The stumbling block seems to be a poor understanding of
formal, deductive mathematics and the unstated, but implicit attitude
that
\begin{quote}
  \em formal manipulation of a physical quantity requires that it be
  represented by one or more numbers.
\end{quote}
Great historical irony is at work here.  The classical, Greek approach
to geometry was purely synthetic, based on idealized notions like
point and line, and on various axioms.  Analytic geometry, a la
\PMlinkexternal{Descartes}{http://www-groups.dcs.st-and.ac.uk/~history/Mathematicians/Descartes.html},
arose much later, but became the dominant mode of thought in
scientific applications and largely overshadowed the synthetic method.
The pendulum began to swing back at the end of the nineteenth century
as mathematics became more formal and important new axiomatic systems,
such as vector spaces, fields, and topology, were developed.  The cost
of increased abstraction in modern mathematics was more than justified
by the improvement in clarity and organization of mathematical
knowledge.

Alas, to a large extent physical science and engineering continue to
dwell in the $19\supth$ century.  The axioms and the formal theory of
vector spaces allow one to manipulate formal geometric entities, such
as physical vectors, without first turning everything into numbers.
The increased level of abstraction, however, poses a formidable
obstacle toward the acceptance of this approach.  Indeed, mainstream
physicists and engineers do not seem in any great hurry to accept the
definition of {\em vector} as something that dwells in a vector space.
Until this attitude changes, {\em vector} will retain the ambiguous
meaning of being both a list numbers, and a physical quantity that
transforms with respect to matrix multiplication.
\begin{thebibliography}{99}
\bibitem{gump} R. Zemeckis, ``Forrest Gump'', Paramount Pictures.
\end{thebibliography}
%%%%%
%%%%%
\end{document}
