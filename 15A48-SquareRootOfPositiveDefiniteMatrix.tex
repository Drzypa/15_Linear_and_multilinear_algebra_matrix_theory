\documentclass[12pt]{article}
\usepackage{pmmeta}
\pmcanonicalname{SquareRootOfPositiveDefiniteMatrix}
\pmcreated{2013-03-22 15:16:42}
\pmmodified{2013-03-22 15:16:42}
\pmowner{rspuzio}{6075}
\pmmodifier{rspuzio}{6075}
\pmtitle{square root of positive definite matrix}
\pmrecord{12}{37067}
\pmprivacy{1}
\pmauthor{rspuzio}{6075}
\pmtype{Definition}
\pmcomment{trigger rebuild}
\pmclassification{msc}{15A48}
\pmrelated{CholeskyDecomposition}

% this is the default PlanetMath preamble.  as your knowledge
% of TeX increases, you will probably want to edit this, but
% it should be fine as is for beginners.

% almost certainly you want these
\usepackage{amssymb}
\usepackage{amsmath}
\usepackage{amsfonts}
\usepackage{amsthm}

\usepackage{mathrsfs}

% used for TeXing text within eps files
%\usepackage{psfrag}
% need this for including graphics (\includegraphics)
%\usepackage{graphicx}
% for neatly defining theorems and propositions
%
% making logically defined graphics
%%%\usepackage{xypic}

% there are many more packages, add them here as you need them

% define commands here

\newcommand{\sR}[0]{\mathbb{R}}
\newcommand{\sC}[0]{\mathbb{C}}
\newcommand{\sN}[0]{\mathbb{N}}
\newcommand{\sZ}[0]{\mathbb{Z}}

 \usepackage{bbm}
 \newcommand{\Z}{\mathbbmss{Z}}
 \newcommand{\C}{\mathbbmss{C}}
 \newcommand{\F}{\mathbbmss{F}}
 \newcommand{\R}{\mathbbmss{R}}
 \newcommand{\Q}{\mathbbmss{Q}}



\newcommand*{\norm}[1]{\lVert #1 \rVert}
\newcommand*{\abs}[1]{| #1 |}



\newtheorem{thm}{Theorem}
\newtheorem{defn}{Definition}
\newtheorem{prop}{Proposition}
\newtheorem{lemma}{Lemma}
\newtheorem{cor}{Corollary}
\begin{document}
Suppose $M$ is a positive definite Hermitian matrix. Then $M$ has a diagonalization 
$$
   M= P^* \operatorname{diag}(\lambda_1, \ldots, \lambda_n) P
$$
where $P$ is a unitary matrix and 
   $\lambda_1, \ldots, \lambda_n$ are the eigenvalues of $M$, which are all positive. 

We can now define the \emph{squar{e} roo{t}} of $M$ as the matrix
$$
   M^{1/2}= P^* \operatorname{diag}(\sqrt{\lambda_1}, \ldots, \sqrt{\lambda_n}) P.
$$
The following properties are clear
\begin{enumerate}
\item $M^{1/2} M^{1/2}=M$,
\item $M^{1/2}$ is Hermitian and positive definite.
\item $M^{1/2}$ and $M$ commute
\item $(M^{1/2})^T=(M^T)^{1/2}$.
\item $(M^{1/2})^{-1}=(M^{-1})^{1/2}$, so one can write $M^{-1/2}$
\item If the eigenvalues of $M$ are $(\lambda_1, \ldots, \lambda_n)$, then
    the eigenvalues of $M^{1/2}$ are 
    $(\sqrt{\lambda_1}, \ldots, \sqrt{\lambda_n})$.
\end{enumerate}
%%%%%
%%%%%
\end{document}
