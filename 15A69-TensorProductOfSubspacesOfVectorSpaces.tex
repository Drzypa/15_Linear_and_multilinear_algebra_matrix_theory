\documentclass[12pt]{article}
\usepackage{pmmeta}
\pmcanonicalname{TensorProductOfSubspacesOfVectorSpaces}
\pmcreated{2013-03-22 18:49:16}
\pmmodified{2013-03-22 18:49:16}
\pmowner{joking}{16130}
\pmmodifier{joking}{16130}
\pmtitle{tensor product of subspaces of vector spaces}
\pmrecord{4}{41623}
\pmprivacy{1}
\pmauthor{joking}{16130}
\pmtype{Theorem}
\pmcomment{trigger rebuild}
\pmclassification{msc}{15A69}

% this is the default PlanetMath preamble.  as your knowledge
% of TeX increases, you will probably want to edit this, but
% it should be fine as is for beginners.

% almost certainly you want these
\usepackage{amssymb}
\usepackage{amsmath}
\usepackage{amsfonts}

% used for TeXing text within eps files
%\usepackage{psfrag}
% need this for including graphics (\includegraphics)
%\usepackage{graphicx}
% for neatly defining theorems and propositions
%\usepackage{amsthm}
% making logically defined graphics
%%%\usepackage{xypic}

% there are many more packages, add them here as you need them

% define commands here

\begin{document}
\textbf{Proposition.} Let $V,W$ be vector spaces over a field $k$. Moreover let $A\subseteq V$, $B\subseteq W$ be subspaces. Then $V\otimes B\cap A\otimes W=A\otimes B$.

\textit{Proof.} The inclusion ,,$\supseteq$'' is obvious. We will show the inclusion ,,$\subseteq$''.

Let $\{e_i\}_{i\in I}$ and $\{e'_j\}_{j\in P}$ be bases of $A$ and $B$ respectively. Moreover let $\{e_i\}_{i\in I'}$ be a completion of given basis of $A$ to the basis of $V$, i.e. $\{e_i\}_{i\in I\cup I'}$ is a basis of $V$. Analogously let $\{e'_j\}_{j\in P'}$ be a completion of a basis of $B$ to the basis of $W$. Then each element $q\in V\otimes W$ can be uniquely written in a form 
$$q=\sum_{i\in I,j\in P}\alpha_{i,j}\, e_i\otimes e'_j\ + \ \sum_{i\in I',j\in P}\beta_{i,j}\, e_i\otimes e'_j\ +$$
$$+\ \sum_{i\in I,j\in P'}\delta_{i,j}\, e_i\otimes e'_j\ +\ \sum_{i\in I',j\in P'}\gamma_{i,j}\, e_i\otimes e'_j.$$
Assume that $q\in V\otimes B\cap A\otimes W.$ Let $i\in I'$ and $j\in P'$. Consider the following linear map: $f_i:V\to k$ such that $f_i(e_t)=1$ if $i=t$ and $f_i(e_t)=0$ if $i\neq t$. Analogously we define $g_j:W\to k$. Then we combine these two mappings into one, i.e. $$f_i\otimes g_j:V\otimes W\to k;$$ $$(f_i\otimes g_j)(v\otimes w)=f_i(v)g_j(w).$$ Furthermore we have $$(f_i\otimes g_j)(q)=\gamma_{i,j}.$$ Note that since $q\in V\otimes B$, then $(f_i\otimes g_j)(q)=0$ and thus $$\gamma_{i,j}=0.$$ Similarly we obtain that all $\beta_{i,j}$ and $\delta_{i,j}$ are equal to $0$. Thus $$q=\sum_{i\in I,j\in P}\alpha_{i,j}\, e_i\otimes e'_j\in A\otimes B,$$ which completes the proof. $\square$
%%%%%
%%%%%
\end{document}
