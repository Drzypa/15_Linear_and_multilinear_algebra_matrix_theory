\documentclass[12pt]{article}
\usepackage{pmmeta}
\pmcanonicalname{EquitableMatricesOfOrder2}
\pmcreated{2013-03-22 14:58:30}
\pmmodified{2013-03-22 14:58:30}
\pmowner{matte}{1858}
\pmmodifier{matte}{1858}
\pmtitle{equitable matrices of order $2$}
\pmrecord{4}{36675}
\pmprivacy{1}
\pmauthor{matte}{1858}
\pmtype{Example}
\pmcomment{trigger rebuild}
\pmclassification{msc}{15-00}

% this is the default PlanetMath preamble.  as your knowledge
% of TeX increases, you will probably want to edit this, but
% it should be fine as is for beginners.

% almost certainly you want these
\usepackage{amssymb}
\usepackage{amsmath}
\usepackage{amsfonts}
\usepackage{amsthm}

\usepackage{mathrsfs}

% used for TeXing text within eps files
%\usepackage{psfrag}
% need this for including graphics (\includegraphics)
%\usepackage{graphicx}
% for neatly defining theorems and propositions
%
% making logically defined graphics
%%%\usepackage{xypic}

% there are many more packages, add them here as you need them

% define commands here

\newcommand{\sR}[0]{\mathbb{R}}
\newcommand{\sC}[0]{\mathbb{C}}
\newcommand{\sN}[0]{\mathbb{N}}
\newcommand{\sZ}[0]{\mathbb{Z}}

 \usepackage{bbm}
 \newcommand{\Z}{\mathbbmss{Z}}
 \newcommand{\C}{\mathbbmss{C}}
 \newcommand{\R}{\mathbbmss{R}}
 \newcommand{\Q}{\mathbbmss{Q}}



\newcommand*{\norm}[1]{\lVert #1 \rVert}
\newcommand*{\abs}[1]{| #1 |}



\newtheorem{thm}{Theorem}
\newtheorem{defn}{Definition}
\newtheorem{prop}{Proposition}
\newtheorem{lemma}{Lemma}
\newtheorem{cor}{Corollary}
\begin{document}
The most general $2\times 2$ equitable matrix is of the form 
$$
  M = \begin{pmatrix} 1 & \lambda \\
                      1/\lambda & 1
      \end{pmatrix}
$$
for some $\lambda>0$. 

Let us consider the matrix 
$$
  M = \begin{pmatrix} m_{11} & m_{12} \\
                      m_{21} & m_{22}
      \end{pmatrix}.
$$
A necessary and sufficient condition for $M$ to be an equitable matrix
is that $m_{11}, m_{12}, m_{21}, m_{22}>0$ and 
\begin{eqnarray*}
   m_{11} = m_{11} m_{11}, \\
   m_{11} = m_{12} m_{21}, \\
   m_{12} = m_{11} m_{12}, \\
   m_{12} = m_{12} m_{22}, \\
   m_{21} = m_{21} m_{11}, \\
   m_{21} = m_{22} m_{21}, \\
   m_{22} = m_{21} m_{12}, \\
   m_{22} = m_{22} m_{22}.
\end{eqnarray*}
It follows that $m_{11}=m_{22}=1$, and $m_{12} = 1/m_{21}$.
%%%%%
%%%%%
\end{document}
