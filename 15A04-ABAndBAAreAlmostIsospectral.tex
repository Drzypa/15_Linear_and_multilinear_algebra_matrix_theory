\documentclass[12pt]{article}
\usepackage{pmmeta}
\pmcanonicalname{ABAndBAAreAlmostIsospectral}
\pmcreated{2013-03-22 14:44:51}
\pmmodified{2013-03-22 14:44:51}
\pmowner{asteroid}{17536}
\pmmodifier{asteroid}{17536}
\pmtitle{$AB$ and $BA$ are almost isospectral}
\pmrecord{14}{36386}
\pmprivacy{1}
\pmauthor{asteroid}{17536}
\pmtype{Corollary}
\pmcomment{trigger rebuild}
\pmclassification{msc}{15A04}
\pmclassification{msc}{47A10}
\pmclassification{msc}{16B99}

% this is the default PlanetMath preamble.  as your knowledge
% of TeX increases, you will probably want to edit this, but
% it should be fine as is for beginners.

% almost certainly you want these
\usepackage{amssymb}
\usepackage{amsmath}
\usepackage{amsfonts}

% used for TeXing text within eps files
%\usepackage{psfrag}
% need this for including graphics (\includegraphics)
%\usepackage{graphicx}
% for neatly defining theorems and propositions
\usepackage{amsthm}
% making logically defined graphics
%%%\usepackage{xypic}

% there are many more packages, add them here as you need them

% define commands here
\begin{document}
\PMlinkescapeword{parent}
\PMlinkescapeword{similar}
\PMlinkescapeword{argument}

\subsection{General case}

Let $A$ and $B$ be endomorphisms of a vector space $V$. Let $\sigma(AB)$ and $\sigma(BA)$ denote, respectively, the \PMlinkname{spectra}{spectrum} of $AB$ and $BA$.

The next result shows that $AB$ and $BA$ are ``almost'' isospectral, in the sense that their spectra is the same except possibly the value $0$.

{\bf Theorem -} Let $A$ and $B$ be as above. We have
\begin{enumerate}
\item $\sigma(AB) \cup \{0\} = \sigma(BA) \cup \{0\}$, and moreover
\item $AB$ and $BA$ have the same eigenvalues, except possibly the zero eigenvalue.
\end{enumerate}

{\bf Proof :} Let  $\lambda \ne 0$.
\begin{enumerate}
\item  If $\lambda \in \sigma(AB)$ then $\lambda^{-1} AB - I$ is not invertible.  By the result in the parent entry, this implies that $\lambda^{-1} BA - I$ is not invertible either, hence $\lambda \in \sigma(BA)$.

A similar argument proves that every non-zero element of $\sigma(BA)$ also belongs to $\sigma(AB)$. Hence $\sigma(AB) \cup \{0\} = \sigma(BA) \cup \{0\}$.

\item If $\lambda$ is an eigenvalue of $AB$, then $I-\lambda^{-1}AB$ is not injective. By the result in the parent entry, this implies that $I-\lambda^{-1}BA$ is also not injective, hence $\lambda$ is an eigenvalue of $BA$.

A similar argument proves that non-zero eigenvalues of $BA$ are also eigenvalues of $AB$. $\square$
\end{enumerate}

{\bf Remark :} Note that for infinite dimensional vector spaces the spectrum of a linear mapping does not consist solely  of its eigenvalues. Hence, 1 and 2 above are two different statements.

\subsection{Finite dimensional case}

When the vector space $V$ is finite dimensional we can strengthen the above result.

{\bf Theroem -} $AB$ and $BA$ are isospectral, i.e. they have the same spectrum. Since $V$ is finite dimensional, this means that $AB$ and $BA$ have the same eigenvalues.

{\bf Proof :} By the above result we only need to prove that: $AB$ is invertible if and only if $BA$ is invertible.

Suppose $AB$ is not invertible. Hence, $A$ is not invertible or $B$ is not invertible.

For finite dimensional vector spaces invertibility, injectivity and surjectivity are the same thing. Thus, the above statement can be rewritten as: $A$ is not injective or $B$ is not surjective.

Either way $BA$ is not invertible.

A similar argument shows that if $BA$ is not invertible, then $AB$ is also not invertible, which concludes the proof. $\square$

\subsection{Comments}

The first theorem can be proven in a more general context : If $A$ and $B$ are elements of an arbitrary unital algebra, then \begin{center}
$\sigma(AB) \cup \{0\} = \sigma(BA) \cup \{0\}$.
\end{center}

This humble result plays an important role in the spectral theory of operator algebras.
%%%%%
%%%%%
\end{document}
