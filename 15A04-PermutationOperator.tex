\documentclass[12pt]{article}
\usepackage{pmmeta}
\pmcanonicalname{PermutationOperator}
\pmcreated{2013-03-22 16:15:38}
\pmmodified{2013-03-22 16:15:38}
\pmowner{Mathprof}{13753}
\pmmodifier{Mathprof}{13753}
\pmtitle{permutation operator}
\pmrecord{7}{38368}
\pmprivacy{1}
\pmauthor{Mathprof}{13753}
\pmtype{Definition}
\pmcomment{trigger rebuild}
\pmclassification{msc}{15A04}

% this is the default PlanetMath preamble.  as your knowledge
% of TeX increases, you will probably want to edit this, but
% it should be fine as is for beginners.

% almost certainly you want these
\usepackage{amssymb}
\usepackage{amsmath}
\usepackage{amsfonts}

% used for TeXing text within eps files
%\usepackage{psfrag}
% need this for including graphics (\includegraphics)
%\usepackage{graphicx}
% for neatly defining theorems and propositions
%\usepackage{amsthm}
% making logically defined graphics
%%%\usepackage{xypic}

% there are many more packages, add them here as you need them

% define commands here

\begin{document}
Let $V$ be a vector space over a field. Let $\sigma \in S_n$, the symmetric group on $\{1, \ldots, n\}$ and define
a multilinear map $\phi: V \times \cdots \times V \to V^{\otimes n} =\overbrace{V\otimes \cdots \otimes
V}^{n\text{ times}}$ by
$$
\phi ( v_1, \ldots , v_n ) = v_{\sigma^{-1}(1)} \otimes \cdots \otimes v_{\sigma^{-1}(n)}.
$$
Then by the \PMlinkname{universal factorization property}{TensorProduct} for a \PMlinkname{tensor product}{TensorProduct} there is a
unique linear map $P(\sigma) : V^{\otimes n} \to V^{\otimes n}$ such that
$P(\sigma)\otimes = \phi$. Then of course, 
$$
P(\sigma)v_1 \otimes \cdots \otimes v_n = v_{\sigma^{-1}(1)} \otimes \cdots \otimes v_{\sigma^{-1}(n)}.
$$

$P(\sigma)$ is called the \emph{permutation operator} associated with $\sigma$.

\section{Properties}
\begin{enumerate}
\item
$P(\sigma\tau) = P(\sigma)P(\tau)$
\item
$P(e) = I$ , where $I$ is the identity mapping on $V^{\otimes n}$
\item
$P(\sigma)$ is nonsingular and $P(\sigma)^{-1} = P(\sigma^{-1})$
\end{enumerate}

%%%%%
%%%%%
\end{document}
