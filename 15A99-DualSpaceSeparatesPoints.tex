\documentclass[12pt]{article}
\usepackage{pmmeta}
\pmcanonicalname{DualSpaceSeparatesPoints}
\pmcreated{2013-03-22 17:30:55}
\pmmodified{2013-03-22 17:30:55}
\pmowner{asteroid}{17536}
\pmmodifier{asteroid}{17536}
\pmtitle{dual space separates points}
\pmrecord{6}{39906}
\pmprivacy{1}
\pmauthor{asteroid}{17536}
\pmtype{Corollary}
\pmcomment{trigger rebuild}
\pmclassification{msc}{15A99}

% this is the default PlanetMath preamble.  as your knowledge
% of TeX increases, you will probably want to edit this, but
% it should be fine as is for beginners.

% almost certainly you want these
\usepackage{amssymb}
\usepackage{amsmath}
\usepackage{amsfonts}

% used for TeXing text within eps files
%\usepackage{psfrag}
% need this for including graphics (\includegraphics)
%\usepackage{graphicx}
% for neatly defining theorems and propositions
%\usepackage{amsthm}
% making logically defined graphics
%%%\usepackage{xypic}

% there are many more packages, add them here as you need them

% define commands here

\begin{document}
The following result is a corollary of the Hahn-Banach theorem.

{\bf Theorem -} Let $X$ be a normed vector space. Given a linearly independent set $\{x_1,\dots ,x_n\} \subset X$ there exist continuous linear functionals $f_1, \dots , f_n \in X'$ such that
\begin{displaymath}
f_j(x_k)=\delta_{jk}\;\;\;\; , 1\leq j, k \leq n
\end{displaymath}

If $x \in span\{x_1, \dots , x_n\}$, then $\displaystyle x= \sum_{j=1}^n f_j(x)x_j$.

The above theorem shows that if $f(x)=f(y)$ for every continuous linear functional $f$ then $x=y$, therefore the dual space $X'$ separates the points of $X$.
%%%%%
%%%%%
\end{document}
