\documentclass[12pt]{article}
\usepackage{pmmeta}
\pmcanonicalname{RuleOfSarrus}
\pmcreated{2013-03-22 17:32:49}
\pmmodified{2013-03-22 17:32:49}
\pmowner{pahio}{2872}
\pmmodifier{pahio}{2872}
\pmtitle{rule of Sarrus}
\pmrecord{9}{39950}
\pmprivacy{1}
\pmauthor{pahio}{2872}
\pmtype{Result}
\pmcomment{trigger rebuild}
\pmclassification{msc}{15A15}
\pmsynonym{Sarrus rule}{RuleOfSarrus}
\pmsynonym{Sarrus' rule}{RuleOfSarrus}
%\pmkeywords{mnemonic}
\pmrelated{LaplaceExpansion}

% this is the default PlanetMath preamble.  as your knowledge
% of TeX increases, you will probably want to edit this, but
% it should be fine as is for beginners.

% almost certainly you want these
\usepackage{amssymb}
\usepackage{amsmath}
\usepackage{amsfonts}

% used for TeXing text within eps files
%\usepackage{psfrag}
% need this for including graphics (\includegraphics)
%\usepackage{graphicx}
% for neatly defining theorems and propositions
 \usepackage{amsthm}
% making logically defined graphics
%%%\usepackage{xypic}

% there are many more packages, add them here as you need them

% define commands here

\theoremstyle{definition}
\newtheorem*{thmplain}{Theorem}

\begin{document}
For calculating the value of a determinant 
$$D = 
\left|\begin{matrix}
a_{11} & a_{12} & a_{13}\\
a_{21} & a_{22} & a_{23}\\
a_{31} & a_{32} & a_{33}
\end{matrix}\right|
$$
with three rows, it is comfortable to use the {\em rule of Sarrus} (invented 1833 by the French mathematician P. F. Sarrus). 

The rule comprises that first one writes the two first columns of the determinant on the \PMlinkescapetext{right side} of the determinant (seeing thus a $3\!\times\!5$ matrix!): 
$$
\left|\begin{matrix}
a_{11} & a_{12} & a_{13}\\
a_{21} & a_{22} & a_{23}\\
a_{31} & a_{32} & a_{33}
\end{matrix}\right|
\begin{matrix}
\,a_{11} & a_{12}\\
\,a_{21} & a_{22}\\
\,a_{31} & a_{32}
\end{matrix}
$$
Here one sums the products on all lines parallel to the main diagonal of $D$ and subtracts the products on the lines parallel to the second diagonal of $D$.  Accordingly, one obtains the expression
$$a_{11}a_{22}a_{33}+a_{12}a_{23}a_{31}+a_{13}a_{21}a_{32}
 -a_{13}a_{22}a_{31}-a_{11}a_{23}a_{32}-a_{12}a_{21}a_{33},$$
which gives the value of the determinant $D$.

There is no corresponding rule for determinants with more or less rows.   


%%%%%
%%%%%
\end{document}
