\documentclass[12pt]{article}
\usepackage{pmmeta}
\pmcanonicalname{SkewHadamardMatrix}
\pmcreated{2013-03-22 16:13:02}
\pmmodified{2013-03-22 16:13:02}
\pmowner{Mathprof}{13753}
\pmmodifier{Mathprof}{13753}
\pmtitle{skew Hadamard matrix}
\pmrecord{13}{38314}
\pmprivacy{1}
\pmauthor{Mathprof}{13753}
\pmtype{Definition}
\pmcomment{trigger rebuild}
\pmclassification{msc}{15-00}

\endmetadata

% this is the default PlanetMath preamble.  as your knowledge
% of TeX increases, you will probably want to edit this, but
% it should be fine as is for beginners.

% almost certainly you want these
\usepackage{amssymb}
\usepackage{amsmath}
\usepackage{amsfonts}

% used for TeXing text within eps files
%\usepackage{psfrag}
% need this for including graphics (\includegraphics)
%\usepackage{graphicx}
% for neatly defining theorems and propositions
%\usepackage{amsthm}
% making logically defined graphics
%%%\usepackage{xypic}

% there are many more packages, add them here as you need them

% define commands here

\begin{document}
\PMlinkescapeword{rows}
\PMlinkescapeword{row}
\PMlinkescapeword{collection}
\PMlinkescapeword{columns}
\PMlinkescapeword{column}
\PMlinkescapeword{contain}
\PMlinkescapeword{property}

\PMlinkescapeword{order}
\PMlinkescapeword{states}
\PMlinkescapeword{satisfies}
\PMlinkescapeword{integer}
A Hadamard matrix $H$ is \emph{skew Hadamard} if $H+H^T=2I$.

A collection of skew Hadamard matrices, including at least one example of every order $n \le 100$
and also including every equivalence class of order $\le 28$, is available
\PMlinkexternal{at this web page}{http://www.rangevoting.org/SkewHad.html}.
It has been conjectured that one exists for every positive order divisible by 4.

Reid and Brown in 1972 showed that there exists a 
``doubly regular tournament of order n''
if and only if there exists a skew Hadamard matrix of order n+1.

\begin{thebibliography}{9}
\bibitem{GeorgiouKS}
S. Georgiou, C. Koukouvinos, J. Seberry, \emph{Hadamard matrices, orthogonal designs and construction algorithms}, pp. 133-205 in DESIGNS 2002: Further computational and constructive design theory, Kluwer 2003.

\bibitem{ReidB}
K.B. Reid, E. Brown, \emph{Doubly regular tournaments are equivalent to skew Hadamard matrices}, J. Combinatorial Theory A 12 (1972) 332-338.

\bibitem{SeberryY}
J. Seberry, M.Yamada, \emph{Hadamard matrices, sequences, and block designs}, pp. 431-560 in Contemporary Design Theory, a collection of surveys (J.H.Dinitz \& D.R.Stinson eds.), Wiley 1992.

\end{thebibliography}

\end{document}

%%%%%
%%%%%
\end{document}
