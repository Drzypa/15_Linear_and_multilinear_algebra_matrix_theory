\documentclass[12pt]{article}
\usepackage{pmmeta}
\pmcanonicalname{FiniteDimensionalProperSubspacesOfANormedSpaceAreNowhereDense}
\pmcreated{2013-03-22 14:58:59}
\pmmodified{2013-03-22 14:58:59}
\pmowner{asteroid}{17536}
\pmmodifier{asteroid}{17536}
\pmtitle{finite dimensional proper subspaces of a normed space are nowhere dense}
\pmrecord{9}{36687}
\pmprivacy{1}
\pmauthor{asteroid}{17536}
\pmtype{Result}
\pmcomment{trigger rebuild}
\pmclassification{msc}{15A03}
\pmclassification{msc}{46B99}
\pmclassification{msc}{54E52}

\endmetadata

\usepackage{amssymb}
\usepackage{amsmath}
\usepackage{amsfonts}

\def\emptyset{\varnothing}
\begin{document}
{\bf \PMlinkescapetext{Proposition} -} Let $V$ be a normed space. If $S \subseteq V$ is a finite dimensional proper subspace, then $S$ is nowhere dense.

\textbf{Proof :}

It is known that for any topological vector space (in particular, normed spaces) \PMlinkname{every proper subspace has empty interior}{ProperSubspacesOfATopologicalVectorSpaceHaveEmptyInterior}.

From the \PMlinkescapetext{parent} \PMlinkname{entry}{EveryFiniteDimensionalSubspaceOfANormedSpaceIsClosed} we also know that finite dimensional subspaces of $V$ are closed.

Then, $\operatorname{int}(\overline{S}) = \operatorname{int}(S) = \emptyset$, which shows that $S$ is nowhere dense. $\square$
%%%%%
%%%%%
\end{document}
