\documentclass[12pt]{article}
\usepackage{pmmeta}
\pmcanonicalname{DimensionTheoremForSymplecticComplementproof}
\pmcreated{2013-03-22 13:32:52}
\pmmodified{2013-03-22 13:32:52}
\pmowner{matte}{1858}
\pmmodifier{matte}{1858}
\pmtitle{dimension theorem for symplectic complement (proof)}
\pmrecord{5}{34149}
\pmprivacy{1}
\pmauthor{matte}{1858}
\pmtype{Proof}
\pmcomment{trigger rebuild}
\pmclassification{msc}{15A04}

\endmetadata

% this is the default PlanetMath preamble.  as your knowledge
% of TeX increases, you will probably want to edit this, but
% it should be fine as is for beginners.

% almost certainly you want these
\usepackage{amssymb}
\usepackage{amsmath}
\usepackage{amsfonts}

% used for TeXing text within eps files
%\usepackage{psfrag}
% need this for including graphics (\includegraphics)
%\usepackage{graphicx}
% for neatly defining theorems and propositions
%\usepackage{amsthm}
% making logically defined graphics
%%%\usepackage{xypic}

% there are many more packages, add them here as you need them

% define commands here
\begin{document}
\newcommand{\image}{\mathop{\mathrm{img}}}
We denote by $V^\star$ the dual space of $V$, i.e.,
linear mappings from $V$ to $\mathbb{R}$. Moreover, we assume known that
$\dim V = \dim V^\ast$ for any vector space $V$.

We begin by showing that the
mapping $S: V \to V^*$, $a \mapsto \omega(a,\cdot)$
is an linear isomorphism. First, linearity is clear, and since
$\omega$ is non-degenerate, $\ker S=\{0\}$, so $S$ is injective.
To show that $S$ is surjective, we apply the 
\PMlinkid{rank-nullity theorem}{2238} to
$S$, which yields $\dim V = \dim \image S$.
We now have
$\image S \subset V^*$
and
$\dim \image S = \dim V^\ast$.
(The first assertion follows directly from the definition of $S$.)
Hence $\image S = V^\ast$ (see \PMlinkname{this page}{VectorSubspace}),
and $S$ is a surjection. We have shown that
$S$ is a linear isomorphism.

Let us next define the mapping $T: V\to W^*$, $a\mapsto \omega(a,\cdot)$.
Applying the  \PMlinkid{rank-nullity theorem}{2238} to $T$ yields
\begin{eqnarray}
\label{eq0}
\dim V &=& \dim \ker T + \dim \image T.
\end{eqnarray}
Now  $\ker T = W^\omega$ and $ \image T = W^*$.
To see the latter assertion, first note that from the definition of $T$, we
have $\image T \subset W^*$. Since $S$ is a linear
isomorphism, we also have $\image T \supset W^*$.
Then, since $\dim W= \dim W^*$,
the result follows from equation \ref{eq0}. $\Box$
%%%%%
%%%%%
\end{document}
