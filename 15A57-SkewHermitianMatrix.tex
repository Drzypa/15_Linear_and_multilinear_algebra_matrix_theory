\documentclass[12pt]{article}
\usepackage{pmmeta}
\pmcanonicalname{SkewHermitianMatrix}
\pmcreated{2013-03-22 13:36:14}
\pmmodified{2013-03-22 13:36:14}
\pmowner{matte}{1858}
\pmmodifier{matte}{1858}
\pmtitle{skew-Hermitian matrix}
\pmrecord{21}{34231}
\pmprivacy{1}
\pmauthor{matte}{1858}
\pmtype{Definition}
\pmcomment{trigger rebuild}
\pmclassification{msc}{15A57}
\pmsynonym{anti-Hermitian matrix}{SkewHermitianMatrix}
\pmrelated{HermitianMatrix}
\pmrelated{SymmetricMatrix}
\pmrelated{SkewSymmetricMatrix}

% this is the default PlanetMath preamble.  as your knowledge
% of TeX increases, you will probably want to edit this, but
% it should be fine as is for beginners.

% almost certainly you want these
\usepackage{amssymb}
\usepackage{amsmath}
\usepackage{amsfonts}

% used for TeXing text within eps files
%\usepackage{psfrag}
% need this for including graphics (\includegraphics)
%\usepackage{graphicx}
% for neatly defining theorems and propositions
%\usepackage{amsthm}
% making logically defined graphics
%%%\usepackage{xypic}

% there are many more packages, add them here as you need them

% define commands here
\begin{document}
\newcommand{\ccj}[1]{\overline{#1}}
\def\dtra{\hspace{0.04cm} ^{\mbox{\scriptsize{T}}} \hspace{0.02cm}}

{\bf Definition.} A square matrix $A$ with complex entries is 
\emph{skew-Hermitian}, if
 $$ A^* = -A. $$
Here $A^\ast=\ccj{A\dtra}$, $A\dtra$ is the transpose of $A$, and $\ccj{A}$ is
is the complex conjugate of the matrix $A$.

\subsection*{Properties.}
\begin{enumerate}
\item The trace of a skew-Hermitian matrix is \PMlinkid{imaginary}{2017}. 
\item The eigenvalues of a skew-Hermitian matrix are
 \PMlinkid{imaginary}{2017}. 
\end{enumerate}


\emph{Proof.} Property (1) follows directly from property (2) since the
trace is the sum of the eigenvalues. But one can also give a simple proof
as follows.  Let $x_{ij}$ and $y_{ij}$ be the 
real respectively imaginary parts of the elements in $A$. 
Then the diagonal elements of $A$ are of the
form $x_{kk} + i y_{kk}$, and the diagonal elements in $A^\ast$ 
are of the form $-x_{kk} + iy_{kk}$. Hence $x_{kk}$, i.e., the real
part for the diagonal elements in $A$ must vanish, and 
property (1) follows.
For property (2), suppose 
 $A$ is a skew-Hermitian matrix, and $x$ an 
eigenvector corresponding to the eigenvalue $\lambda$, i.e.,  
\begin{eqnarray}
\label{gugg}
Ax &=& \lambda x.
\end{eqnarray}
Here, $x$ is a complex column vector. 
Since $x$ is an eigenvector, $x$ is not the zero vector, and
$x^\ast x > 0$. Without loss of generality we can assume $x^\ast x =1$.
 Thus
\begin{eqnarray*}
\ccj{\lambda} &=& x^\ast \ccj{\lambda} x\\
&=& ( x^\ast \lambda x )^\ast \\
&=& (x^\ast A x )^\ast \\
&=& x^\ast A^\ast x \\
&=& x^\ast (-A) x \\
&=& -x^\ast \lambda x \\
&=& - \lambda .
\end{eqnarray*}
Hence the eigenvalue $\lambda$ corresponding
to $x$ is \PMlinkid{imaginary}{2017}. $\Box$
%%%%%
%%%%%
\end{document}
