\documentclass[12pt]{article}
\usepackage{pmmeta}
\pmcanonicalname{QuadraticSpace}
\pmcreated{2013-03-22 15:05:55}
\pmmodified{2013-03-22 15:05:55}
\pmowner{CWoo}{3771}
\pmmodifier{CWoo}{3771}
\pmtitle{quadratic space}
\pmrecord{14}{36827}
\pmprivacy{1}
\pmauthor{CWoo}{3771}
\pmtype{Definition}
\pmcomment{trigger rebuild}
\pmclassification{msc}{15A63}
\pmclassification{msc}{11E88}
\pmsynonym{non-degenerate quadratic space}{QuadraticSpace}
\pmrelated{QuadraticForm}
\pmrelated{QuaternionAlgebra}
\pmdefines{norm form}
\pmdefines{isomorphic quadratic spaces}
\pmdefines{isometric quadratic spaces}
\pmdefines{generalized quaternion algebra}
\pmdefines{regular quadratic space}

% this is the default PlanetMath preamble.  as your knowledge
% of TeX increases, you will probably want to edit this, but
% it should be fine as is for beginners.

% almost certainly you want these
\usepackage{amssymb,amscd}
\usepackage{amsmath}
\usepackage{amsfonts}

% used for TeXing text within eps files
%\usepackage{psfrag}
% need this for including graphics (\includegraphics)
%\usepackage{graphicx}
% for neatly defining theorems and propositions
%\usepackage{amsthm}
% making logically defined graphics
%%%\usepackage{xypic}

% there are many more packages, add them here as you need them

% define commands here
\begin{document}
A \emph{quadratic space} (over a field) is a vector space $V$ equipped with a quadratic form $Q$ on $V$.  It is denoted by $(V,Q)$.  The dimension of the quadratic space is the dimension of the underlying vector space.  Any vector space admitting a bilinear form has an induced quadratic form and thus is a quadratic space.

Two quadratic spaces $(V_1,Q_1)$ and $(V_2,Q_2)$ are said to be \emph{isomorphic} if there exists an isomorphic linear transformation $T:V_1\to V_2$ such that for any $v\in V_1$, $Q_1(v)=Q_2(Tv)$.  Since $T$ is easily seen to be an isometry between $V_1$ and $V_2$ (over the symmetric bilinear forms induced by $Q_1$ and $Q_2$ respectively), we also say that $(V_1,Q_1)$ and $(V_2,Q_2)$ are isometric.

A quadratic space equipped with a regular quadratic form is called a \emph{regular quadratic space}.

\textbf{Example of a Qudratic Space.}  The \emph{Generalized Quaternion Algebra}.

Let $F$ be a field and $a,b\in \dot{F}:=F-\lbrace 0\rbrace$.  Let $H$ be the algebra over $F$ generated by $i,j$ with the following defining relations:
\begin{enumerate}
\item $i^2=a$,
\item $j^2=b$, and
\item $ij=-ji$.
\end{enumerate}
Then $\lbrace 1,i,j,k\rbrace$, where $k:=ij$, forms a basis for the vector space $H$ over $F$.  For a direct proof, first note $(ij)^2=(ij)(ij)=i(ji)j=i(-ij)j=-ab\neq 0$, so that $k\in\dot{F}$.  It's also not hard to show that $k$ anti-commutes with both $i,j$: $ik=-ki$ and $jk=-kj$.  Now, suppose $0=r+si+tj+uk$.  Multiplying both sides of the equation on the right by $i$ gives $0=ri+sa+tji+uki$.  Multiplying both sides on the left by $i$ gives $0=ri+sa+tij+uik$.  Adding the two results and reduce, we have $0=ri+sa$.  Multiplying this again by $i$ gives us $0=ra+sai$, or $0=r+si$.  Similarly, one shows that $0=r+tj$, so that $si=tj$.  This leads to two equations, $sa=tij$ and $sa=tji$, if one multiplies it on the left and right by $i$.  Adding the results then dividing by 2 gives $sa=0$.  Since $a\ne 0$, $s=0$.  Therefore, $0=r+si=r$.  Same argument shows that $t=u=0$ as well.

Next, for any element $\alpha=r+si+tj+uk\in H$, define its conjugate $\overline{\alpha}$ by $r-si-tj-uk$.  Note that $\alpha=\overline{\alpha}$ iff $\alpha\in F$.  Also, it's not hard to see that 
\begin{itemize}
\item $\overline{\overline{\alpha}}=\alpha$,
\item $\overline{\alpha+\beta}=\overline{\alpha}+\overline{\beta}$,
\item $\overline{\alpha\beta}=\overline{\beta}\overline{\alpha}$,
\end{itemize}

We next define the norm $N$ on $H$ by $N(\alpha)=\alpha\overline{\alpha}$.  Since $\overline{N(\alpha)}=\overline{\alpha\overline{\alpha}}=
\overline{\overline{\alpha}}\space \overline{\alpha}=\alpha\overline{\alpha}=N(\alpha)$, $N(\alpha)\in F$.  It's easy to see that $N(r\alpha)=r^2N(\alpha)$ for any $r\in F$.

Finally, if we define the trace $T$ on $H$ by $T(\alpha)=\alpha+ \overline{\alpha}$, we have that $N(\alpha+\beta)-N(\alpha)-N(\beta)=T(\alpha \overline{\beta})$ is bilinear (linear each in $\alpha$ and $\beta$).

Therefore, $N$ defines a quadratic form on $H$ ($N$ is commonly called a \emph{norm form}), and $H$ is thus a quadratic space over $F$.  $H$ is denoted by $$\Big( \frac{a,b}{F} \Big).$$
It can be shown that $H$ is a central simple algebra over $F$.  Since $H$ is four dimensional over $F$, it is a quaternion algebra.  It is a direct generalization of the quaternions $\mathbb{H}$ over the reals 
$$\Big( \frac{-1,-1}{\mathbb{R}} \Big).$$
In fact, every quaternion algebra (over a field $F$) is of the form $\displaystyle{\Big( \frac{a,b}{F} \Big)}$ for some $a,b\in F$.
%%%%%
%%%%%
\end{document}
