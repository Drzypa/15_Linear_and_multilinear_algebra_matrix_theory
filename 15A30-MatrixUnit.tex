\documentclass[12pt]{article}
\usepackage{pmmeta}
\pmcanonicalname{MatrixUnit}
\pmcreated{2013-03-22 18:30:35}
\pmmodified{2013-03-22 18:30:35}
\pmowner{CWoo}{3771}
\pmmodifier{CWoo}{3771}
\pmtitle{matrix unit}
\pmrecord{8}{41194}
\pmprivacy{1}
\pmauthor{CWoo}{3771}
\pmtype{Definition}
\pmcomment{trigger rebuild}
\pmclassification{msc}{15A30}
\pmclassification{msc}{16S50}
\pmrelated{ElementaryMatrix}
\pmdefines{full set of matrix units}

\usepackage{amssymb,amscd}
\usepackage{amsmath}
\usepackage{amsfonts}
\usepackage{mathrsfs}

% used for TeXing text within eps files
%\usepackage{psfrag}
% need this for including graphics (\includegraphics)
%\usepackage{graphicx}
% for neatly defining theorems and propositions
\usepackage{amsthm}
% making logically defined graphics
%%\usepackage{xypic}
\usepackage{pst-plot}

% define commands here
\newcommand*{\abs}[1]{\left\lvert #1\right\rvert}
\newtheorem{prop}{Proposition}
\newtheorem{thm}{Theorem}
\newtheorem{ex}{Example}
\newcommand{\real}{\mathbb{R}}
\newcommand{\pdiff}[2]{\frac{\partial #1}{\partial #2}}
\newcommand{\mpdiff}[3]{\frac{\partial^#1 #2}{\partial #3^#1}}
\begin{document}
A \emph{matrix unit} is a matrix (over some ring with $1$) whose entries are all $0$ except in one cell, where it is $1$.

For example, among the $3\times 2$ matrices, 
$$
\begin{pmatrix}
1 & 0 \\
0 & 0 \\
0 & 0 
\end{pmatrix},\quad
\begin{pmatrix}
0 & 1 \\
0 & 0 \\
0 & 0 
\end{pmatrix},\quad
\begin{pmatrix}
0 & 0 \\
1 & 0 \\
0 & 0 
\end{pmatrix},\quad
\begin{pmatrix}
0 & 0 \\
0 & 1 \\
0 & 0 
\end{pmatrix},\quad
\begin{pmatrix}
0 & 0 \\
0 & 0 \\
1 & 0 
\end{pmatrix},\quad
\begin{pmatrix}
0 & 0 \\
0 & 0 \\
0 & 1 
\end{pmatrix}
$$
are the matrix units.

Let $A$ and $B$ be $m\times n$ and $p\times q$ matrices over $R$, and $U_{ij}$ an $n\times p$ matrix unit (over $R$).  Then 
\begin{enumerate}
\item
$AU_{ij}$ is the $m\times p$ matrix whose $j$th column is the $i$th column of $A$, and $0$ everywhere else, and
\item
$U_{ij}B$ is the $n\times q$ matrix whose $i$th row is the $j$th row of $B$ and $0$ everywhere else.
\end{enumerate}

\textbf{Remarks}.  Let $M=M_{m\times n}(R)$ be the set of all $m$ by $n$ matrices with entries in a ring $R$ (with $1$).  Denote $U_{ij}$ the matrix unit in $M$ whose cell $(i,j)$ is $1$.
\begin{itemize}
\item $M$ is a (left or right) $R$-module generated by the $m\times n$ matrix units.
\item When $m=n$, $M$ has the structure of an algebra over $R$.  The matrix units have the following properties:
\begin{enumerate}
\item $U_{ij}U_{k\ell}=\delta_{jk}U_{i\ell}$, and
\item $U_{11}+\cdots+U_{nn}=I_n$,
\end{enumerate}
where $\delta_{ij}$ is the Kronecker delta and $I_n$ is the identity matrix.  Note that the $U_{ii}$ form a complete set of pairwise orthogonal idempotents, meaning $U_{ii}U_{ii}=U_{ii}$ and $U_{ii}U_{jj}=0$ if $i\ne j$.
\item In general, in a matrix ring $S$ (consisting of, say, all $n\times n$ matrices), any set of $n$ matrices satisfying the two properties above is called a \emph{full set of matrix units} of $S$.  
\item For example, if $\lbrace U_{ij}\mid 1\le i,j\le 2\rbrace$ is the set of $2\times 2$ matrix units over $\mathbb{R}$, then for any invertible matrix $T$, $\lbrace TU_{ij}T^{-1}\mid 1\le i,j\le 2\rbrace$ is a full set of matrix units.
\item If we embed $R$ as a subring of $M_n(R)$, then $R$ is the centralizer of the matrix units of $M_n(R)$, meaning that the only elements in $M_n(R)$ that commute with the matrix units are the elements in $R$.
\end{itemize}

\begin{thebibliography}{9}
\bibitem{tylam} T. Y. Lam, \emph{Lectures on Modules and Rings}, Springer, New York, 1998.
\end{thebibliography}
%%%%%
%%%%%
\end{document}
