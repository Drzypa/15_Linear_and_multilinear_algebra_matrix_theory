\documentclass[12pt]{article}
\usepackage{pmmeta}
\pmcanonicalname{NeumannSeries}
\pmcreated{2013-03-22 15:25:49}
\pmmodified{2013-03-22 15:25:49}
\pmowner{georgiosl}{7242}
\pmmodifier{georgiosl}{7242}
\pmtitle{Neumann series}
\pmrecord{9}{37277}
\pmprivacy{1}
\pmauthor{georgiosl}{7242}
\pmtype{Theorem}
\pmcomment{trigger rebuild}
\pmclassification{msc}{15-00}

% this is the default PlanetMath preamble.  as your knowledge
% of TeX increases, you will probably want to edit this, but
% it should be fine as is for beginners.

% almost certainly you want these
\usepackage{amssymb}
\usepackage{amsmath}
\usepackage{amsfonts}

% used for TeXing text within eps files
%\usepackage{psfrag}
% need this for including graphics (\includegraphics)
%\usepackage{graphicx}
% for neatly defining theorems and propositions
%\usepackage{amsthm}
% making logically defined graphics
%%%\usepackage{xypic}

% there are many more packages, add them here as you need them

% define commands here
\begin{document}
If $A$ is a square matrix, $\|A\|<1$, then $I-A$ is nonsingular and 
$(I-A)^{-1}=I+A+A^2+\cdots=\sum_{k=0}^{\infty}A^k$. This is the \emph{Neumann series}. 
\\It provides approximations of $(I-A)^{-1}$ 
when $A$ has entries of small magnitude.
For example, a first-order approximation is $(I-A)^{-1}\approx I+A$.
\\It is obvious that this is a generalization of the geometric series.
\begin{thebibliography}{6}
\bibitem{AW}
Carl D. Meyer, {\it Matrix Analysis and Applied Linear Algebra}.
\end{thebibliography}
%%%%%
%%%%%
\end{document}
