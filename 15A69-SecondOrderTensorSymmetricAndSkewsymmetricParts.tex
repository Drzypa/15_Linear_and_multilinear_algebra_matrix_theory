\documentclass[12pt]{article}
\usepackage{pmmeta}
\pmcanonicalname{SecondOrderTensorSymmetricAndSkewsymmetricParts}
\pmcreated{2013-03-22 15:51:32}
\pmmodified{2013-03-22 15:51:32}
\pmowner{rspuzio}{6075}
\pmmodifier{rspuzio}{6075}
\pmtitle{second order tensor: symmetric and skew-symmetric parts}
\pmrecord{18}{37847}
\pmprivacy{1}
\pmauthor{rspuzio}{6075}
\pmtype{Theorem}
\pmcomment{trigger rebuild}
\pmclassification{msc}{15A69}

\endmetadata

% this is the default PlanetMath preamble.  as your knowledge
% of TeX increases, you will probably want to edit this, but
% it should be fine as is for beginners.

% almost certainly you want these
\usepackage{amssymb}
\usepackage{amsmath}
\usepackage{amsfonts}
\usepackage{amsthm}

% used for TeXing text within eps files
%\usepackage{psfrag}
% need this for including graphics (\includegraphics)
%\usepackage{graphicx}
% for neatly defining theorems and propositions
%\usepackage{amsthm}
% making logically defined graphics
%%%\usepackage{xypic}

% there are many more packages, add them here as you need them

% define commands here
\newtheorem{theorem*}{Theorem}

\begin{document}
We shall prove the following theorem on existence and uniqueness.  (Here,
we assime that the ground field has characteristic different from 2. 
This hypothesis is satisfied for the cases of greatest interest, 
namely real and complex ground fields.)
\begin{theorem*} Every covariant and contravariant tensor of second rank may be expressed univocally as the sum of a symmetric and skew-symmetric tensor.
\end{theorem*}
\begin{proof} 
Let us consider a contravariant tensor. 

{\em 1. Existence.}\, Put
\begin{align*}
U^{ij}=\frac{1}{2}(T^{ij}+T^{ji}), \qquad  W^{ij}=\frac{1}{2}(T^{ij}-T^{ji}).
\end{align*}
Then $U^{ij}=U^{ji}$ is symmetric,  $W^{ij}=-W^{ji}$ is skew-symmetric, and
\begin{align*}
T^{ij}=U^{ij}+W^{ij}.
\end{align*}
{\em 2. Uniqueness.}\, Let us suppose that $T^{ij}$ admits the decompositions
\begin{align*}
T^{ij}=U^{ij}+W^{ij}=U'^{ij}+W'^{ij}.
\end{align*}
By taking the transposes
\begin{align*}
T^{ji}=U^{ji}+W^{ji}=U'^{ji}+W'^{ji},
\end{align*}
we separate the symmetric and skew-symmetric parts in both equations and making use of their symmetry properties, we have
\begin{eqnarray*}
U^{ij}-U'^{ij} & = & W'^{ij}-W^{ij} \\
=U^{ji}-U'^{ji} & = & W'^{ji}-W^{ji} \\
=W^{ij}-W'^{ij} & = & U'^{ij}-U^{ij} \\
=-(U^{ij}-U'^{ij}) & = & 0,
\end{eqnarray*}
which shows uniqueness of each part. {\em mutatis mutandis}\, for a covariant tensor $T_{ij}$.
\end{proof}
%%%%%
%%%%%
\end{document}
