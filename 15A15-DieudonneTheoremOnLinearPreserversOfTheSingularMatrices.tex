\documentclass[12pt]{article}
\usepackage{pmmeta}
\pmcanonicalname{DieudonneTheoremOnLinearPreserversOfTheSingularMatrices}
\pmcreated{2013-03-22 19:19:49}
\pmmodified{2013-03-22 19:19:49}
\pmowner{kammerer}{26336}
\pmmodifier{kammerer}{26336}
\pmtitle{Dieudonn\'e theorem on linear preservers of the singular matrices}
\pmrecord{8}{42276}
\pmprivacy{1}
\pmauthor{kammerer}{26336}
\pmtype{Theorem}
\pmcomment{trigger rebuild}
\pmclassification{msc}{15A15}
\pmclassification{msc}{15A04}
%\pmkeywords{linear preserver}
%\pmkeywords{singular matrix}
\pmrelated{FundamentalTheoremOfProjectiveGeometry}
\pmrelated{FrobeniusTheoremOnLinearDeterminantPreservers}

\endmetadata

% this is the default PlanetMath preamble.  as your knowledge
% of TeX increases, you will probably want to edit this, but
% it should be fine as is for beginners.

% almost certainly you want these
\usepackage{amssymb}
\usepackage{amsmath}
\usepackage{amsfonts}

% used for TeXing text within eps files
%\usepackage{psfrag}
% need this for including graphics (\includegraphics)
%\usepackage{graphicx}
% for neatly defining theorems and propositions
\usepackage{amsthm}
% making logically defined graphics
%%%\usepackage{xypic}
\newtheorem{thm}{Theorem}
% there are many more packages, add them here as you need them

% define commands here

\begin{document}
Let $\mathbb{F}$ be an arbitrary field. Consider $\mathcal{M}_n (\mathbb{F})$, the vector space of all $n \times n$ matrices over $\mathbb{F}$. Moreover, let $\mathcal{GL}_n (\mathbb{F})$ be the full linear group of nonsingular $n \times n$ matrices over $\mathbb{F}$.
\begin{thm}
For a linear automorphism $\varphi : \mathcal{M}_n (\mathbb{F}) \longrightarrow \mathcal{M}_n (\mathbb{F})$ the following conditions are equivalent:\\
\begin{tabular}{cl}
(i)&$\displaystyle \forall\, A \in \mathcal{M}_n (\mathbb{F}) :\, \det (A) = 0\, \Rightarrow\, \det (\varphi (A)) = 0$,\\
(ii)&either $\displaystyle \exists\, P, Q \in \mathcal{GL}_n (\mathbb{F})\, \forall\, A \in \mathcal{M}_n (\mathbb{F}) :\, \varphi (A) = P A Q$, or $\displaystyle \exists\, P, Q \in \mathcal{GL}_n (\mathbb{F})\, \forall\, A \in \mathcal{M}_n (\mathbb{F}) :\, \varphi (A) = P A^\top Q$.\\
\end{tabular}
\end{thm}

The original proof \cite{dieud} of the nontrivial implication (i) $\Rightarrow$ (ii) is based on the fundamental theorem of projective geometry.   
\begin{thebibliography}{99}
\bibitem[D]{dieud}
J. Dieudonn\'e, Sur une g\'en\'eralisation du groupe orthogonal \`a quatre variables, \emph{Arch. Math.} {\bf 1}: 282--287 (1949).
\end{thebibliography}
%%%%%
%%%%%
\end{document}
