\documentclass[12pt]{article}
\usepackage{pmmeta}
\pmcanonicalname{CommonEigenvectorOfADiagonalElementCrosssection}
\pmcreated{2013-03-22 15:30:41}
\pmmodified{2013-03-22 15:30:41}
\pmowner{lars_h}{9802}
\pmmodifier{lars_h}{9802}
\pmtitle{common eigenvector of a diagonal element cross-section}
\pmrecord{4}{37378}
\pmprivacy{1}
\pmauthor{lars_h}{9802}
\pmtype{Theorem}
\pmcomment{trigger rebuild}
\pmclassification{msc}{15A18}
%\pmkeywords{common eigenvector}
%\pmkeywords{triangular matrix}
\pmrelated{CommutingMatrices}

\endmetadata

\usepackage{amsmath,amsfonts,amssymb,amsthm}

\newtheorem{theorem}{Theorem}
\newtheorem{lemma}[theorem]{Lemma}


\newcommand{\mc}{\mathcal}
\newcommand{\vek}{\mathbf}

\newcommand{\Mat}{\mathrm{M}}
\begin{document}
Denote by $\Mat_n(\mc{K})$ the set of all $n \times n$ matrices over 
$\mc{K}$. Let \(d_i\colon \Mat_n(\mc{K}) \longrightarrow \mc{K}\) be 
the function which extracts the $i$th diagonal element of a matrix, 
and let \(\varepsilon_i\colon \mc{K}^n \longrightarrow \mc{K}\) be 
the function which extracts the $i$th \PMlinkescapetext{component} 
of a vector. Finally denote by $[n]$ the set $\{1,\dotsc,n\}$. 

\PMlinkescapeword{row}
\PMlinkescapeword{column}
\PMlinkescapeword{rows}
\PMlinkescapeword{columns}
\PMlinkescapeword{index}


\begin{theorem}
  Let $\mc{K}$ be a field. 
  For any sequence \(A_1,\dotsc,A_r \in \Mat_n(\mc{K})\) of upper 
  triangular pairwise commuting matrices and every row index 
  \(i \in [n]\), there exists \(\vek{u} \in \mc{K}^n \setminus \{0\}\) 
  such that
  \begin{equation}
    A_k \vek{u} = d_i(A_k) \vek{u}
    \quad\text{for all \(k \in [r]\).}
  \end{equation}
\end{theorem}
\begin{proof}
  Let \(\lambda_k = d_i(A_k)\) for all \(k \in [r]\), so that the 
  problem is to find a common eigenvector $\vek{u}$ of $A_1,\dotsc,A_r$ 
  whose corresponding eigenvalue for $A_k$ is $\lambda_k$. It is 
  sufficient to find such a common eigenvector in the case that 
  \(i=n\) is the least \(i \in [n]\) for which \(d_i(A_k) = \lambda_k\) 
  for all \(k \in [r]\), because if some smaller $i$ also 
  has this property then one can solve the corresponding problem for 
  the $i \times i$ submatrices consisting of rows and columns $1$ 
  through $i$ of $A_1,\dotsc,A_r$, and then pad the common eigenvector 
  of these submatrices with zeros to get a common eigenvector of the 
  original $A_1,\dotsc,A_r$.
  
  By the existence of a 
  \PMlinkname{characteristic matrix of a diagonal element 
  cross-section}{CharacteristicMatrixOfDiagonalElementCrossSection} 
  there exists a matrix $C$ in the unital algebra 
  generated by $A_1,\dotsc,A_r$ such that \(d_i(C)=1\) if 
  \(d_i(A_k)=\lambda_k\) for all \(k \in [r]\), and \(d_i(C)=0\) 
  otherwise; in other words that matrix $C$ satisfies \(d_n(C)=1\) and 
  \(d_i(C)=0\) for all \(i<n\). Since it is also upper triangular it 
  follows that the matrix $I-C$ has 
  \PMlinkname{rank}{RankLinearMapping} 
  $n-1$, so the kernel of this 
  matrix is one-dimensional. Let \(\vek{u} = (u_1,\dotsc,u_n) \in 
  \ker(I -\nobreak C)\) be such that \(u_n=1\); it is easy to see that 
  this is always possible (indeed, the only vector in this nullspace 
  with $n$th 
  \PMlinkescapetext{component} 
  $0$ is the zero vector). This $\vek{u}$ is the 
  wanted eigenvector.
  
  To see that it is an eigenvector of $A_k$, one may first observe 
  that $C$ commutes with this $A_k$, since the unital algebra of 
  matrices to which $C$ belongs is 
  \PMlinkname{commutative}{Commutative}. This implies that 
  \(A_k \vek{u} \in \ker (I -\nobreak C)\) since \(\vek{0} = 
  A_k \vek{0} = A_k (I -\nobreak C) \vek{u} = 
  (I -\nobreak C) A_k \vek{u}\). As $\ker (I -\nobreak C)$ is 
  one-dimensional it follows that \(A_k \vek{u} = \lambda \vek{u}\) 
  for some \(\lambda \in \mc{K}\). Since $A_k$ is upper triangular 
  and \(u_n=1\) this $\lambda$ must furthermore satisfy 
  \(\lambda = \lambda u_n = \varepsilon_n(\lambda\vek{u}) = 
  \varepsilon_n(A_k\vek{u}) = d_n(A_k) u_n = d_n(A_k)\), which is 
  indeed what the eigenvalue was claimed to be.
\end{proof}
%%%%%
%%%%%
\end{document}
