\documentclass[12pt]{article}
\usepackage{pmmeta}
\pmcanonicalname{ProofOfNecessaryAndSufficientConditionForDiagonalizability}
\pmcreated{2013-03-22 14:15:45}
\pmmodified{2013-03-22 14:15:45}
\pmowner{rspuzio}{6075}
\pmmodifier{rspuzio}{6075}
\pmtitle{proof of necessary and sufficient condition for diagonalizability}
\pmrecord{13}{35712}
\pmprivacy{1}
\pmauthor{rspuzio}{6075}
\pmtype{Proof}
\pmcomment{trigger rebuild}
\pmclassification{msc}{15A04}

% this is the default PlanetMath preamble.  as your knowledge
% of TeX increases, you will probably want to edit this, but
% it should be fine as is for beginners.

% almost certainly you want these
\usepackage{amssymb}
\usepackage{amsmath}
\usepackage{amsfonts}

% used for TeXing text within eps files
%\usepackage{psfrag}
% need this for including graphics (\includegraphics)
%\usepackage{graphicx}
% for neatly defining theorems and propositions
%\usepackage{amsthm}
% making logically defined graphics
%%%\usepackage{xypic}

% there are many more packages, add them here as you need them

% define commands here
\begin{document}
First, suppose that $T$ is diagonalizable. Then $V$ has a basis whose elements $\{v_{1}, \ldots, v_{n}\}$ are eigenvectors of $T$ associated with the eigenvalues $\{\lambda_{1}, \ldots, \lambda_{n}\}$ respectively. For each $i=1,...,n$, as $v_{i}$ is an eigenvector, its annihilator polynomial is $m_{v_{i}}=X-\lambda_{i}$. As these vectors form a basis of $V$, we have that the \PMlinkname{minimal polynomial}{MinimalPolynomialEndomorphism} of $T$ is $m_{T}=\operatorname{lcm}(X-\lambda_{1}, \ldots, X-\lambda_{n})$ which is trivially a product of linear factors.

Now, suppose that $m_{T}=(X-\lambda_{1}) \ldots (X-\lambda_{p})$ for some $p$.
Let $v \in V$. Consider the $T$ - cyclic subspace generated by $v$, $Z(v,T)=\left<v, Tv, \ldots, T^{r}v\right>$. Let $T_{v}$ be the restriction of $T$ to $Z(v,T)$. Of course, $v$ is a cyclic vector of $Z(v,T_{v})$, and then $m_{v}=m_{T_{v}}=\chi_{T}$. This is really easy to see: the dimension of $Z(v,T)$ is $r+1$, and it's also the degree of $m_{v}$. But as $m_{v}$ divides $m_{T_{v}}$ (because $m_{T_{v}}v=0$), and $m_{T}$ divides $\chi_{T_{v}}$ (Cayley-Hamilton theorem), we have that $m_{v}$ divides $\chi_{T_{v}}$. As these are two monic polynomials of degree $r+1$ and one divides the other, they are equal. And then by the same reasoning $m_{v}=m_{T_{v}}=\chi_{T}$.
But as $m_{v}$ divides $m_{T}$, then as $m_{v}=m_{T_{v}}$, we have that $m_{T_{v}}$ divides $m_{T}$, and then $m_{T_{v}}$ has no multiple roots and they all lie in $k$. But then so does $\chi_{T_{v}}$. Suppose that these roots are $\lambda_{1}, \ldots, \lambda_{r+1}$. Then $Z(v,T)=\bigoplus_{\lambda_{i}}E_{\lambda_{i}}$, where $E_{\lambda_{i}}$ is the eigenspace associated to $\lambda_{i}$. Then $v$ is a sum of eigenvectors. QED.
%%%%%
%%%%%
\end{document}
