\documentclass[12pt]{article}
\usepackage{pmmeta}
\pmcanonicalname{RanknullityTheorem}
\pmcreated{2013-03-22 16:35:40}
\pmmodified{2013-03-22 16:35:40}
\pmowner{yark}{2760}
\pmmodifier{yark}{2760}
\pmtitle{rank-nullity theorem}
\pmrecord{7}{38790}
\pmprivacy{1}
\pmauthor{yark}{2760}
\pmtype{Theorem}
\pmcomment{trigger rebuild}
\pmclassification{msc}{15A03}
\pmrelated{RankLinearMapping}
\pmrelated{Nullity}

\endmetadata

\usepackage{amsmath}

\def\dim{\operatorname{dim}}
\def\codim{\operatorname{codim}}
\def\im{\operatorname{im}}
\def\ker{\operatorname{ker}}

\def\floor#1{\lfloor#1\rfloor}
\def\ceiling#1{\lceil#1\rceil}
\begin{document}
\PMlinkescapeword{canonical}
\PMlinkescapeword{sum}
\PMlinkescapeword{terms}
\PMlinkescapeword{words}

Let $V$ and $W$ be vector spaces over the same field.
If $\phi\colon V\to W$ is a linear mapping, then
\[
  \dim V = \dim(\ker\phi) + \dim(\im\phi).
\]
In other words, the dimension of $V$
is equal to the \PMlinkname{sum}{CardinalArithmetic}
of the \PMlinkname{rank}{RankLinearMapping} and nullity of $\phi$.

Note that if $U$ is a subspace of $V$, then this
(applied to the canonical mapping $V\to V/U$) says that
\[
  \dim V = \dim U + \dim(V/U),
\]
that is,
\[
  \dim V = \dim U + \codim U,
\]
where $\codim$ denotes codimension.

An alternative way of stating the rank-nullity theorem is
by saying that if
\[
  0\to U\to V\to W\to 0
\]
is a short exact sequence of vector spaces, then
\[
  \dim(V) = \dim(U) + \dim(W).
\]
In fact, if
\[
  0\to V_1\to\cdots\to V_n\to 0
\]
is an exact sequence of vector spaces, then
\[
  \sum_{i=1}^{\floor{n/2}}V_{2i}=\sum_{i=1}^{\ceiling{n/2}}V_{2i-1},
\]
that is, the sum of the dimensions of even-numbered terms
is the same as the sum of the dimensions of the odd-numbered terms.

%%%%%
%%%%%
\end{document}
