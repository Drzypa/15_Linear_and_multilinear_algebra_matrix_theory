\documentclass[12pt]{article}
\usepackage{pmmeta}
\pmcanonicalname{SqueezingmathbbRn}
\pmcreated{2013-03-22 17:10:02}
\pmmodified{2013-03-22 17:10:02}
\pmowner{pahio}{2872}
\pmmodifier{pahio}{2872}
\pmtitle{squeezing $\mathbb{R}^n$}
\pmrecord{20}{39479}
\pmprivacy{1}
\pmauthor{pahio}{2872}
\pmtype{Topic}
\pmcomment{trigger rebuild}
\pmclassification{msc}{15-00}
\pmclassification{msc}{15A04}
\pmrelated{circle}
\pmrelated{Ellipse2}
\pmrelated{Circle}
\pmrelated{ConicSection}
\pmrelated{VolumeOfEllipsoid}

% this is the default PlanetMath preamble.  as your knowledge
% of TeX increases, you will probably want to edit this, but
% it should be fine as is for beginners.

% almost certainly you want these
\usepackage{amssymb}
\usepackage{amsmath}
\usepackage{amsfonts}
\usepackage{amsthm}

\usepackage{mathrsfs}
\usepackage{pstricks}
\usepackage{pst-plot}

% used for TeXing text within eps files
%\usepackage{psfrag}
% need this for including graphics (\includegraphics)
%\usepackage{graphicx}
% for neatly defining theorems and propositions
%
% making logically defined graphics
%%%\usepackage{xypic}

% there are many more packages, add them here as you need them

% define commands here

\newcommand{\sR}[0]{\mathbb{R}}
\newcommand{\sC}[0]{\mathbb{C}}
\newcommand{\sN}[0]{\mathbb{N}}
\newcommand{\sZ}[0]{\mathbb{Z}}

 \usepackage{bbm}
 \newcommand{\Z}{\mathbbmss{Z}}
 \newcommand{\C}{\mathbbmss{C}}
 \newcommand{\F}{\mathbbmss{F}}
 \newcommand{\R}{\mathbbmss{R}}
 \newcommand{\Q}{\mathbbmss{Q}}



\newcommand*{\norm}[1]{\lVert #1 \rVert}
\newcommand*{\abs}[1]{| #1 |}



\newtheorem{thm}{Theorem}
\newtheorem{defn}{Definition}
\newtheorem{prop}{Proposition}
\newtheorem{lemma}{Lemma}
\newtheorem{cor}{Corollary}
\begin{document}
Squeezing the vector space $\mathbb{R}^n$ in the direction of one coordinate axis, \PMlinkname{i.e.}{Ie} multiplying a certain component $x_i$ of all vectors by a non-zero real number $k$, is a linear transformation of $\mathbb{R}^n$.\\

A concrete example of such squeezing and its results is obtained if we squeeze in $\mathbb{R}^2$, i.e. in the Euclidean plane formed by all pairs\, $(x,\,y)$\, of real numbers, every $y$-coordinate by a positive number\, 
$k = \frac{b}{a}$\, where\, $a > b > 0$.\, We may look how this procedure \PMlinkescapetext{acts} on the circle
\begin{align}
x^2+y^2 = a^2.
\end{align}
Since all ordinates of this equation are shrinked by the factor $\displaystyle \frac{b}{a}$ which is less than 1, we must must multiply the new $y$ in equation (1) by the inverse number $\displaystyle\frac{a}{b}$ in \PMlinkescapetext{order} to keep the equation satisfied; then the new $y$ no longer denotes the ordinate of the circle, but rather the ordinate of the squeezed circle. Thus, the equation of the squeezed curve is
$$x^2+\left(\frac{a}{b}\!\cdot\!y\right)^2 = a^2.$$
Simplifying we first obtain
$$x^2+\frac{a^2y^2}{b^2} = a^2,$$
and dividing all \PMlinkescapetext{terms} by $a^2$ yields
\begin{align}
   \frac{x^2}{a^2}+\frac{y^2}{b^2} = 1.
\end{align}
So the resulting curve is an ellipse with semiaxes $a$ and $b$.\\

In the picture below, the circle\, $x^2\!+\!y^2=a^2$\! is drawn in red and the ellipse\, $\displaystyle \frac{x^2}{a^2}+\frac{y^2}{b^2}=1$\, in blue.  The angle $t$ is the eccentric anomaly at the point $P$ of the ellipse, which has the \PMlinkname{parametric presentation}{Parameter}\, $x = a\cos{t}$,\, $y = b\sin{t}$.

\begin{center}
\begin{pspicture}(-3.2,-2.5)(3.5,3.5)
\psaxes[Dx=9,Dy=9]{->}(0,0)(-3.5,-3.2)(3.5,3.5)
\psellipse[linecolor=blue](0,0)(3,2)
\pscircle[linecolor=red](0,0){3}
\rput[b](3.5,-0.3){$x$}
\rput[r](-0.2,3.5){$y$}
\rput(-0.2,2.77){$a$}
\rput(-0.2,1.75){$b$}
\rput(2.8,-0.2){$a$}
\psline[linecolor=green](2,0)(2,2.236)
\psline[linecolor=green](0,0)(2,2.236)
\psarc[linecolor=green](0,0){0.5}{0}{48.19}
\psdot(2,2.20)
\psdot(2,1.49)
\rput(1.8,1.3){$P$}
\rput[b](0.6,0.27){$t$}
\end{pspicture}
\end{center}

Squeezing $\mathbb{R}^3$ in the directions of the $y$-axis and $z$-axis one can get from the sphere
                     $$x^2\!+\!y^2\!+\!z^2 = a^2$$
the ellipsoid
               $$\frac{x^2}{a^2}+\frac{y^2}{b^2}+\frac{z^2}{c^2} = 1.$$


%%%%%
%%%%%
\end{document}
