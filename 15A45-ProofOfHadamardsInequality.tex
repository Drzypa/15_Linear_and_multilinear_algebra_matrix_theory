\documentclass[12pt]{article}
\usepackage{pmmeta}
\pmcanonicalname{ProofOfHadamardsInequality}
\pmcreated{2013-03-22 15:37:02}
\pmmodified{2013-03-22 15:37:02}
\pmowner{Andrea Ambrosio}{7332}
\pmmodifier{Andrea Ambrosio}{7332}
\pmtitle{proof of Hadamard's inequality}
\pmrecord{17}{37541}
\pmprivacy{1}
\pmauthor{Andrea Ambrosio}{7332}
\pmtype{Proof}
\pmcomment{trigger rebuild}
\pmclassification{msc}{15A45}

% this is the default PlanetMath preamble.  as your knowledge
% of TeX increases, you will probably want to edit this, but
% it should be fine as is for beginners.

% almost certainly you want these
\usepackage{amssymb}
\usepackage{amsmath}
\usepackage{amsfonts}

% used for TeXing text within eps files
%\usepackage{psfrag}
% need this for including graphics (\includegraphics)
%\usepackage{graphicx}
% for neatly defining theorems and propositions
%\usepackage{amsthm}
% making logically defined graphics
%%%\usepackage{xypic}

% there are many more packages, add them here as you need them

% define commands here
\begin{document}
Let's first prove the second inequality. If $A$ is singular, the thesis is
trivially verified, since for a Hermitian positive semidefinite matrix the
right-hand side is always nonnegative, all the diagonal entries being nonnegative. Let's thus assume $\det (A)\neq 0$, which means, $A$ being
Hermitian positive semidefinite, $\det (A)>0$. Then no diagonal entry of $A$
can be $0$ (otherwise, since for a Hermitian positive semidefinite matrix, $0\leq \lambda_{min}\leq a_{ii}\leq \lambda _{max}$, $\lambda_{min}$ and $\lambda_{max}$ being respectively the minimal and the maximal eigenvalue, this would imply $\lambda_{min}=0$, that is $A$ is singular); for this reason we can define $D=diag(d_{11},d_{22},\ldots ,d_{nn})$, with $d_{ii}=a_{ii}^{-\frac{1}{2}}\in \mathbb{R}$, since all $a_{ii}\in \mathbb{R}^{+}$. Let's furthermore define $B=DAD$. It's easy to check that $B$ too is Hermitian positive semidefinite, so its eigenvalues $\lambda_{B}$ are all non-negative (actually, since $A=A^{H}$ and since $D$ is real and diagonal, $B^{H}=(DAD)^{H}=D^{H}(DA)^{H}=D^{H}A^{H}D^{H}=DAD=B$; on the other hand, for any $\mathbf{x}\neq \mathbf{0}$, $\mathbf{x}^{H}B\mathbf{x}=\mathbf{x}^{H}DAD\mathbf{x}=(\mathbf{x}^{H}D)A(D\mathbf{x})=(D^{H}\mathbf{x})^{H}A(D\mathbf{x})=(D\mathbf{x})^{H}A(D\mathbf{x})=\mathbf{y}^{H}A\mathbf{y}\geq 0$).
Moreover, we have obviously $b_{ii}=d_{ii}a_{ii}d_{ii}=1$ so that $tr(B)=n$
and, recalling the \PMlinkname{geometric-arithmetic mean inequality}{ArithmeticGeometricMeansInequality}, which holds in this
case because the eigenvalues of $B$ are all non-negative,

$\det (B)=\prod_{i=1}^{n}\lambda _{B}^{(i)}\leq \left( \frac{1}{n}%
\sum_{i=1}^{n}\lambda _{B}^{(i)}\right) ^{n}=\left( \frac{1}{n}tr(B\right)
)^{n}=1$,

and since $\det (B)=\det (D)^{2}\det (A)=\left(\prod_{i=1}^{n}a_{ii}\right) ^{-1}\det (A)$, we have the thesis.
Since $\det (A)=\prod_{i=1}^{n}a_{ii}$ if and only if $\det
(B)=1$ and since in the geometric-arithmetic inequality equality holds if
and only if all terms are equal, we must have $\lambda _{B}^{(i)}=\lambda
_{B}$, so that $\prod_{i=1}^{n}\lambda _{B}^{(i)}=\lambda _{B}^{n}=1$,
whence $\lambda _{B}=1$ ($\lambda _{B}$ having to be non-negative), and since 
$B$ is Hermitian and hence is diagonalizable, we obtain $B=I$, and so $%
A=D^{-1}BD^{-1}=D^{-2}=diag(a_{11},a_{22},\ldots ,a_{nn})$. So we can
conclude that equality holds if and only if $A$ is diagonal.

Let's now derive the more general first inequality.
Let $A$ be a complex-valued $n\times n$ matrix. If $A$ is singular,
the thesis is trivially verified. Let's thus assume $\det (A)\neq 0$; then $B=AA^{H}$ is a Hermitian positive semidefinite matrix (actually, $B^{H}=(AA^{H})^{H}=(A^{H})^{H}A^{H}=AA^{H}=B$ and, for any $\mathbf{x}\neq\mathbf{0}$,$\mathbf{x}^{H}B\mathbf{x}=\mathbf{x}^{H}AA^{H}\mathbf{x}=(\mathbf{x}^{H}A)(A^{H}\mathbf{x})=(A^{H}\mathbf{x})^{H}(A^{H}\mathbf{x})=\mathbf{y}^{H}\mathbf{y}=\Vert \mathbf{y}\Vert _{2}^{2}\geq 0$). Therefore, the second inequality can be applied to $B$, yielding:

$\left\vert \det (A)\right\vert ^{2}=\det (A)\det^{\ast }(A)=\det (A)\det
(A^{H})=\det (AA^{H})=$\\
$=\det (B)\leq\prod_{i=1}^{n}b_{ii}=\prod_{i=1}^{n}\sum_{j=1}^{n}a_{ij}(a^{H})_{ji}=\prod_{i=1}^{n}\sum_{j=1}^{n}a_{ij}a_{ij}^{\ast
}=\prod_{i=1}^{n}\sum_{j=1}^{n}\left\vert a_{ij}\right\vert ^{2}$.

As we proved above, for $\det (B)$ to be equal to $\prod_{i=1}^{n}b_{ii}$, $B$ must be diagonal, which means that $\sum_{k=1}^{n}a_{ik}a_{jk}^{\ast }=|a_{ii}|^{2}\delta _{ij}$. So we can conclude that equality holds if and only if the rows of $A$ are orthogonal.
$\square $

\begin{thebibliography}{6}
\bibitem{Horn} R. A. Horn, C. R. Johnson,
\emph{Matrix Analysis}, Cambridge University Press, 1985
\end{thebibliography}
%%%%%
%%%%%
\end{document}
