\documentclass[12pt]{article}
\usepackage{pmmeta}
\pmcanonicalname{ConformalPartitioning}
\pmcreated{2013-03-22 16:04:16}
\pmmodified{2013-03-22 16:04:16}
\pmowner{Mathprof}{13753}
\pmmodifier{Mathprof}{13753}
\pmtitle{conformal partitioning}
\pmrecord{9}{38127}
\pmprivacy{1}
\pmauthor{Mathprof}{13753}
\pmtype{Definition}
\pmcomment{trigger rebuild}
\pmclassification{msc}{15-00}
\pmdefines{block multiplication}

\endmetadata

% this is the default PlanetMath preamble.  as your knowledge
% of TeX increases, you will probably want to edit this, but
% it should be fine as is for beginners.

% almost certainly you want these
\usepackage{amssymb}
\usepackage{amsmath}
\usepackage{amsfonts}

% used for TeXing text within eps files
%\usepackage{psfrag}
% need this for including graphics (\includegraphics)
%\usepackage{graphicx}
% for neatly defining theorems and propositions
%\usepackage{amsthm}
% making logically defined graphics
%%%\usepackage{xypic}

% there are many more packages, add them here as you need them

% define commands here

\begin{document}
Let $R$ be a ring.
Let the matrices $A \in M_{m,n}(R)$ and $B \in M_{n,p}(R)$ be partitioned into 
submatrices $A^{i,j}$ and $B^{i,j}$ respectively as follows:

$$A=\begin{matrix} 
 \begin{matrix}
n_{1} & n_{2} & \quad \cdots & n_{h}  
\end{matrix}  & 
\begin{matrix}
\quad
\end{matrix} 
\\
\left[ \begin{matrix} 
\overbrace{A^{1,1}} & \overbrace{A^{1,2}} & \cdots & \overbrace{A^{1,h}} \\
A^{2,1} & A^{2,2} & \cdots & A^{2,h} \\
\vdots & \vdots & \ \ & \vdots \\
A^{g,1} & A^{g,2} & \cdots & A^{g,h} 
\end{matrix} \right] & 
\begin{matrix}
\}m_{1} \\
\}m_{2} \\
\vdots \\
\}m_{g} 
\end{matrix}
\end{matrix}
$$
where $A^{i,j}$ is $m_{i} \times n_{j}, \sum_{i=1}^{g} m_{i} = m$, 
$\sum_{j=1}^{h} n_{j} = n$; 
$$B=\begin{matrix} 
 \begin{matrix}
p_{1} & p_{2} & \quad \cdots & p_{k}  
\end{matrix}  & 
\begin{matrix}
\quad
\end{matrix} 
\\
\left[ \begin{matrix} 
\overbrace{B^{1,1}} & \overbrace{B^{1,2}} & \cdots & \overbrace{B^{1,k}} \\
B^{2,1} & B^{2,2} & \cdots & B^{2,k} \\
\vdots & \vdots & \ \ & \vdots \\
B^{h,1} & B^{h,2} & \cdots & B^{h,k} 
\end{matrix} \right] & 
\begin{matrix}
\}n_{1} \\
\}n_{2} \\
\vdots \\
\}n_{h} 
\end{matrix}
\end{matrix}
$$
where $B^{i,j}$ is $n_{i} \times p_{j}, $ 
$\sum_{j=1}^{k} p_{j} = p$. Then $A$ and $B$ (in this \PMlinkescapetext{order}) are said to be
{\it conformally partitioned} for multiplication.  

Now suppose that $A$ and $B$ are conformally partitioned for multiplication.
Let $C=AB$ be partitioned as follows:
$$C=\begin{matrix} 
 \begin{matrix}
p_{1} & p_{2} & \quad \cdots & p_{k}  
\end{matrix}  & 
\begin{matrix}
\quad
\end{matrix} 
\\
\left[ \begin{matrix} 
\overbrace{C^{1,1}} & \overbrace{C^{1,2}} & \cdots & \overbrace{C^{1,k}} \\
C^{2,1} & C^{2,2} & \cdots & C^{2,k} \\
\vdots & \vdots & \ \ & \vdots \\
C^{g,1} & C^{g,2} & \cdots & C^{g,k} 
\end{matrix} \right] & 
\begin{matrix}
\}m_{1} \\
\}m_{2} \\
\vdots \\
\}m_{g} 
\end{matrix}
\end{matrix}
$$
where $C^{i,j}$ is $m_{i} \times p_{j}$, $i=1, \cdots ,g$, $j=1,\cdots ,k$.
Then
$$
C^{i,j} = \sum_{t=1}^{k} A^{i,t}B^{t,j}, \quad i=1,\cdots , g, \quad j=1, \cdots , k.
$$
This method of computing $AB$ is sometimes called {\it block multiplication}.


%%%%%
%%%%%
\end{document}
