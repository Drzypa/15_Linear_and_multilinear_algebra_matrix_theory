\documentclass[12pt]{article}
\usepackage{pmmeta}
\pmcanonicalname{CirculantMatrix}
\pmcreated{2013-03-22 13:53:38}
\pmmodified{2013-03-22 13:53:38}
\pmowner{bwebste}{988}
\pmmodifier{bwebste}{988}
\pmtitle{circulant matrix}
\pmrecord{9}{34640}
\pmprivacy{1}
\pmauthor{bwebste}{988}
\pmtype{Definition}
\pmcomment{trigger rebuild}
\pmclassification{msc}{15-01}
\pmclassification{msc}{15A99}

\usepackage{amssymb}
\usepackage{amsmath}
\usepackage{amsfonts}
\begin{document}
\PMlinkescapeword{term} \PMlinkescapeword{connection}
\PMlinkescapeword{simple} \PMlinkescapeword{images}
A square matrix $M:A\times A\to C$ is said to be $g$-\emph{circulant} for an integer $g$ if each row other than the first is obtained
from the preceding row by shifting the elements cyclically g columns to the right (g>0) or -g columns to the left (g < 0).

That is, if $A=[a_{ij}]$ then
$a_{i,j} = a_{i+1,j+g}$
where the subscripts are computed modulo d.
A 1-circulant is commonly called a circulant
and a -1-circulant is called a back circulant.

More explicitly, a matrix of the form
$$
\left[
\begin{array}{ccccc}
M_1 & M_2 & M_3 &\ldots & M_{d} \\
M_d & M_1 & M_2 & \ldots & M_{d-1}\\
M_{d-1} & M_d & M_1 & \ldots & M_{d-2}\\
\vdots & \vdots& \vdots & \ddots & \vdots \\
M_2 & M_3 & M_4 & \ldots & M_1
\end{array}\right]
$$
is called circulant. 

%The determinant of M' is
%$$\prod_{i=1}^n\sum_{j=1}^nM_j\omega_i^{j-1}$$
%where \omega is exp(2pi i/n).

Because the \PMlinkname{Jordan decomposition}{JordanCanonicalFormTheorem} of a
circulant matrix is rather simple, circulant matrices have some
interest in connection with the approximation of eigenvalues of
more general matrices. In particular, they have become part of the
standard apparatus in the computerized analysis of signals and images.
%%%%%
%%%%%
\end{document}
