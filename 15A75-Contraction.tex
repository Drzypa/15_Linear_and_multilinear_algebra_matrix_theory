\documentclass[12pt]{article}
\usepackage{pmmeta}
\pmcanonicalname{Contraction}
\pmcreated{2013-03-22 13:37:28}
\pmmodified{2013-03-22 13:37:28}
\pmowner{mathcam}{2727}
\pmmodifier{mathcam}{2727}
\pmtitle{contraction}
\pmrecord{4}{34261}
\pmprivacy{1}
\pmauthor{mathcam}{2727}
\pmtype{Definition}
\pmcomment{trigger rebuild}
\pmclassification{msc}{15A75}
\pmclassification{msc}{58A10}

% this is the default PlanetMath preamble.  as your knowledge
% of TeX increases, you will probably want to edit this, but
% it should be fine as is for beginners.

% almost certainly you want these
\usepackage{amssymb}
\usepackage{amsmath}
\usepackage{amsfonts}

% used for TeXing text within eps files
%\usepackage{psfrag}
% need this for including graphics (\includegraphics)
%\usepackage{graphicx}
% for neatly defining theorems and propositions
%\usepackage{amsthm}
% making logically defined graphics
%%%\usepackage{xypic}

% there are many more packages, add them here as you need them

% define commands here
\begin{document}
{\bf Definition} Let $\omega$ be a smooth $k$-form on a smooth manifold $M$,
and let $\xi$ be a smooth vector field on $M$. The \emph{contraction}
of $\omega$ with $\xi$ is the smooth $(k-1)$-form that maps $x\in M$ to
$\omega_x(\xi_x, \cdot)$.
In other words, $\omega$ is
point-wise evaluated with $\xi$ in the first slot.
We shall denote this $(k-1)$-form by  $\iota_\xi\omega$.
If $\omega$ is a $0$-form, we set $\iota_\xi \omega = 0$ for all $\xi$.

{\bf Properties} Let $\omega$ and $\xi$ be as above. Then
the following properties hold:
\begin{enumerate}
\item For any real number $k$
$$\iota_{k \xi} \omega = k\iota_\xi \omega.$$
\item For vector fields $\xi$ and $\eta$
\begin{eqnarray*}
\iota_{\xi+\eta} \omega &=& \iota_\xi \omega + \iota_\eta \omega, \\
\iota_{\xi} \iota_\eta \omega &=& -\iota_\eta \iota_\xi \omega, \\
\iota_{\xi} \iota_\xi \omega &=& 0.
\end{eqnarray*}
\item Contraction is an anti-derivation \cite{frankel}. If
$\omega^1$ is a $p$-form, and $\omega^2$ is a $q$-form, then
$$ \iota_\xi \big(\omega^1\wedge \omega^2\big) = (\iota_\xi \omega^1) \wedge \omega^2 + (-1)^p\ \omega^1\wedge (\iota_\xi \omega^2).$$
 \end{enumerate}

\begin{thebibliography}{9}
\bibitem {frankel} T. Frankel,
        \emph{Geometry of physics},
        Cambridge University press,
        1997.
\end{thebibliography}
%%%%%
%%%%%
\end{document}
