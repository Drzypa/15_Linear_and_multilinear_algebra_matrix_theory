\documentclass[12pt]{article}
\usepackage{pmmeta}
\pmcanonicalname{CyclicSubspace}
\pmcreated{2013-03-22 14:05:03}
\pmmodified{2013-03-22 14:05:03}
\pmowner{CWoo}{3771}
\pmmodifier{CWoo}{3771}
\pmtitle{cyclic subspace}
\pmrecord{12}{35447}
\pmprivacy{1}
\pmauthor{CWoo}{3771}
\pmtype{Definition}
\pmcomment{trigger rebuild}
\pmclassification{msc}{15A04}
\pmclassification{msc}{47A16}
\pmsynonym{cyclic vector subspace}{CyclicSubspace}
\pmrelated{CyclicDecompositionTheorem}
\pmrelated{CyclicVectorTheorem}
\pmdefines{cyclic vector}

% this is the default PlanetMath preamble.  as your knowledge
% of TeX increases, you will probably want to edit this, but
% it should be fine as is for beginners.

% almost certainly you want these
\usepackage{amssymb}
\usepackage{amsmath}
\usepackage{amsfonts}

% used for TeXing text within eps files
%\usepackage{psfrag}
% need this for including graphics (\includegraphics)
%\usepackage{graphicx}
% for neatly defining theorems and propositions
%\usepackage{amsthm}
% making logically defined graphics
%%%\usepackage{xypic}

% there are many more packages, add them here as you need them

% define commands here
\begin{document}
Let $V$ be a vector space over a field $k$, and $x \in V$. Let $T:V\to V$ be a linear transformation. The \emph{$T$-cyclic subspace generated by} $x$ is the smallest $T$-invariant subspace which contains $x$, and is denoted by $Z(x, T)$. 

Since $x,T(x),\ldots, T^n(x),\ldots \in Z(x,T)$, we have that $$W:=\operatorname{span}\lbrace x,T(x),\ldots,T^n(x),\ldots\rbrace \subseteq Z(x,T).$$  On the other hand, since $W$ is $T$-invariant, $Z(x,T)\subseteq W$.  Hence $Z(x,T)$ is the subspace generated by $x,T(x),\ldots, T^n(x),\ldots$  In other words, $Z(x,T)=\{p(T)(x) \mid p \in k[X]\}$.

\textbf{Remark}.  If $Z(x,T)=V$ we say that $x$ is a \emph{cyclic vector} of $T$.  
%%%%%
%%%%%
\end{document}
