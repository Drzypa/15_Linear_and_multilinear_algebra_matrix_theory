\documentclass[12pt]{article}
\usepackage{pmmeta}
\pmcanonicalname{ProofThatdetEAEoperatornametrA}
\pmcreated{2013-03-22 15:51:56}
\pmmodified{2013-03-22 15:51:56}
\pmowner{cvalente}{11260}
\pmmodifier{cvalente}{11260}
\pmtitle{proof that $\det e^A = e^{\operatorname{tr}A}$}
\pmrecord{7}{37858}
\pmprivacy{1}
\pmauthor{cvalente}{11260}
\pmtype{Proof}
\pmcomment{trigger rebuild}
\pmclassification{msc}{15-00}
\pmclassification{msc}{15A15}
%\pmkeywords{exponential}
%\pmkeywords{determinant}
%\pmkeywords{trace}
%\pmkeywords{Liouville}
\pmrelated{SchurDecomposition}

\endmetadata

% this is the default PlanetMath preamble.  as your knowledge
% of TeX increases, you will probably want to edit this, but
% it should be fine as is for beginners.

% almost certainly you want these
\usepackage{amssymb}
\usepackage{amsmath}
\usepackage{amsfonts}

% used for TeXing text within eps files
%\usepackage{psfrag}
% need this for including graphics (\includegraphics)
%\usepackage{graphicx}
% for neatly defining theorems and propositions
%\usepackage{amsthm}
% making logically defined graphics
%%%\usepackage{xypic}

% there are many more packages, add them here as you need them

% define commands here
\newcommand{\trace}{\operatorname{tr}}
\begin{document}
According to Schur decomposition the matrix $A$ can be written after a suitable change of basis as $A=D+N$ where $D$ is a diagonal matrix and $N$ is a strictly upper triangular matrix.

The formula we aim to prove

$$ \det e^A = e^{\trace A} $$

is invariant under a change of basis and thus we can carry out the computation of the exponential in any basis we choose.

By definition

\begin{equation}
\label{def}
e^A = \sum_{n=0}^{\infty} \frac{A^n}{n!}
\end{equation}

By the properties of diagonal and strictly upper triangular matrices we know that both $DN$ and $ND$ will also be strictly upper triangular matrices and so will their sum.

Thus the powers of $A$ are of the form:

\begin{eqnarray}
A   &=& (D+N) = D+N_1\\
A^2 &=& (D+N)(D+N) = D^2 + N_2 \\
A^3 &=& (D+N)(D^2 + N_2) = D^3 + N_3\\
    &\vdots& \\
A^k &=& D^k + N_k \\
    &\vdots& 
\end{eqnarray}

where all the $N_i$ matrices are strictly upper triangular.
Explicitly, $N_2 = DN_1+N_1D+N_1^2$ and by recursion $N_{n+1} = DN_n + N_n D + N_1 N_n$.

Using equation \ref{def} we can write

\begin{equation}
e^A = e^D + \tilde{N}
\end{equation}

where $\tilde{N} = \sum_{n=1}^{\infty}\frac{N_n}{n!}$ is strictly upper triangular and $e^D = \operatorname{diag}(e^{\lambda_1}, \cdots, e^{\lambda_n})$, where $D=\operatorname{diag}(\lambda_1, \cdots,\lambda_n)$.

$e^A$ will thus be an upper triangular matrix.
Since the determinant of an upper triangular matrix is just the product of the elements in its diagonal, we can write:

\begin{equation}
\label{result}
\det e^A = \prod_{i=1}^{n} e^{\lambda_i} = e^{\sum_{i=1}^n \lambda_i} = e^{\trace A}
\end{equation}
%%%%%
%%%%%
\end{document}
