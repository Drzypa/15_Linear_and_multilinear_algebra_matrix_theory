\documentclass[12pt]{article}
\usepackage{pmmeta}
\pmcanonicalname{Frame}
\pmcreated{2013-03-22 12:39:42}
\pmmodified{2013-03-22 12:39:42}
\pmowner{rmilson}{146}
\pmmodifier{rmilson}{146}
\pmtitle{frame}
\pmrecord{13}{32931}
\pmprivacy{1}
\pmauthor{rmilson}{146}
\pmtype{Definition}
\pmcomment{trigger rebuild}
\pmclassification{msc}{15A03}
\pmrelated{Vector2}
\pmrelated{TensorArray}
\pmrelated{BasicTensor}
\pmdefines{coframe}
\pmdefines{component}
\pmdefines{coordinate}
\pmdefines{transition matrix}
\pmdefines{polyad}

\newcommand{\cL}{\mathcal{L}}
\newcommand{\cU}{\mathcal{U}}
\newcommand{\bF}{\mathbf{F}}
\newcommand{\bG}{\mathbf{G}}
\newcommand{\tmat}{\mathcal{M}}
\newcommand{\hbF}{\hat{\bF}}
\newcommand{\ba}{\mathbf{a}}
\newcommand{\bu}{\mathbf{u}}
\newcommand{\bx}{\mathbf{x}}
\newcommand{\by}{\mathbf{y}}
\newcommand{\bv}{\mathbf{v}}
\newcommand{\bzero}{\mathbf{0}}
\newcommand{\du}[2]{^{#2}_{\hphantom{#2}\!#1}}
\newcommand{\ud}[2]{_{#2}^{\hphantom{#2}\!#1}}
\newcommand{\supp}[1]{^{(#1)}}
\newcommand{\subp}[1]{_{(#1)}}
\newcommand{\kfield}{\mathbb{K}}

\usepackage{amsmath}
\usepackage{amsfonts}
\usepackage{amssymb}
\newcommand{\reals}{\mathbb{R}}
\newcommand{\natnums}{\mathbb{N}}
\newcommand{\cnums}{\mathbb{C}}
\newcommand{\znums}{\mathbb{Z}}
\newcommand{\lp}{\left(}
\newcommand{\rp}{\right)}
\newcommand{\lb}{\left[}
\newcommand{\rb}{\right]}
\newcommand{\supth}{^{\text{th}}}
\newtheorem{proposition}{Proposition}
\newtheorem{definition}[proposition]{Definition}
\newcommand{\nl}[1]{\PMlinkescapetext{{#1}}}
\newcommand{\pln}[2]{\PMlinkname{#1}{#2}}
\begin{document}
\paragraph{Introduction}
Frames and coframes are notions closely related to the notions of
basis and dual basis.  As such, frames and coframes are needed to
describe the connection between \PMlinkname{list vectors}{Vector2} and
the more general \PMlinkname{abstract vectors}{VectorSpace}.  


\paragraph{Frames and bases.}
Let $\cU$ be a finite-dimensional vector space over a field $\kfield$,
and let $I$ be a finite, totally ordered set of indices\footnote{It is
  advantageous to allow general indexing sets, because one can
  indicate the use of multiple frames of reference by employing
  multiple, disjoint sets of indices.}, e.g.  $(1,2,\ldots,n)$.  We
will call a mapping $\bF:I\rightarrow\cU$ a reference frame, or simply
a frame.  To put it plainly, $\bF$ is just a list of elements of $\cU$
with indices belong to $I$.  We will adopt a notation to reflect this
and write $\bF_i$ instead of $\bF(i)$.  Subscripts are used when
writing the frame elements because it is best to regard a frame as a
row-vector\footnote{It is customary to use superscripts for the
  components of a column vector, and subscripts for the components of
  a row vector.  This is fully described in the
\PMlinkname{vector entry}{Vector2}.}  whose entries happen to be elements of $\cU$,
and write
$$\bF = (\bF_1,\ldots,\bF_n).$$
This is appropriate because
every reference frame $\bF$ naturally corresponds to a linear
mapping $\hbF:\kfield^I\rightarrow \cU$ defined by
$$
\ba \mapsto \sum_{i\in I} \ba^i\, \bF_i,\quad \ba\in \kfield^I.$$
In other words, $\hbF$ is a linear form on $\kfield^I$ that takes
values in $\cU$ instead of $\kfield$.  We use row vectors to represent
linear forms, and that's why we write the frame as a row vector.

We call $\bF$ a coordinate frame (equivalently, a basis), if $\hbF$ is
an isomorphism of vector spaces.  Otherwise we call $\bF$ degenerate,
incomplete, or both, depending on whether $\hbF$ fails to be,
respectively, injective and surjective.

\paragraph{Coframes and coordinates.}
In cases where $\bF$ is a basis, the inverse isomorphism
$$\hbF^{-1}:\cU\rightarrow \kfield^I$$
is called the coordinate
mapping.  It is cumbersome to work with this inverse explicitly, and
instead we introduce linear forms $\bx^i\in \cU^*,\; i\in I$ defined
by
$$\bx^i:\bu\mapsto \hbF^{-1}(\bu)(i),\quad \bu\in \cU.$$
Each
$\bx^i,\; i\in I$ is called the $i\supth$ coordinate function
relative to $\bF$, or simply the $i\supth$
coordinate\footnote{Strictly speaking, we should be denote the coframe
  by $\bx_\bF$ and the coordinate functions by $\bx^i_\bF$ so as
  to reflect their dependence on the choice of reference frame.
  Historically, writers have been loath to do this, preferring a
  couple of different notational tricks to avoid ambiguity.  The
  cleanest approach is to use different symbols, e.g.  $\bx^i$ versus
  $\by^j$, to distinguish coordinates coming from different frames.
  Another approach is to use distinct indexing sets; in this way the
  indices themselves will indicate the choice of frame.  Say we have
  two frames $\bF: I\rightarrow \cU$ and $\bG:J\rightarrow \cU$ with
  $I$ and $J$ distinct finite sets.  We stipulate that the symbol $i$
  refers to elements of $I$ and that $j$ refers to elements of $J$,
  and write $\bx^i$ for coordinates relative to $\bF$ and $\bx^j$ for
  coordinates relative to $\bG$.  That's the way it was done in all
  the old-time geometry and physics papers, and is still largely the
  way physicists go about writing coordinates.  Be that as it may, the
  notation has its problems and is the subject of long-standing
  controversy, named by mathematicians the debauche of indices. The
  problem is that the notation employs the same symbol, namely $\bx$,
  to refer to two different objects, namely a map with domain $I$ and
  another map with domain $J$.  In practice, ambiguity is 
  avoided because the old-time notation {\em never} refers to the
  coframe (or indeed any tensor) without also writing the indices.
  This is the classical way of the dummy variable, a cousin to the
  $f(x)$ notation.  It creates some confusion for beginners, but with
  a little practice it's a perfectly serviceable and useful way to
  communicate.}.  In this way we obtain a mapping
$$\bx:I\rightarrow \cU^*,\quad i\mapsto \bx^i$$
called the {\em
  coordinate coframe} or simply a {\em coframe}.  The forms $\bx^i,\,i
\in I$ give a basis of $\cU^*$. It is the dual basis of $\bF_i,\,i \in
I$, i.e.
$$\bx^i(\bF_j) = \delta^i_j,\quad i,j\in I,$$
where $\delta^i_j$ is the well-known Kronecker symbol.

In full duality to the custom of writing frames as row-vectors, we
write the coframe as a column vector whose components are the
coordinate functions:
$$
\begin{pmatrix}
  \bx^1\\ \bx^2\\ \vdots \\ \bx^n
\end{pmatrix}.
$$
We identify of $\hbF^{-1}$ and $\bx$ with the above column-vector.
This is quite natural because all of these objects are in natural
correspondence with a $\kfield$-valued functions of two arguments,
$$\cU\times I \rightarrow \kfield,$$
that maps an abstract vector
$\bu\in \cU$ and an index $i\in I$ to a scalar $\bx^i(\bu)$, called
the $i\supth$ component of $\bu$ relative to the reference frame
$\bF$.
\paragraph{Change of frame.}
Given two coordinate frames $\bF:I\rightarrow \cU$ and
$\bG:J\rightarrow \cU$, one can easily show that $I$ and $J$ must
have the same cardinality.  Letting $\bx^i,\,i\in I$ and $\by^j,\,j\in
J$ denote the coordinates functions relative to $\bF$ and $\bG$,
respectively, we define the transition matrix from $\bF$ to $\bG$ to
be the matrix  
$$\tmat:I\times J\rightarrow \kfield$$
with entries
$$\tmat^j_i = \by^j(\bF_i),\quad i\in I,\; j\in J.$$
An equivalent description of the transition matrix is given by
$$
\by^j = \sum_{i\in I} \tmat^j_i\, \bx^i,\quad \mbox{for all }j\in J.
$$


It is also the custom to regard the elements of $I$ as indexing the
columns of the matrix, while the elements of $J$ label the rows.
Thus, for $I=(1,2,\ldots,n)$ and
$J=(\bar{1},\bar{2},\ldots,\bar{n})$, we can write
$$
\begin{pmatrix}
  \tmat^{\bar{1}}_1 & \ldots &\tmat^{\bar{1}}_n \\
  \vdots & \ddots & \vdots \\
  \tmat^{\bar{n}}_1 & \ldots &\tmat^{\bar{n}}_n
\end{pmatrix} = 
\begin{pmatrix}
  \by^{\bar{1}} \\ \vdots \\ \by^{\bar{n}}
\end{pmatrix}
\begin{pmatrix}
  \bF_1 & \ldots & \bF_n
\end{pmatrix}.
$$
In this way we can describe the relation between coordinates
relative to the two frames in terms of ordinary matrix multiplication.
To wit, we can write
$$
\begin{pmatrix}
  \by^{\bar{1}} \\ \vdots \\ \by^{\bar{n}}
\end{pmatrix}
=
\begin{pmatrix}
  \tmat^{\bar{1}}_1 & \ldots &\tmat^{\bar{1}}_n \\
  \vdots & \ddots & \vdots \\
  \tmat^{\bar{n}}_1 & \ldots &\tmat^{\bar{n}}_n
\end{pmatrix}
\begin{pmatrix}
  \bx^1 \\ \vdots \\ \bx^n
\end{pmatrix}
$$

\paragraph{Notes.}
The term frame is often used to refer to objects that
should properly be called a {\em moving frame}.  The latter can be
thought of as a \PMlinkescapetext{field of frames}, or functions taking
values in the space of all frames, and are fully described elsewhere.
The confusion in terminology is unfortunate but quite common, and is
related to the questionable practice of using the word {\em scalar}
when referring to a {\em scalar field} (a.k.a. scalar-valued
functions) and using the word {\em vector} when referring to a {\em
  vector field.}

We also mention that in the world of theoretical physics, the
preferred terminology seems to be {\em polyad} and related
specializations, rather than frame.  Most commonly used are {\em
  dyad}, for a frame of two elements, and {\em tetrad} for a frame of
four elements.
%%%%%
%%%%%
\end{document}
