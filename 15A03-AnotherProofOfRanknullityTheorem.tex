\documentclass[12pt]{article}
\usepackage{pmmeta}
\pmcanonicalname{AnotherProofOfRanknullityTheorem}
\pmcreated{2013-03-22 18:06:14}
\pmmodified{2013-03-22 18:06:14}
\pmowner{CWoo}{3771}
\pmmodifier{CWoo}{3771}
\pmtitle{another proof of rank-nullity theorem}
\pmrecord{4}{40647}
\pmprivacy{1}
\pmauthor{CWoo}{3771}
\pmtype{Proof}
\pmcomment{trigger rebuild}
\pmclassification{msc}{15A03}
\pmrelated{ProofOfRankNullityTheorem}

\usepackage{amssymb,amscd}
\usepackage{amsmath}
\usepackage{amsfonts}
\usepackage{mathrsfs}

% used for TeXing text within eps files
%\usepackage{psfrag}
% need this for including graphics (\includegraphics)
%\usepackage{graphicx}
% for neatly defining theorems and propositions
\usepackage{amsthm}
% making logically defined graphics
%%\usepackage{xypic}
\usepackage{pst-plot}

% define commands here
\newcommand*{\abs}[1]{\left\lvert #1\right\rvert}
\newtheorem{prop}{Proposition}
\newtheorem{thm}{Theorem}
\newtheorem{ex}{Example}
\newcommand{\real}{\mathbb{R}}
\newcommand{\pdiff}[2]{\frac{\partial #1}{\partial #2}}
\newcommand{\mpdiff}[3]{\frac{\partial^#1 #2}{\partial #3^#1}}
\def\dim{\operatorname{dim}}
\def\rank{\operatorname{rank}}
\def\ker{\operatorname{ker}}
\def\nullity{\operatorname{nullity}}
\begin{document}
Let $\phi: V\to W$ be a linear transformation from vector spaces $V$ to $W$.  Recall that the rank of $\phi$ is the dimension of the image of $\phi$ and the nullity of $\phi$ is the dimension of the kernel of $\phi$.

\begin{prop} $\dim(V)=\rank(\phi)+\nullity(\phi)$. \end{prop}

\begin{proof}  Let $K=\ker(\phi)$.  $K$ is a subspace of $V$ so it has a unique algebraic complement $L$ such that $V=K\oplus L$.  It is evident that $$\dim(V)=\dim(K)+\dim(L)$$ since $K$ and $L$ have disjoint bases and the union of their bases is a basis for $V$.  

Define $\phi': L\to \phi(V)$ by restriction of $\phi$ to the subspace $L$.  $\phi'$ is obviously a linear transformation.  If $\phi'(v)=0$, then $\phi(v)=\phi'(v)=0$ so that $v\in K$.  Since $v\in L$ as well, we have $v\in K\cap L=\lbrace 0\rbrace$, or $v=0$.  This means that $\phi'$ is one-to-one.  Next, pick any $w\in \phi(V)$.  So there is some $v\in V$ with $\phi(v)=w$.  Write $v=x+y$ with $x\in K$ and $y\in L$.  So $\phi'(y)=\phi(y)=0+\phi(y) =\phi(x)+ \phi(y) =\phi(v) =w$, and therefore $\phi'$ is onto.  This means that $L$ is isomorphic to $\phi(V)$, which is equivalent to saying that $\dim(L)=\dim(\phi(V))=\rank(\phi)$.  Finally, we have $$\dim(V)=\dim(K)+\dim(L)=\nullity(\phi)+\rank(\phi).$$
\end{proof}

\textbf{Remark}.  The dimension of $V$ is not assumed to be finite in this proof.  For another approach (where finite dimensionality of $V$ is assumed), please see \PMlinkname{this entry}{ProofOfRankNullityTheorem}.
%%%%%
%%%%%
\end{document}
