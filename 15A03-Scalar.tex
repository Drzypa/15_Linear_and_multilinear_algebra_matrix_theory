\documentclass[12pt]{article}
\usepackage{pmmeta}
\pmcanonicalname{Scalar}
\pmcreated{2013-03-22 11:58:55}
\pmmodified{2013-03-22 11:58:55}
\pmowner{mathcam}{2727}
\pmmodifier{mathcam}{2727}
\pmtitle{scalar}
\pmrecord{8}{30870}
\pmprivacy{1}
\pmauthor{mathcam}{2727}
\pmtype{Definition}
\pmcomment{trigger rebuild}
\pmclassification{msc}{15A03}
%\pmkeywords{scalar}
\pmrelated{Vector}
\pmrelated{EigenvalueOfALinearOperator}
\pmrelated{AxialVector3}

\usepackage{amssymb}
\usepackage{amsmath}
\usepackage{amsfonts}
\usepackage{graphicx}
%%%\usepackage{xypic}
\begin{document}
A scalar is a quantity that is invariant under coordinate transformation, also known as a tensor of rank 0. For example, the number 1 is a scalar, so is any number or variable $n\in\mathbb{R}$. The point $(3,4)$ is not a scalar because it is variable under rotation.
As such, a scalar can be an element of a field over which a vector space is defined.
%%%%%
%%%%%
%%%%%
\end{document}
