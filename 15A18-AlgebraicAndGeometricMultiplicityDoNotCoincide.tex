\documentclass[12pt]{article}
\usepackage{pmmeta}
\pmcanonicalname{AlgebraicAndGeometricMultiplicityDoNotCoincide}
\pmcreated{2013-03-22 15:15:18}
\pmmodified{2013-03-22 15:15:18}
\pmowner{matte}{1858}
\pmmodifier{matte}{1858}
\pmtitle{algebraic and geometric multiplicity do not coincide}
\pmrecord{5}{37037}
\pmprivacy{1}
\pmauthor{matte}{1858}
\pmtype{Example}
\pmcomment{trigger rebuild}
\pmclassification{msc}{15A18}

\endmetadata

% this is the default PlanetMath preamble.  as your knowledge
% of TeX increases, you will probably want to edit this, but
% it should be fine as is for beginners.

% almost certainly you want these
\usepackage{amssymb}
\usepackage{amsmath}
\usepackage{amsfonts}
\usepackage{amsthm}

\usepackage{mathrsfs}

% used for TeXing text within eps files
%\usepackage{psfrag}
% need this for including graphics (\includegraphics)
%\usepackage{graphicx}
% for neatly defining theorems and propositions
%
% making logically defined graphics
%%%\usepackage{xypic}

% there are many more packages, add them here as you need them

% define commands here

\newcommand{\sR}[0]{\mathbb{R}}
\newcommand{\sC}[0]{\mathbb{C}}
\newcommand{\sN}[0]{\mathbb{N}}
\newcommand{\sZ}[0]{\mathbb{Z}}

 \usepackage{bbm}
 \newcommand{\Z}{\mathbbmss{Z}}
 \newcommand{\C}{\mathbbmss{C}}
 \newcommand{\F}{\mathbbmss{F}}
 \newcommand{\R}{\mathbbmss{R}}
 \newcommand{\Q}{\mathbbmss{Q}}



\newcommand*{\norm}[1]{\lVert #1 \rVert}
\newcommand*{\abs}[1]{| #1 |}



\newtheorem{thm}{Theorem}
\newtheorem{defn}{Definition}
\newtheorem{prop}{Proposition}
\newtheorem{lemma}{Lemma}
\newtheorem{cor}{Corollary}
\begin{document}
Zero is an eigenvalue of
$$
  A = \begin{pmatrix} 0 & 1 \\ 0 & 0 \end{pmatrix}
$$
with algebraic multiplicity $2$ and geometric multiplicity $1$. 

Indeed, as
$$
   \det (A-\lambda I) = \lambda^2
$$
it follows that $0\,\!$ 
is an eigenvalue of $A$ with algebraic multiplicity $2$. 
To find the geometric multiplicity of $A$ we need to calculate 
$\operatorname{ker} A$. Thus, suppose 
$$
\begin{pmatrix} 0 & 1 \\ 0 & 0 \end{pmatrix} \begin{pmatrix} a \\ b \end{pmatrix}= \begin{pmatrix} 0  \\ 0 \end{pmatrix}.
$$
This implies $b=0$, so 
$$
   \ker A = \operatorname{span} \begin{pmatrix} 1  \\ 0 \end{pmatrix},
$$
and the geometric multiplicity of $0\,\!$ is $1$.
%%%%%
%%%%%
\end{document}
