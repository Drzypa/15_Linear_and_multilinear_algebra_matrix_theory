\documentclass[12pt]{article}
\usepackage{pmmeta}
\pmcanonicalname{Point}
\pmcreated{2013-03-22 16:06:30}
\pmmodified{2013-03-22 16:06:30}
\pmowner{Wkbj79}{1863}
\pmmodifier{Wkbj79}{1863}
\pmtitle{point}
\pmrecord{16}{38173}
\pmprivacy{1}
\pmauthor{Wkbj79}{1863}
\pmtype{Definition}
\pmcomment{trigger rebuild}
\pmclassification{msc}{15-00}
\pmclassification{msc}{54-00}
\pmclassification{msc}{51-00}

\endmetadata

% this is the default PlanetMath preamble.  as your knowledge
% of TeX increases, you will probably want to edit this, but
% it should be fine as is for beginners.

% almost certainly you want these
\usepackage{amssymb}
\usepackage{amsmath}
\usepackage{amsfonts}

% used for TeXing text within eps files
%\usepackage{psfrag}
% need this for including graphics (\includegraphics)
%\usepackage{graphicx}
% for neatly defining theorems and propositions
%\usepackage{amsthm}
% making logically defined graphics
%%%\usepackage{xypic}

% there are many more packages, add them here as you need them

% define commands here

\begin{document}
In {\sl The \PMlinkescapetext{Elements}\/}, Euclid defines a point as that which has no part.

In a vector space, an affine space, or, more generally, an incidence geometry, a {\sl point\/} is a \PMlinkname{zero}{Zero} \PMlinkname{dimensional}{Dimension3} \PMlinkescapetext{object}.

In a projective geometry, a {\sl point\/} is a one-dimensional subspace of the vector space underlying the projective geometry.

In a topology, a {\sl point\/} is an element of a topological space.

In function theory, a {\sl point\/} usually means a complex number as an element of the complex plane.

Note that there is also the possibility for a point-free approach to geometry in which points are not assumed as a primitive. Instead, points are defined by suitable abstraction processes.  (See point-free geometry.)



%%%%%
%%%%%
\end{document}
