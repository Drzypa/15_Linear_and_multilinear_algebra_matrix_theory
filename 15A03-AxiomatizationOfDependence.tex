\documentclass[12pt]{article}
\usepackage{pmmeta}
\pmcanonicalname{AxiomatizationOfDependence}
\pmcreated{2013-03-22 16:27:46}
\pmmodified{2013-03-22 16:27:46}
\pmowner{rspuzio}{6075}
\pmmodifier{rspuzio}{6075}
\pmtitle{axiomatization of dependence}
\pmrecord{8}{38621}
\pmprivacy{1}
\pmauthor{rspuzio}{6075}
\pmtype{Definition}
\pmcomment{trigger rebuild}
\pmclassification{msc}{15A03}
\pmrelated{DependenceRelation}

\endmetadata

% this is the default PlanetMath preamble.  as your knowledge
% of TeX increases, you will probably want to edit this, but
% it should be fine as is for beginners.

% almost certainly you want these
\usepackage{amssymb}
\usepackage{amsmath}
\usepackage{amsfonts}

% used for TeXing text within eps files
%\usepackage{psfrag}
% need this for including graphics (\includegraphics)
%\usepackage{graphicx}
% for neatly defining theorems and propositions
%\usepackage{amsthm}
% making logically defined graphics
%%%\usepackage{xypic}

% there are many more packages, add them here as you need them

% define commands here

\newtheorem{axiom}{Axiom}
\begin{document}
As noted by van der Waerden, it is possible to define the notion of dependence axiomatically in such a way that one can deal with linear dependence, algebraic dependence, and other sorts of dependence via a general theory.  In this general theoretical framework, one can prove results about bases, dimension, and the like.

Let $S$ be a set.  The basic object of this theory is a relation $D$ between $S$ and the power set of $S$.  This relation satisfies the following three axioms:

\begin{axiom} 
If $Y$ is a subset of $S$ and $x \in Y$, then $D(x,Y)$.
\end{axiom}

\begin{axiom}
If, for some set $X \subseteq S$ and some $y,z \in S$, it happens that $D(y,X \cup \{z\})$ but not $D(y,X)$, then $D(z, X \cup \{y\})$.
\end{axiom}

\begin{axiom}
If,  for some sets $Y,Z \subseteq S$ and some $x \in S$, it happens that $D(x,Y)$ and, for every $y \in Y$, it is the case that $D(y,Z)$, then $D(x,Z)$.
\end{axiom}
%%%%%
%%%%%
\end{document}
