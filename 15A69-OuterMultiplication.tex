\documentclass[12pt]{article}
\usepackage{pmmeta}
\pmcanonicalname{OuterMultiplication}
\pmcreated{2013-03-22 12:40:31}
\pmmodified{2013-03-22 12:40:31}
\pmowner{rmilson}{146}
\pmmodifier{rmilson}{146}
\pmtitle{outer multiplication}
\pmrecord{4}{32951}
\pmprivacy{1}
\pmauthor{rmilson}{146}
\pmtype{Definition}
\pmcomment{trigger rebuild}
\pmclassification{msc}{15A69}
\pmrelated{TensorProductClassical}
\pmrelated{TensorProduct}

\endmetadata

\usepackage{amsmath}
\usepackage{amsfonts}
\usepackage{amssymb}

\newcommand{\reals}{\mathbb{R}}
\newcommand{\natnums}{\mathbb{N}}
\newcommand{\cnums}{\mathbb{C}}
\newcommand{\znums}{\mathbb{Z}}

\newcommand{\lp}{\left(}
\newcommand{\rp}{\right)}
\newcommand{\lb}{\left[}
\newcommand{\rb}{\right]}

\newcommand{\supth}{^{\text{th}}}


\newtheorem{proposition}{Proposition}
\newtheorem{definition}[proposition]{Definition}

\newcommand{\kfield}{\mathbb{K}}
\newcommand{\rC}{\mathrm{C}}
\newcommand{\rT}{\mathrm{T}}
\newcommand{\tspace}[1]{\rT^{#1}}
\begin{document}
Note: the present entry employs the terminology and notation defined
and described in the entry on tensor arrays.  To keep things
reasonably self contained we mention that the symbol $\tspace{p,q}$ refers
to the vector space of type $(p,q)$ tensor arrays, i.e. maps
$$I^p\times I^q\rightarrow \kfield,$$
where $I$ is some finite list of
index labels, and where $\kfield$ is a field.


Let $p_1,p_2,q_1,q_2$ be natural numbers.  Outer multiplication is a
bilinear operation
$$\tspace{p_1,q_1} \times \tspace{p_2,q_2} \rightarrow
\tspace{p_1+p_2,q_1+q_2}$$
that combines a type $(p_1,q_1)$
tensor array $X$ and a type $(p_2,q_2)$ tensor array $Y$ to
produce a type $(p_1+p_2,q_1+q_2)$ tensor array $XY$ (also
written as $X\otimes Y$), defined by
$$
(XY)^{i_1\ldots i_{p_1} i_{p_1+1} \ldots i_{p_1+p_2}}_{j_1\ldots
    j_{q_1} j_{q_1+1} \ldots j_{q_1+q_2} }
=
X^{i_1\ldots i_{p_1}}_{j_1\ldots j_{q_1}}
Y^{i_{p_1+1}\ldots i_{p_1+p_2}}_{j_{q_1+1}\ldots j_{q_1+q_2}}
$$
Speaking informally, what is going on above is that we multiply
every value of the $X$ array by every possible value of the $Y$ array,
to create a new array, $XY$. Quite obviously then, the size of $XY$ is
the size of $X$ times the size of $Y$, and the index slots of the
product $XY$ are just the union of the index slots of $X$ and of $Y$.

Outer multiplication is a non-commutative, associative operation.  The
type $(0,0)$ arrays are the scalars, i.e. elements of
$\kfield$; they commute with everything. Thus, we can embed $\kfield$ into
the direct sum
$$\bigoplus_{p,q\in\natnums} \tspace{p,q},$$
and thereby endow the latter
with the structure of an $\kfield$-algebra\footnote{We will not pursue this
line of thought here, because the topic of algebra structure is best
dealt with in the a more abstract context.  The same comment applies
to the use of the tensor product sign $\otimes$ in denoting outer
multiplication.  These topics are dealt with in the entry pertaining
to abstract tensor algebra.}.

By way of illustration we mention that the outer product of a column
vector, i.e. a type $(1,0)$ array, and a row vector, i.e. a type
$(0,1)$ array, gives a matrix, i.e. a type $(1,1)$
tensor array.  For instance:
$$
\begin{pmatrix}
a \\ b \\ c
\end{pmatrix}\otimes
\begin{pmatrix}
x & y & z 
\end{pmatrix} 
= 
\begin{pmatrix}
ax & ay & az \\
bx & by & bz \\
cx & cy & cz
\end{pmatrix}
,\quad
a,b,c,x,y,z\in \kfield
$$
%%%%%
%%%%%
\end{document}
