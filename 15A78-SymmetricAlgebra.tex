\documentclass[12pt]{article}
\usepackage{pmmeta}
\pmcanonicalname{SymmetricAlgebra}
\pmcreated{2013-03-22 15:46:23}
\pmmodified{2013-03-22 15:46:23}
\pmowner{CWoo}{3771}
\pmmodifier{CWoo}{3771}
\pmtitle{symmetric algebra}
\pmrecord{4}{37730}
\pmprivacy{1}
\pmauthor{CWoo}{3771}
\pmtype{Definition}
\pmcomment{trigger rebuild}
\pmclassification{msc}{15A78}

\usepackage{amssymb,amscd}
\usepackage{amsmath}
\usepackage{amsfonts}

% used for TeXing text within eps files
%\usepackage{psfrag}
% need this for including graphics (\includegraphics)
%\usepackage{graphicx}
% for neatly defining theorems and propositions
%\usepackage{amsthm}
% making logically defined graphics
%%%\usepackage{xypic}

% define commands here
\begin{document}
Let $M$ be a module over a commutative ring $R$.  Form the tensor algebra $T(M)$ over $R$.  Let $I$ be the ideal of $T(M)$ generated by elements of the form
$$u\otimes v-v\otimes u$$
where $u,v\in M$.
Then the quotient algebra defined by $$S(M):=T(M)/I$$
is called the \emph{symmetric algebra} over the ring $R$.

\textbf{Remark.}  Let $R$ be a field, and $M$ a finite dimensional vector space over $R$.  Suppose $\lbrace e_1,e_2,\ldots,e_n\rbrace$ is a basis of $M$ over $R$.  Then $T(M)$ is nothing more than a free algebra on the basis elements $e_i$.  Alternatively, the basis elements $e_i$ can be viewed as non-commuting indeterminates in the non-commutative polynomial ring $R\langle e_1,e_2,\ldots,e_n \rangle$.  This then implies that $S(M)$ is isomorphic to the ``commutative'' polynomial ring $R[e_1,e_2,\ldots,e_n]$, where $e_ie_j=e_je_i$.
%%%%%
%%%%%
\end{document}
