\documentclass[12pt]{article}
\usepackage{pmmeta}
\pmcanonicalname{MoorePenroseGeneralizedInverse}
\pmcreated{2013-03-22 14:31:31}
\pmmodified{2013-03-22 14:31:31}
\pmowner{CWoo}{3771}
\pmmodifier{CWoo}{3771}
\pmtitle{Moore-Penrose generalized inverse}
\pmrecord{8}{36067}
\pmprivacy{1}
\pmauthor{CWoo}{3771}
\pmtype{Definition}
\pmcomment{trigger rebuild}
\pmclassification{msc}{15A09}
\pmclassification{msc}{60J10}
\pmsynonym{Moore-Penrose pseudoinverse}{MoorePenroseGeneralizedInverse}
\pmrelated{DrazinInverse}
\pmrelated{Pseudoinverse}

% this is the default PlanetMath preamble.  as your knowledge
% of TeX increases, you will probably want to edit this, but
% it should be fine as is for beginners.

% almost certainly you want these
\usepackage{amssymb,amscd}
\usepackage{amsmath}
\usepackage{amsfonts}

% used for TeXing text within eps files
%\usepackage{psfrag}
% need this for including graphics (\includegraphics)
%\usepackage{graphicx}
% for neatly defining theorems and propositions
%\usepackage{amsthm}
% making logically defined graphics
%%%\usepackage{xypic}

% there are many more packages, add them here as you need them

% define commands here
\begin{document}
Let $A$ be an $m\times n$ matrix with entries in $\mathbb{C}$. The \emph{Moore-Penrose generalized inverse}, denoted by $A^{\dagger}$, is an $n\times m$ matrix with entries in $\mathbb{C}$, such that
\begin{enumerate}
\item $AA^{\dagger}A=A$
\item $A^{\dagger}AA^{\dagger}=A^{\dagger}$
\item $AA^{\dagger}$ and $A^{\dagger}A$ are both Hermitian
\end{enumerate}

\textbf{Remarks}
\begin{itemize}
\item The Moore-Penrose generalized inverse of a given matrix is unique.
\item If $A^{\dagger}$ is the Moore-Penrose generalized inverse of $A$, then $(A^{\dagger})^{\operatorname{T}}$ is the Moore-Penrose generalized inverse of $A^{\operatorname{T}}$.
\item If $A=BC$ such that
\begin{enumerate}
\item $A\in\mathbb{C}^{m\times n}$, $B\in\mathbb{C}^{m\times r}$, and $C\in\mathbb{C}^{r\times n}$,
\item $r=\operatorname{rank}(A)=\operatorname{rank}(B)=\operatorname{rank}(C)$, then $$A^{\dagger}=C^{\ast}(CC^{\ast})^{-1}(B^{\ast}B)^{-1}B^{\ast}.$$
\end{enumerate}
\end{itemize}

For example, let
$$A=\begin{pmatrix} 1&1&i \\ 0&1&0 \end{pmatrix}.$$
Transform $A$ to its row echelon form to get a decomposition of $A=BC$, where $$B=\begin{pmatrix} 1&1 \\ 0&1 \end{pmatrix}\mbox{ and }C=\begin{pmatrix} 1&0&i \\ 0&1&0 \end{pmatrix}.$$
It is readily verified that $2=\operatorname{rank}(A)=\operatorname{rank}(B)=\operatorname{rank}(C)$.  
So $$A^{\dagger}=\frac{1}{2}\begin{pmatrix} 1&-1 \\ 0&2 \\ -i&i \end{pmatrix}.$$
We check that $$AA^{\dagger}=I\mbox{ and }A^{\dagger}A=\frac{1}{2}\begin{pmatrix} 1&0&i \\ 0&2&0 \\ -i&0&1 \end{pmatrix}$$ are both Hermitian.  Furthermore, $AA^{\dagger}A=A$ and $A^{\dagger}AA^{\dagger}=A^{\dagger}$.  So, $A^{\dagger}$ is the Moore-Penrose generalized inverse of $A$.
%%%%%
%%%%%
\end{document}
