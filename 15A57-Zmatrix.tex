\documentclass[12pt]{article}
\usepackage{pmmeta}
\pmcanonicalname{Zmatrix}
\pmcreated{2013-03-22 15:24:51}
\pmmodified{2013-03-22 15:24:51}
\pmowner{kshum}{5987}
\pmmodifier{kshum}{5987}
\pmtitle{Z-matrix}
\pmrecord{11}{37255}
\pmprivacy{1}
\pmauthor{kshum}{5987}
\pmtype{Definition}
\pmcomment{trigger rebuild}
\pmclassification{msc}{15A57}

% this is the default PlanetMath preamble.  as your knowledge
% of TeX increases, you will probably want to edit this, but
% it should be fine as is for beginners.

% almost certainly you want these
\usepackage{amssymb}
\usepackage{amsmath}
\usepackage{amsfonts}

% used for TeXing text within eps files
%\usepackage{psfrag}
% need this for including graphics (\includegraphics)
%\usepackage{graphicx}
% for neatly defining theorems and propositions
%\usepackage{amsthm}
% making logically defined graphics
%%%\usepackage{xypic}

% there are many more packages, add them here as you need them

% define commands here
\begin{document}
A square matrix is called a \emph{$Z$-matrix} if all off-diagonal entries are less than or equal to zero.


Z-matrix arises naturally in solving Dirichlet problem
numerically, and linear modeling of economy. 


{\bf Reference:}

M. Fiedler, {\it Special matrices and their applications in
numerical mathematics,} 1986, Martinus Nijhoff Publishers,
Dordrecht.

R. A. Horn and C. R. Johnson, {\it Topics in matrix analysis,}
1991, Cambridge University Press, United Kingdom.

R. B. Bapat and T. E. S. Raghavan, {\it Nonnegative matrices and
applications,} 1997, Cambridge University Press, United Kingdom.
%%%%%
%%%%%
\end{document}
