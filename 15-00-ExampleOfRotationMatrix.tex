\documentclass[12pt]{article}
\usepackage{pmmeta}
\pmcanonicalname{ExampleOfRotationMatrix}
\pmcreated{2013-03-22 15:09:16}
\pmmodified{2013-03-22 15:09:16}
\pmowner{swiftset}{1337}
\pmmodifier{swiftset}{1337}
\pmtitle{example of rotation matrix}
\pmrecord{5}{36902}
\pmprivacy{1}
\pmauthor{swiftset}{1337}
\pmtype{Example}
\pmcomment{trigger rebuild}
\pmclassification{msc}{15-00}
\pmrelated{Slope}
\pmrelated{RotationMatrix}

\endmetadata

% this is the default PlanetMath preamble.  as your knowledge
% of TeX increases, you will probably want to edit this, but
% it should be fine as is for beginners.

% almost certainly you want these
\usepackage{amssymb}
\usepackage{amsmath}
\usepackage{amsfonts}

% used for TeXing text within eps files
%\usepackage{psfrag}
% need this for including graphics (\includegraphics)
%\usepackage{graphicx}
% for neatly defining theorems and propositions
%\usepackage{amsthm}
% making logically defined graphics
%%%\usepackage{xypic}

% there are many more packages, add them here as you need them

% define commands here
\newcommand{\R}{\ensuremath{\mathcal{R}}}
\begin{document}
You can use rotation matrices to show that if the slope of one line is $m$, then the slope of the line perpendicular to it is $\frac{-1}{m}$:

Let $L$ be a line with a slope of $m$ passing through the origin. The rotation matrix $R_{\frac{\pi}{2}}$ rotates $L$ into a line $L^\prime$ perpendicular to $L$:

$$ R_{\pi/2} = 
\begin{pmatrix}
0 & -1 \\
1 & 0 \\
\end{pmatrix}
$$

Every point on $L$ can be represented as a multiple of the point 
$ \vec{p} = \begin{pmatrix} 1 \\ m \end{pmatrix} $.

Notice $ \vec{p}^\prime = R_{\frac{\pi}{2}} \vec{p} = \begin{pmatrix} -m \\ 1 \end{pmatrix} $. Since every point on $L^\prime$ can be represented as a multiple of the point $\vec{p}^\prime$, the slope of $L^\prime$ is $\frac{-1}{m}$.
%%%%%
%%%%%
\end{document}
