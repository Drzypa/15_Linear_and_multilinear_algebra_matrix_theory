\documentclass[12pt]{article}
\usepackage{pmmeta}
\pmcanonicalname{Trace}
\pmcreated{2013-03-22 12:17:57}
\pmmodified{2013-03-22 12:17:57}
\pmowner{mhale}{572}
\pmmodifier{mhale}{572}
\pmtitle{trace}
\pmrecord{10}{31844}
\pmprivacy{1}
\pmauthor{mhale}{572}
\pmtype{Definition}
\pmcomment{trigger rebuild}
\pmclassification{msc}{15A15}
\pmclassification{msc}{15A04}
\pmrelated{FrobeniusMatrixNorm}

\endmetadata

% this is the default PlanetMath preamble.  as your knowledge
% of TeX increases, you will probably want to edit this, but
% it should be fine as is for beginners.

% almost certainly you want these
\usepackage{amssymb}
\usepackage{amsmath}
\usepackage{amsfonts}

% used for TeXing text within eps files
%\usepackage{psfrag}
% need this for including graphics (\includegraphics)
%\usepackage{graphicx}
% for neatly defining theorems and propositions
%\usepackage{amsthm}
% making logically defined graphics
%%%\usepackage{xypic} 

% there are many more packages, add them here as you need them

% define commands here
\begin{document}
The {\em trace} $\operatorname{Tr}(A)$ of a square matrix $A$ is defined to be the sum of the diagonal entries of $A$. It satisfies the following formulas:
\begin{itemize}
\item $\operatorname{Tr}(A+B) = \operatorname{Tr}(A) + \operatorname{Tr}(B)$
\item $\operatorname{Tr}(AB) = \operatorname{Tr}(BA)$\quad\quad (\PMlinkescapetext{cyclic property})
\end{itemize}
where $A$ and $B$ are square matrices of the same size.

The {\em trace} $\operatorname{Tr}(T)$ of a linear transformation $T\colon V \longrightarrow V$ from any finite dimensional vector space $V$ to itself is defined to be the trace of any matrix representation of $T$ with respect to a basis of $V$. This scalar is independent of the choice of basis of $V$, and in fact is equal to the sum of the eigenvalues of $T$ (over a splitting field of the characteristic polynomial), including multiplicities.

The following link presents some examples for calculating the trace of a matrix.

A {\em trace} on a $C^*$-algebra $A$ is a positive linear functional $\phi\colon A\to\mathbb{C}$ that has the \PMlinkescapetext{cyclic property}.
%%%%%
%%%%%
\end{document}
