\documentclass[12pt]{article}
\usepackage{pmmeta}
\pmcanonicalname{BasicTensor}
\pmcreated{2013-03-22 12:40:37}
\pmmodified{2013-03-22 12:40:37}
\pmowner{rmilson}{146}
\pmmodifier{rmilson}{146}
\pmtitle{basic tensor}
\pmrecord{7}{32953}
\pmprivacy{1}
\pmauthor{rmilson}{146}
\pmtype{Derivation}
\pmcomment{trigger rebuild}
\pmclassification{msc}{15A69}
\pmsynonym{characteristic array}{BasicTensor}
\pmrelated{TensorArray}
\pmrelated{Basis}
\pmrelated{Frame}
\pmrelated{SimpleTensor}

\newcommand{\kfield}{\mathbb{K}}
\newcommand{\rT}{\mathrm{T}}
\newcommand{\tspace}[1]{\rT^{#1}}
\newcommand{\ca}{\varepsilon}
\newcommand{\ud}[2]{^{#1}_{\!\hphantom{#1}#2}}

\usepackage{amsmath}
\usepackage{amsfonts}
\usepackage{amssymb}
\newcommand{\reals}{\mathbb{R}}
\newcommand{\natnums}{\mathbb{N}}
\newcommand{\cnums}{\mathbb{C}}
\newcommand{\znums}{\mathbb{Z}}
\newcommand{\lp}{\left(}
\newcommand{\rp}{\right)}
\newcommand{\lb}{\left[}
\newcommand{\rb}{\right]}
\newcommand{\supth}{^{\text{th}}}
\newtheorem{proposition}{Proposition}
\newtheorem{definition}[proposition]{Definition}
\newcommand{\nl}[1]{\PMlinkescapetext{{#1}}}
\newcommand{\pln}[2]{\PMlinkname{#1}{#2}}
\begin{document}
The present entry employs the terminology and notation defined and
described in the entry on tensor arrays.  To keep things reasonably
self-contained we mention that the symbol $\tspace{p,q}$ refers to the
vector space of type $(p,q)$ tensor arrays, i.e. maps $$I^p\times I^q\rightarrow \kfield,$$ where $I$ is some finite list of
index labels, and where $\kfield$ is a field.  

We say that a tensor array is a {\em characteristic array}, a.k.a. a
{\em basic tensor}, if all but one of its values are $0$, and the
remaining non-zero value is equal to $1$.  For tuples $A\in I^p$ and
$B\in I^q$, we let
$$\ca^B_A:I^p\times I^q\rightarrow\kfield,$$
denote the characteristic
array defined by
$$(\ca^B_A)^{i_1\ldots i_p}_{j_1\ldots j_q} = 
\left\{ 
  \begin{array}{rl}
    1 & \mbox{ if $(i_1,\ldots,i_p) = A$ and $(j_1,\ldots, j_p)=B$},\\
    0 & \mbox{ otherwise.}
  \end{array}
  \right.
$$
The type $(p,q)$ characteristic arrays form a natural basis for
$\tspace{p,q}$.  

Furthermore the outer multiplication of two characteristic arrays
gives a characteristic array of larger valence.  In other words, for
$$
A_1\in I^{p_1},\;
B_1\in I^{q_1},\;
A_2\in I^{p_2},\;
B_2\in I^{q_2},
$$
we have that
$$\ca^{B_1}_{A_1} \ca^{B_2}_{A_2} = \ca^{B_1 B_2}_{A_1 A_2},$$
where
the product on the left-hand side is performed by outer
multiplication, and where $A_1 A_2$ on the right-hand side refers to
the element of $I^{p_1+p_2}$ obtained by concatenating the tuples
$A_1$ and $A_2$, and similarly for $B_1 B_2$.

In this way we see that the type $(1,0)$ characteristic arrays
$\ca_{(i)},\; i\in I$ (the natural basis of $\kfield^I$), and the type
$(0,1)$ characteristic arrays $\ca^{(i)},\; i\in I$ (the natural basis of
$\lp\kfield^I\rp^*$) generate the tensor array algebra relative to the
outer multiplication operation.  

The just-mentioned fact gives us an alternate way of writing and
thinking about tensor arrays.  We introduce the basic symbols
$$\ca_{(i)},\; \ca^{(i)} ,\quad i\in I$$
subject to the commutation relations
$$\ca_{(i)} \ca^{(i')} = \ca^{(i')}\ca_{(i)} ,\quad i, i'\in I,$$
 add and multiply these symbols using coefficients in
$\kfield$, and use
$$\ca^{(i_1 \ldots i_q)}_{(j_1 \ldots j_p)},\quad
i_1,\ldots,i_q,j_1,\ldots,j_p\in I$$ 
as a handy abbreviation for
$$\ca^{(i_1)} \ldots \ca^{(i_q)} \ca_{(j_1)} \ldots \ca_{(j_p)}.$$
We
then interpret the resulting expressions as tensor arrays in the
obvious fashion: the values of the tensor array are just the
coefficients of the $\ca$ symbol matching the given index.  However,
note that in the $\ca$ symbols, the covariant data is written as a
superscript, and the contravariant data as a subscript.  This is done
to facilitate the Einstein summation convention.

By way of illustration, suppose that $I=(1,2)$.  We can now write down
a type $(1,0)$ tensor, i.e. a column vector
$$u=
\begin{pmatrix}
  u^1 \\ u^2
\end{pmatrix}\in \tspace{1,0}
$$
as
$$u = u^1 \ca_{(1)} + u^2 \ca_{(2)}.$$
Similarly, a row-vector
$$\phi = (\phi_1,\phi_2) \in \tspace{0,1}$$
can be written down as
$$\phi = \phi_1 \ca^{(1)} + \phi_2 \ca^{(2)}.$$
In the case of a matrix
$$
M = 
\begin{pmatrix}
  M\ud{1}{1} & M\ud{2}{1} \\
  M\ud{1}{2} & M\ud{2}{2}
\end{pmatrix}\in \tspace{1,1}
$$
we would write
$$ M = 
M\ud{1}{1}\, \ca^{(1)}_{(1)}+
M\ud{1}{2}\, \ca^{(2)}_{(1)}+
M\ud{2}{1}\, \ca^{(1)}_{(2)}+
M\ud{2}{2}\, \ca^{(2)}_{(2)}.
$$
%%%%%
%%%%%
\end{document}
