\documentclass[12pt]{article}
\usepackage{pmmeta}
\pmcanonicalname{GramDeterminant}
\pmcreated{2013-03-22 15:41:37}
\pmmodified{2013-03-22 15:41:37}
\pmowner{CWoo}{3771}
\pmmodifier{CWoo}{3771}
\pmtitle{Gram determinant}
\pmrecord{13}{37637}
\pmprivacy{1}
\pmauthor{CWoo}{3771}
\pmtype{Definition}
\pmcomment{trigger rebuild}
\pmclassification{msc}{15A63}
\pmrelated{GrammianDeterminant}
\pmrelated{GramMatrix}

\endmetadata

\usepackage{amssymb,amscd}
\usepackage{amsmath}
\usepackage{amsfonts}

% used for TeXing text within eps files
%\usepackage{psfrag}
% need this for including graphics (\includegraphics)
%\usepackage{graphicx}
% for neatly defining theorems and propositions
%\usepackage{amsthm}
% making logically defined graphics
%%%\usepackage{xypic}

% define commands here
\begin{document}
Let $V$ be an inner product space over a field $k$ with $\langle \cdot,\cdot\rangle$ the inner product on $V$ (note: since $k$ is not restricted to be either $\mathbb{R}$ or $\mathbb{C}$, the inner product here shall mean a symmetric bilinear form on $V$).  Let $x_1,x_2,\ldots,x_n$ be arbitrary vectors in $V$.  Set $r_{ij}=\langle x_i,x_j\rangle$.  The \emph{Gram determinant} of $x_1,x_2,\ldots,x_n$ is defined to be the determinant of the symmetric matrix

\begin{center}$
\begin{pmatrix}
r_{11} & \cdots & r_{1n} \\
\vdots & \ddots & \vdots \\
r_{n1} & \cdots & r_{nn}
\end{pmatrix}$
\end{center}

Let's denote this determinant by $\operatorname{Gram}[x_1,x_2,\ldots,x_n]$.

\textbf{Properties}.
\begin{enumerate}
\item $\operatorname{Gram}[x_1,\ldots,x_i,\ldots,x_j,\ldots,x_n]=
\operatorname{Gram}[x_1,\ldots,x_j,\ldots,x_i,\ldots,x_n]$.  More generally, $\operatorname{Gram}[x_1,\ldots,x_n]=
\operatorname{Gram}[x_{\sigma(1)},\ldots,x_{\sigma(n)}]$, where $\sigma$ is a permutation on $\lbrace 1,\ldots,n\rbrace$.
\item $\operatorname{Gram}[x_1,\ldots,ax_i+bx_j,\ldots,x_j,\ldots,x_n]=
a^2\operatorname{Gram}[x_1,\ldots,x_i,\ldots,x_j,\ldots,x_n]$, $a,b\in k$.
\item Setting $a=0$ and $b=1$ in Property 2, we get  $\operatorname{Gram}[x_1,\ldots,x_j,\ldots,x_j,\ldots,x_n]=0$.
\item Properties 2 and 3 can be generalized as follows: if $x_i$ (in the $i$th term) is replaced by a linear combination $y=r_1x_1+\cdots+r_nx_n$, then $$\operatorname{Gram}[x_1,\ldots,y,\ldots,x_n]=
r_i^2\operatorname{Gram}[x_1,\ldots,x_i,\ldots,x_n].$$
\item Suppose $k$ is an ordered field.  Then it can be shown that the Gram determinant is at least 0, and at most the product $\langle x_1,x_1\rangle \cdots \langle x_n,x_n\rangle$.
\item Suppose that in addition to $k$ being ordered, that every positive element in $k$ is a square, then the Gram determinant is equal to the square of the volume of the (hyper)parallelepiped generated by $x_1,\ldots,x_n$.  (Recall that an $n$-dimensional parallelepiped is the set of vectors which are linear combinations of the form $r_1x_1+\ldots+r_nx_n$ where $0\le r_i\le 1$.)
\item It's now easy to see that in Property 5, the Gram determinant is 0 if the $x_i$'s are linearly dependent, and attains its maximum if the $x_i$'s are pairwise orthogonal (a quick proof: in the above matrix, $r_{ij}=0$ if $i\neq j$), which corresponds exactly to the square of the volume of the hyperparallelepiped spanned by the $x_i$'s.
\item If $e_1,\ldots,e_n$ are basis elements of a quadratic space $V$ over an order field whose positive elements are squares, then $V$ is \PMlinkescapetext{regular}, or \PMlinkescapetext{non-singular}, iff $\operatorname{Gram}[e_1,\ldots,e_n]=0$.
\end{enumerate}


\begin{thebibliography}{99}
\bibitem{GS} Georgi E. Shilov, ``An Introduction to the Theory of Linear Spaces'', translated from Russian by Richard A. Silverman, 2nd Printing, Prentice-Hall, 1963.
\end{thebibliography}
%%%%%
%%%%%
\end{document}
