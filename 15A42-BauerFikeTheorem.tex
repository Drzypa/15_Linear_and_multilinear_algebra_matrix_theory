\documentclass[12pt]{article}
\usepackage{pmmeta}
\pmcanonicalname{BauerFikeTheorem}
\pmcreated{2013-03-22 14:48:31}
\pmmodified{2013-03-22 14:48:31}
\pmowner{Andrea Ambrosio}{7332}
\pmmodifier{Andrea Ambrosio}{7332}
\pmtitle{Bauer-Fike theorem}
\pmrecord{12}{36465}
\pmprivacy{1}
\pmauthor{Andrea Ambrosio}{7332}
\pmtype{Theorem}
\pmcomment{trigger rebuild}
\pmclassification{msc}{15A42}

% this is the default PlanetMath preamble.  as your knowledge
% of TeX increases, you will probably want to edit this, but
% it should be fine as is for beginners.

% almost certainly you want these
\usepackage{amssymb}
\usepackage{amsmath}
\usepackage{amsfonts}

% used for TeXing text within eps files
%\usepackage{psfrag}
% need this for including graphics (\includegraphics)
%\usepackage{graphicx}
% for neatly defining theorems and propositions
%\usepackage{amsthm}
% making logically defined graphics
%%%\usepackage{xypic}

% there are many more packages, add them here as you need them

% define commands here
\begin{document}
Let $\tilde{\lambda}$ be a complex number and $\tilde{u}$ be a vector with $ \Vert\tilde{u}\Vert _p = 1$, and let $r = A \tilde{u} - \tilde{\lambda} \tilde{u}$ (usually, $\tilde{\lambda}$ and $\tilde{u}$ are considered to be approximation of an eigenvalue and of an eigenvector of $A$).  Assume $A$ is diagonalizable and $A = XDX^{-1}$, with $D$ a diagonal matrix.  Then the matrix $A$ has an eigenvalue $\lambda$ which satisfies the inequality: 
\[
|\lambda - \tilde{\lambda}| \leq \kappa_p(X) \|r\|_p
\]

\textbf{see also:}
\begin{itemize}
\item Wikipedia, \PMlinkexternal{Bauer-Fike Theorem}{http://en.wikipedia.org/wiki/Bauer-Fike_Theorem}
\end{itemize}

%%%%%
%%%%%
\end{document}
