\documentclass[12pt]{article}
\usepackage{pmmeta}
\pmcanonicalname{UniquenessOfASparseSolution}
\pmcreated{2013-03-22 19:36:26}
\pmmodified{2013-03-22 19:36:26}
\pmowner{kammerer}{26336}
\pmmodifier{kammerer}{26336}
\pmtitle{uniqueness of a sparse solution}
\pmrecord{54}{42600}
\pmprivacy{1}
\pmauthor{kammerer}{26336}
\pmtype{Theorem}
\pmcomment{trigger rebuild}
\pmclassification{msc}{15A06}
%\pmkeywords{Hamming weight}
%\pmkeywords{spark of a matrix}
%\pmkeywords{system of linear equations}

\endmetadata

% this is the default PlanetMath preamble.  as your knowledge
% of TeX increases, you will probably want to edit this, but
% it should be fine as is for beginners.

% almost certainly you want these
\usepackage{amssymb}
\usepackage{amsmath}
\usepackage{amsfonts}

% used for TeXing text within eps files
%\usepackage{psfrag}
% need this for including graphics (\includegraphics)
%\usepackage{graphicx}
% for neatly defining theorems and propositions
%\usepackage{amsthm}
% making logically defined graphics
%%%\usepackage{xypic}

% there are many more packages, add them here as you need them

% define commands here
\newtheorem{theorem}{Theorem}

\begin{document}
Let $\mathbb{F}$ be a field, and let $m, n \in \mathbb{N} \setminus \{0\}.$ We denote by $\displaystyle \left\| x \right\|_0$ the Hamming weight of a column vector $\displaystyle x = [x_1, \ldots , x_n]^{\rm T} \in \mathbb{F}^n,$ i.e., $\displaystyle \left\| x \right\|_0 = \# \{j \in \{1, \ldots , n\} :\, x_j \neq 0\}.$ Consider a non-negative integer $k,$ and an $m \times n$ matrix $A$ whose entries belong to $\mathbb{F}.$  
\begin{theorem}[Donoho and Elad]
The following conditions are equivalent:\\
\begin{tabular}{cl}
(1)&for each column vector $\displaystyle y \in \mathbb{F}^m$ there exists at most one $\displaystyle x \in \mathbb{F}^n$ such that $A x = y$ and $\displaystyle \left\| x \right\|_0 \leq k,$\\
(2)&$k < \frac{1}{2} {\rm spark} (A).$\\ 
\end{tabular}
\end{theorem}

The standard proof of the theorem (see, for instance, \cite{ek}) goes as follows.

\noindent
{\bf Proof.} First, suppose that condition {\it (1)} is not satisfied. Then there exist column vectors $\displaystyle v, w \in \mathbb{F}^n$ such that $v \neq w,\, A v = A w,$ and $\displaystyle \max \left\{\left\| v \right\|_0, \left\| w \right\|_0\right\} \leq k.$ Consequently, $\displaystyle v - w \in \mathbb{F}^n \setminus \{{\bf 0}\}$ and $A (v - w) = {\bf 0}.$ Moreover, by the definition of the Hamming weight, $\displaystyle \left\| v - w \right\|_0 \leq 2 k.$ Thus, $\displaystyle {\rm spark} (A) = \inf \left\{\left\| x \right\|_0 :\, x \in \mathbb{F}^n \setminus \{{\bf 0}\},\, A x = {\bf 0}\right\} \leq 2 k,$ which means that condition {\it (2)} is not satisfied. 
\newline
Next, suppose that {\it (2)} is not satisfied. Then there exists a column vector $\displaystyle u \in \mathbb{F}^n \setminus \{{\bf 0}\}$ such that $A u = {\bf 0}$ and $\displaystyle \left\| u \right\|_0 \leq 2 k.$ It is easy to see that $u = p - q$ for some $\displaystyle p, q \in \mathbb{F}^n$ with $\displaystyle \max \left\{\left\| p \right\|_0, \left\| q \right\|_0\right\} \leq k.$ (If $\displaystyle h := \left\| u \right\|_0 > k,\, u = [u_1, \ldots , u_n]^{\rm T},$ and $\displaystyle \{j \in \{1, \ldots , n\} :\, u_j \neq 0\} = \{j_1, \ldots , j_h\},$ define $\displaystyle p = [p_1, \ldots , p_n]^{\rm T}$ by
$$
p_j = \left\{
\begin{array}{ll}
u_j,&\mbox{if $j \in \{j_1, \ldots , j_k\},$}\\
0,&\mbox{otherwise.} 
\end{array}
\right.
$$
If $\displaystyle \left\| u \right\|_0 \leq k,$ define $p = u.)$ Since $p \neq q$ and $A p = A q,$ condition {\it (1)} is not satisfied. $\Box$   

\begin{thebibliography}{99}
\bibitem{de}
D. L. Donoho and M. Elad, Optimally sparse representation in general (nonorthogonal) dictionaries via $\ell^1$ minimization, {\it Proc. Natl. Acad. Sci. USA} {\bf 100}, No. 5: 2197--2202 (2003). 
\bibitem{ek}
{\it Compressed Sensing: Theory and Applications}, edited by Y. C. Eldar and G. Kutyniok, Cambridge University Press, 2012.
\end{thebibliography}

%%%%%
%%%%%
\end{document}
