\documentclass[12pt]{article}
\usepackage{pmmeta}
\pmcanonicalname{VectorSubspace}
\pmcreated{2013-03-22 11:55:24}
\pmmodified{2013-03-22 11:55:24}
\pmowner{yark}{2760}
\pmmodifier{yark}{2760}
\pmtitle{vector subspace}
\pmrecord{20}{30624}
\pmprivacy{1}
\pmauthor{yark}{2760}
\pmtype{Definition}
\pmcomment{trigger rebuild}
\pmclassification{msc}{15-00}
\pmsynonym{subspace}{VectorSubspace}
\pmsynonym{linear subspace}{VectorSubspace}
%\pmkeywords{vector spaces}
\pmrelated{VectorSpace}
\pmrelated{LinearManifold}
\pmdefines{dimension theorem for subspaces}
\pmdefines{proper vector subspace}

\usepackage{amssymb}
\usepackage{amsmath}
\usepackage{amsfonts}
\usepackage{graphicx}
%%%%\usepackage{xypic}
\begin{document}
{\bf Definition} 
Let $V$ be a vector space over a field $F$, 
and let $W$ be a subset of $V$.
If $W$ is itself a vector space,
then $W$ is said to be a \emph{vector subspace} of $V$.
If in addtition $V\neq W$, then $W$ is a \emph{proper vector subspace} of $V$. 

If $W$ is a nonempty subset of $V$,
then a necessary and sufficient condition for $W$ to be a subspace
is that $a+\gamma b \in W$ 
for all $a,b \in W$ and all $\gamma \in F$.

\subsubsection{Examples}
\begin{enumerate}
\item Every vector space is a vector subspace of itself.
\item In every vector space, $\{0\}$ is a vector subspace.
\item If $S$ and $T$ are vector subspaces of a vector space $V$, 
then the vector sum 
\[
  S+T=\{s+t \in V \mid s\in S, t\in T\}
\]
and the intersection 
\[
  S\cap T = \{u \in V \mid u\in S, u\in T \}
\]
are vector subspaces of $V$.
\item Suppose $S$ and $T$ are vector spaces,
and suppose $L$ is a linear
mapping $L\colon S\to T$.
Then $\operatorname{Im}L$ is a vector subspace of $T$,
and $\operatorname{Ker}L$ is a vector subspace of $S$.
\item If $V$ is an inner product space,
then the orthogonal complement of any subset of $V$
is a vector subspace of $V$.
\end{enumerate}

\subsubsection{Results for vector subspaces}

{\bf Theorem 1} \cite{lang}
 Let $V$ be a finite dimensional vector space.
 If $W$ is a vector subspace of $V$ and $\dim W=\dim V$, then $W=V$.
 
 {\bf Theorem 2} \cite{deskins} (Dimension theorem for subspaces)
 Let $V$ be a vector space with
 subspaces $S$ and $T$. Then
 \begin{eqnarray*}
 \dim (S+T) + \dim (S\cap T) &=& \dim S + \dim T.
 \end{eqnarray*}
 
 \begin{thebibliography}{9}
 \bibitem {lang} S. Lang,
 \emph{Linear Algebra},
 Addison-Wesley, 1966.
 \bibitem {deskins} W.E. Deskins,
 \emph{Abstract Algebra},
 Dover publications,
 1995.
 \end{thebibliography}
%%%%%
%%%%%
%%%%%
%%%%%
\end{document}
