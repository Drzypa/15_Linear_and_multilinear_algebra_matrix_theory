\documentclass[12pt]{article}
\usepackage{pmmeta}
\pmcanonicalname{ProofOfCayleyHamiltonTheoremInACommutativeRing}
\pmcreated{2013-03-22 16:03:16}
\pmmodified{2013-03-22 16:03:16}
\pmowner{Mathprof}{13753}
\pmmodifier{Mathprof}{13753}
\pmtitle{proof of Cayley-Hamilton theorem in a commutative ring}
\pmrecord{11}{38106}
\pmprivacy{1}
\pmauthor{Mathprof}{13753}
\pmtype{Proof}
\pmcomment{trigger rebuild}
\pmclassification{msc}{15A18}
\pmclassification{msc}{15A15}

\endmetadata

% this is the default PlanetMath preamble.  as your knowledge
% of TeX increases, you will probably want to edit this, but
% it should be fine as is for beginners.

% almost certainly you want these
\usepackage{amssymb}
\usepackage{amsmath}
\usepackage{amsfonts}

% used for TeXing text within eps files
%\usepackage{psfrag}
% need this for including graphics (\includegraphics)
%\usepackage{graphicx}
% for neatly defining theorems and propositions
%\usepackage{amsthm}
% making logically defined graphics
%%%\usepackage{xypic}

% there are many more packages, add them here as you need them

% define commands here

\begin{document}
Let $R$ be a commutative ring with identity and let $A$ be an order $n$ matrix
with elements from $R[x]$. 
For example, if $A$ is
$\begin{pmatrix}
x^2+2x & 7x^2 \\
x+1 & 5
\end{pmatrix}$

then we can also associate with $A$ the following polynomial having matrix coefficents:
\begin{displaymath}
A^\sigma = \left[ {0 \atop 1}\quad {0 \atop 5} \right] + \left[{2 \atop 1 } \quad{ 0 \atop 0} \right]x + \left[ {1 \atop 0 } \quad{ 7 \atop 0 } \right]x^2 .
\end{displaymath}

In this way we have a mapping $A\longrightarrow A^\sigma$ which is an isomorphism of the rings $M_{n}(R[x])$ and $M_{n}(R)[x]$. 

Now let $A \in M_{n}(R)$ and 
consider the characteristic polynomial of $A$: $p_{A}(x) = \det(xI - A)$, which is a monic
polynomial of degree $n$ with coefficients in $R$. 
Using a property of the adjugate matrix we have
$$(xI-A)\operatorname{adj}(xI-A) = p_{A}(x)I.$$
Now view this as an equation in $M_{n}(R)[x]$. It says that $xI-A$ is a left factor
of $p_{A}(x)$. So by the factor theorem, the left hand value of $p_{A}(x)$ 
at $x=A$ is 0. The coefficients of $p_{A}(x)$ have the form $cI$, for $c\in R$,
so they commute with $A$. Therefore right and left hand values are the same.

\begin{thebibliography}{9}
\bibitem{Smiley} Malcom F. Smiley. Algebra of Matrices. Allyn and Bacon, Inc., 1965. Boston, Mass. 
\end{thebibliography}

%%%%%
%%%%%
\end{document}
