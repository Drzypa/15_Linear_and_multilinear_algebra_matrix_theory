\documentclass[12pt]{article}
\usepackage{pmmeta}
\pmcanonicalname{DerivationOf2DReflectionMatrix}
\pmcreated{2013-03-22 15:25:05}
\pmmodified{2013-03-22 15:25:05}
\pmowner{stevecheng}{10074}
\pmmodifier{stevecheng}{10074}
\pmtitle{derivation of 2D reflection matrix}
\pmrecord{14}{37262}
\pmprivacy{1}
\pmauthor{stevecheng}{10074}
\pmtype{Derivation}
\pmcomment{trigger rebuild}
\pmclassification{msc}{15-00}
%\pmkeywords{reflection matrix}
%\pmkeywords{slope}
%\pmkeywords{real projective line}
\pmrelated{RotationMatrix}
\pmrelated{DecompositionOfOrthogonalOperatorsAsRotationsAndReflections}
\pmrelated{DerivationOfRotationMatrixUsingPolarCoordinates}

\endmetadata

% this is the default PlanetMath preamble.  as your knowledge
% of TeX increases, you will probably want to edit this, but
% it should be fine as is for beginners.

% almost certainly you want these
\usepackage{amssymb}
\usepackage{amsmath}
\usepackage{amsfonts}

% used for TeXing text within eps files
%\usepackage{psfrag}
% need this for including graphics (\includegraphics)
\usepackage{graphicx}
% for neatly defining theorems and propositions
%\usepackage{amsthm}
% making logically defined graphics
%%%\usepackage{xypic}

% define commands here
\newcommand{\real}{\mathbb{R}}

\providecommand{\abs}[1]{\lvert#1\rvert}
\providecommand{\absW}[1]{\left\lvert#1\right\rvert}
\providecommand{\absB}[1]{\Bigl\lvert#1\Bigr\rvert}
\providecommand{\norm}[1]{\lVert#1\rVert}
\providecommand{\normW}[1]{\left\lVert#1\right\rVert}
\providecommand{\normB}[1]{\Bigl\lVert#1\Bigr\rVert}
\providecommand{\defnterm}[1]{\emph{#1}}
\providecommand{\transpose}[1]{#1^{\textrm{tr}}}



\newcommand{\vx}{\mathbf{x}}
\newcommand{\vy}{\mathbf{y}}
\newcommand{\vu}{\mathbf{u}}
\newcommand{\vv}{\mathbf{v}}
\newcommand{\vw}{\mathbf{w}}

\newcommand{\Otwo}{\mathrm{O}(2)}
\newcommand{\RPone}{\mathbb{RP}^1}
\begin{document}
\section*{Reflection across a line of given angle}
Let $\vx, \vy$ be perpendicular unit vectors in the plane.
Suppose we want to reflect vectors (perpendicularly) over a line that makes an angle $\theta$ with
the positive $\vx$ axis.  More precisely, we are given
a direction direction vector $\vu = \cos \theta \, \vx + \sin \theta \, \vy$ for the line of reflection.
A unit vector perpendicular to $\vu$ is $\vv = -\sin \theta \, \vx + \cos \theta \, \vy$
(as is easily checked).  Then to reflect an arbitrary vector $\vw$,
we write $\vw$ in \PMlinkescapetext{terms} of its components in the $\vu, \vv$ axes:
$\vw = a\vu + b \vv$, and the result of the reflection is to be $\vw' = a\vu - b \vv$.

\begin{center}
\includegraphics{reflection.eps}
\end{center}

We compute the matrix for such a reflection in the original $x,y$ coordinates.

Denote the reflection by $T$.  By the matrix change-of-coordinates formula,
we have
\begin{align*}
[ T ]_{xy} = [I]_{uv}^{xy} \: [T]_{uv} \: [I]_{xy}^{uv}\,,
\end{align*}
where $[T]_{xy}$ and $[T]_{uv}$ denote the matrix representing $T$ with
respect to the $x,y$ and $u,v$ coordinates respectively; $[I]_{uv}^{xy}$ is the
matrix that changes from $u,v$ coordinates to $x,y$ coordinates, and $[I]_{xy}^{uv}$ is the matrix that changes coordinates the other way.

The three matrices on the right-hand side are all easily derived
from the description we gave for the reflection $T$:
\begin{align*}
[I]_{uv}^{xy} = \begin{bmatrix}
\cos \theta & -\sin \theta \\
\sin \theta & \cos \theta
\end{bmatrix}\,, \quad
[T]_{uv} = \begin{bmatrix}
1 & 0 \\
0 & -1
\end{bmatrix}\,,
\quad 
[I]_{xy}^{uv} = \Bigl( [I]_{uv}^{xy} \Bigr)^{-1} = 
\begin{bmatrix}
\cos \theta & \sin \theta \\
-\sin \theta & \cos \theta
\end{bmatrix}\,.
\end{align*}
Computing the matrix product (with the help of the double angle identity) yields:
\begin{equation}\label{theta-matrix}
[T]_{xy} = \begin{bmatrix}
\cos 2\theta & \sin 2\theta \\
\sin 2\theta & -\cos 2\theta
\end{bmatrix}\,.
\end{equation}

For the \PMlinkescapetext{information} of the reader, we note that there are other ways of ``deriving'' this result.
One is by the use of a diagram, which would show that $(1,0)$ gets reflected
to $(\cos 2\theta, \sin 2\theta)$ and $(0,1)$ gets reflected to $(\sin 2\theta, -\cos 2\theta)$.
Another way is to observe that we can rotate an arbitrary mirror line
onto the x-axis, then reflect across the x-axis, and rotate back.
(The matrix product $[T]_{xy}$ can be seen as operating this way.)
We took neither of these two approaches, because to justify them rigorously takes a bit of work,
that is avoided by the pure linear algebra approach.

Note also that $[T]_{uv}$ and $[T]_{xy}$ are orthogonal matrices,
with determinant $-1$, as expected.

\section*{Reflection across a line of given direction vector}
Suppose instead of being given an angle $\theta$,
we are given the unit direction vector $u$ to reflect the vector $w$.
We can derive the matrix for the reflection directly, without involving any trigonometric functions.

In the decomposition $\vw = a \vu + b\vv$,
we note that $b = \vw \cdot \vv$.  Therefore 
\[
\vw' = (a\vu + b\vv) - 2b\vv = \vw - 2(\vw \cdot \vv) \vv\,.
\] 
(In fact, this is the formula used in the \PMlinkescapetext{source code} to draw the diagram in this entry.)
To derive the matrix with respect to $x,y$ coordinates, we resort to a trick:
\[
\vw' = I\vw - 2\vv (\vw \cdot \vv) = I\vw - 2\vv (\transpose{\vv} \vw) = I\vw - 2(\vv \transpose{\vv}) \vw\,.
\]
Therefore the matrix of the transformation is
\[
I - 2\vv \transpose{\vv}  = \begin{bmatrix}
u_x^2 - u_y^2 & 2u_x u_y \\
2u_x u_y & u_y^2 - u_x^2
\end{bmatrix}\,, \quad \vu = \transpose{(u_x, u_y)}\,, \quad \vv = \transpose{(-u_y, u_x)}\,.
\]
If $u$ was not a unit vector to begin with, it of course suffices to divide by its magnitude before proceeding. Taking this into account, we obtain the following matrix for a reflection
about a line with direction $\vu$:
\begin{equation}
\label{u-matrix}
\frac{1}{u_x^2 + u_y^2} 
\begin{bmatrix}
u_x^2 - u_y^2 & 2u_x u_y \\
2u_x u_y & u_y^2 - u_x^2
\end{bmatrix}\,.
\end{equation}
Notice that if we put $u_x = \cos \theta$ and $u_y = \sin \theta$ in matrix \eqref{u-matrix},
we get matrix \eqref{theta-matrix}, as it should be.

\section*{Reflection across a line of given slope}
There is another form for the matrix \eqref{theta-matrix}.
We set $m = \tan \theta$ to be the slope of the line of reflection
and use the identities:
\begin{align*}
\cos^2 \theta &= \frac{1}{\tan^2 \theta + 1}  = \frac{1}{m^2 + 1} \\
\cos 2 \theta &= 2 \cos^2 \theta - 1  \\
\sin 2 \theta &= 2 \sin \theta \cos \theta = 2 \tan \theta \cos^2 \theta = 2m \cos^2 \theta\,.
\end{align*}
When these equations are substituted in matrix \eqref{theta-matrix},
we obtain an alternate expression for it in \PMlinkescapetext{terms} of $m$ only:
\begin{equation}\label{m-matrix}
\frac{1}{m^2 + 1} 
\begin{bmatrix}
1-m^2 & 2m \\
2m & m^2 - 1
\end{bmatrix}\,.
\end{equation}
Thus we have derived the matrix for a reflection about a line of slope $m$.

Alternatively, we could have also substituted $u_x = 1$ and $u_y = m$
in matrix \eqref{u-matrix} to arrive at the same result.

\section*{Topology of reflection matrices}
Of course, formula \eqref{m-matrix} does not work literally when 
$m = \pm \infty$ (the line is vertical).
However, that case may be derived by taking the limit $\abs{m} \to \infty$ ---
this limit operation can be justified by considerations of the topology of 
the space of two-dimensional reflection matrices.

What is this topology? It is the one-dimensional projective plane $\RPone$,
or simply, the ``real projective line''.
It is formed by taking the circle, and identifying opposite points, so
that each pair of opposite points specify a unique mirror line of reflection in $\real^2$.
Formula \eqref{theta-matrix} is a parameterization of $\RPone$.
Note that \eqref{theta-matrix} involves the quantity $2\theta$, not $\theta$,
because for a point $(\cos \theta, \sin \theta)$ on the circle,
its opposite point $(\cos (\theta+\pi), \sin (\theta+\pi))$ specify the same reflection,
so formula \eqref{theta-matrix} has to be invariant when $\theta$ is replaced by $\theta + \pi$.

But \eqref{theta-matrix} might as well be written
\begin{equation}\label{phi-matrix}
[T]_{xy} = \begin{bmatrix}
\cos \phi & \sin \phi \\
\sin \phi & -\cos \phi
\end{bmatrix}\,.
\end{equation}
where $\phi = 2\theta$. For this parameterization of $RP^1$ to be one-to-one,
$\phi$ can range over interval $(0, 2\pi)$, and the endpoints at $\phi = 0, 2\pi$ overlap just
as for a circle, without identifying pairs of opposite points.  What does this mean?  It is the fact that $\RPone$ is homeomorphic to the circle $S^1$.

The real projective line $\RPone$ is also the one-point compactification of $\real$
(i.e. $\RPone = \real \cup \{ \infty \}$),
as shown by formula \eqref{m-matrix}; the number $m = \infty$ corresponds to 
a reflection across the vertical axis.  Note that this ``$\infty$''
is not the same as the usual $\pm \infty$, because here $-\infty$ and $\infty$
are actually the same number, both representing the slope of a vertical line.
%%%%%
%%%%%
\end{document}
