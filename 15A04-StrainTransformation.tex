\documentclass[12pt]{article}
\usepackage{pmmeta}
\pmcanonicalname{StrainTransformation}
\pmcreated{2013-03-22 17:25:45}
\pmmodified{2013-03-22 17:25:45}
\pmowner{CWoo}{3771}
\pmmodifier{CWoo}{3771}
\pmtitle{strain transformation}
\pmrecord{5}{39804}
\pmprivacy{1}
\pmauthor{CWoo}{3771}
\pmtype{Definition}
\pmcomment{trigger rebuild}
\pmclassification{msc}{15A04}
\pmsynonym{strain}{StrainTransformation}
\pmdefines{strain coefficient}

\usepackage{amssymb,amscd}
\usepackage{amsmath}
\usepackage{amsfonts}
\usepackage{mathrsfs}

% used for TeXing text within eps files
%\usepackage{psfrag}
% need this for including graphics (\includegraphics)
%\usepackage{graphicx}
% for neatly defining theorems and propositions
\usepackage{amsthm}
% making logically defined graphics
%%\usepackage{xypic}
\usepackage{pst-plot}
\usepackage{psfrag}

% define commands here
\newtheorem{prop}{Proposition}
\newtheorem{thm}{Theorem}
\newtheorem{ex}{Example}
\newcommand{\real}{\mathbb{R}}
\newcommand{\pdiff}[2]{\frac{\partial #1}{\partial #2}}
\newcommand{\mpdiff}[3]{\frac{\partial^#1 #2}{\partial #3^#1}}
\begin{document}
Let $E$ be a Euclidean plane.  Fix a line $\ell$ in $E$ and a real number $r\ne 0$.  Take any point $p\in E$.  Drop a line $m_p$ from $p$ perpendicular to $\ell$.  Denote $d(p,\ell)$ the distance from $p$ to $\ell$.  Then there is a unique point $p'$ on $m_p$ such that $$d(p',\ell)=r\cdot d(p,\ell).$$  The function $s_r:E\to E$ such that $s_r(p)=p'$ is called a \emph{strain transformation}, or simply a \emph{strain}. 

One can visualize a strain stretches a geometric figure if $|r|>1$ and compresses it if $|r|<1$.  If $r=1$, then $s_r$ is the identity function, the only time when a strain is a rigid motion.  For example, let $\ell$ be the $x$-axis and $C$ be a circle in the upper half plane of the $x$-$y$ plane.  Then the following diagrams show how a strain transforms $C$:

\begin{center}
\psset{unit=1.5cm}
\begin{pspicture}(-4,-2)(5,3)
\psline(-4,0)(4,0)
\rput(4.5,0){$\ell$}
\psellipse(-3,1)(0.5,0.5)
\psellipse(-1,2)(0.5,1)
\psellipse(1,0.5)(0.5,0.25)
\psellipse(3,-1)(0.5,0.5)
\rput(-3,-2){$C$}
\rput(-1,-2){$s_2(C)$}
\rput(1,-2){$s_{\frac{1}{2}}(C)$}
\rput(3,-2){$s_{-1}(C)$}
\end{pspicture}
\end{center}

Again, if $\ell$ is the $x$-axis, then $s_r$ is the function that sends $(x,y)$ to $(x,ry)$.  Representing the ordered pairs as column vectors and $s_r$ as a matrix , we have
\begin{center}$s_r
\begin{pmatrix}
x \\ y 
\end{pmatrix}
=
\begin{pmatrix}
1 & 0 \\
0 & r
\end{pmatrix}
\begin{pmatrix}
x \\ y 
\end{pmatrix}
=
\begin{pmatrix}
x \\ ry 
\end{pmatrix}
.$
\end{center}

Nevertheless, a strain, as a (non-singular) linear transformation, takes lines to lines, and parallel lines to parallel lines.

In general, given any finite dimensional vector space $V$ over a field $k$, a strain $s_r$ is a non-singular diagonalizable linear transformation on $V$ such that $s_r$ leaves a subspace $W$ of codimension $1$ fixed.  $0\ne r\in k$ is called the \emph{strain coefficient}.

\textbf{Remark}.  By choosing an appropriate base for $V$ of dimension $n$, $s_r$ can be represented as a diagonal matrix whose diagonals are $1$ in at least $n-1$ cells and $r$ in at most one cell.  

It is easy to see that every non-singular diagonalizable linear transformation on $V$ can be written as a product of $n$ strains, where $n=\operatorname{dim}(V)$.
%%%%%
%%%%%
\end{document}
