\documentclass[12pt]{article}
\usepackage{pmmeta}
\pmcanonicalname{BerlekampMasseyAlgorithm}
\pmcreated{2013-03-22 14:28:55}
\pmmodified{2013-03-22 14:28:55}
\pmowner{mathcam}{2727}
\pmmodifier{mathcam}{2727}
\pmtitle{Berlekamp-Massey algorithm}
\pmrecord{7}{36015}
\pmprivacy{1}
\pmauthor{mathcam}{2727}
\pmtype{Definition}
\pmcomment{trigger rebuild}
\pmclassification{msc}{15A03}
\pmclassification{msc}{11B37}
\pmrelated{RecurrenceRelation}
\pmrelated{MapleImplementationOfBerlekampMasseyAlgorithm}
\pmdefines{linear recurrent sequence}
\pmdefines{minimal polynomial of a sequence}
\pmdefines{annihilator}

% this is the default PlanetMath preamble.  as your knowledge
% of TeX increases, you will probably want to edit this, but
% it should be fine as is for beginners.

% almost certainly you want these
\usepackage{amssymb}
\usepackage{amsmath}
\usepackage{amsfonts}

% used for TeXing text within eps files
%\usepackage{psfrag}
% need this for including graphics (\includegraphics)
%\usepackage{graphicx}
% for neatly defining theorems and propositions
\usepackage{amsthm}
% making logically defined graphics
%%%\usepackage{xypic}

% there are many more packages, add them here as you need them

% define commands here
\newtheorem{theorem}{Theorem}
\newtheorem{proposition}{Proposition}
\newtheorem{definition}{Definition}
\begin{document}
The Berlekamp-Massey algorithm is used for finding the minimal polynomial of a linearly recurrent sequence. The algorithm itself is presented at the end of this article.

\begin{definition}
Suppose the infinite sequence $a$ with elements from a field $K$ has the property that there exist constants $c_1, \ldots, c_k$ in $K$ such that, for all $t > k$, 
        $$a_t = a_{t-1}c_1 + a_{t-2}c_2 + ... + a_{t-k}c_k.$$
Then $a$ is called a {\textbf{linearly recurrent sequence}}.
\end{definition}

\begin{definition}
Given a linearly recurrent sequence $a$, suppose $c_0 \ldots c_k \in K$ with $c_0 \neq 0$ satisfy, for all $t>k, $
        $$c_0a_t = a_{t-1}c_1 + a_{t-2}c_2 + ... + a_{t-k}c_k.$$
Then the polynomial $$c_0x^k - c_1x^{k-1} - c_2x^{k-2} - ... - c_k$$ is called an \textbf{annihilator} for $a$.
\end{definition}

\begin{proposition}
The annihilators of $a$ form an ideal of $K[x]$.
\end{proposition}

\begin{definition}
Since $K[x]$ is a principal ideal domain, the ideal of $a$'s annihilators have a unique monic generator of minimal degree. This annihilator is called the {\textbf{minimal polynomial}} of $a$.
\end{definition}

To find the minimal polynomial, we need to be given an upper bound $m$ on its degree; having done so, the minimal polynomial is uniquely determined by the first $2m$ elements of $a$ (since we need to get $m$ equations to solve for the unknowns $c_1, \ldots c_m$).



There is another way to determine the minimal polynomial, originally presented by Dornstetter, which uses the Euclidean Algorithm. It can be shown that the characteristic polynomial of a sequence is the unique monic polynomial $C(x)$ of least degree for which the infinite product
 $$C(x)(a_1 + a_2x + a_3x^2 + ...)$$
has finitely many nonzero terms. (In fact, the nonzero terms will have coefficients up to $x^{k-1}$ where $k$ is the degree of $C$). 

We can rewrite this as
$$C(x)\cdot(a_1 + a_2x + ... + a_{2m}x^{2m-1}) - Q(x)\cdot x^{2m} = R(x) $$

where $R(x)$ is a remainder polynomial of degree < $m$, and $Q(x)$ is a quotient polynomial. Denote by $A(x)$ the sum $\Sigma_{i=1}^{2m}a_ix^{i-1}$. 

This is where the Euclidean Algorithm comes in; if we take the GCD of $A(x)$ and $x^{2m}$, keeping track of remainders, we get two sequences $P_i(x), Q_i(x)$ such that
$$P_i(x)\cdot A(x) - Q_i(x)\cdot x^{2m}$$
forms a series of polynomials whose degree is decreasing; as soon as this degree is less than $m$, we have the needed polynomials with $C=P_i, Q=Q_i$.

There is more info about the Extended Euclidean Algorithm in ``Modern Computer Algebra'' by von zur Gathen and Gerhard. 

(Berlekamp's algorithm proper to come)

The original algorithm is from \emph{Algebraic Coding Theory} by Elwyn R. Berlekamp, McGraw-Hill, 1968. Its application to linearly recurrent sequences was noted by J.L.Massey, in ``Shift-register synthesis and BCH decoding'', 1969.
%%%%%
%%%%%
\end{document}
