\documentclass[12pt]{article}
\usepackage{pmmeta}
\pmcanonicalname{AreaOfPolygon}
\pmcreated{2013-03-22 17:32:57}
\pmmodified{2013-03-22 17:32:57}
\pmowner{pahio}{2872}
\pmmodifier{pahio}{2872}
\pmtitle{area of polygon}
\pmrecord{6}{39953}
\pmprivacy{1}
\pmauthor{pahio}{2872}
\pmtype{Result}
\pmcomment{trigger rebuild}
\pmclassification{msc}{15A63}
\pmclassification{msc}{51M25}
\pmrelated{Area}
\pmrelated{Determinant2}
\pmrelated{CentreOfMassOfPolygon}

% this is the default PlanetMath preamble.  as your knowledge
% of TeX increases, you will probably want to edit this, but
% it should be fine as is for beginners.

% almost certainly you want these
\usepackage{amssymb}
\usepackage{amsmath}
\usepackage{amsfonts}

% used for TeXing text within eps files
%\usepackage{psfrag}
% need this for including graphics (\includegraphics)
%\usepackage{graphicx}
% for neatly defining theorems and propositions
 \usepackage{amsthm}
% making logically defined graphics
%%%\usepackage{xypic}

% there are many more packages, add them here as you need them

% define commands here

\theoremstyle{definition}
\newtheorem*{thmplain}{Theorem}

\begin{document}
Let the vertices of a (\PMlinkescapetext{planar}) polygon be\, $(x_1,\,y_1),\,(x_2,\,y_2),\,\ldots,\,(x_n,\,y_n)$,\, enumerated in \PMlinkescapetext{order} when gone round the polygon anticlockwise.  The area of the polygon is equal to
$$
\frac{1}{2}\left(\left|\begin{matrix}x_1&x_2\\y_1&y_2\end{matrix}\right|
+\left|\begin{matrix}x_2&x_3\\y_2&y_3\end{matrix}\right|+\ldots
+\left|\begin{matrix}x_{n-1}&x_n\\y_{n-1}&y_n\end{matrix}\right|+
\left|\begin{matrix}x_n&x_1\\y_n&y_1\end{matrix}\right|\right).
$$
\begin{thebibliography}{9}
\bibitem{NP}{\sc E. Lindel\"of:} {\em Johdatus korkeampaan analyysiin}. Nelj\"as painos.\, Werner S\"oderstr\"om Osakeyhti\"o, Porvoo and Helsinki (1956).
\end{thebibliography}
%%%%%
%%%%%
\end{document}
