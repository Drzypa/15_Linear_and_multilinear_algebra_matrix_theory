\documentclass[12pt]{article}
\usepackage{pmmeta}
\pmcanonicalname{LinearComplexStructure}
\pmcreated{2013-03-22 16:16:48}
\pmmodified{2013-03-22 16:16:48}
\pmowner{Mazzu}{14365}
\pmmodifier{Mazzu}{14365}
\pmtitle{linear complex structure}
\pmrecord{12}{38392}
\pmprivacy{1}
\pmauthor{Mazzu}{14365}
\pmtype{Definition}
\pmcomment{trigger rebuild}
\pmclassification{msc}{15-00}
\pmrelated{ComplexificationOfVectorSpace}
\pmdefines{linear complex structure}

% this is the default PlanetMath preamble.  as your knowledge
% of TeX increases, you will probably want to edit this, but
% it should be fine as is for beginners.

% almost certainly you want these
\usepackage{amssymb}
\usepackage{amsmath}
\usepackage{amsfonts}

% used for TeXing text within eps files
%\usepackage{psfrag}
% need this for including graphics (\includegraphics)
%\usepackage{graphicx}
% for neatly defining theorems and propositions
%\usepackage{amsthm}
% making logically defined graphics
%%%\usepackage{xypic}

% there are many more packages, add them here as you need them

% define commands here

\begin{document}
\newcommand{\mm}{\mathrm{M}}
\newcommand{\oo}{\mathrm{O}}
\newcommand{\R}{\mathbb{R}}
\newcommand{\C}{\mathbb{C}}
\newcommand{\I}{\mathbf{I}}

A \emph{\PMlinkescapetext{(linear) complex structure}}  on a real vector space $V$, with $\mathrm{dim}(V)=m$, is a linear automorphism $J\in\mathrm{Aut}(V)$ such that $J^2=J\circ J=-\mathrm{id}_V$. With a complex structure $J$ we can consider $V$ as a complex vector space with the product $\C\times V\rightarrow V$ given by $$(x+iy)\mathbf v=x\mathbf v+y J(\mathbf v),\ \ \forall x,y\in\R,\ \mathbf v\in V.$$
This implies  that the dimension $m$ of $V$ must be even.

A common example is $V=\mathbb R^{2n}$ with the standard basis $\mathbf e_1,...,\mathbf e_n,\mathbf f_1,...,\mathbf f_n$, for which we can obtain a complex structure $J_0\in\mathrm{Aut}(\mathbb R^{2n})$ represented by the matrix 
$$\left( \begin{array}{cc}
\mathbf 0 & \I_n \\
-\I_n & \mathbf 0
\end{array} \right).$$
Here $\I_n\in\mm_n(\R)$ is the identity $n\times n$ matrix and
$\mathbf 0\in\mm_n(\R)$ is the zero $n\times n$ matrix.
%%%%%
%%%%%
\end{document}
