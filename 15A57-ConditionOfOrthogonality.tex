\documentclass[12pt]{article}
\usepackage{pmmeta}
\pmcanonicalname{ConditionOfOrthogonality}
\pmcreated{2013-03-22 14:48:05}
\pmmodified{2013-03-22 14:48:05}
\pmowner{pahio}{2872}
\pmmodifier{pahio}{2872}
\pmtitle{condition of orthogonality}
\pmrecord{14}{36455}
\pmprivacy{1}
\pmauthor{pahio}{2872}
\pmtype{Result}
\pmcomment{trigger rebuild}
\pmclassification{msc}{15A57}
\pmclassification{msc}{51F20}
\pmsynonym{condition of perpendicularity}{ConditionOfOrthogonality}
%\pmkeywords{straight line}
%\pmkeywords{slope}
\pmrelated{OrthogonalCurves}
\pmrelated{InverseNumber}
\pmrelated{OppositeNumber}
\pmrelated{NormalLine}
\pmrelated{AngleBetweenTwoLines}
\pmrelated{PerpendicularityInEuclideanPlane}
\pmrelated{Evolute2}
\pmrelated{ExampleOfFindingCatacaustic}
\pmdefines{opposite inverse}

\endmetadata

% this is the default PlanetMath preamble.  as your knowledge
% of TeX increases, you will probably want to edit this, but
% it should be fine as is for beginners.

% almost certainly you want these
\usepackage{amssymb}
\usepackage{amsmath}
\usepackage{amsfonts}
\usepackage{graphicx}
\begin{document}
Let two straight lines of the $xy$-plane have the slopes $m_1$ and $m_2$. \,The lines are at right angles to each other iff $m_1$ and $m_2$ are the {\em opposite inverses} of each other, i.e. \PMlinkname{iff}{Biconditional} 
                             \[m_1m_2 = -1.\]

\textbf{Example.} \,The lines \,$y = (1+\sqrt{2})x$\, and \,$y = (1-\sqrt{2})x$\, are at right angles to each other.
\begin{center}
\includegraphics{orthogon}
\end{center}
%%%%%
%%%%%
\end{document}
