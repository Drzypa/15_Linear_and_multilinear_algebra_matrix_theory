\documentclass[12pt]{article}
\usepackage{pmmeta}
\pmcanonicalname{IfAinMnRAndAIsSupertriangularThenAn0}
\pmcreated{2013-03-22 13:44:39}
\pmmodified{2013-03-22 13:44:39}
\pmowner{Daume}{40}
\pmmodifier{Daume}{40}
\pmtitle{If $A \in M_n(R)$ and $A$ is supertriangular then $A^n=0$}
\pmrecord{12}{34438}
\pmprivacy{1}
\pmauthor{Daume}{40}
\pmtype{Theorem}
\pmcomment{trigger rebuild}
\pmclassification{msc}{15-00}

% this is the default PlanetMath preamble.  as your knowledge
% of TeX increases, you will probably want to edit this, but
% it should be fine as is for beginners.

% almost certainly you want these
\usepackage{amssymb}
\usepackage{amsmath}
\usepackage{amsfonts}

% used for TeXing text within eps files
%\usepackage{psfrag}
% need this for including graphics (\includegraphics)
%\usepackage{graphicx}
% for neatly defining theorems and propositions
%\usepackage{amsthm}
% making logically defined graphics
%%%\usepackage{xypic} 

% there are many more packages, add them here as you need them

% define commands here
\begin{document}
\textbf{theorem:}  Let $R$ be commutative ring with identity.
If an n-square matrix $A \in Mat_{n}(R)$
 is supertriangular then $A^n = 0$.\\\\
\textbf{proof:}  Find the characteristic polynomial of $A$ by computing the determinant of $A-tI$.  The square matrix $A-tI$ is a triangular matrix.  The determinant of a triangular matrix is the product of the diagonal element of the matrix.  Therefore the characteristic polynomial is $p(t) =t^n$ and by the Cayley-Hamilton theorem the matrix $A$ satisfies the polynomial.  That is $A^n = 0$.\\
QED
%%%%%
%%%%%
\end{document}
