\documentclass[12pt]{article}
\usepackage{pmmeta}
\pmcanonicalname{DerivativeOfMatrix}
\pmcreated{2013-03-22 15:00:28}
\pmmodified{2013-03-22 15:00:28}
\pmowner{matte}{1858}
\pmmodifier{matte}{1858}
\pmtitle{derivative of matrix}
\pmrecord{10}{36713}
\pmprivacy{1}
\pmauthor{matte}{1858}
\pmtype{Definition}
\pmcomment{trigger rebuild}
\pmclassification{msc}{15-01}
\pmrelated{NthDerivativeOfADeterminant}

% this is the default PlanetMath preamble.  as your knowledge
% of TeX increases, you will probably want to edit this, but
% it should be fine as is for beginners.

% almost certainly you want these
\usepackage{amssymb}
\usepackage{amsmath}
\usepackage{amsfonts}
\usepackage{amsthm}

\usepackage{mathrsfs}

% used for TeXing text within eps files
%\usepackage{psfrag}
% need this for including graphics (\includegraphics)
%\usepackage{graphicx}
% for neatly defining theorems and propositions
%
% making logically defined graphics
%%%\usepackage{xypic}

% there are many more packages, add them here as you need them

% define commands here

\newcommand{\sR}[0]{\mathbb{R}}
\newcommand{\sC}[0]{\mathbb{C}}
\newcommand{\sN}[0]{\mathbb{N}}
\newcommand{\sZ}[0]{\mathbb{Z}}

 \usepackage{bbm}
 \newcommand{\Z}{\mathbbmss{Z}}
 \newcommand{\C}{\mathbbmss{C}}
 \newcommand{\R}{\mathbbmss{R}}
 \newcommand{\Q}{\mathbbmss{Q}}



\newcommand*{\norm}[1]{\lVert #1 \rVert}
\newcommand*{\abs}[1]{| #1 |}



\newtheorem{thm}{Theorem}
\newtheorem{defn}{Definition}
\newtheorem{prop}{Proposition}
\newtheorem{lemma}{Lemma}
\newtheorem{cor}{Corollary}
\begin{document}
Suppose $I$ is an open set of $\R$, and for each $t\in I$, $A(t)$ is 
an $n\times m$ matrix. If each element in $A(t)$ is a differentiable function 
of $t$, we say that $A$ is a differentiable, and define the 
derivative of $A$ componentwise. This derivative we shall write as
$\frac{d}{dt}A$ or $\frac{dA}{dt}$. 

\subsubsection*{Properties}
In the below we assume that all matrices are dependent on a parameter $t$
and the matrices are differentiable with respect to $t$. 
\begin{enumerate}
\item For any $n\times m$ matrix $A$,
\begin{eqnarray*}
   \left(\frac{dA}{dt}\right)^T &=& \frac{d}{dt}\left(A^T\right), 
\end{eqnarray*}
where $^T$ is the matrix transpose.
\item If $A(t),B(t)$ are matrices such that $AB$ is defined, then 
$$
   \frac{d}{dt}(AB) = \frac{dA}{dt} B + A\frac{dB}{dt}.
$$
\item When $A(t)$ is invertible,
$$
   \frac{d}{dt}(A^{-1}) = -A^{-1} \frac{dA}{dt} A^{-1}.
$$
\item For a square matrix $A(t)$, 
\begin{eqnarray*}
   \operatorname{tr}(\frac{dA}{dt}) &=& \frac{d}{dt}\operatorname{tr}(A),
\end{eqnarray*}
where $\operatorname{tr}$ is the matrix trace.
\item If $A(t),B(t)$ are $n\times m$ matrices and $A\circ B$ is the
Hadamard product of $A$ and $B$, then 
$$
   \frac{d}{dt}(A\circ B) = \frac{dA}{dt}\circ  B + A\circ \frac{dB}{dt}.
$$
%\operatorname{exp} is the matrix exponential.
\end{enumerate}
%%%%%
%%%%%
\end{document}
