\documentclass[12pt]{article}
\usepackage{pmmeta}
\pmcanonicalname{AntidiagonalMatrix}
\pmcreated{2013-03-22 15:12:20}
\pmmodified{2013-03-22 15:12:20}
\pmowner{matte}{1858}
\pmmodifier{matte}{1858}
\pmtitle{anti-diagonal matrix}
\pmrecord{8}{36964}
\pmprivacy{1}
\pmauthor{matte}{1858}
\pmtype{Definition}
\pmcomment{trigger rebuild}
\pmclassification{msc}{15-00}

\endmetadata

% this is the default PlanetMath preamble.  as your knowledge
% of TeX increases, you will probably want to edit this, but
% it should be fine as is for beginners.

% almost certainly you want these
\usepackage{amssymb}
\usepackage{amsmath}
\usepackage{amsfonts}
\usepackage{amsthm}

\usepackage{mathrsfs}

% used for TeXing text within eps files
%\usepackage{psfrag}
% need this for including graphics (\includegraphics)
%\usepackage{graphicx}
% for neatly defining theorems and propositions
%
% making logically defined graphics
%%%\usepackage{xypic}

% there are many more packages, add them here as you need them

% define commands here

\newcommand{\sR}[0]{\mathbb{R}}
\newcommand{\sC}[0]{\mathbb{C}}
\newcommand{\sN}[0]{\mathbb{N}}
\newcommand{\sZ}[0]{\mathbb{Z}}

 \usepackage{bbm}
 \newcommand{\Z}{\mathbbmss{Z}}
 \newcommand{\C}{\mathbbmss{C}}
 \newcommand{\F}{\mathbbmss{F}}
 \newcommand{\R}{\mathbbmss{R}}
 \newcommand{\Q}{\mathbbmss{Q}}



\newcommand*{\norm}[1]{\lVert #1 \rVert}
\newcommand*{\abs}[1]{| #1 |}



\newtheorem{thm}{Theorem}
\newtheorem{defn}{Definition}
\newtheorem{prop}{Proposition}
\newtheorem{lemma}{Lemma}
\newtheorem{cor}{Corollary}
\begin{document}
Let $A$ be a square matrix (over any field $\F$). An entry in $A$ 
is an \emph{anti-diagonal entry} if it is on the line going from the 
lower left corner of $A$ to the upper right corner. If all 
entries in $A$ are zero except on the anti-diagonal, then $A$ is an 
\emph{anti-diagonal matrix}.

If $a_1, \ldots, a_n \in \F$, let 
$$
  \operatorname{adiag}(a_1, \ldots, a_n) = \begin{pmatrix}
0 & 0     & 0     & 0   & a_1 \\
0 & 0     & 0     & a_2 & 0  \\
0 & 0     & a_{3} &     & 0  \\
0 & \cdot & 0     &     & 0  \\
a_n & 0   & 0     &     &  0
\end{pmatrix}. 
$$

\subsubsection*{Properties of anti-diagonal matrices}
\begin{enumerate}
\item If $A$ and $D$ are $n\times n$ anti-diagonal and 
   diagonal matrices, respectively, then $AD, DA$ are anti-diagonal.
\item The product of two anti-diagonal matrices is an diagonal matrix.
\end{enumerate}
%%%%%
%%%%%
\end{document}
