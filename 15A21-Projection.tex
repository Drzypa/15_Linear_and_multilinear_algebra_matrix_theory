\documentclass[12pt]{article}
\usepackage{pmmeta}
\pmcanonicalname{Projection}
\pmcreated{2013-03-22 12:52:13}
\pmmodified{2013-03-22 12:52:13}
\pmowner{rmilson}{146}
\pmmodifier{rmilson}{146}
\pmtitle{projection}
\pmrecord{8}{33208}
\pmprivacy{1}
\pmauthor{rmilson}{146}
\pmtype{Definition}
\pmcomment{trigger rebuild}
\pmclassification{msc}{15A21}
\pmclassification{msc}{15A57}
\pmdefines{orthogonal projection}

\endmetadata

\usepackage{amsmath}
\usepackage{amsfonts}
\usepackage{amssymb}
\newcommand{\reals}{\mathbb{R}}
\newcommand{\natnums}{\mathbb{N}}
\newcommand{\cnums}{\mathbb{C}}
\newcommand{\znums}{\mathbb{Z}}
\newcommand{\lp}{\left(}
\newcommand{\rp}{\right)}
\newcommand{\lb}{\left[}
\newcommand{\rb}{\right]}
\newcommand{\supth}{^{\text{th}}}
\newtheorem{proposition}{Proposition}
\newtheorem{definition}[proposition]{Definition}
\newcommand{\nl}[1]{{\PMlinkescapetext{{#1}}}}
\newcommand{\pln}[2]{{\PMlinkname{{#1}}{#2}}}

\newcommand{\adj}{^{\displaystyle \star}}
\newcommand{\img}{\mathop{\mathrm{img}}\nolimits}
\begin{document}
A linear transformation $P:V\rightarrow V$ of a vector space $V$ is called a
\emph{projection} if it acts like the identity on its image. This
condition can be more succinctly expressed by the equation
\begin{equation}
  \label{eq:proj}
  P^2 = P.  
\end{equation}
\begin{proposition}
  If $P:V\rightarrow V$ is a projection, then
  its image and the kernel are complementary subspaces, namely
  \begin{equation}
    \label{eq:comp}
    V = \ker P \oplus \img P.
  \end{equation}
\end{proposition}
\emph{Proof. }
  Suppose that $P$ is a projection.  Let $v\in V$ be given, and set 
  $$u=v-Pv.$$
  The projection condition \eqref{eq:proj} then implies
  that $u\in \ker P$, and we can write $v$ as the sum of an image and
  kernel vectors:
  $$v = u + Pv.$$
  This decomposition is unique, because the
  intersection of the image and the kernel is the trivial subspace.
  Indeed, suppose that $v\in V$ is in both the image and the kernel of $P$.
  Then, $Pv=v$ and $Pv=0$, and hence $v=0$. QED


Conversely, every direct sum decomposition
$$V = V_1 \oplus V_2$$
corresponds to a projection $P:V\rightarrow V$ defined by
$$
Pv =\begin{cases}
v & v\in V_1 \\
0 & v\in V_2
\end{cases}$$

Specializing somewhat, suppose that the ground field is $\reals$ or
$\cnums$ and that $V$ is equipped with a positive-definite inner
product.  In this setting we call an endomorphism
$P:V\rightarrow V$ an \emph{orthogonal projection} if it is self-dual
$$P\adj = P,$$
in addition to  satisfying the projection condition \eqref{eq:proj}.
\begin{proposition}
  The kernel and image of an orthogonal projection are orthogonal subspaces.
\end{proposition}
\emph{Proof. }
  Let $u\in\ker P$ and $v\in \img P$ be given.  Since $P$
  is self-dual we have
  $$0 = \langle Pu,v\rangle = \langle u,Pv\rangle = \langle
  u,v\rangle.$$
QED

Thus we see that a orthogonal projection $P$ projects a $v \in V$ onto
$Pv$ in an orthogonal fashion, i.e.
$$\langle v-Pv,u\rangle = 0$$
for all $u\in \img P$.
%%%%%
%%%%%
\end{document}
