\documentclass[12pt]{article}
\usepackage{pmmeta}
\pmcanonicalname{VectorProductInGeneralVectorSpaces}
\pmcreated{2013-03-22 14:32:32}
\pmmodified{2013-03-22 14:32:32}
\pmowner{mathwizard}{128}
\pmmodifier{mathwizard}{128}
\pmtitle{vector product in general vector spaces}
\pmrecord{5}{36089}
\pmprivacy{1}
\pmauthor{mathwizard}{128}
\pmtype{Definition}
\pmcomment{trigger rebuild}
\pmclassification{msc}{15A72}
\pmsynonym{vector product}{VectorProductInGeneralVectorSpaces}

\endmetadata

% this is the default PlanetMath preamble.  as your knowledge
% of TeX increases, you will probably want to edit this, but
% it should be fine as is for beginners.

% almost certainly you want these
\usepackage{amssymb}
\usepackage{amsmath}
\usepackage{amsfonts}

% used for TeXing text within eps files
%\usepackage{psfrag}
% need this for including graphics (\includegraphics)
%\usepackage{graphicx}
% for neatly defining theorems and propositions
%\usepackage{amsthm}
% making logically defined graphics
%%%\usepackage{xypic}

% there are many more packages, add them here as you need them

% define commands here
\begin{document}
The vector product can be defined in any finite dimensional vector space $V$ with $\dim V=n$. Let $v_1,\dots,v_n$ be a basis of $V$, we then define the vector product of the vectors $w_1,\dots,w_{n-1}$ in the following way:
$$w_1\times\dots\times w_{n-1}=\sum_{j=1}^nv_j\det(w_1,\dots,w_{n-1},v_j).$$
One can easily see that some of the properties of the vector product are the same as in $\mathbb{R}^3$:
\begin{itemize}
\item If one of the $w_i$ is equal to $0$, then the vector product is $0$.
\item If $w_i$ are linearly dependent, then the vector product is $0$.
\item In a Euclidean vector space $w_1\times\dots\times w_{n-1}$ is perpendicular to all $w_i$.
\end{itemize}
%%%%%
%%%%%
\end{document}
