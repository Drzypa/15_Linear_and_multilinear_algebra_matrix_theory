\documentclass[12pt]{article}
\usepackage{pmmeta}
\pmcanonicalname{NilpotentMatrix}
\pmcreated{2013-03-22 13:05:56}
\pmmodified{2013-03-22 13:05:56}
\pmowner{jgade}{861}
\pmmodifier{jgade}{861}
\pmtitle{nilpotent matrix}
\pmrecord{17}{33520}
\pmprivacy{1}
\pmauthor{jgade}{861}
\pmtype{Definition}
\pmcomment{trigger rebuild}
\pmclassification{msc}{15-00}

\usepackage{amsmath, amssymb, amsthm}
\newtheorem*{thm}{Theorem}
\begin{document}
The square matrix $A$ is said to be nilpotent if $A^n = \underbrace{AA\cdots A}_{\textrm{n times}} = \mathbf{0}$ for some positive integer $n$ (here $\mathbf{0}$ 
denotes the matrix where every entry is 0).

\begin{thm}[Characterization of nilpotent matrices]
A matrix is nilpotent iff its eigenvalues are all 0.
\end{thm}

\begin{proof}
Let $A$ be a nilpotent matrix. Assume $A^n = \mathbf{0}$. Let $\lambda$ be an eigenvalue of $A$.
Then $A\mathbf{x} = \lambda \mathbf{x}$ for some nonzero vector $\mathbf{x}$.
By induction $\lambda^n \mathbf{x} = A^n \mathbf{x} = 0$, so $\lambda = 0$.

Conversely, suppose that all eigenvalues of $A$ are zero. Then the chararacteristic
polynomial of $A$: $\det(\lambda I - A) = \lambda^n$. It now follows from the
Cayley-Hamilton theorem that $A^n = \mathbf{0}$.
\end{proof}

Since the determinant is the product of the eigenvalues it follows that a nilpotent matrix has determinant 0. Similarly, since the trace of a square matrix is the sum of the eigenvalues, it follows that it has trace 0. 

One class of nilpotent matrices are the \PMlinkid{strictly triangular matrices}{4381} (lower or upper), this follows from the fact that the eigenvalues of a triangular matrix are the diagonal elements, and thus are all zero in the case of \emph{strictly} triangular matrices.

%\PMlinkid{strictly triangular matrices}{4381}

Note for $2\times2$ matrices $A$ the theorem implies that $A$ is nilpotent iff $A=\mathbf{0}$ or $A^2=\mathbf{0}$.

Also it is worth noticing that any matrix that is similar to a nilpotent matrix is nilpotent.
%%%%%
%%%%%
\end{document}
