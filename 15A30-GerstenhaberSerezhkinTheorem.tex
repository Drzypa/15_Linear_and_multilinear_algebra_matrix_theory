\documentclass[12pt]{article}
\usepackage{pmmeta}
\pmcanonicalname{GerstenhaberSerezhkinTheorem}
\pmcreated{2013-03-22 19:20:05}
\pmmodified{2013-03-22 19:20:05}
\pmowner{kammerer}{26336}
\pmmodifier{kammerer}{26336}
\pmtitle{Gerstenhaber - Serezhkin theorem}
\pmrecord{7}{42281}
\pmprivacy{1}
\pmauthor{kammerer}{26336}
\pmtype{Theorem}
\pmcomment{trigger rebuild}
\pmclassification{msc}{15A30}
%\pmkeywords{nilpotent matrix}
%\pmkeywords{linear subspace}
%\pmkeywords{dimension}
%\pmkeywords{simultaneous triangularizability}
\pmrelated{BottaPierceWatkinsTheorem}

\endmetadata

% this is the default PlanetMath preamble.  as your knowledge
% of TeX increases, you will probably want to edit this, but
% it should be fine as is for beginners.

% almost certainly you want these
\usepackage{amssymb}
\usepackage{amsmath}
\usepackage{amsfonts}

% used for TeXing text within eps files
%\usepackage{psfrag}
% need this for including graphics (\includegraphics)
%\usepackage{graphicx}
% for neatly defining theorems and propositions
%\usepackage{amsthm}
% making logically defined graphics
%%%\usepackage{xypic}

% there are many more packages, add them here as you need them

% define commands here
\newtheorem{thm}{Theorem}
\begin{document}
Let $\mathbb{F}$ be an arbitrary field. Consider $\mathcal{M}_n (\mathbb{F}),$ the vector space of all $n \times n$ matrices over $\mathbb{F}.$ Define
\begin{itemize}
\item
$\mathcal{N} = \{A \in \mathcal{M}_n (\mathbb{F}):\, A\, \, \mbox{is nilpotent}\},$
\item
$\mathcal{GL}_n (\mathbb{F}) = \{A \in \mathcal{M}_n (\mathbb{F}): \det (A) \neq 0\},$
\item
$\mathcal{T} = \{A \in \mathcal{M}_n (\mathbb{F}):\, A\, \, \mbox{is strictly upper triangular}\}.$
\end{itemize}

Notice that $\mathcal{T}$ is a linear subspace of $\mathcal{M}_n (\mathbb{F}).$ Moreover, $\mathcal{T} \subseteq \mathcal{N}$ and $\dim \mathcal{T} = n (n - 1) / 2.$

The Gerstenhaber -- Serezhkin theorem on linear subspaces contained in the nilpotent cone \cite{G, S} reads as follows.
\begin{thm}
Let $\mathcal{L}$ be a linear subspace of $\mathcal{M}_n (\mathbb{F}).$ Assume that $\mathcal{L} \subseteq \mathcal{N}.$ Then\\
\begin{tabular}{cl}
(i)&$\dim \mathcal{L} \leq n (n - 1) / 2,$\\
(ii)&$\dim \mathcal{L} = n (n - 1) / 2$ if and only if there exists $U \in \mathcal{GL}_n (\mathbb{F})$ such that $\{U A U^{-1}:\, A \in \mathcal{L}\} = \mathcal{T}.$\\
\end{tabular} 
\end{thm}

An alternative simple proof of inequality {\it (i)} can be found in \cite{M}.

\begin{thebibliography}{99}
\bibitem[G]{G}
M. Gerstenhaber, On nilalgebras and linear varieties of nilpotent matrices, I, {\it Amer. J. Math.} {\bf 80}: 614--622 (1958).
\bibitem[M]{M}
B. Mathes, M. Omladi\v{c}, H. Radjavi, Linear Spaces of Nilpotent Matrices, {\it Linear Algebra Appl.} {\bf 149}: 215--225 (1991).
\bibitem[S]{S}
V. N. Serezhkin, On linear transformations preserving nilpotency, {\it Vests$\bar{\iota}$ Akad. Navuk BSSR Ser. F$\bar{\iota}$z.-Mat. Navuk} {\bf 1985}, no. 6: 46--50 (Russian).  
\end{thebibliography}
%%%%%
%%%%%
\end{document}
