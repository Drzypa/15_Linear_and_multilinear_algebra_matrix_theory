\documentclass[12pt]{article}
\usepackage{pmmeta}
\pmcanonicalname{HasseMinkowskiTheorem}
\pmcreated{2013-03-22 15:19:47}
\pmmodified{2013-03-22 15:19:47}
\pmowner{SirJective}{9710}
\pmmodifier{SirJective}{9710}
\pmtitle{Hasse-Minkowski theorem}
\pmrecord{8}{37143}
\pmprivacy{1}
\pmauthor{SirJective}{9710}
\pmtype{Theorem}
\pmcomment{trigger rebuild}
\pmclassification{msc}{15A63}
\pmclassification{msc}{14G05}
\pmrelated{HassePrinciple}
\pmrelated{QuadraticForm}
\pmdefines{regular quadratic form}

% this is the default PlanetMath preamble.  as your knowledge
% of TeX increases, you will probably want to edit this, but
% it should be fine as is for beginners.

% almost certainly you want these
\usepackage{amssymb}
\usepackage{amsmath}
\usepackage{amsfonts}

% used for TeXing text within eps files
%\usepackage{psfrag}
% need this for including graphics (\includegraphics)
%\usepackage{graphicx}
% for neatly defining theorems and propositions
%\usepackage{amsthm}
% making logically defined graphics
%%%\usepackage{xypic}

% there are many more packages, add them here as you need them

% define commands here
%\theoremstyle{definition}
\newtheorem{thmplain}{Theorem}
\newtheorem{defn}{Definition}
\begin{document}
The \emph{Hasse-Minkowski theorem} is a classical example of the Hasse principle.

Let $F$ be a global field, i.e. a number field or a rational function field over a finite field of characteristic not $2$, $X$ a finite dimensional vector space over $F$ and $\phi$ a regular quadratic form over $X$.

A regular quadratic form $\phi$ over $X$ is a quadratic form such that for every $x\neq 0$ in $X$ there is a $y$ in $X$ with $b(x,y)\neq0$. Here $b(x,y)=\frac{1}{2} (q(x+y)-q(x)-q(y))$ is the associated bilinear form.

To every completion $F_v$ of $F$ with respect to a nontrivial valuation $v$ we assign the vector space $X_v := F_v \otimes_F X$ and the induced quadratic form $\phi_v$ on $X_v$.

A quadratic form $\phi$ over $X$ is an isotropic quadratic form if there is a nonzero vector $x\in X$ with $\phi(x)=0$.

The Hasse-Minkowski theorem can now be stated as:

\begin{thmplain}
A regular quadratic form $\phi$ over a global field $F$ is isotropic if and only if every completion $\phi_v$ is isotropic, where $v$ runs through the nontrivial valuations of $F$.
\end{thmplain}

The case of $\mathbb{Q}$ was first proved by Minkowski. It can be proved using the Hilbert symbol and Dirichlet's theorem on primes in arithmetic progressions.

The general case was proved by Hasse. It can be proved using two local-global principles of class field theory, namely the \emph{Hasse norm theorem}:
For a cyclic field extension $E/F$ of global fields an element $a\in F$ is a norm of $E/F$ and only if it is a norm of $E_v/F_v$ for every valuation $v$ of $E$.

and the \emph{Global square theorem}:
An element $a$ of a global field $F$ is a square if and only if it is a square in every $F_v$.
%%%%%
%%%%%
\end{document}
