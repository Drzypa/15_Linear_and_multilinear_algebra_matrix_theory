\documentclass[12pt]{article}
\usepackage{pmmeta}
\pmcanonicalname{CliffordAlgebra}
\pmcreated{2013-03-22 13:18:05}
\pmmodified{2013-03-22 13:18:05}
\pmowner{rmilson}{146}
\pmmodifier{rmilson}{146}
\pmtitle{Clifford algebra}
\pmrecord{9}{33803}
\pmprivacy{1}
\pmauthor{rmilson}{146}
\pmtype{Definition}
\pmcomment{trigger rebuild}
\pmclassification{msc}{15A66}
\pmclassification{msc}{11E88}

\endmetadata

% this is the default PlanetMath preamble.  as your knowledge
% of TeX increases, you will probably want to edit this, but
% it should be fine as is for beginners.

% almost certainly you want these
\usepackage{amssymb}
\usepackage{amsmath}
\usepackage{amsfonts}

\newcommand{\Cliff}{\operatorname{Cliff}}
\newcommand{\mc}[1]{\mathcal{#1}}
\newcommand{\R}{\mathbb{R}}
\newcommand{\C}{\mathbb{C}}
\newcommand{\Z}{\mathbb{Z}}
\newcommand{\Gr}{\operatorname{Gr}}
\begin{document}
Let $V$ be a vector space over a field $k$, and $Q:V\times V\to k$ a
symmetric bilinear form.  Then the Clifford algebra $\Cliff(Q,V)$ is
the quotient of the tensor algebra $\mc{T}(V)$ by the relations

$$v\otimes w+w\otimes v=-2Q(v,w)\qquad \forall v,w\in V.$$

Since the above relationship is not homogeneous in the usual
$\Z$-grading on $\mc{T}(V)$, $\Cliff(Q,V)$ does not inherit a
$\Z$-grading.  However, by reducing mod 2, we also have a
$\Z_2$-grading on $\mc{T}(V)$, and the relations above are homogeneous
with respect to this, so $\Cliff(Q,V)$ has a natural $\Z_2$-grading,
which makes it into a superalgebra.

In addition, we do have a filtration on $\Cliff(Q,V)$ (making it a
filtered algebra), and the associated graded algebra $\Gr\Cliff(Q,V)$
is simply $\Lambda^*V$, the exterior algebra of $V$.  In
particular, $$\dim\Cliff(Q,V)=\dim\Lambda^*V=2^{\dim V}.$$

The most commonly used Clifford algebra is the case $V=\R^n$, and $Q$
is the standard inner product with orthonormal basis $e_1,\ldots,e_n$.
In this case, the algebra is generated by $e_1,\ldots,e_n$ and the
identity of the algebra $1$, with the relations
\begin{align*}
e_i^2&=-1\\
e_ie_j&=-e_je_i \quad (i\neq j)
\end{align*}
Trivially, $\Cliff(\R^0)=\R$, and it can be seen from the relations
above that $\Cliff(\R)\cong\C$, the complex numbers, and
$\Cliff(\R^2)\cong\mathbb{H}$, the quaternions.

On the other ha
nd, for $V=\C^n$ we get the particularly \PMlinkescapetext{simple} answer of
$$\Cliff(\C^{2k}) \cong \mathrm{M}_{2^k}(\C) \qquad \Cliff(\C^{2k+1})
= \mathrm{M}_{2^k}(\C) \oplus \mathbf{M}_{2^k}(\C).$$
%%%%%
%%%%%
\end{document}
