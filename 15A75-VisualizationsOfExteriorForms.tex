\documentclass[12pt]{article}
\usepackage{pmmeta}
\pmcanonicalname{VisualizationsOfExteriorForms}
\pmcreated{2013-03-22 15:28:12}
\pmmodified{2013-03-22 15:28:12}
\pmowner{rmilson}{146}
\pmmodifier{rmilson}{146}
\pmtitle{visualizations of exterior forms}
\pmrecord{6}{37323}
\pmprivacy{1}
\pmauthor{rmilson}{146}
\pmtype{Definition}
\pmcomment{trigger rebuild}
\pmclassification{msc}{15A75}
\pmclassification{msc}{58A10}
\pmrelated{DifferentialForms}

\endmetadata


\begin{document}
There are (relatively) easy ways to visualize low-dimensional differential forms \cite{mwt}:

A 1-form is locally like a stack of papers; given a vector, it returns a number: how many sheets the arrow pierces.

A 2-form takes a pair of arrows and returns the "area" of the parallelogram they define.

A 3-form takes a triple of arrows and returns the "volume" of the parallelliped they span. This explains why in three dimensions there's only a one-dimensional space of 3-forms, and why a global one-form tells you about orientation.


\begin{thebibliography}{99}
\bibitem{mwt}
Misner, Thorne, and Wheeler, ``Gravitation'', Freeman, 1973.
\end{thebibliography}

\textbf{Editorial note:} Descriptions of these with pictures would be nice (especially for helping to visualize de Rham cohomology). Maybe they would be better off in an attached entry, though.

%%%%%
%%%%%
\end{document}
