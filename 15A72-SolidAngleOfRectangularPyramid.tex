\documentclass[12pt]{article}
\usepackage{pmmeta}
\pmcanonicalname{SolidAngleOfRectangularPyramid}
\pmcreated{2013-03-22 19:16:02}
\pmmodified{2013-03-22 19:16:02}
\pmowner{pahio}{2872}
\pmmodifier{pahio}{2872}
\pmtitle{solid angle of rectangular pyramid}
\pmrecord{16}{42198}
\pmprivacy{1}
\pmauthor{pahio}{2872}
\pmtype{Example}
\pmcomment{trigger rebuild}
\pmclassification{msc}{15A72}
\pmclassification{msc}{51M25}
\pmrelated{CyclometricFunctions}

\endmetadata

% this is the default PlanetMath preamble.  as your knowledge
% of TeX increases, you will probably want to edit this, but
% it should be fine as is for beginners.

% almost certainly you want these
\usepackage{amssymb}
\usepackage{amsmath}
\usepackage{amsfonts}

% used for TeXing text within eps files
%\usepackage{psfrag}
% need this for including graphics (\includegraphics)
%\usepackage{graphicx}
% for neatly defining theorems and propositions
 \usepackage{amsthm}
% making logically defined graphics
%%%\usepackage{xypic}
\usepackage{pstricks}
\usepackage{pst-plot}

% there are many more packages, add them here as you need them

% define commands here

\theoremstyle{definition}
\newtheorem*{thmplain}{Theorem}

\begin{document}
\PMlinkescapeword{formula} \PMlinkescapeword{length}
We calculate the apical solid angle of a \PMlinkescapetext{right} rectangular pyramid, as an example of using the \PMlinkid{formula of van Oosterom and Strackee}{7266} for determining the solid angle $\Omega$ subtended at the origin by a triangle:
\begin{align}
\tan\frac{\Omega}{2} \;=\; \frac{\vec{r}_1\!\times\!\vec{r}_2\!\cdot\!\vec{r}_3}
{(\vec{r}_1\!\cdot\!\vec{r}_2)r_3+(\vec{r}_2\!\cdot\!\vec{r}_3)r_1+(\vec{r}_3\!\cdot\!\vec{r}_1)r_2+r_1r_2r_3}
\end{align}
Here, $\vec{r}_1$, $\vec{r}_2$, $\vec{r}_3$ are the position vectors of the vertices of the triangle and 
$r_1,\,r_2,\,r_3$ their \PMlinkescapetext{lengths}.\\

Let the apex of the pyramid be in the origin and the vertices of the base rectangle be
$$(\pm a,\, \pm b,\, h)$$
where $a$, $b$ and $h$ are positive numbers.\, We take the half-triangle of the base determined by the three vertices
$$(a,\,b,\,h), \;\, (-a,\,b,\,h), \;\, (a,\,-b,\,h),$$
with the position vectors $\vec{r}_1$, $\vec{r}_2$, $\vec{r}_3$, respectively.\, Then we have in the numerator of (1) the scalar triple product 
$$\vec{r}_1\!\times\!\vec{r}_2\!\cdot\!\vec{r}_3 \;=\; 
\left|\begin{matrix}
a & b & h\\
-a & b & h\\
a & -b & h 
\end{matrix}\right| \;=\; 
a\left|\begin{matrix} 
b & h\\
-b & h
\end{matrix}\right|+ 
b\left|\begin{matrix} 
h & -a\\
h & a
\end{matrix}\right|+
h\left|\begin{matrix}
-a & b\\
a & -b
\end{matrix}\right| \;=\; 4abh.
$$
The vectors have the common length $\sqrt{a^2\!+\!b^2\!+\!h^2}$, and the denominator of (1) then attains the value 
$4h^2\sqrt{a^2\!+\!b^2\!+\!h^2}$.\, Thus the formula (1) gives
$$\tan\frac{\Omega}{2} \;=\; \frac{ab}{h\sqrt{a^2\!+\!b^2\!+\!h^2}}$$
which result may be reformulated by using the goniometric formula
$$\sin x \;=\; \frac{\tan x}{\sqrt{1+\tan^2 x}}$$
as
\begin{align}
\sin\frac{\Omega}{2} \;=\; \frac{ab}{\sqrt{(a^2\!+\!h^2)(b^2\!+\!h^2)}}.
\end{align}
Thus the whole apical solid angle of the \PMlinkid{right}{7357} rectangular pyramid is
\begin{align}
\Omega \;=\; 4\arcsin\frac{ab}{\sqrt{(a^2\!+\!h^2)(b^2\!+\!h^2)}}.
\end{align}
A variant of (3) is found in [3].\\

In the special case of a regular pyramid we have simply
\begin{align}
\Omega \;=\; 4\arcsin\frac{a^2}{a^2\!+\!h^2}
\end{align}
where $2a$ is the \PMlinkname{side}{Polygon} of the base square.\\

Note that in (2), the quotients $\frac{a}{\sqrt{a^2+h^2}}$ and $\frac{b}{\sqrt{b^2+h^2}}$ are sines of certain angles in the pyramid.

\begin{center}
\begin{pspicture}(-2.5,-0.5)(2.5,5.5)
\psdot(0,4)
\pspolygon(-2,3.5)(1,3.5)(2,4.5)(-1,4.5)
\psline(-2,3.5)(2,4.5)
\psline(1,3.5)(-1,4.5)
\psline(-2,3.5)(0,0)
\psline(1,3.5)(0,0)
\psline(2,4.5)(0,0)
\psline[linestyle=dotted](-1,4.5)(0,0)
\psline[linestyle=dashed](0,4)(0,0)
\rput(0.5,4.75){$2b$}
\rput(-1.8,4.1){$2a$}
\rput(0.2,2.4){$h$}
\rput(-2.4,-0.4){.}
\rput(+2.4,+5.4){.}
\end{pspicture}
\end{center}


\begin{thebibliography}{8}
\bibitem{O}{\sc A. van Oosterom \& J. Strackee}:\, A solid angle of a plane triangle.\; -- \emph{IEEE Trans. Biomed. Eng.} \textbf{30}:2 (1983); 125--126.
\bibitem{G}{\sc M. S. Gossman \& A. J. Pahikkala \& M. B. Rising \& P. H. McGinley}:\, Providing solid angle formalism for skyshine calculations.\; -- \emph{Journal of Applied Clinical Medical Physics} \textbf{11}:4 (2010); 278--282.
\bibitem{L}{\sc M. S. Gossman \& A. J. Pahikkala \& M. B. Rising \& P. H. McGinley}:\, Letter to the editor.\; -- \emph{Journal of Applied Clinical Medical Physics} \textbf{12}:1 (2011); 242--243.
\bibitem{R}{\sc M. S. Gossman \& M. B. Rising \& P. H. McGinley \& A. J. Pahikkala}:\, Radiation skyshine from a 6 MeV medical accelerator.\; -- \emph{Journal of Applied Clinical Medical Physics} \textbf{11}:3 (2010); 259--264.
\end{thebibliography}

%%%%%
%%%%%
\end{document}
