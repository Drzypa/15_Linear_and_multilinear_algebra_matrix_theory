\documentclass[12pt]{article}
\usepackage{pmmeta}
\pmcanonicalname{CauchyBinetFormula}
\pmcreated{2013-03-22 14:07:04}
\pmmodified{2013-03-22 14:07:04}
\pmowner{CWoo}{3771}
\pmmodifier{CWoo}{3771}
\pmtitle{Cauchy-Binet formula}
\pmrecord{11}{35523}
\pmprivacy{1}
\pmauthor{CWoo}{3771}
\pmtype{Theorem}
\pmcomment{trigger rebuild}
\pmclassification{msc}{15A15}
\pmsynonym{Binet-Cauchy formula}{CauchyBinetFormula}
\pmrelated{MinorOfAMatrix}

% this is the default PlanetMath preamble.  as your knowledge
% of TeX increases, you will probably want to edit this, but
% it should be fine as is for beginners.

% almost certainly you want these
\usepackage{amssymb}
\usepackage{amsmath}
\usepackage{amsfonts}
\usepackage{amsthm}

% used for TeXing text within eps files
%\usepackage{psfrag}
% need this for including graphics (\includegraphics)
%\usepackage{graphicx}
% for neatly defining theorems and propositions
%\usepackage{amsthm}
% making logically defined graphics
%%%\usepackage{xypic}

% there are many more packages, add them here as you need them

% define commands here
\def\sse{\subseteq}
\def\bigtimes{\mathop{\mbox{\Huge $\times$}}}
\def\impl{\Rightarrow}
\def\sgn{\operatorname{sgn}}
\begin{document}
\PMlinkescapeword{term}

Let $A$ be an $m\times n$ matrix and $B$ an $n\times m$ matrix. Then
the determinant of their product $C=AB$ can be written as a sum of products of
minors of $A$ and $B$:
\begin{equation*}
  |C| = \sum_{1\le k_1<k_2<\cdots<k_m\le n}
    A\begin{pmatrix}
      1 & 2 & \cdots & m \\
      k_1 & k_2 & \cdots & k_m
    \end{pmatrix}
    B\begin{pmatrix}
      k_1 & k_2 & \cdots & k_m \\
      1 & 2 & \cdots & m
    \end{pmatrix}.
\end{equation*}
Basically, the sum is over the maximal ($m$-th order) minors of $A$ and $B$.
See the entry on \PMlinkname{minors}{MinorOfAMatrix} for notation.

If $m>n$, then neither $A$ nor $B$ have minors of rank $m$, so $|C|=0$.
If $m=n$, this formula reduces to the usual multiplicativity of determinants
$|C|=|AB|=|A||B|$.

\begin{proof}

Since $C=AB$, we can write its elements as $c_{ij} = \sum_{k=1}^n
a_{ik} b_{kj}$. Then its determinant is
\begin{align*}
  |C| &= \begin{vmatrix}
      \sum_{k_1=1}^n a_{1 k_1} b_{k_1 1} & \cdots &
        \sum_{k_m=1}^n a_{1 k_m} b_{k_m m} \\
      \vdots & \ddots & \vdots \\
      \sum_{k_1=1}^n a_{m k_1} b_{k_1 1} & \cdots &
        \sum_{k_m=1}^n a_{m k_m} b_{k_m m}
    \end{vmatrix} \\
  &= \sum_{k_1,\ldots,k_m=1}^n \begin{vmatrix}
      a_{1 k_1} b_{k_1 1} & \cdots & a_{1k_m} b_{k_m m} \\
      \vdots & \ddots & \vdots \\
      a_{m k_1} b_{k_1 1} & \cdots & a_{mk_m} b_{k_m m}
    \end{vmatrix} \\
   &= \sum_{k_1,\ldots,k_m=1}^n
     A \begin{pmatrix} 1 & 2 & \cdots & m \\ k_1 & k_2 & \cdots
       & k_m \end{pmatrix} b_{k_1 1} b_{k_2 2} \cdots b_{k_m m}.
\end{align*}
In both steps above, we have used the property that the determinant is
multilinear in the colums of a matrix.

Note that the terms in the last sum with any two $k$'s the same will
make the minor of $A$ vanish. And, for $\{k_1,\cdots,k_m\}$'s
that differ only by a permutation, the minor of $A$ will simply change
sign according to the parity of the permutation. Hence the determinant of
$C$ can be rewritten as
\begin{align*}
  |C| &= \sum_{1\le k_1<\cdots<k_m\le n}
    A \begin{pmatrix} 1 & 2 & \cdots & m \\ k_1 & k_2 & \cdots
      & k_m \end{pmatrix}
    \sum_{\sigma\in S_m} \sgn(\sigma)\, b_{k_{\sigma(1)} 1}
      b_{k_{\sigma(2)} 2} \cdots b_{k_{\sigma(m)} m},
\end{align*}
where $S_m$ is the permutation group on $m$ elements.
But the last sum is none other than the determinant
$B \left(\begin{smallmatrix}
  k_1 & k_2 & \cdots & k_m \\
  1 & 2 & \cdots & m
\end{smallmatrix}\right)$.
Hence we write
\begin{equation*}
  |C| = \sum_{1\le k_1<\cdots<k_m\le n}
    A \begin{pmatrix} 1 & 2 & \cdots & m \\ k_1 & k_2 & \cdots
      & k_m \end{pmatrix}
    B \begin{pmatrix} k_1 & k_2 & \cdots & k_m \\
      1 & 2 & \cdots & m\end{pmatrix},
\end{equation*}
which is the Cauchy-Binet formula.
\end{proof}
%%%%%
%%%%%
\end{document}
