\documentclass[12pt]{article}
\usepackage{pmmeta}
\pmcanonicalname{RighthandedSystemOfVectors}
\pmcreated{2013-03-22 14:59:52}
\pmmodified{2013-03-22 14:59:52}
\pmowner{juanman}{12619}
\pmmodifier{juanman}{12619}
\pmtitle{right-handed system of vectors}
\pmrecord{8}{36704}
\pmprivacy{1}
\pmauthor{juanman}{12619}
\pmtype{Definition}
\pmcomment{trigger rebuild}
\pmclassification{msc}{15A72}
\pmsynonym{dextral system of vectors}{RighthandedSystemOfVectors}
\pmrelated{Helix}
\pmdefines{right-handed}

\endmetadata

% this is the default PlanetMath preamble.  as your knowledge
% of TeX increases, you will probably want to edit this, but
% it should be fine as is for beginners.

% almost certainly you want these
\usepackage{amssymb}
\usepackage{amsmath}
\usepackage{amsfonts}

% used for TeXing text within eps files
%\usepackage{psfrag}
% need this for including graphics (\includegraphics)
\usepackage{graphicx}
% for neatly defining theorems and propositions
%\usepackage{amsthm}
% making logically defined graphics
%%%\usepackage{xypic}

% there are many more packages, add them here as you need them

% define commands here
\begin{document}
Three ordered non-coplanar vectors $\vec{u}$, $\vec{v}$ and $\vec{w}$, which have a common \PMlinkescapetext{initial point}, are said to form a {\em right-handed} or {\em dextral system} (Latin {\em dexter} = \PMlinkescapetext{`right'}), if a right-threaded screw rotated through an angle less than $180^{\rm{o}}$ from $\vec{u}$ to $\vec{v}$ will advance in the direction of $\vec{w}$.\, For instance, the usual basis vectors $\vec{i}$, $\vec{j}$ and $\vec{k}$ form a dextral system.

\begin{figure}
\begin{center}
\includegraphics{rhs.eps}
\end{center}
\caption{A right-handed system of vectors}
\end{figure}

%%%%%
%%%%%
\end{document}
