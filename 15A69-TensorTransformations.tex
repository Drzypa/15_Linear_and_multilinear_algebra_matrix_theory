\documentclass[12pt]{article}
\usepackage{pmmeta}
\pmcanonicalname{TensorTransformations}
\pmcreated{2013-03-22 12:40:33}
\pmmodified{2013-03-22 12:40:33}
\pmowner{rmilson}{146}
\pmmodifier{rmilson}{146}
\pmtitle{tensor transformations}
\pmrecord{4}{32952}
\pmprivacy{1}
\pmauthor{rmilson}{146}
\pmtype{Derivation}
\pmcomment{trigger rebuild}
\pmclassification{msc}{15A69}

\newcommand{\kfield}{\mathbb{K}}
\newcommand{\rT}{\mathrm{T}}
\newcommand{\tspace}[1]{\rT^{#1}}
\newcommand{\ca}{\varepsilon}
\newcommand{\ud}[2]{^{#1}_{\!\hphantom{#1}#2}}

\usepackage{amsmath}
\usepackage{amsfonts}
\usepackage{amssymb}
\newcommand{\reals}{\mathbb{R}}
\newcommand{\natnums}{\mathbb{N}}
\newcommand{\cnums}{\mathbb{C}}
\newcommand{\znums}{\mathbb{Z}}
\newcommand{\lp}{\left(}
\newcommand{\rp}{\right)}
\newcommand{\lb}{\left[}
\newcommand{\rb}{\right]}
\newcommand{\supth}{^{\text{th}}}
\newtheorem{proposition}{Proposition}
\newtheorem{definition}[proposition]{Definition}
\newcommand{\nl}[1]{\PMlinkescapetext{{#1}}}
\newcommand{\pln}[2]{\PMlinkname{#1}{#2}}
\begin{document}
The present entry employs the terminology and notation defined
and described in the entry on tensor arrays and basic tensors.  To keep things
reasonably self contained we mention that the symbol $\tspace{p,q}$ refers
to the vector space of type $(p,q)$ tensor arrays, i.e. maps
$$I^p\times I^q\rightarrow \kfield,$$
where $I$ is some finite list of
index labels, and where $\kfield$ is a field.  The symbols $\ca_{(i)},
\ca^{(i)},\; i\in I$ refer to the column and row vectors giving the
natural basis of $\tspace{1,0}$ and $\tspace{0,1}$, respectively.

Let $I$ and $J$ be two finite lists of equal cardinality, and let
 $$T:\kfield^I\rightarrow\kfield^J$$ be a linear isomorphism.  Every
such isomorphism is uniquely represented by an invertible matrix
$$M:J\times I\rightarrow \kfield$$ with entries given by $$M\ud{j}{i}
= \lp T\ca_{(i)}\rp^j,\quad i\in I,\; j\in J.$$ In other words, the
action of $T$ is described by the following substitutions
\begin{equation}
  \label{eq:vsubrule}
  \ca_{(i)} \mapsto \sum_{j\in J} M\ud{j}{i}\, \ca_{(j)},\quad i\in I.  
\end{equation}
Equivalently, the action of $T$ is given by matrix-multiplication
of column vectors in $\kfield^I$ by $M$.  

The corresponding substitutions relations for the type $(0,1)$ tensors
involve the inverse matrix $M^{-1}: I\times J\rightarrow \kfield,$ and
take the form\footnote{The above relations describe the action of the dual homomorphism of the
inverse transformation
$$\lp T^{-1} \rp ^*: \lp \kfield^I \rp^* \rightarrow \lp \kfield^J
\rp^*.$$}
\begin{equation}
  \label{eq:lfsubrule}
  \ca^{(i)} \mapsto \sum_{j\in J} \lp M^{-1} \rp\ud{i}{j}\,
\ca^{(j)},\quad i\in I.  
\end{equation}

The rules for type $(0,1)$ substitutions are what they are, because of
the requirement that the $\ca_{(i)}$ and $\ca^{(i)}$ remain dual bases
even after the substitution.  In other words we want the substitutions
to preserve the relations
$$\ca^{(i_1)}\ca_{(i_2)} = \delta^{\,i_1}_{\;i_2},\quad i_1, i_2\in
I,$$
where the left-hand side of the above equation features the inner
product and the right-hand side the Kronecker delta.  Given that the
vector basis transforms as in \eqref{eq:vsubrule} and given the above
constraint, the substitution rules for the linear form basis, shown in
\eqref{eq:lfsubrule}, are the only such possible.






The classical terminology of contravariant and covariant indices is
motivated by thinking in term of substitutions.  Thus, suppose we
perform a linear substitution and change a vector, i.e. a type $(1,0)$
tensor, $X\in \kfield^I$ into a vector $Y\in \kfield^J$.  The indexed
values of the former and of the latter are related by
\begin{equation}
  \label{eq:visubrule}
  Y^j = \sum_{i\in I} M\ud{j}{i}\, X^i,\quad j\in J.
\end{equation}
Thus, we see
that the ``transformation rule'' for indices is contravariant to the
substitution rule \eqref{eq:vsubrule} for basis vectors.


In modern terms, this contravariance is best described by saying that
the dual space space construction is a contravariant functor\footnote{See the entry on the dual homomorphism.}.  In other
words, the substitution rule for the linear forms, i.e. the type
$(0,1)$ tensors, is contravariant to the substitution rule for
vectors:
\begin{equation}
  \label{eq:ilfsubrule}
  \ca^{(j)} \mapsto \sum_{i\in I} M\ud{j}{i}\, \ca^{(i)},\quad j\in
J,  
\end{equation}
in full agreement with the relation shown in \eqref{eq:lfsubrule}.
Everything comes together, and equations \eqref{eq:visubrule} and
\eqref{eq:ilfsubrule} are seen to be one and the same, once we remark
that tensor array values can be obtained by contracting with
characteristic arrays.  For example,
$$ X^i = \ca^{(i)}(X),\quad i\in I; \qquad Y^j = \ca^{(j)}(Y),\quad j\in J.$$


Finally we must remark that the transformation rule for covariant
indices involves the inverse matrix $M^{-1}$.   Thus if $\alpha\in
\tspace{0,1}(I)$ is transformed to a  $\beta\in \tspace{0,1}$ the
indices will be related by
$$\beta_j = \sum_{i\in I} \lp M^{-1}\rp\ud{i}{j}\, \alpha_i,\quad j\in
J.$$
The most general transformation rule for tensor array indices is
therefore the following: the indexed values of a tensor array $X\in
\tspace{p,q}(I)$  and the values of the transformed tensor array
$Y\in\tspace{p,q}(J)$ are related by
$$Y^{\!j_1\ldots j_p}_{\;l_1\ldots l_q} = \!\!
\sum_{\genfrac{}{}{0pt}{}{i_1,\ldots,i_p\in I^p}{k_1,\ldots,k_q\in
    I^q}} \!\!
 M^{j_1}_{\,i_1}\cdot\cdot\cdot M^{j_p}_{\,i_p} \lp M^{-1} \rp^{\!k_1}_{l_1}
\cdot\cdot\cdot \lp M^{-1} \rp^{\!k_q}_{l_q} X^{i_1\ldots i_p}_{k_1\ldots k_q},
$$
for all possible choice of indices
$\quad j_1,\ldots j_p,l_1,\ldots, l_q \in J.$
Debauche of indices, indeed!
%%%%%
%%%%%
\end{document}
