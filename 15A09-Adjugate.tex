\documentclass[12pt]{article}
\usepackage{pmmeta}
\pmcanonicalname{Adjugate}
\pmcreated{2013-03-22 13:09:42}
\pmmodified{2013-03-22 13:09:42}
\pmowner{rmilson}{146}
\pmmodifier{rmilson}{146}
\pmtitle{adjugate}
\pmrecord{17}{33604}
\pmprivacy{1}
\pmauthor{rmilson}{146}
\pmtype{Definition}
\pmcomment{trigger rebuild}
\pmclassification{msc}{15A09}
\pmsynonym{classical adjoint}{Adjugate}

\usepackage{amssymb}
\usepackage{amsmath}
\usepackage{amsfonts}

\newcommand{\adj}{\operatorname{adj}}
\newcommand{\adjA}{\adj(A)}
\begin{document}
The \emph{adjugate}, $\adjA$, of an $n\times n$
matrix $A$, is the $n\times n$ matrix
\begin{equation}
  \label{eq:def1}
  \adjA_{ij} = (-1)^{i+j}\, M_{\!ji}(A) 
\end{equation}
where $M_{\!ji}(A)$ is the indicated minor of $A$ (the determinant
obtained by deleting row $j$ and column $i$ from $A$).  The adjugate
is also known as the \emph{classical adjoint}, to distinguish it from
the \PMlinkname{usual usage of ``adjoint''}{AdjointEndomorphism} which
denotes the conjugate transpose operation.


An equivalent characterization of the adjugate is the following:
\begin{equation}
  \label{eq:def2}
  \adjA A = \det(A) I.
\end{equation}
The equivalence of \eqref{eq:def1} and \eqref{eq:def2} follows easily
from the \PMlinkname{multi-linearity
  properties}{DeterminantAsAMultilinearMapping} of the determinant.
Thus, the adjugate operation is closely related to the matrix inverse.
Indeed, if $A$ is invertible, the adjugate can be defined as
\[ \adjA = \det(A)A^{-1} \]

Yet another definition of the adjugate is the following:
\begin{align}
  \label{eq:def3}
  \adjA = p_{n-1}(A) I &- p_{n-2}(A) A + p_{n-3}(A) A^2 - \cdots \\ \nonumber
  & +  (-1)^{n-2}p_1(A) A^{n-2} + (-1)^{n-1}A^{n-1},
\end{align}
where $p_1(A)=\operatorname{tr}(A), p_2(A),\ldots, p_n(A) = \det(A)$
are the elementary invariant polynomials of
$A$.  The latter arise as
coefficients in the
characteristic polynomial $p(t)$ of $A$, namely
\[p(t) = \det(t I - A) = t^n - p_1(A) t^{n-1} + \cdots + (-1)^n p_n(A).\]
The equivalence of \eqref{eq:def2} and \eqref{eq:def3} follows from
the Cayley-Hamilton theorem.  The latter states that $p(A)=0$, which
in turn implies that
 \[A ( A^{n-1} - p_1(A) A^{n-2} + \cdots + (-1)^{n-1} p_{n-1}(A) ) =
 (-1)^{n-1} \det(A) I\]

The adjugate operation enjoys  a number of notable
properties:
\begin{align}
  &\adj(AB) =\adj(B)\adj(A),\\
  &\adj(A^t) = \adjA^t,\\
  &\det(\adjA) = \det(A)^{n-1}.
\end{align}
%%%%%
%%%%%
\end{document}
