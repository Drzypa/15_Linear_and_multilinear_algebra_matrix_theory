\documentclass[12pt]{article}
\usepackage{pmmeta}
\pmcanonicalname{EuclideanVectorSpace}
\pmcreated{2013-03-22 15:38:24}
\pmmodified{2013-03-22 15:38:24}
\pmowner{rmilson}{146}
\pmmodifier{rmilson}{146}
\pmtitle{Euclidean vector space}
\pmrecord{9}{37571}
\pmprivacy{1}
\pmauthor{rmilson}{146}
\pmtype{Definition}
\pmcomment{trigger rebuild}
\pmclassification{msc}{15A63}
\pmrelated{InnerProductSpace}
\pmrelated{UnitarySpace}
\pmrelated{PositiveDefinite}
\pmrelated{EuclideanDistance}
\pmrelated{Vector}
\pmrelated{EuclideanVectorSpace}

\usepackage{amsmath}
\usepackage{amsfonts}
\usepackage{amssymb}
\newcommand{\reals}{\mathbb{R}}
\newcommand{\natnums}{\mathbb{N}}
\newcommand{\cnums}{\mathbb{C}}
\newcommand{\znums}{\mathbb{Z}}
\newcommand{\lp}{\left(}
\newcommand{\rp}{\right)}
\newcommand{\lb}{\left[}
\newcommand{\rb}{\right]}
\newcommand{\supth}{^{\text{th}}}
\newtheorem{proposition}{Proposition}
\newtheorem{definition}[proposition]{Definition}

\newtheorem{theorem}[proposition]{Theorem}
\begin{document}
\section{Definition}
The term \emph{Euclidean vector space} is synonymous with \emph{finite-dimensional, real, positive definite, inner product space}.  The canonical example is $\reals^n$, equipped with the usual dot product.  Indeed, every Euclidean vector space $V$  is isomorphic to $\reals^n$, up to a choice of orthonormal basis of $V$.    As well, every  Euclidean vector space $V$ carries a natural metric space structure given by
$$ d(u,v) = \sqrt{\left< u-v, u-v \right>},\quad u,v\in V.$$

\section{Remarks.}
\begin{itemize}
\item An analogous object with complex numbers as the base field is called a unitary space.
\item Dropping the assumption of finite-dimensionality we arrive at the class of  real pre-Hilbert spaces.
\item  If we drop the inner product and the vector space structure, but retain the metric space structure, we arrive at the notion of a Euclidean space.
\end{itemize}
%%%%%
%%%%%
\end{document}
