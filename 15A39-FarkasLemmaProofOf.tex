\documentclass[12pt]{article}
\usepackage{pmmeta}
\pmcanonicalname{FarkasLemmaProofOf}
\pmcreated{2013-03-22 14:12:51}
\pmmodified{2013-03-22 14:12:51}
\pmowner{CWoo}{3771}
\pmmodifier{CWoo}{3771}
\pmtitle{Farkas lemma, proof of}
\pmrecord{13}{35649}
\pmprivacy{1}
\pmauthor{CWoo}{3771}
\pmtype{Proof}
\pmcomment{trigger rebuild}
\pmclassification{msc}{15A39}

% this is the default PlanetMath preamble.  as your knowledge
% of TeX increases, you will probably want to edit this, but
% it should be fine as is for beginners.

% almost certainly you want these
\usepackage{amssymb}
\usepackage{amsmath}
\usepackage{amsfonts}

% used for TeXing text within eps files
%\usepackage{psfrag}
% need this for including graphics (\includegraphics)
%\usepackage{graphicx}
% for neatly defining theorems and propositions
%\usepackage{amsthm}
% making logically defined graphics
%%%\usepackage{xypic}

% there are many more packages, add them here as you need them

% define commands here
\begin{document}
We begin by showing that at least one of the systems 
has a solution.  

Suppose that system 2
has no solution.  Let $S$ be the cone in $\mathbb{R}^n$ generated by nonnegative linear combinations of the rows $a_1,\dots, a_m$ of $A$.  The set $S$ is closed and convex.  Since system 2 is unsolvable, the vector $c$ is not in $S$; therefore, there exist a scalar $\alpha$ and $n$-column vector $x$ such that the hyperplane $x^T v=\alpha$ separates $c$ from $S$ in $\mathbb{R}^n$.  This hyperplane can be selected so that for any point $v\in S$,
\[
x^T c^T > \alpha > x^T v^T.
\]
Since $0\in S$, this implies that $\alpha>0$.  Hence for any $w\ge 0$,
\[
\alpha > x^T (wA)^T = wAx = \sum_{i=0}^m w_i (Ax)_i.
\]

Each $(Ax)_i$ is nonpositive.  Otherwise, by selecting $w$ with $w_i$ sufficiently large and all other $w_j=0$, we would get a contradiction.
We have now shown that $x$ satisfies
$Ax\leq 0$ and $cx =(cx)^T=x^T c^T > \alpha > 0$, which means that $x$ is a solution of system 1.  Thus at least one of the systems is solvable.

We claim that systems 1 and 2 are not simultaneously solvable.  Suppose
that $x$ is a solution of system 1 and $w$ is a solution of system 2.  Then
for each $i$, the inequality $w_i (Ax)_i\le 0$ holds, and so $w(Ax)\le 0$.
However,
\[
(wA)x=cx>0,
\]
a contradiction.  This completes the proof.
%%%%%
%%%%%
\end{document}
