\documentclass[12pt]{article}
\usepackage{pmmeta}
\pmcanonicalname{LevyDesplanquesTheorem}
\pmcreated{2013-03-22 15:34:50}
\pmmodified{2013-03-22 15:34:50}
\pmowner{Andrea Ambrosio}{7332}
\pmmodifier{Andrea Ambrosio}{7332}
\pmtitle{Levy-Desplanques theorem}
\pmrecord{9}{37490}
\pmprivacy{1}
\pmauthor{Andrea Ambrosio}{7332}
\pmtype{Theorem}
\pmcomment{trigger rebuild}
\pmclassification{msc}{15-00}

% this is the default PlanetMath preamble.  as your knowledge
% of TeX increases, you will probably want to edit this, but
% it should be fine as is for beginners.

% almost certainly you want these
\usepackage{amssymb}
\usepackage{amsmath}
\usepackage{amsfonts}

% used for TeXing text within eps files
%\usepackage{psfrag}
% need this for including graphics (\includegraphics)
%\usepackage{graphicx}
% for neatly defining theorems and propositions
%\usepackage{amsthm}
% making logically defined graphics
%%%\usepackage{xypic}

% there are many more packages, add them here as you need them

% define commands here
\begin{document}
A strictly diagonally dominant matrix is non-singular. In other words, let $A\in\mathbf{C}^{n,n}$ be a matrix satisfying the property
\[
\left|a_{ii}\right|>\sum_{j\ne i}\left|a_{ij}\right|\qquad \forall i;
\]
then $\det(A)\ne 0$.

Proof:
Let $\det(A)=0$; then a non-zero vector $\mathbf{x}$ exists such that $A\mathbf{x}=\mathbf{0}$; let $M$ be the index such that $\left|x_M\right|=\max(\left|x_1\right|,\left|x_2\right|,\cdots,\left|x_n\right|)$, so that $\left|x_j\right|\leq\left|x_M\right|\quad \forall j$; we have

$a_{M1}x_1+a_{M2}x_2+\cdots+a_{MM}x_M+\cdots+a_{Mn}x_n=0$

which implies:

$\left|a_{MM}\right|\left|x_M\right|=\left|a_{MM}x_M\right|=\left|\sum_{j\ne M}a_{Mj}x_j\right|\leq\sum_{j\ne M}\left|a_{Mj}\right|\left|x_j\right|\leq\left|x_M\right|\sum_{j\ne M}\left|a_{Mj}\right|$

that is

$\left|a_{MM}\right|\leq\sum_{j\ne M}\left|a_{Mj}\right|,$

in contrast with strictly diagonally dominance definition.$\square$


Remark:
the Levy-Desplanques theorem is equivalent to the well-known Gerschgorin circle theorem. In fact, let's assume Levy-Desplanques theorem is true, and let $A$ a $n\times n$ complex-valued matrix, with an eigenvalue $\lambda$; let's apply Levy-Desplanques theorem to the matrix $B=A-\lambda I$, which is singular by definition of eigenvalue: an index $i$ must exist for which $\left|a_{ii}-\lambda\right|=\left|b_{ii}\right|\leq\sum_{j\ne i}^n \left|b_{ij}\right|=\sum_{j\ne i}^n \left|a_{ij}\right|$, which is Gerschgorin circle theorem.
On the other hand, let's assume Gerschgorin circle theorem is true, and let $A$ be a strictly diagonally dominant $n\times n$ complex matrix. Then, since the absolute value of each disc center $\left|a_{ii}\right|$ is strictly greater than the same disc radius $\sum_{j\ne i}^n \left|a_{ij}\right|$, the point $\lambda=0$ can't belong to any circle, so it doesn't belong to the spectrum of $A$, which therefore can't be singular.
%%%%%
%%%%%
\end{document}
