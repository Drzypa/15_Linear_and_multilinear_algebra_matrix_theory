\documentclass[12pt]{article}
\usepackage{pmmeta}
\pmcanonicalname{ParameterizationOfEquitableMatrices}
\pmcreated{2013-03-22 14:58:36}
\pmmodified{2013-03-22 14:58:36}
\pmowner{rspuzio}{6075}
\pmmodifier{rspuzio}{6075}
\pmtitle{parameterization of equitable matrices}
\pmrecord{6}{36678}
\pmprivacy{1}
\pmauthor{rspuzio}{6075}
\pmtype{Theorem}
\pmcomment{trigger rebuild}
\pmclassification{msc}{15-00}

% this is the default PlanetMath preamble.  as your knowledge
% of TeX increases, you will probably want to edit this, but
% it should be fine as is for beginners.

% almost certainly you want these
\usepackage{amssymb}
\usepackage{amsmath}
\usepackage{amsfonts}

% used for TeXing text within eps files
%\usepackage{psfrag}
% need this for including graphics (\includegraphics)
%\usepackage{graphicx}
% for neatly defining theorems and propositions
%\usepackage{amsthm}
% making logically defined graphics
%%%\usepackage{xypic}

% there are many more packages, add them here as you need them

% define commands here
\begin{document}
A $n \times n$ matrix is \emph{equitable} if and only if it can be expressed in the form
 $$m_{ij} = \exp (\lambda_i - \lambda_j)$$
for real numbers $\lambda_1, \lambda_2, \ldots, \lambda_n$ with $\lambda_1 = 0$.

\PMlinkescapetext{{\sl Proof:}} \,
Assume that $m_{ij}$ are the entries of an equitable matrix.

Since all the elements of an equitable matrix are positive by definition, we can write
 $$m_{ij} = \exp \mu_{ij}$$
with the quantities $\mu_{ij}$ being real numbers (which may be positive, negative or zero).

In terms of this representation, the defining identity for an equitable matrix becomes
 $$\mu_{ik} = \mu_{ij} + \mu_{jk}$$
Since this comprises a system of linear equations for the quantities $\mu_{ij}$, we could solve it using the usual methods of matrix theory.  However, for this particular system of linear equations, there is a much simpler approach.

Consider the special case of the identity when $i=j=k$:
 $$\mu_{ii} = \mu_{ii} + \mu_{ii}.$$
This simplifies to 
 $$\mu_{ii} = 0.$$
In other words, all the diagonal entries are zero.

Consider the case when $i = k$ (but does not equal $j$).
 $$\mu_{ij} + \mu_{ji} = \mu_{ii}$$
By wat we have just shown, the right hand side of this equation equals zero.  Hence, we have
 $$\mu_{ij} = - \mu_{ji}.$$
In other words, the matrix of $\mu$'s is antisymmetric.

We may express any entry in terms of the $n$ entries $\mu_{i1}$:
 $$\mu_{ij} = \mu_{i1} + \mu_{1j} = \mu_{i1} - \mu_{j1}$$

We will conclude by noting that if, given any $n$ numbers $\lambda_i$ with $\lambda_1 = 0$, but the remaining $\lambda$'s arbitrary, we define
 $$\mu_{ij} = \lambda_i - \lambda_j,$$
then
 $$\mu_{ij} + \mu_{jk} = \lambda_i - \lambda_j + \lambda_j - \lambda_k = \lambda_i - \lambda_k = \mu_{ik}$$
Hence, we obtain a solution of the equations
 $$\mu_{ik} = \mu_{ij} + \mu_{jk}.$$
Moreover, by what we what we have seen, if we set $\lambda_i = \mu_{i1}$, all solutions of these equations can be so described.
\rightline{Q.E.D.}
%%%%%
%%%%%
\end{document}
