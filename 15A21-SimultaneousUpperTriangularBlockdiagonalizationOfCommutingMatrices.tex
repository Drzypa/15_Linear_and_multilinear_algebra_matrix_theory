\documentclass[12pt]{article}
\usepackage{pmmeta}
\pmcanonicalname{SimultaneousUpperTriangularBlockdiagonalizationOfCommutingMatrices}
\pmcreated{2013-03-22 15:29:35}
\pmmodified{2013-03-22 15:29:35}
\pmowner{lars_h}{9802}
\pmmodifier{lars_h}{9802}
\pmtitle{simultaneous upper triangular block-diagonalization of commuting matrices}
\pmrecord{4}{37350}
\pmprivacy{1}
\pmauthor{lars_h}{9802}
\pmtype{Theorem}
\pmcomment{trigger rebuild}
\pmclassification{msc}{15A21}
%\pmkeywords{diagonalization}
%\pmkeywords{diagonalisation}
%\pmkeywords{commuting matrices}
\pmrelated{JordanCanonicalFormTheorem}
\pmrelated{IfABInM_nmathbbCABBAThenQHAQT_1QHBQT_2}

\usepackage{amsmath,amsfonts,amssymb,amsthm}

\newtheorem{theorem}{Theorem}
\newtheorem{lemma}[theorem]{Lemma}


\newcommand{\mc}{\mathcal}
\newcommand{\vek}{\mathbf}

\newcommand{\Mat}{\mathrm{M}}
\newcommand{\GL}{\mathrm{GL}}
\newcommand{\Trans}[1]{#1^{\mathrm{T}}\!}
\begin{document}
Let $\vek{e}_i$ denote the (column) vector whose $i$th position is $1$ 
and where all other positions are $0$. Denote by $[n]$ the set 
$\{1,\dotsc,n\}$. Denote by $\Mat_n(\mc{K})$ the set of all $n \times 
n$ matrices over $\mc{K}$, and by $\GL_n(\mc{K})$ the set of all 
invertible elements of $\Mat_n(\mc{K})$.

\begin{theorem}
  Let $\mc{K}$ be a field, let \(A_1,\dotsc,A_r \in \Mat_n(\mc{K})\) 
  be pairwise commuting matrices, and let $\mc{L}$ be a field extension 
  of $\mc{K}$ in which the characteristic polynomials of all $A_k$ 
  split. Then there exists an equivalence relation $\sim$ on $[n]$ and 
  a matrix \(R \in \GL_n(\mc{L})\) such that:
  \begin{enumerate}
    \item
      If \(i \sim j\) and \(i \leqslant k \leqslant j\) then 
      \(k \sim i\).
    \item
      If \(i \sim j\) then \(\Trans{\vek{e}_i} R^{-1} A_k R 
      \vek{e}_i = \Trans{\vek{e}_j} R^{-1} A_k R \vek{e}_j\).
    \item
      If \(\Trans{\vek{e}_i} R^{-1} A_k R \vek{e}_j \neq 0\) then 
      \(i \leqslant j\) and \(i \sim j\).
  \end{enumerate}
  In other words there exists a simultaneous upper triangular 
  block-diagonalisation of the matrices $A_1,\dotsc,A_r$ in which each 
  block is characterised by the particular values of the diagonal 
  elements.
\end{theorem}

The proof of this theorem is the obvious combination of the following 
two lemmas.

\begin{lemma} 
  Let $\mc{K}$ be a field, let \(A_1,\dotsc,A_r \in \Mat_n(\mc{K})\) 
  be pairwise commuting matrices, and let $\mc{L}$ be a field extension 
  of $\mc{K}$ in which the characteristic polynomials of all $A_k$ 
  split. Then there exists some \(P \in \GL_n(\mc{L})\) such that
  \begin{enumerate}
    \item
      \(P^{-1} A_k P\) is upper triangular for all \(k=1,\dotsc,r\),
      and
    \item
      if \(i,j,l \in [n]\) are such that \(i \leqslant l \leqslant j\) 
      and \(\Trans{\vek{e}_i} P^{-1} A_k P \vek{e}_i = 
      \Trans{\vek{e}_j} P^{-1} A_k P \vek{e}_j\) for all 
      \(k=1,\dotsc,r\), then \(\Trans{\vek{e}_l} P^{-1} A_k P 
      \vek{e}_l = \Trans{\vek{e}_j} P^{-1} A_k P \vek{e}_j\) 
      for all \(k=1,\dotsc,r\) as well.
  \end{enumerate}
\end{lemma}

\noindent
Let \(B_k = P^{-1} A_k P\) for all \(k=1,\dotsc,r\) and define
\begin{equation*}
  i \sim j
  \quad\text{if and only if}\quad
  \Trans{\vek{e}_i} P^{-1} A_k P \vek{e}_i = 
    \Trans{\vek{e}_j} P^{-1} A_k P \vek{e}_j
  \text{ for all }k \in [r]\text{.}
\end{equation*}

\begin{lemma}
  Let $\mc{L}$ be a field, let $n$ be a positive integer, and let $\sim$ 
  be an equivalence relation on $[n]$ such that if \(i \sim j\) and 
  \(i \leqslant k \leqslant j\) then \(k \sim i\). Let \(B_1,\dotsc,B_r 
  \in \Mat_n(\mc{L})\) be pairwise commuting upper triangular matrices. 
  If these matrices and $\sim$ are related such that
  \begin{equation*}
    i \sim j
    \quad\text{if and only if}\quad
    \Trans{\vek{e}_i} B_k \vek{e}_i = \Trans{\vek{e}_j} B_k \vek{e}_j
    \text{ for all }k \in [r]\text{,}
  \end{equation*}
  then there exists a matrix \(Q \in \GL_n(\mc{L})\) such that:
  \begin{enumerate}
    \item
      If \(\Trans{\vek{e}_i} Q^{-1} B_k Q \vek{e}_j \neq 0\) then 
      \(i \sim j\) and \(i \leqslant j\).
    \item
      If \(i \sim j\) then \(\Trans{\vek{e}_i} Q^{-1} B_k Q 
      \vek{e}_j = \Trans{\vek{e}_i} B_k \vek{e}_j\).
  \end{enumerate}
\end{lemma}

\noindent The wanted $R$ is then $PQ$.
%%%%%
%%%%%
\end{document}
