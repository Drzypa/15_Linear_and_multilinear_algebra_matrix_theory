\documentclass[12pt]{article}
\usepackage{pmmeta}
\pmcanonicalname{ComplexificationOfVectorSpace}
\pmcreated{2013-03-22 15:24:33}
\pmmodified{2013-03-22 15:24:33}
\pmowner{stevecheng}{10074}
\pmmodifier{stevecheng}{10074}
\pmtitle{complexification of vector space}
\pmrecord{9}{37249}
\pmprivacy{1}
\pmauthor{stevecheng}{10074}
\pmtype{Definition}
\pmcomment{trigger rebuild}
\pmclassification{msc}{15A04}
\pmclassification{msc}{15A03}
%\pmkeywords{complex matrix}
\pmrelated{ComplexStructure2}
\pmdefines{complexification}

% this is the default PlanetMath preamble.  as your knowledge
% of TeX increases, you will probably want to edit this, but
% it should be fine as is for beginners.

% almost certainly you want these
\usepackage{amssymb}
\usepackage{amsmath}
\usepackage{amsfonts}

% used for TeXing text within eps files
%\usepackage{psfrag}
% need this for including graphics (\includegraphics)
%\usepackage{graphicx}
% for neatly defining theorems and propositions
%\usepackage{amsthm}
% making logically defined graphics
%%%\usepackage{xypic}

% there are many more packages, add them here as you need them
\usepackage{enumerate}

% define commands here
\newcommand{\real}{\mathbb{R}}
\newcommand{\complex}{\mathbb{C}}
\newcommand{\rat}{\mathbb{Q}}
\newcommand{\nat}{\mathbb{N}}

\newcommand{\Vc}{V^\mathbb{C}}
\newcommand{\Wc}{W^\mathbb{C}}
\newcommand{\Tc}{T^\mathbb{C}}


\providecommand{\abs}[1]{\lvert#1\rvert}
\providecommand{\absW}[1]{\left\lvert#1\right\rvert}
\providecommand{\absB}[1]{\Bigl\lvert#1\Bigr\rvert}
\providecommand{\norm}[1]{\lVert#1\rVert}
\providecommand{\normW}[1]{\left\lVert#1\right\rVert}
\providecommand{\normB}[1]{\Bigl\lVert#1\Bigr\rVert}
\providecommand{\defnterm}[1]{\emph{#1}}
\begin{document}
\PMlinkescapeword{fix}
\PMlinkescapeword{representation}
\PMlinkescapeword{arguments}
\PMlinkescapeword{bases}
\PMlinkescapeword{mean}

\subsection{Complexification of vector space}

If $V$ is a real vector space,
its \defnterm{complexification} $\Vc$
is the complex vector space consisting of elements $x + iy$, where $x, y \in V$. Vector addition and scalar multiplication by complex numbers
are defined in the obvious way:
\begin{align*}
(x + iy) + (u+iv) &= (x+u) + i(y+v)\,, & x, y, u, v \in V \\
(\alpha + i\beta) (x+iy) &= (\alpha x - \beta y) + i(\beta x + \alpha y), & x, y \in V, \alpha, \beta \in \real\,.
\end{align*}

If $v_1, \dotsc, v_n$ is a basis for $V$, then $v_1 + i0, \dotsc, v_n + i0$
is a basis for $\Vc$.  Naturally, $x + i0 \in \Vc$ is often written just as $x$.

So, for example, the complexification of $\real^n$ is (isomorphic to) $\complex^n$.

\subsection{Complexification of linear transformation}

If $T\colon V \to W$ is a linear transformation between two real vector spaces $V$ and $W$,
its complexification $\Tc\colon \Vc \to \Wc$
is defined by
\begin{align*}
\Tc (x+iy) = Tx + iTy\,.
\end{align*}
It may be readily verified that $\Tc$ is complex-linear.

If $v_1, \dotsc, v_n$ is a basis for $V$, $w_1, \dotsc, w_m$ is a basis for $W$,
and $A$ is the matrix representation of $T$ with respect to these bases,
then $A$, regarded as a complex matrix,
is also the representation of $\Tc$ with respect to the corresponding bases
in $\Vc$ and $\Wc$.

So, the complexification process is a formal, coordinate-free way of saying: 
take the matrix $A$ of $T$, with its real entries,
but operate on it as a complex matrix.  The advantage of making this abstracted definition is that
we are not required to fix a choice of coordinates and use matrix representations
when otherwise there is no need to.
For example, we might want to make arguments about the complex eigenvalues
and eigenvectors for a transformation $T\colon V \to V$, while,
of course, non-real eigenvalues and eigenvectors, by definition, cannot exist
for a transformation between real vector spaces.
What we really mean are the eigenvalues and eigenvectors of $\Tc$.

Also, the complexification process generalizes without change for infinite-dimensional 
spaces.

\subsection{Complexification of inner product}

Finally,
if $V$ is also a real inner product space,
its real inner product can be extended to a complex inner product for $\Vc$ by
the obvious expansion:
\[
\langle x+iy, u+iv \rangle = \langle x, u \rangle + \langle y, v \rangle + i(\langle y, u \rangle - \langle x, v \rangle)\,.
\]
It follows that $\norm{x+iy}^2 = \norm{x}^2 + \norm{y}^2$.

\subsection{Complexification of norm}

More generally, for a real normed vector space $V$, 
the equation 
\[
\norm{x+iy}^2 = \norm{x}^2 + \norm{y}^2
\]
can serve as a definition of the norm for $\Vc$.

\begin{thebibliography}{3}
\bibitem{arnold} Vladimir I. Arnol'd. {\it Ordinary Differential Equations}. Springer-Verlag, 1992.
\end{thebibliography}
%%%%%
%%%%%
\end{document}
