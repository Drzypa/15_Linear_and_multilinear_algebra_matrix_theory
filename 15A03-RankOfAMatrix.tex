\documentclass[12pt]{article}
\usepackage{pmmeta}
\pmcanonicalname{RankOfAMatrix}
\pmcreated{2013-03-22 19:22:42}
\pmmodified{2013-03-22 19:22:42}
\pmowner{CWoo}{3771}
\pmmodifier{CWoo}{3771}
\pmtitle{rank of a matrix}
\pmrecord{15}{42332}
\pmprivacy{1}
\pmauthor{CWoo}{3771}
\pmtype{Definition}
\pmcomment{trigger rebuild}
\pmclassification{msc}{15A03}
\pmclassification{msc}{15A33}
\pmrelated{DeterminingRankOfMatrix}
\pmdefines{left row rank}
\pmdefines{left column rank}
\pmdefines{right row rank}
\pmdefines{right column rank}

\endmetadata

\usepackage{amssymb,amscd}
\usepackage{amsmath}
\usepackage{amsfonts}
\usepackage{mathrsfs}

% used for TeXing text within eps files
%\usepackage{psfrag}
% need this for including graphics (\includegraphics)
%\usepackage{graphicx}
% for neatly defining theorems and propositions
\usepackage{amsthm}
% making logically defined graphics
%%\usepackage{xypic}
\usepackage{pst-plot}

% define commands here
\newcommand*{\abs}[1]{\left\lvert #1\right\rvert}
\newtheorem{prop}{Proposition}
\newtheorem{thm}{Theorem}
\newtheorem{ex}{Example}
\newcommand{\real}{\mathbb{R}}
\newcommand{\pdiff}[2]{\frac{\partial #1}{\partial #2}}
\newcommand{\mpdiff}[3]{\frac{\partial^#1 #2}{\partial #3^#1}}

\begin{document}
Let $D$ be a division ring, and $M$ an $m\times n$ matrix over $D$.  There are four numbers we can associate with $M$:
\begin{enumerate}
\item the dimension of the subspace spanned by the columns of $M$ viewed as elements of the $n$-dimensional right vector space over $D$.
\item the dimension of the subspace spanned by the columns of $M$ viewed as elements of the $n$-dimensional left vector space over $D$.
\item the dimension of the subspace spanned by the rows of $M$ viewed as elements of the $m$-dimensional right vector space over $D$.
\item the dimension of the subspace spanned by the rows of $M$ viewed as elements of the $m$-dimensional left vector space over $D$.
\end{enumerate}
The numbers are respectively called the \emph{right column rank}, \emph{left column rank}, \emph{right row rank}, and \emph{left row rank} of $M$, and they are respectively denoted by $\operatorname{rc.rnk}(M)$, $\operatorname{lc.rnk}(M)$, $\operatorname{rr.rnk}(M)$, and $\operatorname{lr.rnk}(M)$.

Since the columns of $M$ are the rows of its transpose $M^T$, we have
$$\operatorname{lc.rnk}(M)=\operatorname{lr.rnk}(M^T), \qquad \textrm{and} \qquad  \operatorname{rc.rnk}(M)=\operatorname{rr.rnk}(M^T).$$  

In addition, it can be shown that for a given matrix $M$, $$\operatorname{lc.rnk}(M)=\operatorname{rr.rnk}(M), \qquad \textrm{and} \qquad  \operatorname{rc.rnk}(M)=\operatorname{lr.rnk}(M).$$
For any $0\ne r\in D$, it is also easy to see that the left column and row ranks of $rM$ are the same as those of $M$.  Similarly, the right column and row ranks of $Mr$ are the same as those of $M$.

If $D$ is a field, $\operatorname{lc.rnk}(M)=\operatorname{rc.rnk}(M)$, so that all four numbers are the same, and we simply call this number the \emph{rank} of $M$, denoted by $\operatorname{rank}(M)$.  

Rank can also be defined for matrices $M$ (over a fixed $D$) that satisfy the identity $M=rM^T$, where $r$ is in the center of $D$.  Matrices satisfying the identity include symmetric and anti-symmetric matrices.

However, the left column rank is not necessarily the same as the right row rank of a matrix, if the underlying division ring is not commutative, as can be shown in the following example: let $u=(1,j)$ and $v=(i,k)$ be vectors over the Hamiltonian quaternions $\mathbb{H}$.  They are columns in the $2\times 2$ matrix
$$ M:=\begin{pmatrix}
1 & i \\
j & k
\end{pmatrix} $$
Since $iu=(i,ij)=(i,k)=v$, they are left linearly dependent, and therefore the left column rank of $M$ is $1$.  Now, suppose $ur+vs=(0,0)$, with $r,s\in \mathbb{H}$.  Since $ui=(i,ji)=(i,-k)$, then $ui(-ir)+vs=0$, which boils down to two equations $ir=s$ and $-ir=s$, and which imply that $s=r=0$, showing that $u,v$ are right linearly independent.  Thus the right column rank of $M$ is $2$.

%%%%%
%%%%%
\end{document}
