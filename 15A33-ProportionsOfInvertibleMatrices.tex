\documentclass[12pt]{article}
\usepackage{pmmeta}
\pmcanonicalname{ProportionsOfInvertibleMatrices}
\pmcreated{2013-03-22 15:57:35}
\pmmodified{2013-03-22 15:57:35}
\pmowner{Algeboy}{12884}
\pmmodifier{Algeboy}{12884}
\pmtitle{proportions of invertible matrices}
\pmrecord{16}{37973}
\pmprivacy{1}
\pmauthor{Algeboy}{12884}
\pmtype{Theorem}
\pmcomment{trigger rebuild}
\pmclassification{msc}{15A33}
\pmsynonym{proportions invertible linear transformations}{ProportionsOfInvertibleMatrices}

\endmetadata

\usepackage{latexsym}
\usepackage{amssymb}
\usepackage{amsmath}
\usepackage{amsfonts}
\usepackage{amsthm}

%%\usepackage{xypic}

%-----------------------------------------------------

%       Standard theoremlike environments.

%       Stolen directly from AMSLaTeX sample

%-----------------------------------------------------

%% \theoremstyle{plain} %% This is the default

\newtheorem{thm}{Theorem}

\newtheorem{coro}[thm]{Corollary}

\newtheorem{lem}[thm]{Lemma}

\newtheorem{lemma}[thm]{Lemma}

\newtheorem{prop}[thm]{Proposition}

\newtheorem{conjecture}[thm]{Conjecture}

\newtheorem{conj}[thm]{Conjecture}

\newtheorem{defn}[thm]{Definition}

\newtheorem{remark}[thm]{Remark}

\newtheorem{ex}[thm]{Example}



%\countstyle[equation]{thm}



%--------------------------------------------------

%       Item references.

%--------------------------------------------------


\newcommand{\exref}[1]{Example-\ref{#1}}

\newcommand{\thmref}[1]{Theorem-\ref{#1}}

\newcommand{\defref}[1]{Definition-\ref{#1}}

\newcommand{\eqnref}[1]{(\ref{#1})}

\newcommand{\secref}[1]{Section-\ref{#1}}

\newcommand{\lemref}[1]{Lemma-\ref{#1}}

\newcommand{\propref}[1]{Prop\-o\-si\-tion-\ref{#1}}

\newcommand{\corref}[1]{Cor\-ol\-lary-\ref{#1}}

\newcommand{\figref}[1]{Fig\-ure-\ref{#1}}

\newcommand{\conjref}[1]{Conjecture-\ref{#1}}


% Normal subgroup or equal.

\providecommand{\normaleq}{\unlhd}

% Normal subgroup.

\providecommand{\normal}{\lhd}

\providecommand{\rnormal}{\rhd}
% Divides, does not divide.

\providecommand{\divides}{\mid}

\providecommand{\ndivides}{\nmid}


\providecommand{\union}{\cup}

\providecommand{\bigunion}{\bigcup}

\providecommand{\intersect}{\cap}

\providecommand{\bigintersect}{\bigcap}










\begin{document}
Let $GL(d,R)$ denote the invertible $d\times d$-matrices over a ring $R$, and
$M_d(R)$ the set of all $d\times d$-matrices over $R$.  When $R$ is a finite field of order $q$, commonly denoted $GF(q)$ or $\mathbb{F}_q$, we prefer to write simply $q$.  In particular, $q$ is a power of a prime.

\begin{prop}
\[
\frac{|GL(d,q)|}{|M_d(q)|}=\prod_{i=1}^d \left( 1-\frac{1}{q^i}\right).
\]
\end{prop}
\begin{proof}
The number of $d\times d$-matrices over a $GF(q)$ is $q^{d^2}$.  When a matrix
is invertible, its rows form a basis of the vector space $GF(q)^d$ and this 
leads to the following formula 
\[|GL(d,q)|=q^{\binom{d}{2}}\prod_{i=1}^d (q^i-1).\]
(Refer to \PMlinkname{order of the general linear group.}{OrdersAndStructureOfClassicalGroups})

Now we prove the ratio holds:
\[
\prod_{i=1}^d \left( 1-\frac{1}{q^i}\right)
= \prod_{i=1}^d \frac{q^i-1}{q^i}
= \frac{1}{q^{\binom{d+1}{2}}}\prod_{i=1}^d (q^i-1)
= \frac{1}{q^{d^2}}q^{\binom{d}{2}}\prod_{i=1}^d (q^i-1)
   = \frac{|GL(d,q)|}{|M_d(q)|}.
\]
\end{proof}


\begin{coro}
As $q\rightarrow \infty$ with $d$ fixed, the proportion of invertible matrices
to all matrices converges to 1.  That is:
\[
\lim_{q\rightarrow \infty} \frac{|GL(d,q)|}{|M_d(q)|}=1.
\]
\end{coro}

\begin{coro}\label{coro:random-mat}
As $d\rightarrow \infty$ and $q$ is fixed, the proportion of invertible matrices decreases monotonically and converges towards a positive limit.  Furthermore,
\[\frac{1}{4} \leq 1-\frac{q^2-q+1}{q^2 (q-1)} \leq 
\prod_{i=1}^d\left(1-\frac{1}{q^i}\right) \leq 1-\frac{1}{q}.\]
\end{coro}
\begin{proof}
By direct expansion we find
\[
\prod_{i=1}^\infty (1-a_i)=1-\sum_{1\leq i} a_i+\sum_{1\leq i<j} a_i a_j
	- \sum_{1\leq i<j<k} a_i a_j a_k +\cdots.
\]
So setting $a_0=1$ and
\[a_{i+1}=a_i\sum_{j=i+1}^\infty \frac{1}{q^j}=a_i\frac{1}{q^{i}(q-1)}\]
for all $i\in\mathbb{N}$, we have
\[
\prod_{i=1}^\infty\left(1-\frac{1}{q^i}\right) = \sum_{i=0}^\infty (-1)^i a_i.
\]
As $a_i\geq 0$, $a_{i}\geq a_{i+1}$ for all $i\in\mathbb{N}$ and $a_i\rightarrow 0$ 
as $i\rightarrow \infty$, we may use Leibniz's theorem to conclude the alternating
series converges.  Furthermore, we may estimate the error to the $N$-th term with
error within $\pm a_{N+1}$.  Using $N=2$ we have an estimate of $1-1/q$ with error
$\pm \frac{1}{q^2(q-1)}$.  Since $q\geq 2$ this gives $1/2$ with error $\pm 1/4$.  Thus
we have at least $1/4$ chance of choosing an invertible matrix at random.
\end{proof}

\begin{remark}
$q=2$ is the only field size where the proportion of invertible matrices to all matrices is less than $1/2$.
\end{remark}


Acknowledgements: due to discussions with Wei Zhou, silverfish and mathcam.  
%%%%%
%%%%%
\end{document}
