\documentclass[12pt]{article}
\usepackage{pmmeta}
\pmcanonicalname{CoboundaryDefinitionOfExteriorDerivative}
\pmcreated{2013-03-22 15:38:06}
\pmmodified{2013-03-22 15:38:06}
\pmowner{rmilson}{146}
\pmmodifier{rmilson}{146}
\pmtitle{coboundary definition of exterior derivative}
\pmrecord{15}{37564}
\pmprivacy{1}
\pmauthor{rmilson}{146}
\pmtype{Definition}
\pmcomment{trigger rebuild}
\pmclassification{msc}{15A69}
\pmclassification{msc}{58A10}
\pmrelated{LieAlgebraCohomology}

\usepackage{amsmath}
\usepackage{amsfonts}
\usepackage{amssymb}
\begin{document}
Let $M$ be a smooth manifold, and 
\begin{itemize}
\item  let $C^\infty(M)$ denote the algebra of smooth functions on $M$;
\item  let $V(M)$ denote the Lie-algebra of smooth  vector fields;
\item and let $\Omega^k(M)$ denote the vector space of smooth,
  differential $k$-forms.
\end{itemize}
Recall that a differential form $\alpha \in \Omega^k(M)$ is a
multilinear, alternating mapping \[\alpha:V(M)\times \cdots \times
V(M) \text{(k times)}\to C^\infty(M)\] such that, in local coordinates,
$\alpha$ looks like a multilinear combination of its vector field
arguments. Thus, employing the Einstein summation convention and local
coordinates , we have
\[ \alpha(u,v,\dots, w) = \alpha_{ij\dots k }\, u^i v^j \cdots w^k,\]
where $u,v,\dots, w $ is a list of $k$ vector fields.  Recall also
that $C^\infty(M)$ is a $V(M)$ module.  The action is given by
a directional derivative, and takes the form
\[ v(f) = v^i \partial_i f,\quad v\in V(M),\; f\in C^\infty(M).\]

With these preliminaries out of the way, we have the following
description of the exterior derivative operator $d:\Omega^k(M)\to
\Omega^{k+1}(M)$. For $\omega\in \Omega^k(M)$, we have
\begin{align}
\label{eq:domega}
(d \omega)(v_0,v_1,\dots, v_k) =& \sum_{0\leq i\leq k} (-1)^k v_i
\omega(\dots,\widehat{v}_i,\dots) + \\ \nonumber
&\ +\sum_{0\leq i<j\leq k}
(-1)^{i+j} \omega([v_i,v_j],\dots, \widehat{v}_i,\dots
\widehat{v}_j,\dots ),
\end{align}
where $\widehat{v}_i$ indicates the omission of the argument $v_i$.

The above expression \eqref{eq:domega} of $d\omega$ can be taken as
the definition of the exterior derivative. Letting the $v_i$ arguments be coordinate vector fields, it is not hard to show that the above definition is equivalent to the
usual definition of $d$ as a derivation of the exterior algebra of
differential forms, or the local coordinate definition of $d$. The
nice feature of \eqref{eq:domega} is that it is equivalent to the
definition of the coboundary operator for Lie algebra cohomology.
Thus, we see that de Rham cohomology, which is the cohomology of the
cochain complex $d:\Omega^k(M)\to \Omega^{k+1}(M)$, is just
zeroth-order Lie algebra cohomology of $V(M)$ with coefficients in
$C^\infty(M)$. The bit about ``zeroth order'' means that we are
considering cochains that are zeroth order differential operators of
their arguments --- in other words, differential forms.
%%%%%
%%%%%
\end{document}
