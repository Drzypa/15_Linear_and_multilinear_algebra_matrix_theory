\documentclass[12pt]{article}
\usepackage{pmmeta}
\pmcanonicalname{MatrixPnorm}
\pmcreated{2013-03-22 11:43:22}
\pmmodified{2013-03-22 11:43:22}
\pmowner{mathcam}{2727}
\pmmodifier{mathcam}{2727}
\pmtitle{matrix p-norm}
\pmrecord{20}{30108}
\pmprivacy{1}
\pmauthor{mathcam}{2727}
\pmtype{Definition}
\pmcomment{trigger rebuild}
\pmclassification{msc}{15A60}
\pmclassification{msc}{00A69}
%\pmkeywords{frobenius}
\pmrelated{MatrixNorm}
\pmrelated{VectorNorm}
\pmrelated{FrobeniusMatrixNorm}

\usepackage{amssymb}
\usepackage{amsmath}
\usepackage{amsfonts}
\usepackage{graphicx}
%%%%%%%%%\usepackage{xypic}
\begin{document}
A class of matrix norms, denoted $\Vert\cdot\Vert_p$, is defined as

\begin{displaymath}
    \Vert\,A\,\Vert_p = \sup_{x\neq0}\frac{\Vert\,Ax\,\Vert_p}{\Vert\,x\,\Vert_p}
\qquad{}x\in\mathbb{R}^n,A\in\mathbb{R}^{m\times n}.
\end{displaymath}

The matrix $p$-norms are defined in terms of the \emph{\PMlinkname{vector $p$-norms}{VectorPNorm}}.

An alternate definition is

\begin{displaymath}
    \Vert\,A\,\Vert_p = \max_{\Vert\,x\,\Vert_p=1}\Vert\,Ax\,\Vert_p.
\end{displaymath}

As with vector $p$-norms, the most important are the 1, 2, and $\infty$ norms.
The 1 and $\infty$ norms are very easy to calculate for an arbitrary matrix:

\begin{displaymath}
\begin{array}{ll}
    \Vert\,A\,\Vert_1 & = \displaystyle\max_{1\leq j\leq n}\sum_{i=1}^m|a_{ij}| \\
    \Vert\,A\,\Vert_\infty & = \displaystyle\max_{1\leq i\leq m}\sum_{j=1}^n|a_{ij}|.
\end{array}
\end{displaymath}

It directly follows from this that $\Vert\,A\,\Vert_1 = \Vert\,A^T\,\Vert_\infty$.

The calculation of the $2$-norm is more complicated.  However, it can be shown that
the 2-norm of $A$ is the square root of the largest \emph{eigenvalue} of $A^TA$.
There are also various inequalities that allow one to make estimates on the value
of $\Vert\,A\,\Vert_2$:

\begin{equation*}
    \frac{1}{\sqrt{n}}\Vert\,A\,\Vert_\infty \leq \Vert\,A\,\Vert_2 \leq \sqrt{m}\Vert\,A\,\Vert_\infty.
\end{equation*}

\begin{equation*}
    \frac{1}{\sqrt{m}}\Vert\,A\,\Vert_1 \leq \Vert\,A\,\Vert_2 \leq \sqrt{n}\Vert\,A\,\Vert_1 .
\end{equation*}

\begin{equation*}
   \Vert\,A\,\Vert_2^2\leq\Vert\,A\,\Vert_\infty\cdot\Vert\,A\,\Vert_1.
\end{equation*}

\begin{equation*}
    \Vert\,A\,\Vert_2 \leq \Vert\,A\,\Vert_F\leq\sqrt{n}\Vert\,A\,\Vert_2.
\end{equation*}

($\Vert\,A\,\Vert_F$ is the \emph{Frobenius matrix norm})
%%%%%
%%%%%
%%%%%
%%%%%
%%%%%
%%%%%
%%%%%
%%%%%
%%%%%
\end{document}
