\documentclass[12pt]{article}
\usepackage{pmmeta}
\pmcanonicalname{AntisymmetricMapping}
\pmcreated{2013-03-22 12:34:39}
\pmmodified{2013-03-22 12:34:39}
\pmowner{rmilson}{146}
\pmmodifier{rmilson}{146}
\pmtitle{antisymmetric mapping}
\pmrecord{10}{32826}
\pmprivacy{1}
\pmauthor{rmilson}{146}
\pmtype{Definition}
\pmcomment{trigger rebuild}
\pmclassification{msc}{15A69}
\pmclassification{msc}{15A63}
\pmsynonym{skew-symmetric}{AntisymmetricMapping}
\pmsynonym{anti-symmetric}{AntisymmetricMapping}
\pmsynonym{antisymmetric}{AntisymmetricMapping}
\pmsynonym{skew-symmetric mapping}{AntisymmetricMapping}
\pmrelated{SkewSymmetricMatrix}
\pmrelated{SymmetricBilinearForm}
\pmrelated{ExteriorAlgebra}

\endmetadata

\usepackage{amsmath}
\usepackage{amsfonts}
\usepackage{amssymb}

\newcommand{\reals}{\mathbb{R}}
\newcommand{\natnums}{\mathbb{N}}
\newcommand{\cnums}{\mathbb{C}}
\newcommand{\znums}{\mathbb{Z}}

\newcommand{\lp}{\left(}
\newcommand{\rp}{\right)}
\newcommand{\lb}{\left[}
\newcommand{\rb}{\right]}

\newcommand{\supth}{^{\text{th}}}


\newtheorem{proposition}{Proposition}
\begin{document}
Let $U$ and $V$ be a vector spaces over a field $K$.  A bilinear mapping
$B:U\times U\rightarrow V$
is said to be \emph{antisymmetric} if
\begin{equation}
B(u,u)=0  
\end{equation}
for all $u\in U$.  

If $B$ is antisymmetric, then the polarization of the anti-symmetry
relation gives the condition:
\begin{equation}
B(u,v) + B(v,u) = 0  
\end{equation}
for all $u,v \in U$.  If the characteristic of $K$ is not 2, then
the two conditions are equivalent.

A multlinear mapping $M:U^k\rightarrow V$
is said to be \emph{totally antisymmetric}, or simply antisymmetric, if 
for every $u_1,\ldots,u_k\in U$ such that
$$u_{i+1} = u_i$$
for some $i=1,\ldots,k-1$ we have
$$M(u_1,\ldots,u_k)=0.$$
\begin{proposition}
  Let $M:U^k\rightarrow V$ be a totally antisymmetric, multlinear
  mapping, and let $\pi$ be a permutation of $\{1,\ldots,k\}$.  Then,
  for every $u_1,\ldots,u_k\in U$ we have
  $$M(u_{\pi_1},\ldots,u_{\pi_k}) = \mathrm{sgn}(\pi)
  M(u_1,\ldots,u_k),$$
  where $\mathrm{sgn}(\pi)=\pm1$ according to the parity of $\pi$.
\end{proposition}
{\em Proof.}
Let $u_1,\ldots,u_k\in U$ be given.  multlinearity and anti-symmetry
imply that
\begin{align*}
0 &= M(u_1+u_2,u_1+u_2,u_3,\ldots,u_k) \\
&= M(u_1,u_2,u_3,\ldots,u_k) + M(u_2,u_1,u_3,\ldots,u_k)  
\end{align*}
Hence, the proposition is valid for $\pi=(12)$ (see cycle notation).
Similarly, one can show that the proposition holds for all
transpositions 
$$\pi=(i,i+1),\quad i=1,\ldots,k-1.$$
However, such transpositions
generate the group of permutations, and hence the proposition holds in
full generality.

\paragraph{Note.} The determinant is an excellent example of a totally
antisymmetric, multlinear mapping.
%%%%%
%%%%%
\end{document}
