\documentclass[12pt]{article}
\usepackage{pmmeta}
\pmcanonicalname{FullyIndecomposableMatrix}
\pmcreated{2013-03-22 15:58:56}
\pmmodified{2013-03-22 15:58:56}
\pmowner{Mathprof}{13753}
\pmmodifier{Mathprof}{13753}
\pmtitle{fully indecomposable matrix}
\pmrecord{10}{37999}
\pmprivacy{1}
\pmauthor{Mathprof}{13753}
\pmtype{Definition}
\pmcomment{trigger rebuild}
\pmclassification{msc}{15A57}
\pmdefines{nearly decomposable}
\pmdefines{partly decomposable}
\pmdefines{fully indecomposable}

\endmetadata

% this is the default PlanetMath preamble.  as your knowledge
% of TeX increases, you will probably want to edit this, but
% it should be fine as is for beginners.

% almost certainly you want these
\usepackage{amssymb}
\usepackage{amsmath}
\usepackage{amsfonts}

% used for TeXing text within eps files
%\usepackage{psfrag}
% need this for including graphics (\includegraphics)
%\usepackage{graphicx}
% for neatly defining theorems and propositions
%\usepackage{amsthm}
% making logically defined graphics
%%%\usepackage{xypic}

% there are many more packages, add them here as you need them

% define commands here

\begin{document}
 An  $n\times n$ matrix $A$   that contains an $s \times (n-s)$ zero submatrix for some positive integer $s$ is said to be {\it partly decomposable}. If no such submatrix exists then $A$
is said to be \emph{ it fully  indecomposable}. 
By convention, a $1 \times 1$ matrix is fully indecomposable if it is nonzero.
$A$ is {\it nearly decomposable} if it fully indecomposable but whenever a nonzero  entry is changed to 0 the resulting matrix is partly decomposable.



%%%%%
%%%%%
\end{document}
