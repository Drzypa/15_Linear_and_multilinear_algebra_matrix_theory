\documentclass[12pt]{article}
\usepackage{pmmeta}
\pmcanonicalname{ExampleOfConstructionOfASchauderBasis}
\pmcreated{2013-03-22 17:49:18}
\pmmodified{2013-03-22 17:49:18}
\pmowner{perucho}{2192}
\pmmodifier{perucho}{2192}
\pmtitle{example of construction of a Schauder basis}
\pmrecord{5}{40285}
\pmprivacy{1}
\pmauthor{perucho}{2192}
\pmtype{Example}
\pmcomment{trigger rebuild}
\pmclassification{msc}{15A03}
\pmclassification{msc}{42-00}

% this is the default PlanetMath preamble.  as your knowledge
% of TeX increases, you will probably want to edit this, but
% it should be fine as is for beginners.

% almost certainly you want these
\usepackage{amssymb}
\usepackage{amsmath}
\usepackage{amsfonts}
\usepackage{amsthm}

% used for TeXing text within eps files
%\usepackage{psfrag}
% need this for including graphics (\includegraphics)
%\usepackage{graphicx}
% for neatly defining theorems and propositions
%\usepackage{amsthm}
% making logically defined graphics
%%%\usepackage{xypic}

% there are many more packages, add them here as you need them

% define commands here
\newtheorem{theorem}{Theorem}
\newtheorem{defn}{Definition}
\newtheorem{prop}{Proposition}
\newtheorem{lemma}{Lemma}
\newtheorem{cor}{Corollary}

\begin{document}
Consider an uniformly continuous function $f:[0,1]\rightarrow\mathbb{R}$. A Schauder basis $\{f_n(x)\}_0^\infty\in C[0,1]$ is constructed. For this purpose we set $f_0(x)=1$, $f_1(x)=x$. Let us consider the sequence of semi-open intervals in $[0,1]$
\begin{equation*}
I_n=[2^{-k}(2n-2),2^{-k}(2n-1)), \qquad J_n=[2^{-k}(2n-1),2^{-k}2n),
\end{equation*}
where $2^{k-1}<n\leq 2^k$, $k\geq 1$. Define now
\begin{eqnarray*}
f_n(x) &=& \left\{\begin{array}{ll}
2^k[x-(2^{-k}(2n-2)-1)] & \text{if}\,\, x\in I_n, \\
1-2^k[x-(2^{-k}(2n-1)-1)] & \text{if}\,\, x\in J_n, \\
0 & \text{otherwise.}
\end{array} \right.
\end{eqnarray*}
Geometrically these functions form a sequence of triangular functions of height {\em one} and width $2^{-(k-1)}$, sweeping $[0,1]$. So that if $f\in C([0,1])$, it is expressible in Fourier series $f(x)\sim \sum_{n=0}^\infty c_nf_n(x)$ and computing the coefficients $c_n$ by equating the values of $f(x)$ and the series at the points $x=2^{-k}m$, $m=0,1,\ldots,2^k$. The resulting series converges uniformly to $f(x)$ by the imposed premise.

%%%%%
%%%%%
\end{document}
