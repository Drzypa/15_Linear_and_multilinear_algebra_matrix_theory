\documentclass[12pt]{article}
\usepackage{pmmeta}
\pmcanonicalname{SchurDecomposition}
\pmcreated{2013-03-22 13:42:12}
\pmmodified{2013-03-22 13:42:12}
\pmowner{Daume}{40}
\pmmodifier{Daume}{40}
\pmtitle{Schur decomposition}
\pmrecord{8}{34380}
\pmprivacy{1}
\pmauthor{Daume}{40}
\pmtype{Theorem}
\pmcomment{trigger rebuild}
\pmclassification{msc}{15-00}
\pmrelated{AnExampleForSchurDecomposition}
\pmrelated{ProofThatDetEAEoperatornametrA}

\endmetadata

% this is the default PlanetMath preamble.  as your knowledge
% of TeX increases, you will probably want to edit this, but
% it should be fine as is for beginners.

% almost certainly you want these
\usepackage{amssymb}
\usepackage{amsmath}
\usepackage{amsfonts}

% used for TeXing text within eps files
%\usepackage{psfrag}
% need this for including graphics (\includegraphics)
%\usepackage{graphicx}
% for neatly defining theorems and propositions
%\usepackage{amsthm}
% making logically defined graphics
%%%\usepackage{xypic} 

% there are many more packages, add them here as you need them

% define commands here
\begin{document}
If $A$ is a complex square matrix of order n \textit{(i.e. $A\in\mathrm{Mat}_n(\mathbb{C})$)}, then there exists a unitary matrix $Q \in \mathrm{Mat}_n(\mathbb{C})$ such that\\
\begin{center}
$Q^HAQ = T = D + N$
\end{center}
where $^H$ is the conjugate transpose, $D = \operatorname{diag}(\lambda_1,   \dots, \lambda_n)$ \textit{(the $\lambda_i$ are eigenvalues of $A$)}, and $N \in \mathrm{Mat}_n(\mathbb{C})$ is strictly upper triangular matrix.  Furthermore, $Q$ can be chosen such that the eigenvalues $\lambda_i$ appear in any order along the diagonal.  \cite{1}
\begin{thebibliography}{1}
\bibitem[GVL]{1} Golub, H. Gene, Van Loan F. Charles:  Matrix Computations \textit{(Third Edition)}.  The Johns Hopkins University Press, London, 1996.
\end{thebibliography}
%%%%%
%%%%%
\end{document}
