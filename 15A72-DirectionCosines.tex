\documentclass[12pt]{article}
\usepackage{pmmeta}
\pmcanonicalname{DirectionCosines}
\pmcreated{2013-03-22 17:16:32}
\pmmodified{2013-03-22 17:16:32}
\pmowner{pahio}{2872}
\pmmodifier{pahio}{2872}
\pmtitle{direction cosines}
\pmrecord{8}{39615}
\pmprivacy{1}
\pmauthor{pahio}{2872}
\pmtype{Definition}
\pmcomment{trigger rebuild}
\pmclassification{msc}{15A72}
\pmclassification{msc}{51N20}
\pmrelated{MutualPositionsOfVectors}
\pmrelated{EquationOfPlane}
\pmdefines{direction numbers}

\endmetadata

% this is the default PlanetMath preamble.  as your knowledge
% of TeX increases, you will probably want to edit this, but
% it should be fine as is for beginners.

% almost certainly you want these
\usepackage{amssymb}
\usepackage{amsmath}
\usepackage{amsfonts}

% used for TeXing text within eps files
%\usepackage{psfrag}
% need this for including graphics (\includegraphics)
%\usepackage{graphicx}
% for neatly defining theorems and propositions
 \usepackage{amsthm}
% making logically defined graphics
%%%\usepackage{xypic}

% there are many more packages, add them here as you need them

% define commands here

\theoremstyle{definition}
\newtheorem*{thmplain}{Theorem}

\begin{document}
If the non-zero vector \,$\vec{r} = x\vec{i}+y\vec{j}+z\vec{k}$\, of $\mathbb{R}^3$ forms the angles $\alpha$, $\beta$ and $\gamma$ with the positive directions of $x$-axis, $y$-axis and $z$-axis, respectively, then the numbers
$$\cos{\alpha},\; \cos{\beta},\; \cos{\gamma}$$
are the {\em direction cosines} of the vector.  Any triple $l,\,m,\,n$ of numbers, which are \PMlinkname{proportional}{Variation} to the direction cosines, are {\em direction numbers} of the vector.

If\, $r = \sqrt{x^2+y^2+z^2}$\, is the \PMlinkescapetext{length} of $\vec{r}$, we see easily that
$$\cos{\alpha} = \frac{x}{r},\;\; \cos{\beta} = \frac{y}{r},\;\; \cos{\gamma} = \frac{z}{r}.$$

Conversely, the components of the vector on the coordinate axes may be obtained  from
$$x = r\cos{\alpha},\;\; y = r\cos{\beta},\;\; z = r\cos{\gamma}.$$

We also see that the direction cosines satisfy
$$\cos^2\alpha+\cos^2\beta+\cos^2\gamma = 1.$$
%%%%%
%%%%%
\end{document}
