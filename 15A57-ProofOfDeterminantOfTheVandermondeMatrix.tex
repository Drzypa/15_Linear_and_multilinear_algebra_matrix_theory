\documentclass[12pt]{article}
\usepackage{pmmeta}
\pmcanonicalname{ProofOfDeterminantOfTheVandermondeMatrix}
\pmcreated{2013-03-22 15:44:50}
\pmmodified{2013-03-22 15:44:50}
\pmowner{rspuzio}{6075}
\pmmodifier{rspuzio}{6075}
\pmtitle{proof of determinant of the Vandermonde matrix}
\pmrecord{10}{37699}
\pmprivacy{1}
\pmauthor{rspuzio}{6075}
\pmtype{Proof}
\pmcomment{trigger rebuild}
\pmclassification{msc}{15A57}
\pmclassification{msc}{65F99}
\pmclassification{msc}{65T50}

% this is the default PlanetMath preamble.  as your knowledge
% of TeX increases, you will probably want to edit this, but
% it should be fine as is for beginners.

% almost certainly you want these
\usepackage{amssymb}
\usepackage{amsmath}
\usepackage{amsfonts}

% used for TeXing text within eps files
%\usepackage{psfrag}
% need this for including graphics (\includegraphics)
%\usepackage{graphicx}
% for neatly defining theorems and propositions
%\usepackage{amsthm}
% making logically defined graphics
%%%\usepackage{xypic}

% there are many more packages, add them here as you need them

% define commands here
\begin{document}
To begin, note that the determinant of the $n \times n$ Vandermonde
matrix (which we shall denote as `$\Delta$') is a homogeneous
polynomial of order $n(n-1)/2$ because every term in the determinant
is, up to sign, the product of a zeroth power of some variable times the first
power of some other variable , $\ldots$, the $n-1$-st power of some
variable and $0 + 1 + \cdots + (n-1) = n(n-1)/2$.

Next, note that if $a_i = a_j$ with $i \neq j$, then $\Delta = 0$
because two columns of the matrix would be equal.  Since $\Delta$ is a
polynomial, this implies that $a_i - a_j$ is a factor of $\Delta$.
Hence, 
 \[ \Delta = C \prod_{1 \leq i < j \leq n}(a_j - a_i) \]
where C is some polynomial.  However, since both $\Delta$ and the
product on the right hand side have the same degree, $C$ must have
degree zero, i.e. $C$ must be a constant.  So all that remains is the
determine the value of this constant.

One way to determine this constant is to look at the coefficient of
the leading diagonal, $\prod_n (a_n)^{n-1}$.  Since it equals 1 in both
the determinant and the product, we conclude that $C = 1$, hence
 \[ \Delta = \prod_{1 \leq i < j \leq n}(a_j - a_i). \]
%%%%%
%%%%%
\end{document}
