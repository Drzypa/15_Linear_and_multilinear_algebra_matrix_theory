\documentclass[12pt]{article}
\usepackage{pmmeta}
\pmcanonicalname{DirectSumOfHermitianAndSkewHermitianMatrices}
\pmcreated{2013-03-22 13:36:30}
\pmmodified{2013-03-22 13:36:30}
\pmowner{mathcam}{2727}
\pmmodifier{mathcam}{2727}
\pmtitle{direct sum of Hermitian and skew-Hermitian matrices}
\pmrecord{5}{34238}
\pmprivacy{1}
\pmauthor{mathcam}{2727}
\pmtype{Example}
\pmcomment{trigger rebuild}
\pmclassification{msc}{15A03}
\pmclassification{msc}{15A57}
\pmrelated{DirectSumOfEvenoddFunctionsExample}

% this is the default PlanetMath preamble.  as your knowledge
% of TeX increases, you will probably want to edit this, but
% it should be fine as is for beginners.

% almost certainly you want these
\usepackage{amssymb}
\usepackage{amsmath}
\usepackage{amsfonts}

% used for TeXing text within eps files
%\usepackage{psfrag}
% need this for including graphics (\includegraphics)
%\usepackage{graphicx}
% for neatly defining theorems and propositions
%\usepackage{amsthm}
% making logically defined graphics
%%%\usepackage{xypic}

% there are many more packages, add them here as you need them

% define commands here
\begin{document}
\newcommand{\ccj}[1]{\overline{#1}}
\def\dtra{\hspace{0.04cm} ^{\mbox{\scriptsize{T}}} \hspace{0.02cm}}

\newcommand{\matC}[0]{M}
\newcommand{\matCp}[0]{M_+}
\newcommand{\matCm}[0]{M_-}
 

In this example, we show that any square matrix with complex
entries can uniquely be decomposed into the sum of one Hermitian matrix and 
one skew-Hermitian matrix. A fancy way to say this is that
complex square matrices is the direct sum of Hermitian and skew-Hermitian 
matrices. 

Let us denote the vector space (over $\mathbb{C}$) of 
complex square $n\times n$ matrices by $\matC$.
Further, we denote by $\matCp$ respectively $\matCm$ the vector
subspaces of Hermitian and skew-Hermitian matrices. 
We claim that 
 \begin{eqnarray}
 \label{eqp}
 \matC &=&  \matCp  \oplus \matCm.
 \end{eqnarray}
Since $\matCp$ and $\matCm$ are vector subspaces of $\matC$, it is clear
that $\matCp +\matCm$ is a vector subspace of $\matC$. Conversely, suppose 
$A\in \matC$. We can then define
 \begin{eqnarray*}
 A_+ &=& \frac{1}{2}\big( A + A^\ast \big), \\
 A_- &=& \frac{1}{2}\big( A - A^\ast \big).
 \end{eqnarray*}
Here $A^\ast = \ccj{A}\dtra$, and $\ccj{A}$ is the complex conjugate of $A$,
and $A\dtra$ is the transpose of $A$. It follows that $A_+$ is Hermitian 
and $A_-$ is anti-Hermitian. Since $A=A_+ + A_-$, any element 
in $\matC$ can be written as
the sum of one element in $\matCp$ and one element in $\matCm$. Let us check 
that this decomposition is unique. If $A\in \matCp\cap \matCm$, then 
$A=A^\ast=-A$, so $A=0$. 
We have established equation \ref{eqp}.

{\bf Special cases}
\begin{itemize} 
\item In the special case of $1\times 1$ matrices, we obtain the 
decomposition of a complex number into its real and imaginary components. 

\item In the special case of real matrices, we obtain the decomposition of 
a $n\times n$ matrix into a symmetric matrix and anti-symmetric matrix. 

\end{itemize}
%%%%%
%%%%%
\end{document}
