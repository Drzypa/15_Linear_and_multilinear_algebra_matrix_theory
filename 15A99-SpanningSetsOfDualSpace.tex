\documentclass[12pt]{article}
\usepackage{pmmeta}
\pmcanonicalname{SpanningSetsOfDualSpace}
\pmcreated{2013-03-22 17:17:28}
\pmmodified{2013-03-22 17:17:28}
\pmowner{stevecheng}{10074}
\pmmodifier{stevecheng}{10074}
\pmtitle{spanning sets of dual space}
\pmrecord{6}{39634}
\pmprivacy{1}
\pmauthor{stevecheng}{10074}
\pmtype{Theorem}
\pmcomment{trigger rebuild}
\pmclassification{msc}{15A99}

\endmetadata

% The standard font packages
\usepackage{amssymb}
\usepackage{amsmath}
\usepackage{amsfonts}

% For neatly defining theorems and definitions
\usepackage{amsthm}

% Including EPS/PDF graphics (\includegraphics)
%\usepackage{graphicx}

% Making matrix-based graphics
%%%\usepackage{xypic}

% Enumeration lists with different styles
%\usepackage{enumerate}

% Set up the theorem environments
%\newtheorem{thm}{Theorem}
\newtheorem*{thm*}{Theorem}

\providecommand{\defnterm}[1]{\emph{#1}}

% The standard number systems
\newcommand{\complex}{\mathbb{C}}
\newcommand{\real}{\mathbb{R}}
\newcommand{\rat}{\mathbb{Q}}
\newcommand{\nat}{\mathbb{N}}
\newcommand{\intset}{\mathbb{Z}}

% Absolute values and norms
% Normal, wide, and big versions of the delimeters
\providecommand{\abs}[1]{\lvert#1\rvert}
\providecommand{\absW}[1]{\left\lvert#1\right\rvert}
\providecommand{\absB}[1]{\Bigl\lvert#1\Bigr\rvert}
\providecommand{\norm}[1]{\lVert#1\rVert}
\providecommand{\normW}[1]{\left\lVert#1\right\rVert}
\providecommand{\normB}[1]{\Bigl\lVert#1\Bigr\rVert}

% Differentiation operators
\providecommand{\od}[2]{\frac{d #1}{d #2}}
\providecommand{\pd}[2]{\frac{\partial #1}{\partial #2}}
\providecommand{\pdd}[2]{\frac{\partial^2 #1}{\partial #2}}
\providecommand{\ipd}[2]{\partial #1 / \partial #2}

% Differentials on integrals
\newcommand{\dx}{\, dx}
\newcommand{\dt}{\, dt}
\newcommand{\dmu}{\, d\mu}

% Inner products
\providecommand{\ip}[2]{\langle {#1}, {#2} \rangle}

% Calligraphic letters
\newcommand{\sF}{\mathcal{F}}
\newcommand{\sD}{\mathcal{D}}

% Standard spaces
\newcommand{\Hilb}{\mathcal{H}}
\newcommand{\Le}{\mathbf{L}}

% Operators and functions occassionally used in my articles
\DeclareMathOperator{\D}{D}
\DeclareMathOperator{\linspan}{span}
\DeclareMathOperator{\rank}{rank}
\DeclareMathOperator{\lindim}{dim}
\DeclareMathOperator{\sinc}{sinc}

% Probability stuff
\newcommand{\PP}{\mathbb{P}}
\newcommand{\E}{\mathbb{E}}

\begin{document}
\PMlinkescapeword{argument}
\PMlinkescapeword{sides}

\begin{thm*}
Let $X$ be a vector space and $\phi_1, \dotsc, \phi_n \in X^*$ be 
functionals belonging to the dual space.
A linear functional $f \in X^*$ belongs to the linear span of 
$\phi_1, \dotsc, \phi_n$ if and only if 
$\ker f \supseteq \bigcap_{i=1}^n \ker \phi_i$.
\end{thm*}

$\ker$ refers to the kernel.
Note that the domain $X$ need not be finite-dimensional.

\begin{proof}
The ``only if'' part is easy: if $f = \sum_{i=1}^n \lambda_i \phi_i$
for some scalars $\lambda_i$, and $x \in X$ 
is such that $\phi_i(x) = 0$ for all $i$, then clearly $f(x) = 0$ too.

The ``if'' part will be proved by induction on $n$.

Suppose $\ker f \supseteq \ker \phi_1$.
If $f = 0$, then the result is trivial.
Otherwise, there exists $y \in X$ such that $f(y) \neq 0$.
By hypothesis, we also have $\phi_1(y) \neq 0$.
Every $z \in X$ can be decomposed into $z = x+ ty$
where $x \in \ker \phi_1 \subseteq \ker f$, and $t$ is a scalar.
Indeed, just set $t = \phi_1(z)/\phi_1(y)$, and $x = z-ty$.
Then we propose that 
\[
f(z) = \frac{f(y)}{\phi_1(y)} \phi_1(z)\,, \text{ for all $z \in X$.}
\]
To check this equation, simply evaluate both sides using the decomposition
$z = x+ty$.

Now suppose we have $\ker f \supseteq \bigcap_{i=1}^n \ker \phi_i$
for $n > 1$.
Restrict each of the functionals
to the subspace $W = \ker \phi_n$, so that 
$\ker f|_W \supseteq \bigcap_{i=1}^{n-1} \ker \phi_i|_W$.
By the induction hypothesis, there exist scalars $\lambda_1, \dotsc, \lambda_{n-1}$
such that $f|_W = \sum_{i=1}^{n-1} \lambda_i \phi_i|_W$.
Then $\ker ( f - \sum_{i=1}^{n-1} \lambda_i \phi_i ) \supseteq 
W = \ker \phi_n$, and the argument for the case $n=1$ 
can be applied anew, to obtain the final $\lambda_n$.
\end{proof}

%%%%%
%%%%%
\end{document}
