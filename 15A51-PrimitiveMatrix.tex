\documentclass[12pt]{article}
\usepackage{pmmeta}
\pmcanonicalname{PrimitiveMatrix}
\pmcreated{2013-03-22 13:18:18}
\pmmodified{2013-03-22 13:18:18}
\pmowner{Mathprof}{13753}
\pmmodifier{Mathprof}{13753}
\pmtitle{primitive matrix}
\pmrecord{12}{33809}
\pmprivacy{1}
\pmauthor{Mathprof}{13753}
\pmtype{Definition}
\pmcomment{trigger rebuild}
\pmclassification{msc}{15A51}
\pmclassification{msc}{15A48}

% this is the default PlanetMath preamble.  as your knowledge
% of TeX increases, you will probably want to edit this, but
% it should be fine as is for beginners.

% almost certainly you want these
\usepackage{amssymb}
\usepackage{amsmath}
\usepackage{amsfonts}

% used for TeXing text within eps files
%\usepackage{psfrag}
% need this for including graphics (\includegraphics)
%\usepackage{graphicx}
% for neatly defining theorems and propositions
%\usepackage{amsthm}
% making logically defined graphics
%%%\usepackage{xypic}

% there are many more packages, add them here as you need them

% define commands here
\begin{document}
A nonnegative square matrix $A=(a_{ij})$ is said to be a \PMlinkescapetext{\emph{primitive} matrix} if there exists $k$ such that $A^k\gg 0$, i.e., if there exists $k$ such that for all $i,j$, the $(i,j)$ entry of $A^k$ is positive.

A sufficient condition for a matrix to be a primitive matrix is for the matrix to 
be a nonnegative, irreducible matrix with a positive element on the main diagonal.
%%%%%
%%%%%
\end{document}
