\documentclass[12pt]{article}
\usepackage{pmmeta}
\pmcanonicalname{SomeFactsAboutInjectiveAndSurjectiveLinearMaps}
\pmcreated{2013-03-22 18:32:22}
\pmmodified{2013-03-22 18:32:22}
\pmowner{joking}{16130}
\pmmodifier{joking}{16130}
\pmtitle{some facts about injective and surjective linear maps}
\pmrecord{6}{41257}
\pmprivacy{1}
\pmauthor{joking}{16130}
\pmtype{Derivation}
\pmcomment{trigger rebuild}
\pmclassification{msc}{15A04}

% this is the default PlanetMath preamble.  as your knowledge
% of TeX increases, you will probably want to edit this, but
% it should be fine as is for beginners.

% almost certainly you want these
\usepackage{amssymb}
\usepackage{amsmath}
\usepackage{amsfonts}

% used for TeXing text within eps files
%\usepackage{psfrag}
% need this for including graphics (\includegraphics)
%\usepackage{graphicx}
% for neatly defining theorems and propositions
%\usepackage{amsthm}
% making logically defined graphics
%%%\usepackage{xypic}

% there are many more packages, add them here as you need them

% define commands here

\begin{document}
Let $k$ be a field and $V,W$ be vector spaces over $k$.

\textbf{Proposition}. Let $f:V\to W$ be an injective linear map. Then there exists a (surjective) linear map $g:W\to V$ such that $g\circ f=\mathrm{id}_{V}$.

\textit{Proof}. Of course $\mathrm{Im}(f)$ is a subspace of $W$ so $f:V\to\mathrm{Im}(f)$ is a linear isomorphism. Let $(e_i)_{i\in I}$ be a basis of $\mathrm{Im}(f)$ and $(e_j)_{j\in J}$ be its completion to the basis of $W$, i.e. $(e_i)_{i\in I\cup J}$ is a basis of $W$. Define $g:W\to V$ on the basis as follows:
$$g(e_i)=f^{-1}(e_i),\ \mbox{if } i\in I;$$
$$g(e_j)=0,\ \mbox{if } j\in J.$$
We will show that $g\circ f=\mathrm{id}_{V}$.

Let $v\in V$. Then $$f(v)=\sum_{i\in I}\alpha_{i}e_i,$$
where $\alpha_{i}\in k$ (note that the indexing set is $I$). Thus we have $$(g\circ f)(v)=g(\sum_{i\in I}\alpha_{i}e_i)=\sum_{i\in I}\alpha_{i}g(e_i)=\sum_{i\in I}\alpha_{i}f^{-1}(e_i)=$$
$$=f^{-1}(\sum_{i\in I}\alpha_{i}e_i)=f^{-1}(f(v))=v.$$
It is clear that the equality $g\circ f=\mathrm{id}_{V}$ implies that $g$ is surjective. $\square$

\textbf{Proposition}. Let $g:W\to V$ be a surjective linear map. Then there exists a (injective) linear map $f:V\to W$ such that $g\circ f=\mathrm{id}_{V}$.

\textit{Proof}. Let $(e_{i})_{i\in I}$ be a basis of $V$. Since $g$ is onto, then for any $i\in I$ there exist $w_{i}\in W$ such that $g(w_{i})=e_{i}$.
Now define $f:V\to W$ by the formula $$f(e_{i})=w_{i}.$$
It is clear that $g\circ f=\mathrm{id}_{V}$, which implies that $f$ is injective. $\square$

If we combine these two propositions, we have the following corollary:

\textbf{Corollary}. There exists an injective linear map $f:V\to W$ if and only if there exists a surjective linear map $g:W\to V$.
%%%%%
%%%%%
\end{document}
