\documentclass[12pt]{article}
\usepackage{pmmeta}
\pmcanonicalname{MutualPositionsOfVectors}
\pmcreated{2013-03-22 14:36:24}
\pmmodified{2013-03-22 14:36:24}
\pmowner{pahio}{2872}
\pmmodifier{pahio}{2872}
\pmtitle{mutual positions of vectors}
\pmrecord{25}{36178}
\pmprivacy{1}
\pmauthor{pahio}{2872}
\pmtype{Definition}
\pmcomment{trigger rebuild}
\pmclassification{msc}{15A72}
%\pmkeywords{angle between two vectors}
\pmrelated{AngleBetweenTwoLines}
\pmrelated{DirectionCosines}
\pmrelated{OrthogonalVectors}
\pmrelated{PerpendicularityInEuclideanPlane}
\pmrelated{MedianOfTrapezoid}
\pmrelated{TriangleMidSegmentTheorem}
\pmrelated{CommonPointOfTriangleMedians}
\pmrelated{FluxOfVectorField}
\pmrelated{NormalOfPlane}
\pmdefines{parallel}
\pmdefines{parallelism}
\pmdefines{perpendicular}
\pmdefines{perpendicularity}
\pmdefines{diverging}
\pmdefines{normal vector}

\endmetadata

% this is the default PlanetMath preamble.  as your knowledge
% of TeX increases, you will probably want to edit this, but
% it should be fine as is for beginners.

% almost certainly you want these
\usepackage{amssymb}
\usepackage{amsmath}
\usepackage{amsfonts}

% used for TeXing text within eps files
%\usepackage{psfrag}
% need this for including graphics (\includegraphics)
%\usepackage{graphicx}
% for neatly defining theorems and propositions
%\usepackage{amsthm}
% making logically defined graphics
%%%\usepackage{xypic}

% there are many more packages, add them here as you need them

% define commands here
\begin{document}
In this entry, we work within a Euclidean space $E$.

\begin{enumerate}
\item Two non-zero Euclidean vectors $\vec{a}$ and $\vec{b}$ are said to be {\em parallel}, denoted by\, $\vec{a}\parallel\vec{b}$,\, iff there exists a real number $k$ such that
$$\vec{a} = k\vec{b}.$$ 

Since both $\vec{a}$ and $\vec{b}$ are non-zero, $k\neq 0$.\, So $\parallel$ is a binary relation on on\, $E\!\smallsetminus\!\lbrace\vec{0}\rbrace$\, and called the {\em parallelism}.\, If\, $k > 0$,\, then $a$ and $b$ are said to be in the \emph{same direction}, and we denote this by\, $\vec{a}\upuparrows\vec{b}$;\, if\, $k < 0$,\, then $a$ and $b$ are said to be in the \emph{opposite} or \emph{contrary directions}, and we denote this by\, $\vec{a}\downarrow\uparrow\vec{b}$.

\textbf{Remarks} 
 \begin{itemize}
 \item Actually, the parallelism is an equivalence relation on\, 
$E\!\smallsetminus\!\lbrace\vec{0}\rbrace$.\, If the zero vector $\vec{0}$ were allowed along, then the relation were not symmetric ($\vec{0} = 0\vec{b}$\, but not necessarily\, $\vec{b} = k\vec{0}$).
 \item When two vectors $\vec{a}$ and $\vec{b}$ are not parallel to one another, written\, $\vec{a}\nparallel\vec{b}$,\, they are said to be {\em diverging}.
 \end{itemize}

\item Two Euclidean vectors $\vec{a}$ and $\vec{b}$ are {\em perpendicular}, denoted by\, $\vec{a}\perp\vec{b}$,\, iff 
                      $$\vec{a}\cdot\vec{b} = 0,$$
i.e. iff their scalar product vanishes.\, Then $\vec{a}$ and $\vec{b}$ are {\em normal vectors} of each other.

\textbf{Remarks} 
\begin{itemize}
 \item We may say that $\vec{0}$ is perpendicular to all vectors, because its direction is \PMlinkescapetext{indefinite} and because\, $\vec{0}\cdot\vec{b} = 0$.
 \item Perpendicularity is not an equivalence relation in the set of all vectors of the space in question, since it is neither reflexive nor transitive.
 \end{itemize}

\item The angle $\theta$ between two non-zero vectors $\vec{a}$ and $\vec{b}$ is obtained from
$$\cos\theta = \frac{\vec{a}\cdot\vec{b}}{|\vec{a}||\vec{b}|}.$$
The angle is chosen so that\, $0 \leqq \theta \leqq \pi$.

\end{enumerate}
%%%%%
%%%%%
\end{document}
