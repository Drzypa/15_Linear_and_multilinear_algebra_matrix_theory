\documentclass[12pt]{article}
\usepackage{pmmeta}
\pmcanonicalname{DiagonalMatrix}
\pmcreated{2013-03-22 13:43:32}
\pmmodified{2013-03-22 13:43:32}
\pmowner{rspuzio}{6075}
\pmmodifier{rspuzio}{6075}
\pmtitle{diagonal matrix}
\pmrecord{12}{34411}
\pmprivacy{1}
\pmauthor{rspuzio}{6075}
\pmtype{Definition}
\pmcomment{trigger rebuild}
\pmclassification{msc}{15-00}
\pmclassification{msc}{15A57}
\pmsynonym{quasi-scalar matrix}{DiagonalMatrix}
\pmsynonym{quasi-scalar matrices}{DiagonalMatrix}
\pmsynonym{diagonal matrices}{DiagonalMatrix}
\pmrelated{DiagonalizationLinearAlgebra}

\endmetadata

% this is the default PlanetMath preamble.  as your knowledge
% of TeX increases, you will probably want to edit this, but
% it should be fine as is for beginners.

% almost certainly you want these
\usepackage{amssymb}
\usepackage{amsmath}
\usepackage{amsfonts}

% used for TeXing text within eps files
%\usepackage{psfrag}
% need this for including graphics (\includegraphics)
%\usepackage{graphicx}
% for neatly defining theorems and propositions
%\usepackage{amsthm}
% making logically defined graphics
%%%\usepackage{xypic}

% there are many more packages, add them here as you need them

% define commands here

\newcommand{\sR}[0]{\mathbb{R}}
\newcommand{\sC}[0]{\mathbb{C}}
\newcommand{\sN}[0]{\mathbb{N}}
\newcommand{\sZ}[0]{\mathbb{Z}}

 \newcommand{\diag}[0]{\operatorname{diag}}
\begin{document}
{\bf Definition}
Let $A$ be a square matrix (with entries in any field).
If all off-diagonal entries of $A$ are zero, then $A$ is a 
\emph{diagonal matrix}. 

From the definition, we see that an $n\times n$ diagonal matrix is 
completely determined by the $n$ entries on the diagonal; all other entries
are zero. If the diagonal entries are $a_1, a_2, \ldots, a_n$, 
then we denote the corresponding diagonal matrix by 
$$ \diag(a_1,\ldots, a_n) = \begin{pmatrix}
 a_{1} & 0 & 0      &  \cdots &  0      \\
 0 & a_{2} & 0      & \cdots & 0      \\
 0 & 0 & a_{3} & \cdots &  0       \\
 \vdots & \vdots & \vdots & \ddots &  \\
 0 & 0 & 0 &        & a_{n}
 \end{pmatrix}. $$

\subsubsection*{Examples}
\begin{enumerate}
\item The identity matrix and zero matrix are diagonal matrices. Also, 
 any $1\times 1$ matrix is a diagonal matrix.
\item A matrix $A$ is a diagonal matrix if and only if $A$ is 
both an upper and lower triangular matrix. 
\end{enumerate}

\subsubsection*{Properties}
\begin{enumerate}
\item If $A$ and $B$ are diagonal matrices of same order, then 
$A+B$ and $AB$ are again a diagonal matrix. Further, diagonal matrices
commute, i.e., $AB=BA$. It follows that real (and complex)
diagonal matrices are normal matrices. 
\item A square matrix is diagonal if and only if it is 
triangular and normal (see \PMlinkname{this page}{TheoremForNormalTriangularMatrices}).
\item The eigenvalues of a diagonal matrix 
$A=\diag(a_1,\ldots, a_n)$ are $a_1, \ldots, a_n$. 
Corresponding eigenvectors are the standard unit vectors in $\sR^n$.
For the determinant, we have $\det A = a_1 a_2 \cdots a_n$, so 
$A$ is invertible if and only if all $a_i$ are non-zero. 
Then the inverse is given by
$$ 
  \big( \diag(a_1,\ldots, a_n)\big)^{-1} = \diag(1/a_1, \ldots, 1/a_n).
$$
\item If $A$ is a diagonal matrix, then the adjugate of $A$ is also a diagonal matrix.
\item The matrix exponential of a diagonal matrix is
$$
  e^{\diag(a_1,\ldots, a_n)} = \diag(e^{a_1}, \ldots, e^{a_n}).
$$ 
\end{enumerate}
More generally, every analytic function of a diagonal matrix can be computed entrywise, i.e.:
\[ f(\diag(a_{11},a_{22},...,a_{nn}))= \diag(f(a_{11}),f(a_{22}),...,f(a_{nn})) \]

\subsubsection*{Remarks}
Diagonal matrices are also sometimes called \emph{quasi-scalar matrices} \cite{eves}.

\begin{thebibliography}{9}
 \bibitem {eves} H. Eves,
 \emph{Elementary Matrix Theory},
 Dover publications, 1980.
\bibitem{wiki_diagonal} Wikipedia, 
\PMlinkexternal{diagonal matrix}{http://www.wikipedia.org/wiki/Diagonal_matrix}.
 \end{thebibliography}
%%%%%
%%%%%
\end{document}
