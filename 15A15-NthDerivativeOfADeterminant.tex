\documentclass[12pt]{article}
\usepackage{pmmeta}
\pmcanonicalname{NthDerivativeOfADeterminant}
\pmcreated{2013-03-22 14:30:25}
\pmmodified{2013-03-22 14:30:25}
\pmowner{GeraW}{6138}
\pmmodifier{GeraW}{6138}
\pmtitle{n'th derivative of a determinant}
\pmrecord{5}{36044}
\pmprivacy{1}
\pmauthor{GeraW}{6138}
\pmtype{Result}
\pmcomment{trigger rebuild}
\pmclassification{msc}{15A15}
\pmrelated{GeneralizedLeibnizRule}
\pmrelated{MultinomialTheorem}
\pmrelated{DerivativeOfMatrix}

\endmetadata

% this is the default PlanetMath preamble.  as your knowledge
% of TeX increases, you will probably want to edit this, but
% it should be fine as is for beginners.

% almost certainly you want these
\usepackage{amssymb}
\usepackage{amsmath}
\usepackage{amsfonts}

% used for TeXing text within eps files
%\usepackage{psfrag}
% need this for including graphics (\includegraphics)
%\usepackage{graphicx}
% for neatly defining theorems and propositions
%\usepackage{amsthm}
% making logically defined graphics
%%%\usepackage{xypic}

% there are many more packages, add them here as you need them

% define commands here
\DeclareMathOperator{\sgn}{\mathrm{sgn}}
\begin{document}
Let $ A= (a_{i,j})$ be a $d \times d$ matrix whose entries are real functions of $t$. Then,

\[\begin{split}
\frac{d^n}{dt^n}\det(A)
&=\sum_{n_1+\cdots+n_d=n} {n \choose n_1,n_2,...,n_d} \sum_{\pi \in S_d} \sgn(\pi) \prod_{i=1}^{d} \frac{d^{n_i}}{dt^{n_i}}a_{i,\pi(i)}\\
\\
&= \sum_{n_1+\cdots+n_d=n} {n \choose n_1,n_2,...,n_d} \det \begin{pmatrix} \frac{ d^{n_1}}{dt^{n_1}}a_{1,1} & \frac{ d^{n_1}}{dt^{n_1}}a_{1,2} &  \cdots
&\frac{d^{n_1}}{dt^{n_1}}a_{1,d} \cr \vdots & \vdots &  & \vdots \cr \frac{d^{n_d}}{dt^{n_d}}a_{d,1} & \frac{d^{n_d}}{dt^{n_d}}a_{d,2} & \cdots &
\frac{d^{n_d}}{dt^{n_d}}a_{d,d}
\end{pmatrix}
\end{split}\]

where $ {n \choose n_1,n_2,...,n_r}$ is the multinomial coefficient, $S_d$ is the symmetric group of permutations and $\sgn(\pi)$ is the sign of a
permutation $\pi$.
%%%%%
%%%%%
\end{document}
