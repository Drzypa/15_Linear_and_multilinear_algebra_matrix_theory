\documentclass[12pt]{article}
\usepackage{pmmeta}
\pmcanonicalname{HomogeneousLinearProblem}
\pmcreated{2013-03-22 12:26:03}
\pmmodified{2013-03-22 12:26:03}
\pmowner{rmilson}{146}
\pmmodifier{rmilson}{146}
\pmtitle{homogeneous linear problem}
\pmrecord{5}{32500}
\pmprivacy{1}
\pmauthor{rmilson}{146}
\pmtype{Definition}
\pmcomment{trigger rebuild}
\pmclassification{msc}{15A06}
\pmsynonym{homogeneous}{HomogeneousLinearProblem}
\pmrelated{LinearProblem}

\usepackage{amsmath}
\usepackage{amsfonts}
\usepackage{amssymb}

\newcommand{\reals}{\mathbb{R}}
\newcommand{\natnums}{\mathbb{N}}
\newcommand{\cnums}{\mathbb{C}}

\newcommand{\lp}{\left(}
\newcommand{\rp}{\right)}
\newcommand{\lb}{\left[}
\newcommand{\rb}{\right]}

\newcommand{\supth}{^{\text{th}}}


\newtheorem{proposition}{Proposition}
\begin{document}
Let $L:U\rightarrow V$ be a linear mapping. A linear equation
is called {\em homogeneous} if it has
the form
$$L(u)=0,\quad u\in U.$$
A homogeneous linear problem always has a
trivial solution, namely $u=0$.  The key issue in homogeneous problems
is, therefore, the question of the existence of non-trivial solutions,
i.e. whether or not the kernel of $L$ is trivial, or equivalently,
whether or not $L$ is injective.
%%%%%
%%%%%
\end{document}
