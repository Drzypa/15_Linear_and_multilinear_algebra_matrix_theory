\documentclass[12pt]{article}
\usepackage{pmmeta}
\pmcanonicalname{TensorAlgebra}
\pmcreated{2013-03-22 13:17:21}
\pmmodified{2013-03-22 13:17:21}
\pmowner{rmilson}{146}
\pmmodifier{rmilson}{146}
\pmtitle{tensor algebra}
\pmrecord{13}{33776}
\pmprivacy{1}
\pmauthor{rmilson}{146}
\pmtype{Definition}
\pmcomment{trigger rebuild}
\pmclassification{msc}{15A69}
\pmrelated{FreeAssociativeAlgebra}
\pmdefines{tensor power}

% this is the default PlanetMath preamble.  as your knowledge
% of TeX increases, you will probably want to edit this, but
% it should be fine as is for beginners.

% almost certainly you want these
\usepackage{amssymb}
\usepackage{amsmath}
\usepackage{amsfonts}
\newcommand{\mc}[1]{\mathcal{#1}}
\begin{document}
Let $R$ be a commutative ring, and $M$ an $R$-module.
The \emph{tensor algebra}
\[ \mc{T}(M) = \bigoplus_{n=0}^\infty \mc{T}_n(M)\]
is the  graded $R$-algebra with $n^{th}$
graded component simply the $n^{th}$ tensor power:
\[ \mc{T}_n(M) = M^{\otimes n} =\overbrace{M\otimes \cdots \otimes
  M}^{n\text{ times}},\quad n=1,2,\ldots,\]
and $\mc{T}_0(M)=R$.
The multiplication $m:\mc{T}(M)\times \mc{T}(M)\to\mc{T}(M)$ is given
by the usual tensor product: 
\[ m(a,b)=a\otimes b,\quad a\in M^{\otimes n},\; b\in M^{\otimes m}.\]

\paragraph{Remark 1.} One can  generalize the above definition to
cover the case where the ground ring $R$ is non-commutative by
requiring that the module $M$ is a bimodule with $R$ acting on both
the left and the right.


\paragraph{Remark 2.} From the point of view of category theory, one
can describe the tensor algebra construction as  a functor $\mc{T}$
from the category of $R$-module to the category of $R$-algebras that
is left-adjoint to the forgetful functor $\mc{F}$ from algebras to
modules.  Thus, for $M$ an $R$-module and $S$ an $R$-algebra, every
module homomorphism $M\to \mc{F}(S)$ extends to a unique algebra
homomorphism $\mc{T}(M)\to S$.
%%%%%
%%%%%
\end{document}
