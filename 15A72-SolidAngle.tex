\documentclass[12pt]{article}
\usepackage{pmmeta}
\pmcanonicalname{SolidAngle}
\pmcreated{2014-03-12 21:18:06}
\pmmodified{2014-03-12 21:18:06}
\pmowner{pahio}{2872}
\pmmodifier{pahio}{2872}
\pmtitle{solid angle}
\pmrecord{26}{37266}
\pmprivacy{1}
\pmauthor{pahio}{2872}
\pmtype{Definition}
\pmcomment{trigger rebuild}
\pmclassification{msc}{15A72}
\pmclassification{msc}{51M25}
\pmrelated{ConvexAngle}
\pmrelated{Radian}
\pmrelated{AreaOfASphericalTriangle}
\pmrelated{DihedralAngle}
\pmdefines{space angle}
\pmdefines{full solid angle}
\pmdefines{steradian}
\pmdefines{trihedral angle}

\endmetadata

% this is the default PlanetMath preamble.  as your knowledge
% of TeX increases, you will probably want to edit this, but
% it should be fine as is for beginners.

% almost certainly you want these
\usepackage{amssymb}
\usepackage{amsmath}
\usepackage{amsfonts}

% used for TeXing text within eps files
%\usepackage{psfrag}
% need this for including graphics (\includegraphics)
%\usepackage{graphicx}
% for neatly defining theorems and propositions
 \usepackage{amsthm}
% making logically defined graphics
%%%\usepackage{xypic}

% there are many more packages, add them here as you need them

% define commands here

\theoremstyle{definition}
\newtheorem*{thmplain}{Theorem}
\begin{document}
A conical surface may contain a certain portion $\Omega$ of the space $\mathbb{R}^3$.\, This portion is called \emph{solid angle} or \emph{space angle}.\, If the conical surface contains a portion $A$ of a spherical surface with radius $R$ and with \PMlinkname{centre}{Sphere} $P$ in the \PMlinkescapetext{vertex} of the solid angle, then the magnitude of the solid angle is given by
                   $$\Omega \;=\; \frac{A}{R^2}$$
which is \PMlinkescapetext{independent} on the radius $R$.\,The spherical surface can be replaced by any surface $a$, through which all the half-lines originating from $P$ and being contained in the solid angle go.\, Then the solid angle may be computed from the \PMlinkescapetext{surface integral}
\begin{align}         
\Omega \;=\; -\int_a \vec{da}\cdot\nabla\frac{1}{r},
\end{align}
where $r$ is the length of the position vector $\vec{r}$ for the points on the surface $a$.
 The \emph{full solid angle}, consisting of all points of $\mathbb{R}^3$, has the magnitude $4\pi$. 

The SI \PMlinkescapetext{unit} of solid angle, analogous to the angle \PMlinkescapetext{unit} radian, is the \emph{steradian} ($= 1\;\mbox{sr}$).\, The steradian takes a proportion $\frac{1}{4\pi}$, or approximately 7.957747 \%, of the surface area of a sphere.\\

If the solid angle is bounded by three planes having exactly one common point, it may be called a \emph{trihedral angle}; cf. the example 2!\\


\textbf{Example 1.}\,  The solid angle determined by a right circular cone with the angle $\alpha$ between its axis and \PMlinkescapetext{side line} is equal to $2\pi(1\!-\cos\alpha)$, i.e. $\displaystyle 4\pi\sin^2{\frac{\alpha}{2}}$.\\

\textbf{Example 2.}\,  Let\, $\vec{r}_1,\,\vec{r}_2,\,\vec{r}_3$ be the position vectors of three points in $\mathbb{R}^3$ and $r_1,\,r_2,\,r_3$ their lengths.\, Then the solid angle $\Omega$ of the tetrahedron \PMlinkescapetext{spanned} by the vectors $\vec{r_i}$ is obtained from the equation
\begin{align}
\tan\frac{\Omega}{2} \;=\; 
\frac{\vec{r}_1\!\times\!\vec{r}_2\!\cdot\!\vec{r}_3}{(\vec{r}_1\!\cdot\!\vec{r}_2)r_3+(\vec{r}_2\!\cdot\!\vec{r}_3)r_1+(\vec{r}_3\!\cdot\!\vec{r}_1)r_2+r_1r_2r_3},
\end{align}
where the numerator of the \PMlinkescapetext{right hand side} is the triple scalar product of the vectors.\, This equation is expressed simplier using the unit vectors $\vec{u}_i$ corresponding $\vec{r}_i$:
$$
\tan\frac{\Omega}{2} \;=\; 
 \frac{\vec{u}_1\!\times\!\vec{u}_2\!\cdot\!\vec{u}_3}
  {1+\vec{u}_2\!\cdot\!\vec{u}_3+\vec{u}_3\!\cdot\!\vec{u}_1+\vec{u}_1\!\cdot\!\vec{u}_2}
$$
The result (2) is due to van Oosterom and Strackee 1983.\\


\textbf{Example 3.}\,  Using (2), one can easily get the \PMlinkname{apical}{ConeInMathbbR3} solid angle of a \PMlinkname{right pyramid}{ConeInMathbbR3} with square base:
        $$\Omega \;=\; 4\arctan{\frac{a^2}{2h\sqrt{2a^2+4h^2}}} \;=\; 4\arcsin\frac{a^2}{a^2+4h^2}$$
Here $a$ is the side of the base square and $h$ is the \PMlinkname{height}{ConeInMathbbR3} of the pyramid.\, Cf. the solid angle of rectangular pyramid.




\begin{thebibliography}{8}
\bibitem{O}{\sc A. van Oosterom, J. Strackee}:\, A solid angle of a plane triangle.\; -- \emph{IEEE Trans. Biomed. Eng.} \textbf{30}:2 (1983); 125--126.
\bibitem{G}{\sc M. S. Gossman, A. J. Pahikkala, M. B. Rising, P. H. McGinley}:\, Providing Solid Angle Formalism for    
Skyshine Calculations.\; -- \emph{Journal of Applied Clinical Medical Physics.} \textbf{11}:4 (2010); 278--282.
\end{thebibliography}


%%%%%
%%%%%
\end{document}
