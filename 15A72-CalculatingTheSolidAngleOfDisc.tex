\documentclass[12pt]{article}
\usepackage{pmmeta}
\pmcanonicalname{CalculatingTheSolidAngleOfDisc}
\pmcreated{2013-03-22 18:19:36}
\pmmodified{2013-03-22 18:19:36}
\pmowner{pahio}{2872}
\pmmodifier{pahio}{2872}
\pmtitle{calculating the solid angle of disc}
\pmrecord{6}{40957}
\pmprivacy{1}
\pmauthor{pahio}{2872}
\pmtype{Example}
\pmcomment{trigger rebuild}
\pmclassification{msc}{15A72}
\pmclassification{msc}{51M25}
\pmrelated{SubstitutionNotation}
\pmrelated{AngleOfViewOfALineSegment}

% this is the default PlanetMath preamble.  as your knowledge
% of TeX increases, you will probably want to edit this, but
% it should be fine as is for beginners.

% almost certainly you want these
\usepackage{amssymb}
\usepackage{amsmath}
\usepackage{amsfonts}

% used for TeXing text within eps files
%\usepackage{psfrag}
% need this for including graphics (\includegraphics)
%\usepackage{graphicx}
% for neatly defining theorems and propositions
%\usepackage{amsthm}
% making logically defined graphics
%%%\usepackage{xypic}

% there are many more packages, add them here as you need them

% define commands here
\newcommand{\sijoitus}[2]%
{\operatornamewithlimits{\Big/}_{\!\!\!#1}^{\,#2}}
\begin{document}
We determine the solid angle formed by a disc when one is looking at it on the normal line of its plane set to the center of it.\\

Let us look the disc from the origin and let the disc with radius $R$ situate such that its plane is parallel to the $xy$-plane and the center is on the $z$-axis at\, $(0,\,0,\,h)$\, with\, $h > 0$.\, Into the \PMlinkescapetext{formula}
\begin{align}         
\Omega = -\int_a \vec{da}\cdot\nabla\frac{1}{r} = \int_a \vec{da}\cdot\frac{\vec{r}}{|\vec{r}|^3}
\end{align}
of the \PMlinkname{parent entry}{SolidAngle}, we may substitute the position vector \,$\vec{r} = x\vec{i}+y\vec{j}+h\vec{k}$\, of the directed surface element \,$d\vec{a} = \vec{k}\,da,$
getting         
$$\Omega = \int_a \frac{h\,da}{(x^2+y^2+h^2)^{3/2}}.$$
Now we can use a \PMlinkname{annulus}{Annulus2}-formed surface element \,$da = 2\pi\varrho\;d\varrho$\, where\, $\varrho^2 = x^2+y^2$, whence the surface integral may be calculated as
$$\Omega = \pi h\int_0^R \frac{2\varrho\;d\varrho}{(\varrho^2+h^2)^{3/2}} = 
\frac{\pi h}{-2}\sijoitus{\varrho=0}{\quad R}\frac{1}{\sqrt{\varrho^2+h^2}}.\\$$
Thus we have the result
$$\Omega = 2\pi h\left(\frac{1}{h}-\frac{1}{\sqrt{R^2+h^2}}\right)\!.$$

%%%%%
%%%%%
\end{document}
