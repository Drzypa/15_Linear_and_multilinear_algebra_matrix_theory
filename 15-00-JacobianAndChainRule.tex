\documentclass[12pt]{article}
\usepackage{pmmeta}
\pmcanonicalname{JacobianAndChainRule}
\pmcreated{2013-03-22 18:59:45}
\pmmodified{2013-03-22 18:59:45}
\pmowner{pahio}{2872}
\pmmodifier{pahio}{2872}
\pmtitle{Jacobian and chain rule}
\pmrecord{4}{41863}
\pmprivacy{1}
\pmauthor{pahio}{2872}
\pmtype{Example}
\pmcomment{trigger rebuild}
\pmclassification{msc}{15-00}
\pmclassification{msc}{26B05}
\pmclassification{msc}{26B10}

\endmetadata

% this is the default PlanetMath preamble.  as your knowledge
% of TeX increases, you will probably want to edit this, but
% it should be fine as is for beginners.

% almost certainly you want these
\usepackage{amssymb}
\usepackage{amsmath}
\usepackage{amsfonts}

% used for TeXing text within eps files
%\usepackage{psfrag}
% need this for including graphics (\includegraphics)
%\usepackage{graphicx}
% for neatly defining theorems and propositions
 \usepackage{amsthm}
% making logically defined graphics
%%%\usepackage{xypic}

% there are many more packages, add them here as you need them

% define commands here

\theoremstyle{definition}
\newtheorem*{thmplain}{Theorem}

\begin{document}
\PMlinkescapeword{connection}

Let $u$, $v$ be differentiable functions of $x$, $y$ and $x$, $y$ be differentiable functions of $s$, $t$.\, Then the connection
\begin{align}
\frac{\partial(u,v)}{\partial(s,t)} \;=\; \frac{\partial(u,v)}{\partial(x,y)}\cdot\frac{\partial(x,y)}{\partial(s,t)}
\end{align}
between the Jacobian determinants is in \PMlinkescapetext{force}.\\

\emph{Proof.}\, Starting from the right hand side of (1), where one can \PMlinkname{multiply the determinants}{Determinant2} similarly as the corresponding \PMlinkname{matrices}{MatrixMultiplication}, we have
$$\left|\begin{matrix}
u_x & u_y \\
v_x & v_y \\
\end{matrix}\right| \cdot
\left|\begin{matrix}
x_s & x_t \\
y_s & y_t \\
\end{matrix}\right| \;=\;
\left|\begin{matrix}
u_xx_s+u_yy_s & u_xx_t+u_yy_t \\
v_xx_s+v_yy_s & v_xx_t+v_yy_t \\
\end{matrix}\right| \;=\; 
\left|\begin{matrix}
u_s & u_t \\
v_s & v_t \\
\end{matrix}\right|. 
$$
Here, the last stage has been written according to the \PMlinkname{general chain rule}{ChainRuleSeveralVariables}.\, But thus we have arrived at the left hand side of the equation (1), which hereby has been proved.\\

\textbf{Remark.}\, The rule (1) is only a visualisation of the more general one concerning the case of functions of $n$ variables.
%%%%%
%%%%%
\end{document}
