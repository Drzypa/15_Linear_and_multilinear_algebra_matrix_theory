\documentclass[12pt]{article}
\usepackage{pmmeta}
\pmcanonicalname{TheoremForNormalTriangularMatrices}
\pmcreated{2013-03-22 13:43:35}
\pmmodified{2013-03-22 13:43:35}
\pmowner{Mathprof}{13753}
\pmmodifier{Mathprof}{13753}
\pmtitle{theorem for normal triangular matrices}
\pmrecord{12}{34412}
\pmprivacy{1}
\pmauthor{Mathprof}{13753}
\pmtype{Theorem}
\pmcomment{trigger rebuild}
\pmclassification{msc}{15A57}
\pmclassification{msc}{15-00}
\pmrelated{NormalMatrix}

\endmetadata

% this is the default PlanetMath preamble.  as your knowledge
% of TeX increases, you will probably want to edit this, but
% it should be fine as is for beginners.

% almost certainly you want these
\usepackage{amssymb}
\usepackage{amsmath}
\usepackage{amsfonts}

% used for TeXing text within eps files
%\usepackage{psfrag}
% need this for including graphics (\includegraphics)
%\usepackage{graphicx}
% for neatly defining theorems and propositions
%\usepackage{amsthm}
% making logically defined graphics
%%%\usepackage{xypic}

% there are many more packages, add them here as you need them

% define commands here

\newcommand{\sR}[0]{\mathbb{R}}
\newcommand{\sC}[0]{\mathbb{C}}
\newcommand{\sN}[0]{\mathbb{N}}
\newcommand{\sZ}[0]{\mathbb{Z}}
\begin{document}
\PMlinkescapeword{row}
\newtheorem{thm}{Theorem}
\begin{thm}
 (\cite{prasolov}, pp. 82) 
A square matrix is diagonal
if and only if it is normal and triangular. 
\end{thm}

\emph{Proof.} If $A$ is a diagonal matrix, then the complex conjugate 
$A^\ast$ is also a diagonal matrix. Since arbitrary diagonal matrices
commute, it follows that $A^\ast A = A A^\ast$. 
Thus
any diagonal matrix is a normal triangular matrix. 

Next, suppose $A=(a_{ij})$ is a normal upper triangular matrix. 
Thus $a_{ij}=0$ for $i>j$, so for the diagonal elements in $A^\ast A$ and
$AA^\ast$, we obtain
\begin{eqnarray*}
(A^\ast A)_{ii} &=& \sum_{k=1}^i |a_{ki}|^2, \\
(AA^\ast)_{ii} &=& \sum_{k=i}^n |a_{ik}|^2. \\
\end{eqnarray*}
For $i=1$, we have
$$ |a_{11}|^2 = |a_{11}|^2+|a_{12}|^2+\cdots + |a_{1n}|^2.$$
It follows that the only non-zero entry on the first row of $A$ is $a_{11}$.
Similarly, for $i=2$, we obtain
$$ |a_{12}|^2 + |a_{22}|^2 = |a_{22}|^2+\cdots + |a_{2n}|^2.$$
Since $a_{12}=0$, it follows that the only non-zero element on the
second row is $a_{22}$. Repeating this \PMlinkescapetext{argument} for all rows, 
we see that $A$ is a diagonal matrix. Thus any normal 
upper triangular matrix is a diagonal matrix. 

Suppose then that $A$ is a normal lower triangular matrix. 
Then it is not difficult to see that $A^\ast$ is a normal 
upper triangular matrix. Thus, by the above, $A^\ast$ is a diagonal matrix,
whence also $A$ is a diagonal matrix. $\Box$


\begin{thebibliography}{9}
\bibitem{prasolov} V.V. Prasolov, 
\emph{Problems and Theorems in Linear Algebra}, 
American Mathematical Society, 1994.
 \end{thebibliography}
%%%%%
%%%%%
\end{document}
