\documentclass[12pt]{article}
\usepackage{pmmeta}
\pmcanonicalname{PermutationMatrix}
\pmcreated{2013-03-22 12:06:39}
\pmmodified{2013-03-22 12:06:39}
\pmowner{Wkbj79}{1863}
\pmmodifier{Wkbj79}{1863}
\pmtitle{permutation matrix}
\pmrecord{20}{31232}
\pmprivacy{1}
\pmauthor{Wkbj79}{1863}
\pmtype{Definition}
\pmcomment{trigger rebuild}
\pmclassification{msc}{15A36}
\pmrelated{MonomialMatrix}

\endmetadata

\usepackage{amssymb}
\usepackage{amsmath}
\usepackage{amsfonts}
\usepackage{graphicx}
%%%\usepackage{xypic}
\begin{document}
\PMlinkescapeword{column}
\PMlinkescapeword{columns}
\PMlinkescapeword{row}
\PMlinkescapeword{rows}

\section{Permutation Matrix}

Let $n$ be a positive integer.  A \emph{permutation matrix} is any $n\times n$ matrix which can be created by rearranging the rows and/or columns of the $n\times n$ identity matrix.  More formally, given a permutation $\pi$ from the symmetric group $S_n$, one can define an $n\times n$ permutation matrix $P_{\pi}$ by $P_{\pi}=(\delta_{i\, \pi(j)})$, where $\delta$ denotes the Kronecker delta symbol.  

Premultiplying an $n\times n$ matrix $A$ by an $n\times n$ permutation matrix results in a rearrangement of the rows of $A$.  For example, if the matrix $P$ is obtained by swapping rows $i$ and $j$ of the $n \times n$ identity matrix, then rows $i$ and $j$ of $A$ will be swapped in the product $PA$.

Postmultiplying an $n\times n$ matrix $A$ by an $n\times n$ permutation matrix results in a rearrangement of the columns of $A$.  For example, if the matrix $P$ is obtained by swapping rows $i$ and $j$ of the $n \times n$ identity matrix, then columns $i$ and $j$ of $A$ will be swapped in the product $AP$.

\section{Properties}

Permutation matrices have the following properties:

\begin{itemize}
\item They are \PMlinkname{orthogonal}{OrthogonalMatrices}.
\item They are invertible.
\item For a \PMlinkname{fixed}{Fixed3} positive integer $n$, the $n \times n$ permutation matrices form a group under matrix multiplication.
\item Since they have a single 1 in each row \emph{and} each column, they are doubly stochastic.
\item They are the extreme points of the convex set of doubly stochastic matrices.
\end{itemize}
%%%%%
%%%%%
%%%%%
\end{document}
