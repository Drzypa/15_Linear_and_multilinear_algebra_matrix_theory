\documentclass[12pt]{article}
\usepackage{pmmeta}
\pmcanonicalname{DualHomomorphismOfTheDerivative}
\pmcreated{2013-03-22 12:35:28}
\pmmodified{2013-03-22 12:35:28}
\pmowner{rmilson}{146}
\pmmodifier{rmilson}{146}
\pmtitle{dual homomorphism of the derivative}
\pmrecord{4}{32842}
\pmprivacy{1}
\pmauthor{rmilson}{146}
\pmtype{Example}
\pmcomment{trigger rebuild}
\pmclassification{msc}{15A04}
\pmclassification{msc}{15A72}

\usepackage{amsmath}
\usepackage{amsfonts}
\usepackage{amssymb}
\newcommand{\dual}{^*}
\newcommand{\cP}[1]{\mathcal{P}_{#1}}
\newcommand{\ev}[1]{\mathrm{Ev}^{(#1)}}
\newcommand{\D}[1]{\mathrm{D}_{#1}}

\newcommand{\reals}{\mathbb{R}}
\newcommand{\natnums}{\mathbb{N}}
\newcommand{\cnums}{\mathbb{C}}
\newcommand{\znums}{\mathbb{Z}}

\newcommand{\lp}{\left(}
\newcommand{\rp}{\right)}
\newcommand{\lb}{\left[}
\newcommand{\rb}{\right]}

\newcommand{\supth}{^{\text{th}}}


\newtheorem{proposition}{Proposition}
\newtheorem{definition}[proposition]{Definition}
\begin{document}
Let $\cP{n}$ denote the vector space of real
polynomials of degree $n$ or less, and let $\D{n}:\cP{n}\rightarrow
\cP{n-1}$ denote the ordinary derivative.  Linear forms on $\cP{n}$
can be given in terms of evaluations, and so we introduce the
following notation.  For every scalar $k\in\reals$, let $\ev{n}_k\in
(\cP{n})\dual$ denote the evaluation functional
$$\ev{n}_k:p\mapsto p(k),\quad p\in \cP{n}.$$
Note: the degree superscript matters!  For example:
$$\ev1_2 = 2\, \ev1_1- \ev1_0,$$
whereas $\ev2_0, \ev2_1, \ev2_2$ are
linearly independent.  Let us consider the dual homomorphism
$\D2\dual$, i.e. the adjoint of $\D2$.  We have the following
relations:
$$
\begin{array}{rrrr}
  \D2\dual\lp\ev1_0\rp =&-\frac{3}{2}\,
  \ev2_0 &+ 2\,\ev2_1 & -\frac{1}{2}\, \ev2_2 ,\\ 
  \D2\dual\lp \ev1_1\rp =& - \frac{1}{2}\, \ev2_0  &&+\frac{1}{2}\, \ev2_2.
\end{array}
$$
In other words, taking $\ev1_0, \ev1_1$ as the basis of
$(\cP1)\dual$ and $\ev2_0, \ev2_1, \ev2_2$ as the basis  of
$(\cP2)\dual$, the matrix that represents $\D2\dual$ is just
$$
\lp
\begin{array}{rr}
  -\frac{3}{2} & - \frac{1}{2} \\
  2 &  0 \\
 - \frac{1}{2} &  \frac{1}{2}
\end{array}\rp
$$
Note the contravariant relationship between $\D2$ and $\D2\dual$. The
former turns second degree polynomials into first degree polynomials,
where as the latter turns first degree evaluations into second degree
evaluations.  The matrix of $\D2\dual$ has 2 columns and 3 rows
precisely because $\D2\dual$ is a homomorphism from a 2-dimensional
vector space to a 3-dimensional vector space.  

By contrast, $\D2$ will be represented by a $2\times 3$ matrix.  The
dual basis of $\cP1$ is
$$-x+1,\quad x$$
and the dual basis of $\cP2$ is
$$\frac{1}{2} (x-1)(x-2),\quad  x(2-x),\quad \frac{1}{2} x(x-1).$$
Relative to these bases, $\D2$ is represented by the transpose of the
matrix for $\D2\dual$, namely
$$
\begin{pmatrix}
  -\frac{3}{2} & 2 & - \frac{1}{2} \\
 - \frac{1}{2} & 0 &  \frac{1}{2}
\end{pmatrix}
$$
This corresponds to the following three relations:
$$
\begin{array}{lcrr}
\D2\lb \frac{1}{2} (x-1)(x-2) \rb &=& -\frac{3}{2}\, (-x+1) &  -
\frac{1}{2}\, x \\
\D2\lb x(2-x) \rb &=& 2\, (-x+1) & + 0\, x \\
\D2\lb \frac{1}{2} x(x-1) \rb &=& - \frac{1}{2}\,(-x+1) & +
\frac{1}{2}\, x
\end{array}
$$
%%%%%
%%%%%
\end{document}
