\documentclass[12pt]{article}
\usepackage{pmmeta}
\pmcanonicalname{DimensionOfTheSpecialOrthogonalGroup}
\pmcreated{2013-03-22 15:24:22}
\pmmodified{2013-03-22 15:24:22}
\pmowner{stevecheng}{10074}
\pmmodifier{stevecheng}{10074}
\pmtitle{dimension of the special orthogonal group}
\pmrecord{8}{37245}
\pmprivacy{1}
\pmauthor{stevecheng}{10074}
\pmtype{Result}
\pmcomment{trigger rebuild}
\pmclassification{msc}{15A04}
\pmclassification{msc}{15A75}
\pmclassification{msc}{20G20}
%\pmkeywords{rotation}
%\pmkeywords{special orthogonal group}
\pmrelated{ProofOfRodriguesRotationFormula}
\pmrelated{DecompositionOfOrthogonalOperatorsAsRotationsAndReflections}
\pmrelated{OrthogonalMatrices}
\pmrelated{ExteriorAlgebra}
\pmrelated{OrthogonalGroup}
\pmdefines{special orthogonal group}

\endmetadata

% this is the default PlanetMath preamble.  as your knowledge
% of TeX increases, you will probably want to edit this, but
% it should be fine as is for beginners.

% almost certainly you want these
\usepackage{amssymb}
\usepackage{amsmath}
\usepackage{amsfonts}

% used for TeXing text within eps files
%\usepackage{psfrag}
% need this for including graphics (\includegraphics)
%\usepackage{graphicx}
% for neatly defining theorems and propositions
%\usepackage{amsthm}
% making logically defined graphics
%%%\usepackage{xypic}

% there are many more packages, add them here as you need them
\usepackage{enumerate}

% define commands here
\newcommand{\real}{\mathbb{R}}
\newcommand{\rat}{\mathbb{Q}}
\newcommand{\nat}{\mathbb{N}}

\newcommand{\GLV}{\mathrm{GL}(V)}

\DeclareMathOperator{\D}{D}

\providecommand{\abs}[1]{\lvert#1\rvert}
\providecommand{\absW}[1]{\left\lvert#1\right\rvert}
\providecommand{\absB}[1]{\Bigl\lvert#1\Bigr\rvert}
\providecommand{\norm}[1]{\lVert#1\rVert}
\providecommand{\normW}[1]{\left\lVert#1\right\rVert}
\providecommand{\normB}[1]{\Bigl\lVert#1\Bigr\rVert}
\providecommand{\defnterm}[1]{\emph{#1}}
\begin{document}
Let $V$ be a $n$-dimensional real inner product space.
The group of orthogonal operators on $V$ with positive determinant
(i.e. the group of ``rotations'' on $V$)
is called the special orthogonal group, denoted $SO(n)$.

The \PMlinkname{theorem on decomposing orthogonal operators as rotations and reflections}{DecompositionOfOrthogonalOperatorsAsRotationsAndReflections}
suggests that all elements of $SO(n)$ are all fundamentally two-dimensional in some sense.
This article is an elementary exploration of this aspect.

\bigskip

First observe that the set of orthogonal operators $O(n)$ is a manifold embedded
in the real vector space $\GLV \simeq \real^{n \times n}$,
defined by
the condition $f(X)  = X X^* - I = 0$, where $X \in GL(V)$, and $X^*$ is the 
transpose of $X$ (represented as a matrix in orthonormal coordinates).
$f(X) = 0$ is simply shorthand for the condition that the columns of $X$ are 
orthonormal --- we see there are $n + n(n-1)/2 = n(n+1)/2$ scalar equations\footnote{$f(X)=0$ viewed as a matrix equation of course has $n^2$ scalar equations,
but many of them are duplicated. For simplicity,
we prefer to view $f$'s codomain to be the space of symmetric matrices, which is $n(n+1)/2$-dimensional.} in $n^2$ variables.  Then if we can show that the derivative $\D f(X)$ as full rank everywhere,
then we will have established that $O(n)$ is a manifold of dimension $n(n-1)/2$.
This is an easy computation if organized in the right way:
for any $y \in \GLV$, by the product rule we have
\[
\D f(X) \cdot Y = X Y^* + Y X^*\,.
\]
Consider first the case that $X = I$.  Then $\D f(I) \cdot Y = 0$ implies $Y^* = - Y$;
in other words, $Y$ is skew-symmetric.  But the skew-symmetric matrices have dimension 
$(n^2 - n)/2$, so the kernel of $\D f(I)$ has this dimension.
By the dimension theorem of linear algebra, the rank of $\D f(I)$ is $n^2 - (n^2 - n)/2 = n(n+1)/2$,
which is the full rank.

For general $X \in O(n)$, let $Y' = X^* Y$.  Then $\D f(X) \cdot Y = 0$ reduces to $Y'^* = -Y$,
again the skew-symmetric condition. 
The set of all such $Y'$ have dimension $(n^2 -n)/2$; but multiplication by $X^*$
is an automorphism of $\GLV$, so the set of all $Y$ also have the same dimension.
Thus $\D f(X)$ has full rank, as before.

$SO(n)$ is of course just the part of $O(n)$ that satisfies the additional condition
$\det X = 1$. It has the same dimension as a manifold.

Notice that $n(n-1)/2 = \binom{n}{2}$, the number of ways to choose $2$ basis elements out of $n$.
We might na\"ively think that an element $X$ of $SO(n)$ can be decomposed
into rotations on each of the coordinate 2-dimensional planes in $n$-dimensional space.
This does not quite work, because rotations do not commute, even though
we expect that the order of the decomposition should not matter (after all,
there is no canonical order on the pairs of basis elements of $V$).

Nevertheless, on an ``infinitesimal level'', the decomposition does work.
Suppose we are given a one-parameter subgroup\footnote{Given $X \in SO(n)$, it can be shown with the orthogonal operator decomposition theorem that there is a one-parameter subgroup containing X. Hint: 
\[
e^A = \begin{pmatrix}
\cos \theta & -\sin \theta \\
\sin \theta & \cos \theta
\end{pmatrix}\,,
\qquad \textrm{if } A = \begin{pmatrix}
0 & -\theta \\
\theta & 0
\end{pmatrix}\,.
\]
} $X^t$ of $SO(n)$.
From ODE theory,
it is known that $X^t$ is given by the matrix exponential $e^{tA}$,
where $A$ is derivative of $X^t$ at $t = 0$.
Also, $A$ must be skew-symmetric, because it is a tangent vector
to the curve $X^t$ and so must lie in the tangent space to $SO(n)$ (the kernel of $\D f(X)$).

Since $X^t = e^{tA}$, we may reasonably ask: in what way does the matrix $A$ represent
the rotations $X^t$?  The answer is suggested by the simpler case of $n=3$;
by Rodrigues' rotation formula, $A$ is the operator $A v = \omega \times v$
where $\omega$ is the angular velocity vector of the rotation.

The wedge product of the exterior algebra on $V$
generalizes the cross product in higher dimensions ($n>3$),
so we should start looking there,
and one cannot but help notice that $\Lambda^2(V)$ also has dimension $\binom{n}{2}$.
The basis elements of this space are of the form $\hat{x} \wedge \hat{y}$,
where $x$ and $y$ are orthonormal basis vectors of $V$, and $\hat{x}, \hat{y} \in V^*$
are defined by $\hat{x}(w) = \langle x, w\rangle$,
$\hat{y}(w) = \langle y, w\rangle$.  By attempting to generalize the cross product representation for $A$, we can derive that the natural isomorphism between elements $\eta \in \Lambda^2(V)$ and the skew-symmetric matrices $A$ should be given by:
\[
\langle Av, w \rangle = \eta(v, w)\,.
\]
If $\eta = \omega \, \hat{x} \wedge \hat{y}$, 
a simple calculation shows $X^t = e^{tA}$ is a rotation on the plane spanned by $x$ and $y$
with angular velocity $\omega$!

So one can think of the infinitesimal decomposition of $X^t$, as the ``angular velocity'' with  its $\binom{n}{2}$ components.  The fact that angular velocity in our physical world
can be represented by one vector, and the fact that rotations in $\real^3$ have a single axis,
is a consequence of the fact that $\binom{3}{2} = 3$. 

So although rotations do not commute in general, the correspondence between rotations
and alternating 2-tensors, shows that rotations do add up
and commute at the infinitesimal level\footnote{This really just boils down to the fact that a manifold looks like a vector space locally.}.
%%%%%
%%%%%
\end{document}
