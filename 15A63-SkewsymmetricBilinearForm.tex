\documentclass[12pt]{article}
\usepackage{pmmeta}
\pmcanonicalname{SkewsymmetricBilinearForm}
\pmcreated{2013-03-22 13:10:47}
\pmmodified{2013-03-22 13:10:47}
\pmowner{sleske}{997}
\pmmodifier{sleske}{997}
\pmtitle{skew-symmetric bilinear form}
\pmrecord{9}{33626}
\pmprivacy{1}
\pmauthor{sleske}{997}
\pmtype{Definition}
\pmcomment{trigger rebuild}
\pmclassification{msc}{15A63}
\pmsynonym{antisymmetric bilinear form}{SkewsymmetricBilinearForm}
\pmsynonym{anti-symmetric bilinear form}{SkewsymmetricBilinearForm}
\pmrelated{AntiSymmetric}
\pmrelated{SymmetricBilinearForm}
\pmrelated{BilinearForm}
\pmdefines{skew symmetric}
\pmdefines{anti-symmetric}
\pmdefines{antisymmetric}

% this is the default PlanetMath preamble.  as your knowledge
% of TeX increases, you will probably want to edit this, but
% it should be fine as is for beginners.

% almost certainly you want these
\usepackage{amssymb}
\usepackage{amsmath}
\usepackage{amsfonts}

% used for TeXing text within eps files
%\usepackage{psfrag}
% need this for including graphics (\includegraphics)
%\usepackage{graphicx}
% for neatly defining theorems and propositions
%\usepackage{amsthm}
% making logically defined graphics
%%%\usepackage{xypic}

% there are many more packages, add them here as you need them

% define commands here
\begin{document}
A {\em skew-symmetric} (or {\em antisymmetric}) {\em bilinear form} is a special case of a bilinear form $B$, namely one which is skew-symmetric in the two coordinates; that is, $B(x,y) = -B(y,x)$ for all vectors $x$ and $y$. Note that this definition only makes sense if $B$ is defined over two identical vector spaces, so we must require this in the formal definition:

a bilinear form $B: V \times V \rightarrow K$ ($V$ a vector
space over a field $K$) is called {\em skew-symmetric} iff
\begin{quote}
$B(x,y) = -B(y,x)$ for all vectors $x, y \in V$.
\end{quote}

Suppose that the characteristic of $K$ is not $2$.  Set $x=y$ in the above equation.  Then $B(x,x)=-B(x,x)$ for all vectors $x \in V$, which means that $2B(x,x)=0$, or $B(x,x)=0$.  Therefore, $B$ is an alternating form.

If, however, $\operatorname{char}(K)=2$, then $B(x,y)=-B(y,x)=B(y,x)$; $B$ is a symmetric bilinear form.

If $V$ is finite-dimensional, then every bilinear form on $V$ can be represented
by a matrix. In this case the following theorem applies:

A bilinear form is skew-symmetric iff its representing matrix is skew-symmetric.
(The fact that the representing matrix is skew-symmetric is independent of the
choice of representing matrix).
%%%%%
%%%%%
\end{document}
