\documentclass[12pt]{article}
\usepackage{pmmeta}
\pmcanonicalname{SpectralValuesClassification}
\pmcreated{2013-03-22 18:52:01}
\pmmodified{2013-03-22 18:52:01}
\pmowner{fernsanz}{8869}
\pmmodifier{fernsanz}{8869}
\pmtitle{spectral values classification}
\pmrecord{5}{41698}
\pmprivacy{1}
\pmauthor{fernsanz}{8869}
\pmtype{Definition}
\pmcomment{trigger rebuild}
\pmclassification{msc}{15A18}
\pmsynonym{eigenvalues}{SpectralValuesClassification}
\pmsynonym{spectrum}{SpectralValuesClassification}
%\pmkeywords{spectrum}
%\pmkeywords{eigenvalues}
%\pmkeywords{vector space}
%\pmkeywords{topological vector space}
%\pmkeywords{matrix}
%\pmkeywords{transformation}
%\pmkeywords{identity transformation}
%\pmkeywords{domain}
%\pmkeywords{dense}
\pmrelated{Eigenvalue}
\pmrelated{SpectrumOfAMuI}
\pmrelated{InvertibleLinearTransformation}
\pmdefines{spectrum}
\pmdefines{point spectrum}
\pmdefines{residual spectrum}
\pmdefines{continuous spectrum}
\pmdefines{resolvent set}
\pmdefines{eigenvalues}
\pmdefines{puntual spectrum}
\pmdefines{point spectral value}
\pmdefines{residual spectral value}
\pmdefines{continuous spectral value}
\pmdefines{resolvent set value}

% this is the default PlanetMath preamble.  as your knowledge
% of TeX increases, you will probably want to edit this, but
% it should be fine as is for beginners.

% almost certainly you want these
\usepackage{amssymb}
\usepackage{amsmath}
\usepackage{amsfonts}

% used for TeXing text within eps files
%\usepackage{psfrag}
% need this for including graphics (\includegraphics)
\usepackage{graphicx}
% for neatly defining theorems and propositions
\usepackage{amsthm}
% making logically defined graphics
%%%\usepackage{xypic}

% there are many more packages, add them here as you need them

% define commands here

\theoremstyle{definition}
\newtheorem{defn}{Definition}
\newtheorem{rem}{Remark}
\numberwithin{equation}{section}
\begin{document}
\title{Spectral points classification}%
\author{Fernando Sanz Gamiz}%

\begin{defn}
Let $X$ a topological vector space and $A: X \supset D_A \longrightarrow X$ a linear transformation with
domain $D_A$. Depending on the properties of\footnote{the notation $(\lambda -A)$ is to be
understood as $\lambda I -A$ with $I$ the identity transformation and $R(\lambda-A)$ is the range
of $(\lambda -A)$} $(\lambda - A)$ the following definitions apply:

\medskip

\begin{center}
\begin{tabular}{cccc}
$(\lambda-A)^{-1}$ & Boundness of $(\lambda-A)^{-1}$ & $R(\lambda-A)$ & Set to which $\lambda$ belongs\\
\hline \hline
exists & bounded & dense in X& resolvent set $\rho(A)$\\
\hline
exists & unbounded & dense in X & continuous spectrum $C\sigma(A)$\\
\hline
exists & bounded or unbounded in X & not dense in X & residual spectrum $R\sigma(A)$\\
\hline
not exists &  & dense or not dense in X & puntual spectrum $P\sigma(A)$\\
\end{tabular}
\end{center}
\end{defn}

\begin{rem}
It is obvious that, if $F$ is the field of possible values for $\lambda$ (usually $F=\mathbb C$ or
$F=\mathbb R$) then $F=\rho(A) \cup C\sigma(A) \cup R\sigma(A) \cup P\sigma(A)$, that is, these
definitions cover all the possibilities for $\lambda$. The complement of the resolvent set is called \emph{spectrum} of the operator A, i.e., $\sigma(A)=C\sigma(A) \cup R\sigma(A) \cup P\sigma(A)$
\end{rem}

\bigskip

\begin{rem}
In the finite dimensional case if $(\lambda-A)^{-1}$ exists it must be bounded, since all finite
dimensional linear mappings are bounded. This existence also implies that the range of
$(\lambda-A)$ must be the whole X. So, in the finite dimensional case the only spectral values we
can encounter are point spectrum values (eigenvalues).
\end{rem}

%%%%%
%%%%%
\end{document}
