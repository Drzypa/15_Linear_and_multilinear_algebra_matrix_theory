\documentclass[12pt]{article}
\usepackage{pmmeta}
\pmcanonicalname{ScalarFactorTransferRules}
\pmcreated{2013-03-22 15:26:41}
\pmmodified{2013-03-22 15:26:41}
\pmowner{pahio}{2872}
\pmmodifier{pahio}{2872}
\pmtitle{scalar factor transfer rules}
\pmrecord{5}{37292}
\pmprivacy{1}
\pmauthor{pahio}{2872}
\pmtype{Topic}
\pmcomment{trigger rebuild}
\pmclassification{msc}{15A72}
\pmsynonym{transfer rules of scalar factor}{ScalarFactorTransferRules}
%\pmkeywords{scalar multiplication}

% this is the default PlanetMath preamble.  as your knowledge
% of TeX increases, you will probably want to edit this, but
% it should be fine as is for beginners.

% almost certainly you want these
\usepackage{amssymb}
\usepackage{amsmath}
\usepackage{amsfonts}

% used for TeXing text within eps files
%\usepackage{psfrag}
% need this for including graphics (\includegraphics)
%\usepackage{graphicx}
% for neatly defining theorems and propositions
 \usepackage{amsthm}
% making logically defined graphics
%%%\usepackage{xypic}

% there are many more packages, add them here as you need them

% define commands here

\theoremstyle{definition}
\newtheorem*{thmplain}{Theorem}
\begin{document}
The different kinds of products between two Euclidean vectors may \PMlinkescapetext{contain} an additional scalar as factor in either vector factor $\vec{u}$, $\vec{v}$.\, Then such a scalar $r$ can be transferred from a vector to the other vector and to the whole product.\, This is true for scalar product,
$$\vec{u}\!\cdot\!(r\vec{v}) = (r\vec{u})\!\cdot\!\vec{v} = r(\vec{u}\!\cdot\!\vec{v}),$$
for vector product,
$$\vec{u}\!\times\!(r\vec{v}) = (r\vec{u})\!\times\!\vec{v} = r(\vec{u}\!\times\!\vec{v}),$$
and also for dyad product,
$$\vec{u}(r\vec{v}) = (r\vec{u})\vec{v} = r(\vec{u}\vec{v}).$$
%%%%%
%%%%%
\end{document}
