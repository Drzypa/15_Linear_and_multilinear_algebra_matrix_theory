\documentclass[12pt]{article}
\usepackage{pmmeta}
\pmcanonicalname{TripleCrossProduct}
\pmcreated{2013-03-22 14:15:53}
\pmmodified{2013-03-22 14:15:53}
\pmowner{pahio}{2872}
\pmmodifier{pahio}{2872}
\pmtitle{triple cross product}
\pmrecord{28}{35714}
\pmprivacy{1}
\pmauthor{pahio}{2872}
\pmtype{Definition}
\pmcomment{trigger rebuild}
\pmclassification{msc}{15A72}
\pmsynonym{vector triple product}{TripleCrossProduct}
\pmsynonym{triple vector product}{TripleCrossProduct}
\pmrelated{PhysicalVector}
\pmdefines{Lagrange's formula}

\endmetadata

% this is the default PlanetMath preamble.  as your knowledge
% of TeX increases, you will probably want to edit this, but
% it should be fine as is for beginners.

% almost certainly you want these
\usepackage{amssymb}
\usepackage{amsmath}
\usepackage{amsfonts}

% used for TeXing text within eps files
%\usepackage{psfrag}
% need this for including graphics (\includegraphics)
%\usepackage{graphicx}
% for neatly defining theorems and propositions
%\usepackage{amsthm}
% making logically defined graphics
%%%\usepackage{xypic}

% there are many more packages, add them here as you need them

% define commands here
\begin{document}
The cross product of a vector with a cross product is called the {\em triple cross product}.

The \PMlinkescapetext{{\em expansion formula}} of the triple cross product or {\em Lagrange's \PMlinkescapetext{formula}} is
  $$\vec{a} \times (\vec{b} \times \vec{c}) = (\vec{a} \cdot    \vec{c})\vec{b}-(\vec{a} \cdot \vec{b})\vec{c}$$
(``exterior dot far times near minus exterior dot near times far'' --- this works also when ``exterior'' is the last \PMlinkescapetext{factor}).

The \PMlinkescapetext{formula shows that this vector is in the plane spanned by} the vectors $\vec{b}$ and $\vec{c}$ (when these are not parallel).

Note that the use of parentheses in the triple cross products is necessary, since the cross product operation is not \PMlinkname{associative}{GeneralAssociativity}, i.e., generally we have
  $$(\vec{a}\times\vec{b})\times\vec{c} \neq\vec{a}\times(\vec{b}\times\vec{c})$$
(for example:\, $(\vec{i}\times\vec{i})\times\vec{j} = \vec{0}$\, but\, $\vec{i}\times(\vec{i}\times\vec{j}) = -\vec{j}$\, when 
$(\vec{i},\,\vec{j},\,\vec{k})$ is a right-handed orthonormal basis of $\mathbb{R}^3$).\, So the \PMlinkname{system}{AlgebraicSystem}\, $(\mathbb{R}^3,\,+,\,\times)$\, is not a ring.

A direct consequence of the \PMlinkescapetext{expansion formula} is the {\em Jacobi identity}
$$\vec{a}\times(\vec{b}\times\vec{c})+\vec{b}\times(\vec{c}\times\vec{a})+
  \vec{c} \times(\vec{a}\times\vec{b}) = \vec{0},$$
which is one of the properties making\, $(\mathbb{R}^3,\,+,\,\times)$\, a Lie algebra.

It follows from the \PMlinkescapetext{expansion formula} also that
  $$(\vec{a}\times\vec{b})\times(\vec{c}\times\vec{d}) = (\vec{a}\vec{b}\vec{d})\vec{c}-(\vec{a}\vec{b}\vec{c})\vec{d}$$
where $(\vec{u}\vec{v}\vec{w})$ means the triple scalar product of $\vec{u}$, $\vec{v}$ and $\vec{w}$.
%%%%%
%%%%%
\end{document}
