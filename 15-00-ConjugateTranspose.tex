\documentclass[12pt]{article}
\usepackage{pmmeta}
\pmcanonicalname{ConjugateTranspose}
\pmcreated{2013-03-22 13:42:18}
\pmmodified{2013-03-22 13:42:18}
\pmowner{Koro}{127}
\pmmodifier{Koro}{127}
\pmtitle{conjugate transpose}
\pmrecord{10}{34382}
\pmprivacy{1}
\pmauthor{Koro}{127}
\pmtype{Definition}
\pmcomment{trigger rebuild}
\pmclassification{msc}{15-00}
\pmclassification{msc}{15A15}
\pmsynonym{adjoint matrix}{ConjugateTranspose}
\pmsynonym{Hermitian conjugate}{ConjugateTranspose}
\pmsynonym{tranjugate}{ConjugateTranspose}
\pmrelated{Transpose}

\endmetadata

% this is the default PlanetMath preamble.  as your knowledge
% of TeX increases, you will probably want to edit this, but
% it should be fine as is for beginners.

% almost certainly you want these
\usepackage{amssymb}
\usepackage{amsmath}
\usepackage{amsfonts}

% used for TeXing text within eps files
%\usepackage{psfrag}
% need this for including graphics (\includegraphics)
%\usepackage{graphicx}
% for neatly defining theorems and propositions
%\usepackage{amsthm}
% making logically defined graphics
%%%\usepackage{xypic}

% there are many more packages, add them here as you need them

% define commands here

\newcommand{\sR}[0]{\mathbb{R}}
\newcommand{\sC}[0]{\mathbb{C}}
\newcommand{\sN}[0]{\mathbb{N}}
\newcommand{\sZ}[0]{\mathbb{Z}}

% The below lines should work as the command
% \renewcommand{\bibname}{References}
% without creating havoc when rendering an entry in 
% the page-image mode.
\makeatletter
\@ifundefined{bibname}{}{\renewcommand{\bibname}{References}}
\makeatother

\newcommand*{\norm}[1]{\lVert #1 \rVert}
\newcommand*{\abs}[1]{| #1 |}
\begin{document}
\def\dtra{\hspace{0.04cm} ^{\mbox{\scriptsize{T}}} \hspace{0.02cm}}
\def\htra{\hspace{0.04cm} ^{\mbox{\scriptsize{H}}} \hspace{0.02cm}}

{\bf Definition} If $A$ is a complex matrix, then the 
\emph{conjugate transpose} $A^\ast$ is the matrix 
$A^\ast = \bar{A}\dtra$, where $\bar{A}$ is
the complex conjugate of $A$, and $A\dtra$ is the 
transpose of $A$. 

It is clear that for real matrices, the conjugate transpose coincides with
the transpose. 

\subsubsection{Properties}
\begin{enumerate}
\item If $A$ and $B$ are complex matrices of same size, and $\alpha,\beta$
are complex constants, then
\begin{eqnarray*}
  (\alpha A + \beta B)^\ast &=& \overline{\alpha} A^\ast + \overline{\beta} B^\ast,\\
  A^{\ast\ast} &=& A.
\end{eqnarray*}

\item If $A$ and $B$ are complex matrices such that $AB$ is defined, then 
$$ (AB)^\ast = B^\ast A^\ast.$$
\item If $A$ is a complex square matrix, then 
\begin{eqnarray*}
 \det (A^\ast) &=& \overline{ \det{A}}, \\
\operatorname{trace}(A^\ast) &=& \overline{ \operatorname{trace}{A}}, \\
(A^\ast)^{-1} &=& (A^{-1})^\ast,
\end{eqnarray*}
where $\operatorname{trace}$ and $\operatorname{det}$ are the trace 
and the determinant operators, and $^{-1}$ is the inverse operator. 
\item Suppose $\langle \cdot, \cdot \rangle$ is the standard inner product on $\sC^n$. 
Then for an arbitrary complex $n\times n$ matrix $A$,  
and vectors $x,y\in \sC^n$, we have 
$$ \langle Ax,y\rangle = \langle x,A^\ast y \rangle.$$
\end{enumerate}

\subsubsection*{Notes}
The conjugate transpose of $A$ is also called the \emph{adjoint matrix} of $A$, 
the \emph{Hermitian conjugate} of $A$ (whence one usually writes $A^\ast = A\htra$).
The notation $A^\dagger$ is also used for the conjugate transpose \cite{pease}. 
In \cite{eves}, $A^\ast$ is also called the \emph{tranjugate} of $A$.


\begin{thebibliography}{9}
 \bibitem {eves} H. Eves, \emph{Elementary Matrix Theory}, Dover publications, 1980.
\bibitem {pease} M. C. Pease,
 \emph{Methods of Matrix Algebra},  Academic Press, 1965.
 \end{thebibliography}


\subsubsection*{See also}
\begin{itemize}
 \item Wikipedia, 
 \PMlinkexternal{conjugate transpose}{http://www.wikipedia.org/wiki/Conjugate_transpose}
\end{itemize}
%%%%%
%%%%%
\end{document}
