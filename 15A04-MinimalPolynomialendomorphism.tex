\documentclass[12pt]{article}
\usepackage{pmmeta}
\pmcanonicalname{MinimalPolynomialendomorphism}
\pmcreated{2013-03-22 13:10:14}
\pmmodified{2013-03-22 13:10:14}
\pmowner{mathcam}{2727}
\pmmodifier{mathcam}{2727}
\pmtitle{minimal polynomial (endomorphism)}
\pmrecord{12}{33615}
\pmprivacy{1}
\pmauthor{mathcam}{2727}
\pmtype{Definition}
\pmcomment{trigger rebuild}
\pmclassification{msc}{15A04}
\pmrelated{ZeroPolynomial2}
\pmrelated{OppositePolynomial}
\pmdefines{zero polynomial}
\pmdefines{minimal polynomial}

\endmetadata

% this is the default PlanetMath preamble.  as your knowledge
% of TeX increases, you will probably want to edit this, but
% it should be fine as is for beginners.

% almost certainly you want these
\usepackage{amssymb}
\usepackage{amsmath}
\usepackage{amsfonts}
\usepackage{amsthm}

% used for TeXing text within eps files
%\usepackage{psfrag}
% need this for including graphics (\includegraphics)
%\usepackage{graphicx}
% for neatly defining theorems and propositions
%\usepackage{amsthm}
% making logically defined graphics
%%%\usepackage{xypic}

% there are many more packages, add them here as you need them

% define commands here
\begin{document}
Let $T$ be an endomorphism of an $n$-dimensional vector space $V$.

{\bf Definitions.}
We define the \emph{\PMlinkescapetext{minimal polynomial}}, $M_T(X)$, to be the unique monic polynomial of \PMlinkescapetext{minimal degree} such that $M_T(T) = 0$. We say that $P(X)$ is a \emph{zero \PMlinkescapetext{polynomial}} for $T$ if $P(T)$ is the zero endomorphism.

Note that the minimal polynomial exists by virtue of the Cayley-Hamilton theorem, which provides a zero polynomial for $T$.

{\bf \PMlinkescapetext{Properties}.}
Firstly, $\operatorname{End}(V)$ is a vector space of dimension $n^2$. Therefore the $n^2 + 1$ vectors, $i_v, T, T^2, \ldots T^{n^2}$, are linearly dependant. So there are coefficients, $a_i$ not all zero such that $\sum_{i=0}^{n^2} a_i T^i = 0$. We conclude that a non-trivial zero polynomial for $T$ exists. We take $M_T(X)$ to be a zero polynomial for $T$ of minimal degree with leading coefficient one.

{\bf \PMlinkescapetext{Lemma}:} If $P(X)$ is a zero polynomial for $T$ then $M_T(X) \mid P(X)$.

\begin{proof}
By the division algorithm for polynomials, $P(X) = Q(X)M_T(X) + R(X)$ with $deg R < deg M_T$. We note that $R(X)$ is also a zero polynomial for $T$ and by minimality of $M_T(X)$, must be just $0$. Thus we have shown $M_T(X) \mid P(X)$.
\end{proof}

The minimal polynomial has a number of interesting properties:

\begin{enumerate}
\item The roots are exactly the eigenvalues of the endomorphism
\item If the minimal polynomial of $T$ splits into linear factors then $T$ is upper-triangular with respect to some basis
\item The minimal polynomial of $T$ splits into \emph{distinct} linear factors (i.e. no repeated roots) if and only if $T$ is diagonal with respect to some basis.
\end{enumerate}


It is then a \PMlinkescapetext{simple} corollary of the fundamental theorem of algebra that every endomorphism of a finite dimensional vector space over $\mathbb{C}$ may be upper-triangularized.

The minimal polynomial is intimately related to the characteristic polynomial for $T$. For let $\chi_T(X)$ be the characteristc polynomial. Since $\chi_T(T)=0$, we have by the above lemma that $M_T(X) \mid \chi_T(X)$. It is also a fact that the eigenvalues of $T$ are exactly the roots of $\chi_T$. So when split into linear factors the only difference between $M_T(X)$ and $\chi_T(X)$ is the algebraic multiplicity of the roots.

In general they may not be the same - for example any diagonal matrix with repeated eigenvalues.
%%%%%
%%%%%
\end{document}
