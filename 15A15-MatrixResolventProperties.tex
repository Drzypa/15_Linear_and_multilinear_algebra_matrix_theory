\documentclass[12pt]{article}
\usepackage{pmmeta}
\pmcanonicalname{MatrixResolventProperties}
\pmcreated{2013-03-22 15:33:52}
\pmmodified{2013-03-22 15:33:52}
\pmowner{Andrea Ambrosio}{7332}
\pmmodifier{Andrea Ambrosio}{7332}
\pmtitle{matrix resolvent properties}
\pmrecord{15}{37468}
\pmprivacy{1}
\pmauthor{Andrea Ambrosio}{7332}
\pmtype{Result}
\pmcomment{trigger rebuild}
\pmclassification{msc}{15A15}

\endmetadata

% this is the default PlanetMath preamble.  as your knowledge
% of TeX increases, you will probably want to edit this, but
% it should be fine as is for beginners.

% almost certainly you want these
\usepackage{amssymb}
\usepackage{amsmath}
\usepackage{amsfonts}

% used for TeXing text within eps files
%\usepackage{psfrag}
% need this for including graphics (\includegraphics)
%\usepackage{graphicx}
% for neatly defining theorems and propositions
\usepackage{amsthm}
% making logically defined graphics
%%%\usepackage{xypic}

% there are many more packages, add them here as you need them

% define commands here
\begin{document}
The matrix resolvent norm for a complex-valued $s$ is related to the proximity of such value to the spectrum of $A$; more precisely, the following simple yet meaningful property holds:
\[
\|R_A(s)\|\geq\frac{1}{\mathrm{dist}(s,\sigma_A)},
\]

where $\|.\|$ is any self consistent matrix norm, $\sigma_A$ is the spectrum of $A$ and the distance between a complex point and the discrete set of the eigenvalues $\lambda_i$ is defined as $\mathrm{dist}(s,\sigma_A)=\min\limits_{1\leq i\leq n}|s-\lambda_i|$.

From this fact it comes immediately, for any $1\leq i\leq n$,
\[
\lim_{s\rightarrow\lambda_i}\|R_A(s)\|=+\infty.
\]

\bigskip

\begin{proof}

Let ($\lambda_i$,$\mathbf{v}$) be an eigenvalue-eigenvector pair of $A$; then
\[
(sI-A)v=sv-Av=(s-\lambda_i)v
\]

which shows $(s-\lambda_i)$ to be an eigenvalue of $(sI-A)$; $(s-\lambda_i)^{-1}$ is then an eigenvalue of $(sI-A)^{-1}$ and , since for any self consistent norm $|\lambda|\leq \|A\|$, we have:
\[
\max\limits_{1\leq i\leq n}\frac{1}{|s-\lambda_i|}\leq\|(sI-A)^{-1}\|
\]
whence
\[
\|(sI-A)^{-1}\|\geq\frac{1}{\min\limits_{1\leq i\leq n}|s-\lambda_i|}=\frac{1}{\mathrm{dist}(s,\sigma_A)}.
\]
\end{proof}
%%%%%
%%%%%
\end{document}
