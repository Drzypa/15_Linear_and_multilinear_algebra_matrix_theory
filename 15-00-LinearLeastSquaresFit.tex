\documentclass[12pt]{article}
\usepackage{pmmeta}
\pmcanonicalname{LinearLeastSquaresFit}
\pmcreated{2013-03-22 17:24:16}
\pmmodified{2013-03-22 17:24:16}
\pmowner{rspuzio}{6075}
\pmmodifier{rspuzio}{6075}
\pmtitle{linear least squares fit}
\pmrecord{13}{39776}
\pmprivacy{1}
\pmauthor{rspuzio}{6075}
\pmtype{Definition}
\pmcomment{trigger rebuild}
\pmclassification{msc}{15-00}
\pmrelated{RegressionModel}
\pmrelated{GaussMarkovTheorem}

% this is the default PlanetMath preamble.  as your knowledge
% of TeX increases, you will probably want to edit this, but
% it should be fine as is for beginners.

% almost certainly you want these
\usepackage{amssymb}
\usepackage{amsmath}
\usepackage{amsfonts}

% used for TeXing text within eps files
%\usepackage{psfrag}
% need this for including graphics (\includegraphics)
%\usepackage{graphicx}
% for neatly defining theorems and propositions
\usepackage{amsthm}
% making logically defined graphics
%%%\usepackage{xypic}

% there are many more packages, add them here as you need them

% define commands here
\newtheorem{thm}{Theorem}
\begin{document}
One of the most common uses of least squares fitting is fitting a straight line to 
data.  Whilst, in general, it is difficult to determine the curve which best fits
the data, in this case there is a relatively simple formula which can be used.

\begin{thm}
Suppose we have a data set $(x_1,y_1), \ldots, (x_n,y_n)$.  Then the straight 
line which best fits this set is given as 
\[ 
y = {ns - pq \over nr - p^2} x + {qr - ps \over nr - p^2}
\]
where
\begin{align}
p &= \sum_{k=1}^n x_k \\
q &= \sum_{k=1}^n y_k \\
r &= \sum_{k=1}^n x_k^2 \\
s &= \sum_{k=1}^n x_k y_k
\end{align}
\end{thm}

\begin{proof}
Being the best fitting line means minimizing the merit function $M$, given as
\[
M(a,b) = \sum_{k=0}^n (a x_k + b - y_k)^2
\]
with respect to the parameters $a$ and $b$.  Expanding the square, this can be 
written as 
\[
M(a,b) = r a^2 + 2pab + nb^2 - 2sa - 2qb + t
\]
where $p,q,r,s$ are as above and 
\[
t = \sum_{k=1}^n y_k^2 .
\]

This function $M$ is a quadratic polynomial; moreover, from its definition as a 
sum of squares, it is clear that the highest order terms are positive definite, 
hence it has a minimum and all that remains is to find that minimum.  To do this, 
we set the derivatives equal to zero to obtain the following equations:
\begin{align}
0 = {\partial M(a,b) \over \partial a} & = 2ar + 2pb - 2s \\
0 = {\partial M(a,b) \over \partial b} & = 2pa + 2nb - 2q
\end{align}
These equations are easily solved to give
\begin{align}
a &= {ns - pq \over nr - p^2} \\
b &= {qr - ps \over nr - p^2} ;
\end{align}
substituting in the equation $y = ax  + b$ for a straight line, we obtain 
the answer given above.
\end{proof}

Because of the ease with which one can make a least squares fit of a line, this
technique is often adapted to fitting other sorts of curves by making a change
of variables.  Two common cases of this practice are power laws and exponentials.

Suppose that one wants to fit some data to a curve of the form $y = c e^{kx}$.
Making a change of variable $y = e^u$ and defining $b = \log c$, the equation of 
the curve becomes $u = kx + b$.  One can therefore fit the data set $(x_1, \log y_1),
\ldots (x_n, \log y_n)$ to a straight line.

Suppose that one wants to fit some data to a curve of the form $y = cx^p$.  Making
a change of variable $x = e^v$, $y = e^u$ and defining $b = \log c$, the equation of 
the curve becomes $u = pv + b$.  One can therefore fit the data set $(\log x_1, \log y_1),
\ldots (\log x_n, \log y_n)$ to a straight line.

Although convenient and common, this procedure can be a cheat because changing variables
and making a least squares fit of a line is not the same as making a least squares fit
to a curve.  The reason for this is that the merit functions are different and will not,
in general have a minimum in the same place.  However, if the data happen to approximately
lie on a power curve or an exponential, then the answer obtained by changing variables
and fitting will be an approximation to the correct answer.  Depending on what one is
doing, this approximation may be good enough or one may use it as a starting point for
some algorithm to compute the correct minimum.
%%%%%
%%%%%
\end{document}
