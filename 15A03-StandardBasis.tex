\documentclass[12pt]{article}
\usepackage{pmmeta}
\pmcanonicalname{StandardBasis}
\pmcreated{2013-03-22 14:20:07}
\pmmodified{2013-03-22 14:20:07}
\pmowner{Mathprof}{13753}
\pmmodifier{Mathprof}{13753}
\pmtitle{standard basis}
\pmrecord{9}{35806}
\pmprivacy{1}
\pmauthor{Mathprof}{13753}
\pmtype{Definition}
\pmcomment{trigger rebuild}
\pmclassification{msc}{15A03}
\pmrelated{BasalUnits}
\pmdefines{standard basis vectors}

\endmetadata

% this is the default PlanetMath preamble.  as your knowledge
% of TeX increases, you will probably want to edit this, but
% it should be fine as is for beginners.

% almost certainly you want these
\usepackage{amssymb}
\usepackage{amsmath}
\usepackage{amsfonts}

% used for TeXing text within eps files
%\usepackage{psfrag}
% need this for including graphics (\includegraphics)
%\usepackage{graphicx}
% for neatly defining theorems and propositions
%\usepackage{amsthm}
% making logically defined graphics
%%%\usepackage{xypic}

% there are many more packages, add them here as you need them

% define commands here
\begin{document}
\PMlinkescapeword{component}

If $R$ is a division ring, then the \PMlinkname{direct sum}{DirectSum} of $n$ copies of $R$,
\[ R^n = R \oplus\dotsb\oplus R\text{  (n times),}\]
is a vector space.


The \emph{standard basis for $R^n$} consists of $n$ elements
\[ e_1 = (1,0,\dotsc ,0), \quad e_2 = (0,1,0,\dotsc ,0),\quad \dotsc \quad e_n = (0,\dotsc ,0,1) \]
where each $e_i$ has $1$ for its $i$th component and $0$ for every other component.  
The $e_i$ are called the \emph{standard basis vectors}.



%%%%%
%%%%%
\end{document}
