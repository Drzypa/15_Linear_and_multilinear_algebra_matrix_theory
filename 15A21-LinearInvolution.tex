\documentclass[12pt]{article}
\usepackage{pmmeta}
\pmcanonicalname{LinearInvolution}
\pmcreated{2013-03-22 13:34:37}
\pmmodified{2013-03-22 13:34:37}
\pmowner{matte}{1858}
\pmmodifier{matte}{1858}
\pmtitle{linear involution}
\pmrecord{14}{34197}
\pmprivacy{1}
\pmauthor{matte}{1858}
\pmtype{Definition}
\pmcomment{trigger rebuild}
\pmclassification{msc}{15A21}
\pmsynonym{involution}{LinearInvolution}
\pmrelated{Projection}
\pmrelated{AntiIdempotent}

\endmetadata

% this is the default PlanetMath preamble.  as your knowledge
% of TeX increases, you will probably want to edit this, but
% it should be fine as is for beginners.

% almost certainly you want these
\usepackage{amssymb}
\usepackage{amsmath}
\usepackage{amsfonts}

% used for TeXing text within eps files
%\usepackage{psfrag}
% need this for including graphics (\includegraphics)
%\usepackage{graphicx}
% for neatly defining theorems and propositions
%\usepackage{amsthm}
% making logically defined graphics
%%%\usepackage{xypic}

% there are many more packages, add them here as you need them

% define commands here

\newcommand{\sR}[0]{\mathbb{R}}
\newcommand{\sC}[0]{\mathbb{C}}
\newcommand{\sN}[0]{\mathbb{N}}
\newcommand{\sZ}[0]{\mathbb{Z}}
\begin{document}
{\bf Definition.}
 Let $V$ be a vector space.
 A \emph{ linear involution} is a linear
 operator $L:V\to V$ such that $L^2$ is the identity operator on $V$.
An equivalent definition is that  a linear involution is a linear operator that 
equals its own inverse.

{\bf Theorem 1.} Let $V$ be a vector space and let $A:V\to V$ be a linear involution.
Then the eigenvalues of $A$ are $\pm 1$. Further,
if $V$ is $\sC^n$, and $A$ is a $n\times n$ complex matrix, then we have that:
\begin{enumerate}
\item $\det A = \pm 1$.
\item The characteristic polynomial of $A$, $p(\lambda) = \det( A-\lambda I)$,
is a reciprocal polynomial, i.e.,
$$ p(\lambda) = \pm \lambda^n p(1/\lambda).$$
\end{enumerate}
(\PMlinkname{proof.}{EigenvaluesOfAnInvolution})

The next theorem gives a correspondence between involution
 operators and projection operators.

{\bf Theorem 2.} Let $L$ and $P$ be linear operators on a
 vector space $V$ over a field of characteristic not 2, and let $I$ be the identity operator on $V$.
 If $L$ is an involution then
 the operators $\frac{1}{2}\big(I\pm L\big)$
 are projection operators.
 Conversely, if $P$ is a projection operator, then
 the operators $\pm(2P-I)$ are involutions. \\

Involutions have important application in expressing hermitian-orthogonal operators, that is, $H^t=\overline{H}=H^{-1}$. In fact, it may be represented as 
$$H=Le^{iS},$$
being $L$ a real symmetric involution operator and $S$ a real skew-symmetric operator permutable with it, i.e.

$$L=\overline{L}=L^t, \qquad L^2=I, \qquad S=\overline{S}=-S^t, \qquad LS=SL.$$
%%%%%
%%%%%
\end{document}
