\documentclass[12pt]{article}
\usepackage{pmmeta}
\pmcanonicalname{MotionInCentralforceField}
\pmcreated{2013-03-22 18:52:41}
\pmmodified{2013-03-22 18:52:41}
\pmowner{pahio}{2872}
\pmmodifier{pahio}{2872}
\pmtitle{motion in central-force field}
\pmrecord{14}{41725}
\pmprivacy{1}
\pmauthor{pahio}{2872}
\pmtype{Derivation}
\pmcomment{trigger rebuild}
\pmclassification{msc}{15A72}
\pmclassification{msc}{51N20}
\pmsynonym{Kepler's first law}{MotionInCentralforceField}
\pmrelated{CommonEquationOfConics}
\pmrelated{PropertiesOfEllipse}

\endmetadata

% this is the default PlanetMath preamble.  as your knowledge
% of TeX increases, you will probably want to edit this, but
% it should be fine as is for beginners.

% almost certainly you want these
\usepackage{amssymb}
\usepackage{amsmath}
\usepackage{amsfonts}
\usepackage[T2A]{fontenc}
\usepackage[russian, english]{babel}


% used for TeXing text within eps files
%\usepackage{psfrag}
% need this for including graphics (\includegraphics)
%\usepackage{graphicx}
% for neatly defining theorems and propositions
%\usepackage{amsthm}
% making logically defined graphics
%%%\usepackage{xypic}

% there are many more packages, add them here as you need them

% define commands here
\newcommand{\sijoitus}[2]%
{\operatornamewithlimits{\Big/}_{\!\!\!#1}^{\,#2}}
\begin{document}
\PMlinkescapeword{constant} \PMlinkescapeword{moment} \PMlinkescapeword{degrees}
\PMlinkescapeword{central} \PMlinkescapeword{force} \PMlinkescapeword{field}
Let us consider a body with \PMlinkescapetext{mass} $m$ in a gravitational force \PMlinkname{field}{VectorField} exerted by the origin and directed always from the body towards the origin.\, Set the plane through the origin and the velocity vector $\vec{v}$ of the body.\, Apparently, the body is forced to move constantly in this plane, i.e. there is a question of a planar motion.\, We want to derive the trajectory of the body.

Equip the plane of the motion with a polar coordinate system $r,\,\varphi$ and denote the position vector of the body by $\vec{r}$.\, Then the velocity vector is
\begin{align}
\vec{v} \;=\; \frac{d\vec{r}}{dt} \;=\; \frac{d}{dt}(r\vec{r}^{\,0}) 
\;=\; \frac{dr}{dt}\vec{r}^{\,0}+r\frac{d\varphi}{dt}\vec{s}^{\,0},
\end{align}
where $\vec{r}^{\,0}$ and $\vec{s}^{\,0}$ are the unit vectors in the direction of $\vec{r}$ and of $\vec{r}$ rotated 90 degrees anticlockwise ($\vec{r}^{\,0} = \vec{i}\cos\varphi+\vec{j}\sin\varphi$,\, whence\, $\frac{\vec{r}^{\,0}}{dt} = 
(-\vec{i}\sin\varphi+\vec{j}\cos\varphi)\frac{d\varphi}{dt} = \frac{d\varphi}{dt}\vec{s}^{\,0}$).\, Thus the kinetic energy of the body is
$$E_k \;=\; \frac{1}{2}m\left|\frac{d\vec{r}}{dt}\right|^2 
\;=\; \frac{1}{2}m\left(\!\left(\frac{dr}{dt}\right)^2\!+\!\left(r\frac{d\varphi}{dt}\right)^2\right)\!.$$
Because the gravitational force on the body is exerted along the position vector, its moment is 0 and therefore the angular momentum 
$$\vec{L} \;=\; \vec{r}\!\times\!m\frac{d\vec{r}}{dt} 
\;=\; mr^2\frac{d\varphi}{dt}\vec{r}^{\,0}\!\times\!\vec{s}^{\,0}$$
of the body is constant; thus its magnitude is a constant,
$$mr^2\frac{d\varphi}{dt} \;=\; G,$$
whence
\begin{align}
\frac{d\varphi}{dt} \;=\; \frac{G}{mr^2}.
\end{align}
The central force\, $\displaystyle\vec{F} := -\frac{k}{r^2}\vec{r}^{\,0}$\, (where $k$ is a constant) has the scalar potential \, $U(r) = -\frac{k}{r}$.\, Thus the total energy\, $E = E_k\!+\!U(r)$ of the body, which is constant, may be written
$$E \;=\; \frac{1}{2}m\!\left(\frac{dr}{dt}\right)^2\!+\frac{1}{2}mr^2\!\left(\frac{G}{mr^2}\right)^2\!-\!\frac{k}{r} 
\;=\; \frac{m}{2}\!\left(\frac{dr}{dt}\right)^2\!+\frac{G^2}{2mr^2}\!-\!\frac{k}{r}.$$
This equation may be revised to
$$\left(\frac{dr}{dt}\right)^2\!+\frac{G^2}{m^2r^2}-\frac{2k}{mr}+\frac{k^2}{G^2} \;=\; \frac{2E}{m}+\frac{k^2}{G^2},$$
i.e. 
$$\left(\frac{dr}{dt}\right)^2\!+\left(\frac{k}{G}-\frac{G}{mr}\right)^2 \;=\; q^2$$
where
$$q \;:=\; \sqrt{\frac{2}{m}\left(\!E\!+\!\frac{mk^2}{2G^2}\right)}$$
is a constant.\, We introduce still an auxiliary angle $\psi$ such that
\begin{align}
\frac{k}{G}-\frac{G}{mr} \;=\; q\cos\psi, \quad \frac{dr}{dt} \;=\; q\sin\psi.
\end{align}
Differentiation of the first of these equations implies
$$\frac{G}{mr^2}\cdot\frac{dr}{dt} \;=\; -q\sin\psi\frac{d\psi}{dt} \;=\; -\frac{dr}{dt}\cdot\frac{d\psi}{dt},$$
whence, by (2),
$$\frac{d\psi}{dt} \;=\; -\frac{G}{mr^2} \;=\; -\frac{d\varphi}{dt}.$$
This means that\, $\psi = C\!-\!\varphi$, where the constant $C$ is determined by the initial conditions.\, We can then solve $r$ from the first of the equations (3), obtaining
\begin{align}
r \;=\; \frac{G^2}{km\left(1-\frac{Gq}{k}\cos(C-\varphi)\right)} \;=\; \frac{p}{1-\varepsilon\cos(\varphi-C)},
\end{align}
where
$$p \;:=\; \frac{G^2}{km}, \quad \varepsilon \;:=\; \frac{Gq}{k}.$$\\

By the \PMlinkid{parent entry}{11724}, the result (4) shows that the trajectory of the body in the gravitational \PMlinkname{field}{VectorField} of one point-like sink is always a conic section whose focus \PMlinkescapetext{contains} the sink causing the field.\\

As for the \PMlinkescapetext{type} of the conic, the most interesting one is an ellipse.\, It occurs, by the 
\PMlinkid{parent entry}{11724}, when\, $\varepsilon < 1$.\, This condition is easily seen to be equivalent with a negative total energy $E$ of the body.\\

One can say that any planet revolves around the Sun along an ellipse having the Sun in one of its foci --- this is \emph{Kepler's first law}.



\begin{thebibliography}{9}
\bibitem{MMP} \CYRYA. \CYRB. \CYRZ\cyre\cyrl\cyrsftsn\cyrd\cyro\cyrv\cyri\cyrch \;\&\, \CYRA. \CYRD. \CYRM\cyrery\cyrsh\cyrk\cyri\cyrs: 
{\em \CYREREV\cyrl\cyre\cyrm\cyre\cyrn\cyrt\cyrery\, \cyrp\cyrr\cyri\cyrk\cyrl\cyra\cyrd\cyrn\cyro\cyrishrt\, \cyrm\cyra\cyrt\cyre\cyrm\cyra\cyrt\cyri\cyrk\cyri}. \,\CYRI\cyrz\cyrd\cyra\cyrt\cyre\cyrl\cyrsftsn\cyrs\cyrt\cyrv\cyro \,
``\CYRN\cyra\cyru\cyrk\cyra''.\, \CYRM\cyro\cyrs\cyrk\cyrv\cyra \,(1976).

\end{thebibliography}
%%%%%
%%%%%
\end{document}
