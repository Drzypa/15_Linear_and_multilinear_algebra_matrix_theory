\documentclass[12pt]{article}
\usepackage{pmmeta}
\pmcanonicalname{AdjointEndomorphism}
\pmcreated{2013-03-22 12:29:36}
\pmmodified{2013-03-22 12:29:36}
\pmowner{rmilson}{146}
\pmmodifier{rmilson}{146}
\pmtitle{adjoint endomorphism}
\pmrecord{12}{32718}
\pmprivacy{1}
\pmauthor{rmilson}{146}
\pmtype{Definition}
\pmcomment{trigger rebuild}
\pmclassification{msc}{15A04}
\pmclassification{msc}{15A63}
\pmsynonym{adjoint}{AdjointEndomorphism}
\pmrelated{Transpose}
\pmdefines{Hermitian adjoint}

\usepackage{amsmath}
\usepackage{amsfonts}
\usepackage{amssymb}
%%\usepackage{xypic}
\newcommand{\Hom}{\mathop{\mathrm{Hom}}\nolimits}
\newcommand{\Mat}{\mathop{\mathrm{Mat}}\nolimits}
\newcommand{\kfield}{\mathbb{K}}
\newcommand{\supt}{^t}
\newcommand{\dual}{^*}
\newcommand{\adj}{^{\displaystyle \star}}


\newcommand{\reals}{\mathbb{R}}
\newcommand{\natnums}{\mathbb{N}}
\newcommand{\cnums}{\mathbb{C}}

\newcommand{\lp}{\left(}
\newcommand{\rp}{\right)}
\newcommand{\lb}{\left[}
\newcommand{\rb}{\right]}

\newcommand{\supth}{^{\text{th}}}

\newcommand{\bu}{\mathbf{u}}

\newtheorem{proposition}{Proposition}
\begin{document}
\paragraph{Definition (the bilinear case).} Let $U$ be a
finite-dimensional vector space over a field $\kfield$, and $B:U\times
U\to\kfield$ a symmetric, non-degenerate bilinear mapping, for example
a real inner product.  For an endomorphism $T:U\rightarrow U$ we
define the adjoint of $T$ relative to $B$ to be the endomorphism
$T\adj:U\rightarrow U$, characterized by
$$B(u,Tv) = B(T\adj u,v),\quad u,v\in U.$$

It is convenient to identify $B$
with a linear isomorphism $B:U\rightarrow U\dual$ in the sense that
$$B(u,v) = (Bu)(v),\quad u,v\in U.$$
We then have
$$T\adj = B^{-1} T\dual B.$$
To put it another way, $B$ gives an
isomorphism between $U$ and
the dual $U^*$, and  the 
adjoint $T\adj$ is  the endomorphism of $U$  that corresponds to the
\PMlinkname{dual homomorphism}{DualHomomorphism}
$T\dual:U\dual\rightarrow U\dual$.  Here is a commutative diagram to
illustrate this idea:
$$
\xymatrix{%
U \ar[r]^{T^\star} \ar[d]^B & U \ar[d]^{B} \\
\;U^* \ar[r]^{T^*} & \;U^{*}
}
$$


\paragraph{Relation to the matrix transpose.} Let $\bu_1,\ldots,\bu_n$
be a basis of $U$, and let $M\in
\Mat_{n,n}(\kfield)$ be the matrix of $T$ relative to this basis, i.e.
$$\sum_j M^j_{\,i}\, \bu_j = T(\bu_i).$$
Let $P\in\Mat_{n,n}(\kfield)$ denote the matrix of the inner product
relative to the same basis, i.e.
$$P_{ij} = B(\bu_i,\bu_j).$$
Then, the representing matrix of $T\adj$ relative to the same basis
is given by $ P^{-1} M\supt P.$ Specializing further, suppose that the
basis in question is orthonormal, i.e. that
$$B(\bu_i,\bu_j) = \delta_{ij}.$$
Then, the matrix of $T\adj$ is
simply the transpose $M\supt$.

\paragraph{The Hermitian (sesqui-linear) case.}
If $T:U\rightarrow U$ is an endomorphism of a unitary space (a complex
vector space equipped with a \PMlinkname{Hermitian inner product}{HermitianForm}). In this setting we can define we define
the Hermitian adjoint $T\adj:U\rightarrow U$ by means of the familiar
adjointness condition
$$\langle u,Tv\rangle = \langle T\adj u,v\rangle,\quad u,v\in U.$$


However, the analogous operation at the matrix level is the conjugate
transpose. Thus, if $M\in \Mat_{n,n}(\cnums)$ is the matrix of $T$
relative to an orthonormal basis, then $\overline{M\supt}$ is the
matrix of $T\adj$ relative to the same basis.
%%%%%
%%%%%
\end{document}
