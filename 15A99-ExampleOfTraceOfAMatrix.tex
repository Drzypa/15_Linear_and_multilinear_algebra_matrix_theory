\documentclass[12pt]{article}
\usepackage{pmmeta}
\pmcanonicalname{ExampleOfTraceOfAMatrix}
\pmcreated{2013-03-22 13:33:20}
\pmmodified{2013-03-22 13:33:20}
\pmowner{Daume}{40}
\pmmodifier{Daume}{40}
\pmtitle{example of trace of a matrix}
\pmrecord{8}{34160}
\pmprivacy{1}
\pmauthor{Daume}{40}
\pmtype{Example}
\pmcomment{trigger rebuild}
\pmclassification{msc}{15A99}

% this is the default PlanetMath preamble.  as your knowledge
% of TeX increases, you will probably want to edit this, but
% it should be fine as is for beginners.

% almost certainly you want these
\usepackage{amssymb}
\usepackage{amsmath}
\usepackage{amsfonts}

% used for TeXing text within eps files
%\usepackage{psfrag}
% need this for including graphics (\includegraphics)
%\usepackage{graphicx}
% for neatly defining theorems and propositions
%\usepackage{amsthm}
% making logically defined graphics
%%%\usepackage{xypic} 

% there are many more packages, add them here as you need them

% define commands here
\begin{document}
Let $A = \left[ \begin{array}{ccc}
2 & 4 & 6 \\
8 & 10 & 12 \\
14 & 16 & 18 
\end{array}\right]$,
$A' = \frac{1}{2}A = \left[\begin{array}{ccc}
1 & 2 & 3 \\
4 & 5 & 6 \\
7 & 8 & 9
\end{array}\right]$
 and $B = \left[\begin{array}{ccc}
9 & 8 & 7 \\
6 & 5 & 4 \\
3 & 2 & 1 
\end{array}\right]$ then
\begin{align*}
\operatorname{trace}(A+B) & = \operatorname{trace}(A) + \operatorname{trace}(B)\\ 
& = (2 + 10 + 18) + (9 + 5 + 1)\\ 
& = 45
\end{align*}
\begin{align*}
\operatorname{trace}(A)
& = \operatorname{trace}(2A')\\
& = 2\cdot \operatorname{trace}(A')\\
& = 2\cdot \operatorname{trace}
\left[\begin{array}{ccc}
1 & 2 & 3 \\
4 & 5 & 6 \\
7 & 8 & 9
\end{array}\right]\\
& = 2\cdot (1 + 5 + 9)\\
& = 30
\end{align*}
%%%%%
%%%%%
\end{document}
