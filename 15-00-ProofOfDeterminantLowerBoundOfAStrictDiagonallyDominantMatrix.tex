\documentclass[12pt]{article}
\usepackage{pmmeta}
\pmcanonicalname{ProofOfDeterminantLowerBoundOfAStrictDiagonallyDominantMatrix}
\pmcreated{2013-03-22 17:01:11}
\pmmodified{2013-03-22 17:01:11}
\pmowner{Andrea Ambrosio}{7332}
\pmmodifier{Andrea Ambrosio}{7332}
\pmtitle{proof of determinant lower bound of a strict diagonally dominant matrix}
\pmrecord{13}{39304}
\pmprivacy{1}
\pmauthor{Andrea Ambrosio}{7332}
\pmtype{Proof}
\pmcomment{trigger rebuild}
\pmclassification{msc}{15-00}

% this is the default PlanetMath preamble.  as your knowledge
% of TeX increases, you will probably want to edit this, but
% it should be fine as is for beginners.

% almost certainly you want these
\usepackage{amssymb}
\usepackage{amsmath}
\usepackage{amsfonts}

% used for TeXing text within eps files
%\usepackage{psfrag}
% need this for including graphics (\includegraphics)
%\usepackage{graphicx}
% for neatly defining theorems and propositions
%\usepackage{amsthm}
% making logically defined graphics
%%%\usepackage{xypic}

% there are many more packages, add them here as you need them

% define commands here

\begin{document}
Let's define, for any $i=1,2,...,n$%
\[
h_{i}=\left\vert a_{ii}\right\vert -\sum_{j=1,j\neq i}\left\vert
a_{ij}\right\vert 
\]
Then, by strict diagonally dominance, one has $h_{i}>0$ \ $\forall i$.
Let $D=diag\{\left( h_{1}\right) ^{-1},\left( h_{2}\right)
^{-1},...,\left( h_{n}\right) ^{-1}\}$ and $B=DA$, so that the i-th row of $B
$ matrix is equal to the corresponding row of $A$ matrix multiplied by $%
\left( h_{i}\right) ^{-1}$. In this way , one has
\begin{eqnarray*}
d_{i} &=&\left\vert b_{ii}\right\vert -\sum_{j=1,j\neq i}\left\vert
b_{ij}\right\vert  \\
&=&\frac{\left\vert a_{ii}\right\vert }{h_{i}}-\sum_{j=1,j\neq i}\frac{%
\left\vert a_{ij}\right\vert }{h_{i}} \\
&=&1
\end{eqnarray*}
Now, let $\lambda $ be an eigenvalue of $B$, and $v=[v_{1},v_{2},...,v_{n}]$
the corresponding eigenvector; let moreover $p$ be the index of the maximal
component of $v$, i.e. 
\[
\left\vert v_{p}\right\vert \geq \left\vert v_{i}\right\vert \text{ \ }%
\forall i
\]
Of course, by definition of eigenvector, $\left\vert v_{p}\right\vert >0$.
Writing the p-th characteristic equation, we have:%
\begin{eqnarray*}
\lambda v_{p} &=&\sum_{j=1}^{n}b_{pj}v_{j} \\
&=&b_{pp}v_{p}+\sum_{j=1,j\neq p}^{n}b_{pj}v_{j}
\end{eqnarray*}
so that, being $\left\vert \frac{v_{j}}{v_{p}}\right\vert \leq 1$,%
\begin{eqnarray*}
\lambda  &=&b_{pp}+\sum_{j=1,j\neq p}^{n}b_{pj}\frac{v_{j}}{v_{p}} \\
\left\vert \lambda \right\vert  &=&\left\vert b_{pp}+\sum_{j=1,j\neq
p}^{n}b_{pj}\frac{v_{j}}{v_{p}}\right\vert  \\
&\geq &\left\vert \left\vert b_{pp}\right\vert -\left\vert \sum_{j=1,j\neq
p}^{n}b_{pj}\frac{v_{j}}{v_{p}}\right\vert \right\vert  \\
&\geq &\left\vert \left\vert b_{pp}\right\vert -\sum_{j=1,j\neq
p}^{n}\left\vert b_{pj}\right\vert \left\vert \frac{v_{j}}{v_{p}}\right\vert
\right\vert \text{ \ \ \ \ \ \ \ \ \ (*)} \\  \\
&\geq &\left\vert \left\vert b_{pp}\right\vert -\sum_{j=1,j\neq
p}^{n}\left\vert b_{pj}\right\vert \right\vert \text{ \ \ \ \ \ \ \ \ \ (**)} \\  \\
&=&\left\vert b_{pp}\right\vert -\sum_{j=1,j\neq p}^{n}\left\vert
b_{pj}\right\vert  \\
&=&d_{p}=1
\end{eqnarray*}
In this way, we found that each eigenvalue of $B$ is greater than one in
absolute value; for this reason,%
\[
\left\vert \det (B)\right\vert =\left\vert \prod_{i=1}^{n}\lambda
_{i}\right\vert \geq 1
\]
Finally,%
\[
\det (D)=\prod_{i=1}^{n}\left( h_{i}\right) ^{-1}=\left(
\prod_{i=1}^{n}h_{i}\right) ^{-1}
\]
so that 
\begin{eqnarray*}
1 &\leq &\left\vert \det (B)\right\vert  \\
&=&\left\vert \det (D)\right\vert \left\vert \det (A)\right\vert  \\
&=&\left( \prod_{i=1}^{n}h_{i}\right) ^{-1}\left\vert \det (A)\right\vert 
\end{eqnarray*}
whence the thesis.\\ \\
\textit{Remark:} Perhaps it could be not immediately evident where the
hypothesis of strict diagonally dominance is employed in this proof; in
fact, inequality (*) and (**) would be, in a general case, \textit{not valid}%
; they can be stated only because we can assure, by virtue of strict
diagonally dominance, that the final argument of the absolute value ($%
\left\vert b_{pp}\right\vert -\sum_{j=1,j\neq p}^{n}\left\vert
b_{pj}\right\vert $) does remain positive.
%%%%%
%%%%%
\end{document}
