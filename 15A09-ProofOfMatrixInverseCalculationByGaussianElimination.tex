\documentclass[12pt]{article}
\usepackage{pmmeta}
\pmcanonicalname{ProofOfMatrixInverseCalculationByGaussianElimination}
\pmcreated{2013-03-22 14:15:10}
\pmmodified{2013-03-22 14:15:10}
\pmowner{rspuzio}{6075}
\pmmodifier{rspuzio}{6075}
\pmtitle{proof of matrix inverse calculation by Gaussian elimination}
\pmrecord{9}{35700}
\pmprivacy{1}
\pmauthor{rspuzio}{6075}
\pmtype{Definition}
\pmcomment{trigger rebuild}
\pmclassification{msc}{15A09}
%\pmkeywords{Gaussian elimination}

% this is the default PlanetMath preamble.  as your knowledge
% of TeX increases, you will probably want to edit this, but
% it should be fine as is for beginners.

% almost certainly you want these
\usepackage{amssymb}
\usepackage{amsmath}
\usepackage{amsfonts}

% used for TeXing text within eps files
%\usepackage{psfrag}
% need this for including graphics (\includegraphics)
%\usepackage{graphicx}
% for neatly defining theorems and propositions
%\usepackage{amsthm}
% making logically defined graphics
%%%\usepackage{xypic}

% there are many more packages, add them here as you need them

% define commands here
\begin{document}
  Let $A$ be an invertible matrix, and $A^{-1}$ its inverse, whose
  columns are $A^{-1}_1,\cdots,A^{-1}_n$.
  Then, by definition of matrix inverse, $AA^{-1}=I_n$.
  But this implies
  $AA^{-1}_1 = e_1,\ldots,AA^{-1}_n = e_n$, 
  with $e_1,\cdots,e_n$ being the first,$\ldots$,$n$-th column of $I_n$
  respectively.

  $A$ being non singular (or invertible), for all $k \leq n$, $AA^{-1}_k = e_k$ has a solution for $A^{-1}_k$, which can
  be found by Gaussian elimination of $[A \mid e_k]$.

  The only part that changes between the augmented matrices
  constructed is the last column, and these last columns, once the
  Gaussian elimination has been performed, correspond to the columns
  of $A^{-1}$.  Because of this, the steps we need to take for the
  Gaussian elimination are the same for each augmented matrix.

  Therefore, we can solve the matrix equation by performing Gaussian
  elimination on $[A \mid e_1 \cdots e_n]$, or $[A \mid I_n]$.
%%%%%
%%%%%
\end{document}
