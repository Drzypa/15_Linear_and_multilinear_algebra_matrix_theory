\documentclass[12pt]{article}
\usepackage{pmmeta}
\pmcanonicalname{GeneralizedBezoutTheoremOnMatrices}
\pmcreated{2013-03-22 17:43:35}
\pmmodified{2013-03-22 17:43:35}
\pmowner{perucho}{2192}
\pmmodifier{perucho}{2192}
\pmtitle{generalized B\'ezout theorem on matrices}
\pmrecord{8}{40173}
\pmprivacy{1}
\pmauthor{perucho}{2192}
\pmtype{Theorem}
\pmcomment{trigger rebuild}
\pmclassification{msc}{15-01}

% this is the default PlanetMath preamble.  as your knowledge
% of TeX increases, you will probably want to edit this, but
% it should be fine as is for beginners.

% almost certainly you want these
\usepackage{amssymb}
\usepackage{amsmath}
\usepackage{amsfonts}
\usepackage{amsthm}

% used for TeXing text within eps files
%\usepackage{psfrag}
% need this for including graphics (\includegraphics)
%\usepackage{graphicx}
% for neatly defining theorems and propositions
%\usepackage{amsthm}
% making logically defined graphics
%%%\usepackage{xypic}

% there are many more packages, add them here as you need them

% define commands here
\newtheorem{theorem}{Theorem}
\newtheorem{defn}{Definition}
\newtheorem{prop}{Proposition}
\newtheorem{lemma}{Lemma}
\newtheorem{cor*}{Corollary}
\newtheorem{theorem*}{Generalized B\'ezout theorem}
\begin{document}
\begin{theorem*} Let $M[x]$ be an arbitrary matrix polynomial of order $n$ and $A$ a square matrix of the same order. Then, when the matrix polynomial is divided on the right (left) by the characteristic polynomial $xI-A$, the remainder is $M(A)$ ($\widehat{M}(A)$). 
\end{theorem*}
\begin{proof} 
Consider $M[x]$ given by
\begin{equation}
M[x] = M_0 x^m + M_1 x^{m-1} + \cdots + M_m, \qquad (M_0\neq 0).
\end{equation}
The polynomial can also be written as
\begin{equation}
M[x]=x^mM_0+x^{m-1}M_1+\cdots +M_m.
\end{equation}
We are now substituting the scalar argument (real or complex) $x$ by the matrix $A$ and therefore (1) and (2) will, in general, be distinct, as the powers of $A$ need not be permutable with the polynomial matrix coefficients. So that,
\begin{equation*}
M(A) = M_0 A^m + M_1 A^{m-1} + \cdots + M_m
\end{equation*}
and
\begin{equation*}
\widehat{M}(A)=A^mM_0+A^{m-1}M_1+\cdots +M_m,
\end{equation*}
calling $M(A)$ ($\widehat{M}(A)$) the right (left) value of $M[x]$ on substitution of $A$ for $x$. \\
If we divide $M[x]$ by the binomial $xI-A$ ($I$ is the correspondent  identity matrix), we shall prove that the right (left) remainder $R$ ($\widehat{R}$) does not depend on $x$. In fact,
\begin{align*}
M[x]=& M_0 x^m + M_1 x^{m-1} + \cdots + M_m \\
    =& M_0 x^{m-1} (xI-A) + (M_0 A + M_1) x^{m-1} + M_2 x^{m-2} + \cdots + M_m \\
    =& [M_0 x^{m-1} + (M_0 A + M_1) x^{m-2}](xI-A) + (M_0A^2 + M_1A + M_2)x^{m-2} + M_3x^{m-3} + \cdots + M_m \\
    =& [M_0 x^{m-1} + (M_0A + M_1)x^{m-2} + (M_0A^2 + M_1A + M_2)x^{m-3}](xI-A) \\
     & + (M_0A^3 + M_1A^2 + M_2A + M_3)x^{m-3} + M_4x^{m-4} + \cdots + M_m \\
    =& [M_0 x^{m-1}+(M_0A+M_1)x^{m-2}+(M_0A^2+M_1A+M_2)x^{m-3}+\cdots \\
     & +(M_0A^{m-1}+M_1A^{m-2}+\cdots+M_{m-1})](xI-A) + M_0A^m+M_1A^{m-1}+\cdots+M_m, 
\end{align*}
whence we have found that
\begin{equation*}
R= M_0A^m+M_1A^{m-1}+\cdots+M_m\equiv M(A),
\end{equation*}
and analogously that
\begin{equation*}
\widehat{R}= A^mM_0+A^{m-1}M_1+\cdots+M_m\equiv \widehat{M}(A),
\end{equation*}
which proves the theorem.
\end{proof}
From this theorem we have the following
\begin{cor*}
A polynomial $M[x]$ is divisible by the characteristic polynomial $xI-A$ on the right (left) without remainder 
if\mbox{}f $M(A)=0$ ($\widehat{M}(A)=0$).
\end{cor*}

%%%%%
%%%%%
\end{document}
