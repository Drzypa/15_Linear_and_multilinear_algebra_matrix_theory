\documentclass[12pt]{article}
\usepackage{pmmeta}
\pmcanonicalname{CharacteristicPolynomialOfAlgebraicNumber}
\pmcreated{2013-03-22 19:08:41}
\pmmodified{2013-03-22 19:08:41}
\pmowner{pahio}{2872}
\pmmodifier{pahio}{2872}
\pmtitle{characteristic polynomial of algebraic number}
\pmrecord{14}{42046}
\pmprivacy{1}
\pmauthor{pahio}{2872}
\pmtype{Definition}
\pmcomment{trigger rebuild}
\pmclassification{msc}{15A18}
\pmclassification{msc}{12F05}
\pmclassification{msc}{11R04}
\pmrelated{RationalIntegersInIdeals}
\pmrelated{DegreeOfAlgebraicNumber}
\pmdefines{characteristic polynomial}
\pmdefines{characteristic equation}
\pmdefines{$\mathbb{Q}(\vartheta)$-conjugates}
\pmdefines{$K$-conjugates}

% this is the default PlanetMath preamble.  as your knowledge
% of TeX increases, you will probably want to edit this, but
% it should be fine as is for beginners.

% almost certainly you want these
\usepackage{amssymb}
\usepackage{amsmath}
\usepackage{amsfonts}

% used for TeXing text within eps files
%\usepackage{psfrag}
% need this for including graphics (\includegraphics)
%\usepackage{graphicx}
% for neatly defining theorems and propositions
 \usepackage{amsthm}
% making logically defined graphics
%%%\usepackage{xypic}

% there are many more packages, add them here as you need them

% define commands here

\theoremstyle{definition}
\newtheorem*{thmplain}{Theorem}

\begin{document}
Let $\vartheta$ be an algebraic number of degree $n$, $f(x)$ its minimal polynomial and
$$\vartheta_1 = \vartheta,\; \vartheta_2,\; \ldots,\; \vartheta_n$$
its algebraic conjugates.

Let $\alpha$ be an element of the number field $\mathbb{Q}(\vartheta)$ and 
$$r(x) \;:=\; c_0+c_1x+\ldots+c_{n-1}x^{n-1}$$
the canonical polynomial of $\alpha$ with respect to $\vartheta$.\, We consider the numbers
\begin{align}
r(\vartheta_1) \;=\; \alpha \;:=\; \alpha^{(1)},\quad r(\vartheta_2) \;:=\; \alpha^{(2)},\quad 
\ldots,\quad r(\vartheta_n) \;:=\; \alpha^{(n)}
\end{align}
and form the equation
$$
g(x) \;:=\; \prod_{i=1}^n[x\!-\!r(\vartheta_i)] \;=\; (x\!-\!\alpha^{(1)})(x\!-\!\alpha^{(2)})\cdots(x\!-\!\alpha^{(n)}) 
\;=\; x^n\!+\!g_1x^{n-1}\!+\!\ldots\!+\!g_n \;=\; 0,
$$
the roots of which are the numbers (1) and only these.\, The coefficients $g_i$ of the polynomial $g(x)$ are symmetric polynomials in the numbers $\vartheta_1,\, \vartheta_2,\, \ldots,\, \vartheta_n$ and also symmetric polynomials in the numbers $\alpha^{(i)}$.\, The fundamental theorem of symmetric polynomials implies now that the symmetric polynomials $g_i$ in the roots $\vartheta_i$ of the equation \,$f(x) = 0$\, belong to the ring determined by the coefficients of the equation and of the canonical polynomial $r(x)$; thus the numbers $g_i$ are rational (whence the degree of $\alpha$ is at most equal to $n$).\\

It is not hard to show (see the entry degree of algebraic number) of that the degree $k$ of $\alpha$ divides $n$ and that the numbers (1) consist of $\alpha$ and its algebraic conjugates $\alpha_2,\,\ldots,\,\alpha_k$, each of which appears in (1) exactly\, $\frac{n}{k} = m$\, times.\, In fact,\, $g(x) = [a(x)]^m$\, where $a(x)$ is the minimal polynomial of $\alpha$ (consequently, the coefficients 
$g_i$ are integers if $\alpha$ is an algebraic integer).\\

The polynomial $g(x)$ is the \emph{characteristic polynomial} (in German \emph{Hauptpolynom}) of the element $\alpha$ of the algebraic number field $\mathbb{Q}(\vartheta)$ and the equation\, $g(x) = 0$ the \emph{characteristic equation} (\emph{Hauptgleichung}) of $\alpha$.\, See the independence of characteristic polynomial on primitive element. \\

So, the roots of the characteristic equation of $\alpha$ are $\alpha^{(1)},\,\alpha^{(2)},\,\ldots,\,\alpha^{(n)}$.\, They are called the $\mathbb{Q}(\vartheta)$\emph{-conjugates} of $\alpha$; they all are algebraic conjugates of 
$\alpha$.


%%%%%
%%%%%
\end{document}
