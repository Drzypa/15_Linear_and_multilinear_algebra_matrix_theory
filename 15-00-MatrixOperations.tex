\documentclass[12pt]{article}
\usepackage{pmmeta}
\pmcanonicalname{MatrixOperations}
\pmcreated{2013-03-22 13:57:54}
\pmmodified{2013-03-22 13:57:54}
\pmowner{djao}{24}
\pmmodifier{djao}{24}
\pmtitle{matrix operations}
\pmrecord{6}{34734}
\pmprivacy{1}
\pmauthor{djao}{24}
\pmtype{Definition}
\pmcomment{trigger rebuild}
\pmclassification{msc}{15-00}
\pmdefines{matrix addition}
\pmdefines{matrix multiplication}
\pmdefines{matrix product}
\pmdefines{standard matrix multiplication}

\endmetadata

% this is the default PlanetMath preamble.  as your knowledge
% of TeX increases, you will probably want to edit this, but
% it should be fine as is for beginners.

% almost certainly you want these
\usepackage{amssymb}
\usepackage{amsmath}
\usepackage{amsfonts}

% used for TeXing text within eps files
%\usepackage{psfrag}
% need this for including graphics (\includegraphics)
%\usepackage{graphicx}
% for neatly defining theorems and propositions
%\usepackage{amsthm}
% making logically defined graphics
%%%\usepackage{xypic} 

% there are many more packages, add them here as you need them

% define commands here
\begin{document}
A {\em matrix} is an array, or a rectangular grid, of numbers. An $m
\times n$ matrix is one which has $m$ rows and $n$ columns. Examples of
matrices include:
\begin{itemize}
\item The $2 \times 3$ matrix
$$
A =
\begin{pmatrix}
1 & -2 & 2 \\
1 & 0 & 2
\end{pmatrix}
$$
\item The $3 \times 3$ matrix
$$
B =
\begin{pmatrix}
1 & 0 & 1 \\
3 & -1 & 5 \\
4 & 2 & 0
\end{pmatrix}
$$
\item The $3 \times 1$ matrix
$$
C =
\begin{pmatrix}
7 \\
7 \\
2
\end{pmatrix}
$$
\item The $1 \times 2$ matrix
$$
D =
\begin{pmatrix}
\frac{1}{2} & 2
\end{pmatrix}
$$
\end{itemize}

All of our example matrices (except the last one) have entries which are
integers. In general, matrices are allowed to have their entries taken
from any ring $R$. The set of all $m \times n$ matrices with entries in
a ring $R$ is denoted $M_{m \times n}(R)$. If a matrix has exactly as
many rows as it has columns, we say it is a square matrix.

Addition of two matrices is allowed provided that both matrices have the
same number of rows and the same number of columns. The sum of two such
matrices is obtained by adding their respective entries. For example,
$$
\begin{pmatrix}
7 \\
7 \\
2
\end{pmatrix} +
\begin{pmatrix}
-1 \\
4.5 \\
0
\end{pmatrix} =
\begin{pmatrix}
7 + (-1) \\
7 + 4.5 \\
2 + 0
\end{pmatrix} =
\begin{pmatrix}
6 \\
11.5 \\
2
\end{pmatrix}
$$

Multiplication of two matrices is allowed provided that the number of
columns of the first matrix equals the number of rows of the second
matrix. (For example, multiplication of a $2 \times 3$ matrix with a $3
\times 3$ is allowed, but multiplication of a $3 \times 3$ matrix with a
$2 \times 3$ matrix is not allowed, since the first matrix has 3
columns, and the second matrix has 2 rows, and 3 doesn't equal 2.)
In this case the matrix multiplication is defined by
$$(AB)_{ij}=\sum_k (A)_{ik} (B)_{kj}\;.$$

We will describe how matrix multiplication works is with an
example. Let
$$
A =
\begin{pmatrix}
1 & -2 & 2 \\
1 & 0 & 2
\end{pmatrix},\ \ \ \
B =
\begin{pmatrix}
1 & 0 & 1 \\
3 & -1 & 5 \\
4 & 2 & 0
\end{pmatrix}
$$
be the two matrices that we used above as our very first two examples of
matrices. Since $A$ is a $2 \times 3$ matrix, and $B$ is a $3 \times 3$
matrix, it is legal to multiply $A$ and $B$, but it is not legal to
multiply $B$ and $A$. The method for computing the product $AB$ is to
place $A$ below and to the left of $B$, as follows:
$$
\begin{matrix}
& \begin{pmatrix}
1 & 0 & 1 \\
3 & -1 & 5 \\
4 & 2 & 0
\end{pmatrix}
\\
\begin{pmatrix}
1 & -2 & 2 \\
1 & 0 & 2
\end{pmatrix} &
\begin{pmatrix}
X & Y & \ \ \\
\ \ &\ \ &\ \
\end{pmatrix}
\end{matrix}
$$
$A$ is always in the bottom left corner, $B$ is in the top right corner,
and the product, $AB$, is always in the bottom right corner. We see from
the picture that $AB$ will be a $2 \times 3$ matrix. (In general, $AB$
has as many rows as $A$, and as many columns as $B$.)

Let us compute the top left entry of $AB$, denoted by $X$ in the above
picture. The way to calculate this entry of $AB$ (or any other entry) is
to take the dot product of the stuff above it [which is $(1,3,4)$] and
the stuff to the left of it [which is $(1,-2,2)$]. In this case, we have
$$
X = 1 \cdot 1 + 3 \cdot (-2) + 4 \cdot 2 = 3.
$$
Similarly, the top middle entry of $AB$ (where the $Y$ is in the above
picture) is gotten by taking the dot product of the stuff above it:
$(0,-1,2)$, and the stuff to the left of it: $(1,-2,2)$, which gives
$$
Y = 0\cdot 1 + (-1) \cdot (-2) + 2 \cdot 2 = 6
$$

Continuing in this way, we can compute every entry of $AB$ one by one to
get
$$
\begin{matrix}
& \begin{pmatrix}
1 & 0 & 1 \\
3 & -1 & 5 \\
4 & 2 & 0
\end{pmatrix}
\\
\begin{pmatrix}
1 & -2 & 2 \\
1 & 0 & 2
\end{pmatrix} &
\begin{pmatrix}
3 & 6 & -9 \\
9 & 4 & 1
\end{pmatrix}
\end{matrix}
$$
and so
$$
AB = \begin{pmatrix}
3 & 6 & -9 \\
9 & 4 & 1
\end{pmatrix}.
$$
If one tries to compute the {\bf illegal} product $BA$ using this
procedure, one winds up with
$$
\begin{matrix}
& \begin{pmatrix}
1 & -2 & 2 \\
1 & 0 & 2
\end{pmatrix} \\
\begin{pmatrix}
1 & 0 & 1 \\
3 & -1 & 5 \\
4 & 2 & 0
\end{pmatrix} &
\begin{pmatrix}
?\ &\ \ & \ \ \\
\ \ &\ \ & \ \ \\
\ \ &\ \ & \ \
\end{pmatrix}
\end{matrix}
$$
The top left entry of this illegal product (marked with a $?$ above)
would have to be the dot product of the stuff above it: $(1,1)$, and the
stuff to the left of it: $(1,0,1)$, but these vectors do not have the same
length, so it is impossible to take their dot product, and consequently
it is impossible to take the product of the matrices $BA$.

Under the correspondence of matrices and linear transformations, one can
show that matrix multiplication is equivalent to composition of linear
transformations, which explains why matrix multiplication is
defined in a manner which is so odd at first sight, and why this strange
manner of multiplication is so useful in mathematics.
%%%%%
%%%%%
\end{document}
