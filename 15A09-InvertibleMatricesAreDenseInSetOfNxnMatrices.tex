\documentclass[12pt]{article}
\usepackage{pmmeta}
\pmcanonicalname{InvertibleMatricesAreDenseInSetOfNxnMatrices}
\pmcreated{2013-03-22 15:38:51}
\pmmodified{2013-03-22 15:38:51}
\pmowner{stevecheng}{10074}
\pmmodifier{stevecheng}{10074}
\pmtitle{invertible matrices are dense in set of nxn matrices}
\pmrecord{5}{37579}
\pmprivacy{1}
\pmauthor{stevecheng}{10074}
\pmtype{Theorem}
\pmcomment{trigger rebuild}
\pmclassification{msc}{15A09}

\endmetadata

\usepackage{amssymb}
\usepackage{amsmath}
\usepackage{amsfonts}
\usepackage{enumerate}

% used for TeXing text within eps files
%\usepackage{psfrag}
% need this for including graphics (\includegraphics)
%\usepackage{graphicx}
% making logically defined graphics
%%%\usepackage{xypic}

% define commands here
\newcommand{\complex}{\mathbb{C}}
\newcommand{\real}{\mathbb{R}}
\newcommand{\rat}{\mathbb{Q}}
\newcommand{\nat}{\mathbb{N}}

\providecommand{\abs}[1]{\lvert#1\rvert}
\providecommand{\absW}[1]{\left\lvert#1\right\rvert}
\providecommand{\absB}[1]{\Bigl\lvert#1\Bigr\rvert}
\providecommand{\norm}[1]{\lVert#1\rVert}
\providecommand{\normW}[1]{\left\lVert#1\right\rVert}
\providecommand{\normB}[1]{\Bigl\lVert#1\Bigr\rVert}
\providecommand{\defnterm}[1]{\emph{#1}}

\DeclareMathOperator{\D}{D}
\DeclareMathOperator{\linspan}{span}
\begin{document}
If $A$ is any $n \times n$ matrix with real or complex entries,
Then there are invertible matrices arbitrarily close to $A$,
under any norm for the $n \times n$ matrices.

This is easily proven as follows. Take any invertible matrix $B$
(e.g. $B = I$), and consider the function
(for $t \in \real$ or $\complex$)
\[
p(t) = \det\bigl((1-t)A + tB \bigr)\,.
\]
Clearly, $p$ is a polynomial function.  It is not identically zero, for $p(1) = \det B  \neq 0$.
But a non-zero polynomial has only finitely many zeroes,
So given any single point $t_0$, if $t$ is close enough but unequal to $t_0$,
$p(t)$ must be non-zero.  In particular, applying this for $t_0 = 0$, 
we see that the matrix $(1-t)A + tB$ is invertible for small $t \neq 0$.
And the distance of this matrix from $A$ is $\abs{t} \, \norm{B-A}$, 
which becomes small as $t$ gets small.
%%%%%
%%%%%
\end{document}
