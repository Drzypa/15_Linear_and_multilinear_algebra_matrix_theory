\documentclass[12pt]{article}
\usepackage{pmmeta}
\pmcanonicalname{NilpotentTransformation}
\pmcreated{2013-03-22 12:19:52}
\pmmodified{2013-03-22 12:19:52}
\pmowner{rmilson}{146}
\pmmodifier{rmilson}{146}
\pmtitle{nilpotent transformation}
\pmrecord{7}{31961}
\pmprivacy{1}
\pmauthor{rmilson}{146}
\pmtype{Definition}
\pmcomment{trigger rebuild}
\pmclassification{msc}{15-00}
\pmsynonym{nilpotent}{NilpotentTransformation}
\pmrelated{LinearTransformation}

\endmetadata

\usepackage{amsmath}
\usepackage{amsfonts}
\usepackage{amssymb}

\newtheorem{proposition}{Proposition}
\begin{document}
A linear transformation $N: U\rightarrow U$ is called nilpotent if there exists a $k\in\mathbb{N}$ such that 
$$N^k = 0.$$
A nilpotent transformation naturally determines a flag of subspaces
$$ \{0\} \subset \ker N^1 \subset \ker N^2 \subset \ldots \subset \ker N^{k-1} \subset \ker N^k = U$$
and a signature
$$0  = n_0 < n_1 < n_2 < \ldots < n_{k-1} < n_k = \dim U,\qquad n_i =
\dim \ker N^i.$$
The signature is governed by the following constraint, and
characterizes $N$ up to linear isomorphism.
\begin{proposition}
A sequence of increasing natural numbers is the signature of a nil-potent transformation if and only if
$$n_{j+1} - n_{j} \leq n_{j} - n_{j-1}$$
for all $j=1,\ldots,k-1$.  Equivalently, there exists a basis of $U$
such that the matrix of $N$ relative to this basis is block diagonal
$$\begin{pmatrix}
N_1 & 0 & 0 & \ldots & 0\\
0 & N_2 & 0 & \ldots & 0 \\
0 & 0 & N_3 & \ldots & 0 \\
\vdots & \vdots & \vdots & \ddots & \vdots \\
0 & 0 & 0 & \ldots & N_k
\end{pmatrix},$$
with each of the blocks having the form
$$
N_i = \begin{pmatrix}
  0 & 1 & 0 & \ldots & 0 & 0 \\
  0 & 0 & 1 & \ldots & 0 & 0 \\
  \vdots & \vdots & \vdots & \ddots & \vdots \\
  0 & 0 & 0 & \ldots & 1 & 0 \\
  0 & 0 & 0 & \ldots & 0 & 1 \\
  0 & 0 & 0 & \ldots & 0 & 0   
\end{pmatrix}
$$
Letting $d_i$ denote the number of blocks of size $i$, the
signature of $N$ is given by
$$n_i = n_{i-1} + d_i + d_{i+1} + \ldots + d_k,\quad i=1,\ldots,k$$
\end{proposition}
%%%%%
%%%%%
\end{document}
