\documentclass[12pt]{article}
\usepackage{pmmeta}
\pmcanonicalname{CentralizerOfMatrixUnits}
\pmcreated{2013-03-22 18:39:58}
\pmmodified{2013-03-22 18:39:58}
\pmowner{asteroid}{17536}
\pmmodifier{asteroid}{17536}
\pmtitle{centralizer of matrix units}
\pmrecord{8}{41410}
\pmprivacy{1}
\pmauthor{asteroid}{17536}
\pmtype{Theorem}
\pmcomment{trigger rebuild}
\pmclassification{msc}{15A30}
\pmclassification{msc}{16S50}

\endmetadata

% this is the default PlanetMath preamble.  as your knowledge
% of TeX increases, you will probably want to edit this, but
% it should be fine as is for beginners.

% almost certainly you want these
\usepackage{amssymb}
\usepackage{amsmath}
\usepackage{amsfonts}

% used for TeXing text within eps files
%\usepackage{psfrag}
% need this for including graphics (\includegraphics)
%\usepackage{graphicx}
% for neatly defining theorems and propositions
%\usepackage{amsthm}
% making logically defined graphics
%%%\usepackage{xypic}

% there are many more packages, add them here as you need them

% define commands here

\begin{document}
{\bf Theorem -} Let $R$ be a ring with identity $1$ and $M_n(R)$ the ring of $n \times n$ matrices with entries in $R$. The centralizer of the matrix units is the set $R\cdot Id$, consisting of all multiples of the identity matrix.

$\,$

{\bf \emph{\PMlinkescapetext{Proof}:}} It is clear that the multiples of the identity matrix commute with all matrix units, and therefore belong to their centralizer. We will now prove the converse.

We will regard the elements of $M_n(R)$ as endomorphisms of the module $\oplus_{i =1}^n R$. We will denote by $\{e_i\}$ the canonical basis of $\oplus_{i =1}^n R$ and by $E_{ij}$ the matrix unit whose entry $(i,j)$ is $1$.

Let $S = [s_{ij}] \in M_n(R)$ be an element of the centralizer of the matrix units. For all $i, j, k$ we must have

\begin{align}
SE_{ij}\, e_k = E_{ij}S\, e_k
\end{align}

But a straightforward computation shows that $SE_{ij\,}e_j = S\, e_i$ and $E_{ij}S\, e_j = s_{jj}\,e_i$. Since $j$ is arbitrary we see, by equality (1), that all $s_{jj}$ are equal, say $s_{jj} = s \in R$.

 Hence, $S\, e_i = s\, e_i$, wich means that $S = s\,Id$. We conclude that $S$ must be a multiple of the identity matrix. $\square$
%%%%%
%%%%%
\end{document}
