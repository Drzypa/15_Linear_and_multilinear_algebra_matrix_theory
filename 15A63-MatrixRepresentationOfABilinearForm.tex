\documentclass[12pt]{article}
\usepackage{pmmeta}
\pmcanonicalname{MatrixRepresentationOfABilinearForm}
\pmcreated{2013-03-22 14:56:22}
\pmmodified{2013-03-22 14:56:22}
\pmowner{vitriol}{148}
\pmmodifier{vitriol}{148}
\pmtitle{matrix representation of a bilinear form}
\pmrecord{5}{36630}
\pmprivacy{1}
\pmauthor{vitriol}{148}
\pmtype{Definition}
\pmcomment{trigger rebuild}
\pmclassification{msc}{15A63}
\pmclassification{msc}{11E39}
\pmclassification{msc}{47A07}

\endmetadata

% this is the default PlanetMath preamble.  as your knowledge
% of TeX increases, you will probably want to edit this, but
% it should be fine as is for beginners.

% almost certainly you want these
\usepackage{amssymb}
\usepackage{amsmath}
\usepackage{amsfonts}

% used for TeXing text within eps files
%\usepackage{psfrag}
% need this for including graphics (\includegraphics)
%\usepackage{graphicx}
% for neatly defining theorems and propositions
%\usepackage{amsthm}
% making logically defined graphics
%%%\usepackage{xypic}

% there are many more packages, add them here as you need them

% define commands here
\begin{document}
Given a bilinear form, $B : U \times V \rightarrow K$, we show how we can represent it with a matrix, with respect to a particular pair of bases for $U$ and $V$

Suppose $U$ and $V$ are finite-dimensional and we have chosen bases, ${{\cal B}_1} = \{e_1, \ldots\}$ and ${{\cal B}_2}=\{f_1, \ldots\}$. Now we define the matrix $C$ with entries $C_{ij} = B(e_i, f_j)$. This will be the matrix associated to $B$ with respect to this basis as follows; If we write $x,y \in V$ as column vectors in terms of the chosen bases, then check $B(x,y) = x^T C y$. Further if we choose the corresponding dual bases for $U^\ast$ and $V^\ast$ then $C$ and $C^T$ are the corresponding matrices for $B_R$ and $B_L$, respectively (in the sense of linear maps). Thus we see that a symmetric bilinear form is represented by a symmetric matrix, and similarly for skew-symmetric forms.

Let ${{\cal B}_1^\prime}$ and ${{\cal B}_2^\prime}$ be new bases, and $P$ and $Q$ the corresponding change of basis matrices. Then the new matrix is $C^\prime = P^{T}CQ$.
%%%%%
%%%%%
\end{document}
