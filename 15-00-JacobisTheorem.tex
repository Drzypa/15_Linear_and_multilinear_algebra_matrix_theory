\documentclass[12pt]{article}
\usepackage{pmmeta}
\pmcanonicalname{JacobisTheorem}
\pmcreated{2013-03-22 13:33:06}
\pmmodified{2013-03-22 13:33:06}
\pmowner{Koro}{127}
\pmmodifier{Koro}{127}
\pmtitle{Jacobi's theorem}
\pmrecord{13}{34156}
\pmprivacy{1}
\pmauthor{Koro}{127}
\pmtype{Theorem}
\pmcomment{trigger rebuild}
\pmclassification{msc}{15-00}

\endmetadata

% this is the default PlanetMath preamble.  as your knowledge
% of TeX increases, you will probably want to edit this, but
% it should be fine as is for beginners.

% almost certainly you want these
\usepackage{amssymb}
\usepackage{amsmath}
\usepackage{amsfonts}

% used for TeXing text within eps files
%\usepackage{psfrag}
% need this for including graphics (\includegraphics)
%\usepackage{graphicx}
% for neatly defining theorems and propositions
%\usepackage{amsthm}
% making logically defined graphics
%%%\usepackage{xypic}

% there are many more packages, add them here as you need them

% define commands here
\begin{document}
{\bf Jacobi's Theorem} Any skew-symmetric matrix of odd order has determinant equal to $0$. 

{\bf Proof.} Suppose $A$ is an $n\times n$ square matrix. 
For the determinant, we then have $\det A = \det A^T$, and
$\det (-A) = (-1)^n \det A$. Thus, since $n$ is odd, and $A^T=-A$, we have
 $\det A = -\det A$, and the theorem follows.  $\Box$

\subsubsection{Remarks}
\begin{enumerate}
\item  According to \cite{eves}, this theorem was given by 
Carl Gustav Jacob Jacobi (1804-1851) \cite{jacobi} in 1827.

\item The $2\times 2$ matrix $\left( \begin{array}{cc}
 0 & 1 \\
 -1 & 0
 \end{array} \right)$ shows that Jacobi's theorem does not hold for $2\times 2$ 
matrices. The determinant of the  $2n\times 2n$ block matrix with 
these $2\times 2$ matrices on the diagonal equals $(-1)^n$. Thus Jacobi's theorem
does not hold for matrices of even order.

\item For $n=3$, any antisymmetric matrix $A$ can be written 
as
$$ A =
\begin{pmatrix}
0 & -v_3 & v_2 \\
v_3 & 0 & -v_1 \\
-v_2 & v_1 & 0
\end{pmatrix}
$$
for some real $v_1,v_2,v_3$, which can be written as a
vector $v=(v_1,v_2,v_3)$. Then $A$ is the matrix representing the
mapping $u\mapsto v\times u$, that is, the cross product with 
respect to $v$. Since $Av=v\times v=0$, we have $\det A=0$. 
\end{enumerate}

\begin{thebibliography}{9}
\bibitem {eves} H. Eves,
        \emph{Elementary Matrix Theory},
        Dover publications, 1980.
\bibitem{jacobi}
 The MacTutor History of Mathematics archive,
 \PMlinkexternal{Carl Gustav Jacob Jacobi}{http://www-gap.dcs.st-and.ac.uk/~history/Mathematicians/Jacobi.html}
\end{thebibliography}
%%%%%
%%%%%
\end{document}
