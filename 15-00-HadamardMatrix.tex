\documentclass[12pt]{article}
\usepackage{pmmeta}
\pmcanonicalname{HadamardMatrix}
\pmcreated{2013-03-22 13:09:45}
\pmmodified{2013-03-22 13:09:45}
\pmowner{Koro}{127}
\pmmodifier{Koro}{127}
\pmtitle{Hadamard matrix}
\pmrecord{13}{33605}
\pmprivacy{1}
\pmauthor{Koro}{127}
\pmtype{Definition}
\pmcomment{trigger rebuild}
\pmclassification{msc}{15-00}
\pmclassification{msc}{05B20}
\pmsynonym{Hadamard}{HadamardMatrix}
\pmrelated{HadamardConjecture}

\endmetadata

\usepackage{amssymb}
\usepackage{amsmath}
\usepackage{amsfonts}
\begin{document}
\PMlinkescapeword{rows}
\PMlinkescapeword{row}
\PMlinkescapeword{columns}
\PMlinkescapeword{column}
\PMlinkescapeword{contain}
\PMlinkescapeword{property}

\PMlinkescapeword{order}
\PMlinkescapeword{states}
\PMlinkescapeword{satisfies}
\PMlinkescapeword{integer}

An $n\times n$ matrix $H = (h_{ij})$ is a \emph{Hadamard matrix} of order $n$ if the entries of $H$ are either 
$+1$ or $-1$ and such that $HH^T = nI,$ where $H^T$ is the transpose of $H$ and $I$ is the order $n$
identity matrix.

In other words, an $n\times n$ matrix with only $+1$ and $-1$ as its elements is Hadamard if the inner product of two distinct rows is 
$0$ and the inner product of a row with itself is $n$.

A few examples of Hadamard matrices are 
$$\begin{bmatrix} 1 & 1 \\ 1 & -1\end{bmatrix} ,   \begin{bmatrix} -1 & 1 & 1 & 1 \\ 1 & -1 & 1 & 1 \\ 1 & 1 & -1 & 1 \\ 1 & 1 & 1 & -1\end{bmatrix},   \begin{bmatrix} 1 & 1 & 1 & 1 \\ 1 & -1 & 1 & -1 \\ 1 & 1 & -1 & -1 \\ 1 & -1 & -1 & 1\end{bmatrix}$$
These matrices were first considered as Hadamard determinants, because the determinant of a Hadamard matrix satisfies equality in Hadamard's determinant theorem, which states that if $X = (x_{ij})$ is a matrix of order $n$ where $|x_{ij}| \leq 1$ for all $i$ and $j,$ then
$$det(X) \leq n^{n/2}$$

\textbf{Property 1:}

The order of a Hadamard matrix is $1, 2$ or $4n,$ where $n$ is an integer.

\textbf{Property 2:}

If the rows and columns of a Hadamard matrix are permuted, the matrix remains Hadamard.

\textbf{Property 3:}

If any row or column is multiplied  by $-1,$ the Hadamard property is retained.

Hence it is always possible to arrange to have the first row and first column of  a Hadamard matrix contain
only $+1$ entries. A Hadamard matrix in this form is said to be \emph{normalized}.

Hadamard matrices are common in signal processing and coding applications.
%%%%%
%%%%%
\end{document}
