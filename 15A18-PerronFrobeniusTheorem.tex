\documentclass[12pt]{article}
\usepackage{pmmeta}
\pmcanonicalname{PerronFrobeniusTheorem}
\pmcreated{2013-03-22 13:18:26}
\pmmodified{2013-03-22 13:18:26}
\pmowner{jarino}{552}
\pmmodifier{jarino}{552}
\pmtitle{Perron-Frobenius theorem}
\pmrecord{5}{33812}
\pmprivacy{1}
\pmauthor{jarino}{552}
\pmtype{Theorem}
\pmcomment{trigger rebuild}
\pmclassification{msc}{15A18}
\pmrelated{FundamentalTheoremOfDemography}

\endmetadata

% this is the default PlanetMath preamble.  as your knowledge
% of TeX increases, you will probably want to edit this, but
% it should be fine as is for beginners.

% almost certainly you want these
\usepackage{amssymb}
\usepackage{amsmath}
\usepackage{amsfonts}

% used for TeXing text within eps files
%\usepackage{psfrag}
% need this for including graphics (\includegraphics)
%\usepackage{graphicx}
% for neatly defining theorems and propositions
%\usepackage{amsthm}
% making logically defined graphics
%%%\usepackage{xypic}

% there are many more packages, add them here as you need them

% define commands here
\begin{document}
Let $A$ be a nonnegative matrix. Denote its spectrum by $\sigma(A)$.
Then the spectral radius $\rho(A)$ is an eigenvalue, that is, $\rho(A)\in \sigma(A)$, and is associated to a nonnegative eigenvector.

If, in addition, $A$ is an irreducible matrix, then $|\rho(A)|\geq |\lambda|$, for all $\lambda\in \sigma(A)$, $\lambda\neq \rho(A)$, and $\rho(A)$ is a simple eigenvalue associated to a positive eigenvector.

If, in addition, $A$ is a primitive matrix, then $\rho(A)>|\lambda|$ for all $\lambda\in\sigma(A)$, $\lambda\neq\rho(A)$.
%%%%%
%%%%%
\end{document}
