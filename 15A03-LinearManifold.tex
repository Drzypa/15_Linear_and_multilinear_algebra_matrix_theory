\documentclass[12pt]{article}
\usepackage{pmmeta}
\pmcanonicalname{LinearManifold}
\pmcreated{2013-03-22 14:04:32}
\pmmodified{2013-03-22 14:04:32}
\pmowner{matte}{1858}
\pmmodifier{matte}{1858}
\pmtitle{linear manifold}
\pmrecord{6}{35435}
\pmprivacy{1}
\pmauthor{matte}{1858}
\pmtype{Definition}
\pmcomment{trigger rebuild}
\pmclassification{msc}{15A03}
\pmclassification{msc}{15-00}
\pmrelated{VectorSubspace}
\pmrelated{LineSegment}
\pmdefines{hyperplane}

% this is the default PlanetMath preamble.  as your knowledge
% of TeX increases, you will probably want to edit this, but
% it should be fine as is for beginners.

% almost certainly you want these
\usepackage{amssymb}
\usepackage{amsmath}
\usepackage{amsfonts}

% used for TeXing text within eps files
%\usepackage{psfrag}
% need this for including graphics (\includegraphics)
%\usepackage{graphicx}
% for neatly defining theorems and propositions
%\usepackage{amsthm}
% making logically defined graphics
%%%\usepackage{xypic}

% there are many more packages, add them here as you need them

% define commands here

\newcommand{\sR}[0]{\mathbb{R}}
\newcommand{\sC}[0]{\mathbb{C}}
\newcommand{\sN}[0]{\mathbb{N}}
\newcommand{\sZ}[0]{\mathbb{Z}}

% The below lines should work as the command
% \renewcommand{\bibname}{References}
% without creating havoc when rendering an entry in 
% the page-image mode.
\makeatletter
\@ifundefined{bibname}{}{\renewcommand{\bibname}{References}}
\makeatother

\newcommand*{\norm}[1]{\lVert #1 \rVert}
\newcommand*{\abs}[1]{| #1 |}
\begin{document}
{\bf Definition}  
Suppose $V$ is a vector space and suppose that $L$ is a 
non-empty subset of $V$. If there exists a $v\in V$ such that $L+v=\{ v+l \mid l\in L\}$
is a vector subspace of $V$, then $L$ is a {\bf linear manifold} of $V$. Then we
say that the dimension of $L$ is the dimension of $L+v$ and write $\dim L = \dim (L+v)$.
In the important case $\dim L = \dim V -1$,  $L$ is called a {\bf hyperplane}.

A linear manifold is, in other words, a linear subspace that has possibly been 
shifted away from the origin. 
For instance, in $\sR^2$ examples of linear 
manifolds are points, lines (which are hyperplanes), and $\sR^2$ itself. 
In $\sR^n$ hyperplanes naturally describe tangent planes to a smooth 
hyper surface. 

\begin{thebibliography}{9}
 \bibitem{cristescu} R. Cristescu, \emph{Topological vector spaces},
 Noordhoff International Publishing, 1977.
\end{thebibliography}
%%%%%
%%%%%
\end{document}
