\documentclass[12pt]{article}
\usepackage{pmmeta}
\pmcanonicalname{SquareMatrix}
\pmcreated{2013-03-22 12:11:27}
\pmmodified{2013-03-22 12:11:27}
\pmowner{akrowne}{2}
\pmmodifier{akrowne}{2}
\pmtitle{square matrix}
\pmrecord{9}{31472}
\pmprivacy{1}
\pmauthor{akrowne}{2}
\pmtype{Definition}
\pmcomment{trigger rebuild}
\pmclassification{msc}{15-00}

\endmetadata

\usepackage{amssymb}
\usepackage{amsmath}
\usepackage{amsfonts}

%\usepackage{psfrag}
%\usepackage{graphicx}
%%%%\usepackage{xypic}
\begin{document}
A \emph{square matrix} has the same number of rows as columns.

Examples:

\begin{enumerate}

\item

$\begin{pmatrix}
  1.00000 & 0.50000 & 0.33333 & 0.25000 \\
  0.50000 & 0.33333 & 0.25000 & 0.20000 \\
  0.33333 & 0.25000 & 0.20000 & 0.16667 \\
  0.25000 & 0.20000 & 0.16667 & 0.14286
\end{pmatrix} $

\item

$\begin{pmatrix}1\end{pmatrix}$

\item
 
$\begin{pmatrix}
  0.94 & 0.37 & 0.71 & 0.32 & 0.58 \\
  0.90 & 0.16 & 0.74 & 0.83 & 0.27 \\
  0.50 & 0.03 & 0.07 & 0.49 & 0.55 \\
  0.15 & 0.59 & 0.43 & 0.03 & 0.76 \\
  0.04 & 0.64 & 0.61 & 0.17 & 0.29 
\end{pmatrix}$

\item

$\begin{pmatrix} 89 & 38 & 50 \\  64 & 26 & 98 \\  40 & 96 & 83 \end{pmatrix}$

\end{enumerate}

The notation $\operatorname{Mat}_n(\mathbb{K})$ is often used to signify the set of square matrices which are of order $n$ (size $n \times n$) with elements drawn from a field $\mathbb{K}$.  Thus, one would use $a \in  \operatorname{Mat}_3(\mathbb{C})$ to declare that $a$ is a three-by-three matrix   with elements that are complex numbers.

\textbf{Property}: Suppose $A$ and $B$ are matrices such that $AB$ is a square matrix.  Then the product $BA$ is defined and also a square matrix.
%%%%%
%%%%%
%%%%%
\end{document}
