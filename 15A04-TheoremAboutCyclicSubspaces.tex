\documentclass[12pt]{article}
\usepackage{pmmeta}
\pmcanonicalname{TheoremAboutCyclicSubspaces}
\pmcreated{2013-03-22 14:15:16}
\pmmodified{2013-03-22 14:15:16}
\pmowner{Mathprof}{13753}
\pmmodifier{Mathprof}{13753}
\pmtitle{theorem about cyclic subspaces}
\pmrecord{12}{35702}
\pmprivacy{1}
\pmauthor{Mathprof}{13753}
\pmtype{Theorem}
\pmcomment{trigger rebuild}
\pmclassification{msc}{15A04}

\endmetadata

% this is the default PlanetMath preamble.  as your knowledge
% of TeX increases, you will probably want to edit this, but
% it should be fine as is for beginners.

% almost certainly you want these
\usepackage{amssymb}
\usepackage{amsmath}
\usepackage{amsfonts}

% used for TeXing text within eps files
%\usepackage{psfrag}
% need this for including graphics (\includegraphics)
%\usepackage{graphicx}
% for neatly defining theorems and propositions
%\usepackage{amsthm}
% making logically defined graphics
%%%\usepackage{xypic}

% there are many more packages, add them here as you need them

% define commands here
\begin{document}
Let $k$ be field, $V$ a vector space over $k$, $\dim V=n$, and $T:V \to V$ a linear operator. Let $W$ be a subspace of $V$. And let $v_{1},\ldots,v_{r} \in V$ such that $W=Z(v_{1},T) \bigoplus \cdots \bigoplus Z(v_{r},T)$ (see the cyclic subspace definition), and $(m_{v_{i}}, m_{v_{j}})=1$ if $i \neq j$, where $m_{v}$ denotes the minimal polynomial of $v$ (or in other words, its annihilator polynomial). Then, $Z(v_{1}+\cdots+v_{r},T)=Z(v_{1},T) \bigoplus \cdots \bigoplus Z(v_{r}, T)$, and $m_{v_{1}+ \cdots +v_{r}}=m_{v_{1}}  \cdots  m_{v_{r}}$.
%%%%%
%%%%%
\end{document}
