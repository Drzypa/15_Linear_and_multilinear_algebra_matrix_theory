\documentclass[12pt]{article}
\usepackage{pmmeta}
\pmcanonicalname{JordanCanonicalFormTheorem}
\pmcreated{2013-03-22 12:59:21}
\pmmodified{2013-03-22 12:59:21}
\pmowner{Mathprof}{13753}
\pmmodifier{Mathprof}{13753}
\pmtitle{Jordan canonical form theorem}
\pmrecord{16}{33364}
\pmprivacy{1}
\pmauthor{Mathprof}{13753}
\pmtype{Theorem}
\pmcomment{trigger rebuild}
\pmclassification{msc}{15A18}
\pmsynonym{Jordan canonical form}{JordanCanonicalFormTheorem}
\pmrelated{PartitionedMatrix}
\pmrelated{SimultaneousUpperTriangularBlockDiagonalizationOfCommutingMatrices}
\pmrelated{Diagonalizable2}
\pmdefines{Jordan block}
\pmdefines{Jordan matrix}

% this is the default PlanetMath preamble.  as your knowledge
% of TeX increases, you will probably want to edit this, but
% it should be fine as is for beginners.

% almost certainly you want these
\usepackage{amssymb}
\usepackage{amsmath}
\usepackage{amsfonts}

% used for TeXing text within eps files
%\usepackage{psfrag}
% need this for including graphics (\includegraphics)
%\usepackage{graphicx}
% for neatly defining theorems and propositions
\usepackage{amsthm}
% making logically defined graphics
%%%\usepackage{xypic}

% there are many more packages, add them here as you need them

% define commands here
\newtheorem*{thms}{Theorem}
\begin{document}
A \textbf{Jordan block} or \textbf{Jordan matrix} is a matrix of the form

$$\begin{pmatrix}
\lambda & 1 & 0 & \cdots & 0\\
0 & \lambda & 1 & \cdots & 0\\
0 & 0 & \lambda & \cdots & 0\\
\vdots & \vdots & \vdots &  \ddots & 1\\
0 & 0 & 0 & \cdots & \lambda
\end{pmatrix}$$

with a constant value $\lambda$ along the diagonal and 1's on the superdiagonal.  Some texts \PMlinkescapetext{place} the 1's on the subdiagonal instead.




\begin{thms} Let $V$ be a finite-dimensional vector space over a field $F$ and $t:V \to V$ be a linear transformation.  Then, if the characteristic polynomial factors completely over $F$, there will exist a basis of $V$ with respect to which the matrix of $t$ is of the form

$$\begin{pmatrix}
J_{1} & 0 & \cdots& 0\\
0 & J_{2} & \cdots & 0\\
 & & \cdots & \\
0 & 0 & \cdots & J_{k}
\end{pmatrix}$$

where each $J_{i}$ is a Jordan block in which $\lambda = \lambda_{i}$.
\end{thms}
The matrix in Theorem 1 is called a \emph{Jordan canonical form} for the transformation \emph{t}.


%%%%%
%%%%%
\end{document}
