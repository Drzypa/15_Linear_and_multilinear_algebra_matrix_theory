\documentclass[12pt]{article}
\usepackage{pmmeta}
\pmcanonicalname{ProofOfWielandtHoffmanTheorem}
\pmcreated{2013-03-22 14:58:51}
\pmmodified{2013-03-22 14:58:51}
\pmowner{Andrea Ambrosio}{7332}
\pmmodifier{Andrea Ambrosio}{7332}
\pmtitle{proof of Wielandt-Hoffman theorem}
\pmrecord{6}{36682}
\pmprivacy{1}
\pmauthor{Andrea Ambrosio}{7332}
\pmtype{Proof}
\pmcomment{trigger rebuild}
\pmclassification{msc}{15A18}
\pmclassification{msc}{15A42}

% this is the default PlanetMath preamble.  as your knowledge
% of TeX increases, you will probably want to edit this, but
% it should be fine as is for beginners.

% almost certainly you want these
\usepackage{amssymb}
\usepackage{amsmath}
\usepackage{amsfonts}

% used for TeXing text within eps files
%\usepackage{psfrag}
% need this for including graphics (\includegraphics)
%\usepackage{graphicx}
% for neatly defining theorems and propositions
%\usepackage{amsthm}
% making logically defined graphics
%%%\usepackage{xypic}

% there are many more packages, add them here as you need them

% define commands here
\def\sse{\subseteq}
\def\bigtimes{\mathop{\mbox{\Huge $\times$}}}
\def\impl{\Rightarrow}
\def\tr{\operatorname{tr}}
\def\Re{\operatorname{Re}}
\begin{document}
Since both $A$ and $B$ are normal, they can be diagonalized by unitary transformations:
\[
  A = V^\dagger C V \quad \text{and} \quad B = W^\dagger D W,
\]
where $C$ and $D$ are diagonal, $V$ and $W$ are unitary, and $(~)^\dagger$
denotes the conjugate transpose. The Frobenius matrix norm is defined by the quadratic form $\|A\|_F^2 = \tr[A^\dagger A]$ and is invariant
under unitary transformations, hence
\[
  \|A-B\|_F^2 = \|V^\dagger C V - W^\dagger D W\|_F^2
    = \|C\|_F^2 + \|D\|_F^2
      - 2\Re\tr[C^\dagger U^\dagger D U],
\]
where $U=W V^\dagger$. The matrix $U$ is also unitary, let its matrix elements
be given by $(U)_{ij} = u_{ij}$. Unitarity implies that the matrix with
elements $|u_{ij}|^2$ has its row and column sums equal to $1$, in other words,
it is doubly stochastic.

The diagonal elements $C_{ii} = a_i$ are eigenvalues of $A$ and
$D_{ii} = b_i$ are those of $B$. Writing out the Frobenius norm explicitly,
we get
\[
  \|A-B\|_F^2 = \sum_{i} (|a_i|^2+|b_i|^2)
    - 2\Re\sum_{ij} \overline{a}_i |u_{ij}|^2 b_j
    \ge \sum_{i} (|a_i|^2+|b_i|^2)
      - 2\min_{S} \Re \sum_{ij} \overline{a}_i s_{ij} b_j,
\]
where the minimum is taken over all doubly stochastic matrices $S$, whose
elements are $(S)_{ij} = s_{ij}$. By the Birkoff-von Neumann theorem, doubly
stochastic matrices form a closed \PMlinkname{convex}{ConvexSet} polyhedron
with permutation matrices at the vertices. The expression
$\sum_{ij} \overline{a}_i s_{ij} b_j$ is a linear functional on this polyhedron,
hence its minimum is achieved at one of the vertices, that is when $S$
is a permutation matrix.

If $S$ represents the permutation $\sigma$, its action can be written as
$\sum_{j} s_{ij} b_j = b_{\sigma(i)}$. Finally, we can write the last
inequality as
\[
  \|A-B\|_F^2 \ge \sum_{i} (|a_i|^2+|b_{\sigma(i)}|^2)
    - 2\min_{\sigma} \Re\sum_{ij} \overline{a}_i b_{\sigma(i)}
    = \min_{\sigma} |a_i-b_{\sigma(i)}|^2,
\]
which is exactly the statement of the Wielandt-Hoffman theorem.
%%%%%
%%%%%
\end{document}
