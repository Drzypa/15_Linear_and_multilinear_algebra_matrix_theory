\documentclass[12pt]{article}
\usepackage{pmmeta}
\pmcanonicalname{DirectSumOfMatrices}
\pmcreated{2013-03-22 17:36:48}
\pmmodified{2013-03-22 17:36:48}
\pmowner{CWoo}{3771}
\pmmodifier{CWoo}{3771}
\pmtitle{direct sum of matrices}
\pmrecord{8}{40030}
\pmprivacy{1}
\pmauthor{CWoo}{3771}
\pmtype{Definition}
\pmcomment{trigger rebuild}
\pmclassification{msc}{15-01}
\pmrelated{DirectSum}

\endmetadata

\usepackage{amssymb,amscd}
\usepackage{amsmath}
\usepackage{amsfonts}
\usepackage{mathrsfs}

% used for TeXing text within eps files
%\usepackage{psfrag}
% need this for including graphics (\includegraphics)
%\usepackage{graphicx}
% for neatly defining theorems and propositions
\usepackage{amsthm}
% making logically defined graphics
%%\usepackage{xypic}
\usepackage{pst-plot}
\usepackage{psfrag}

% define commands here
\newtheorem{prop}{Proposition}
\newtheorem{thm}{Theorem}
\newtheorem{ex}{Example}
\newcommand{\real}{\mathbb{R}}
\newcommand{\pdiff}[2]{\frac{\partial #1}{\partial #2}}
\newcommand{\mpdiff}[3]{\frac{\partial^#1 #2}{\partial #3^#1}}
\begin{document}
\subsubsection*{Direct sum of matrices}

Let $A$ be an $m\times n$ matrix and $B$ be a $p\times q$ matrix.  By the direct sum of $A$ and $B$, written $A\oplus B$, we mean the $(m+p)\times (n+q)$ matrix of the form 
$$
\begin{pmatrix}
A & O \\
O & B 
\end{pmatrix}
$$
where the $O$'s represent zero matrices.  The $O$ on the top right is an $m\times q$ matrix, while the $O$ on the bottom left is $n\times p$.

For example, if $A=\begin{pmatrix} 3 &-1\\ 2&5\end{pmatrix}$ and $B=\begin{pmatrix} 1&2\\ 4&0\\ -7&8 \end{pmatrix}$, then 
$$
\begin{pmatrix}
A & O \\
O & B 
\end{pmatrix}=
\begin{pmatrix}
3&-1 & 0&0 \\
2&5 & 0&0 \\
0&0 & 1&2 \\
0&0 & 4&2 \\
0&0 & -7&8 \\
\end{pmatrix}
$$
\textbf{Remark}.  
It is not hard to see that the $\oplus$ operation on matrices is associative: $$(A\oplus B)\oplus C = A \oplus (B\oplus C),$$ because both sides lead to  
$$
\begin{pmatrix}
A & O & O \\
O & B & O \\
O & O & C 
\end{pmatrix}
$$
In fact, we can inductively define the direct sum of $n$ matrices unambiguously.

\subsubsection*{Direct sums of linear transformations}
The direct sum of matrices is closely related to the direct sum of vector spaces and linear transformations.  Let $A$ and $B$ be as above, over some field $k$.  We may view $A$ and $B$ as linear transformations $T_A:k^n\to k^m$ and $T_B: k^q\to k^p$ using the standard ordered bases.  Then $A\oplus B$ may be viewed as the linear transformation $$T_{A\oplus B}: k^{n+q}\to k^{m+p}$$ using the standard ordered basis, such that 
\begin{itemize}
\item the restriction of $T_{A\oplus B}$ to the subspace $k^n$ (embedded in $k^{n+q}$) is $T_A$, and 
\item the restriction of $T_{A\oplus B}$ to $k^q$ is $T_B$.
\end{itemize}

The above suggests that we can define direct sums on linear transformations.  Let $T_1:V_1\to W_1$ and $T_2:V_2\to W_2$ be linear transformations, where $V_i$ and $W_j$ are finite dimensional vector spaces over some field $k$ such that $V_1\cap V_2=0$.  Then define $T_1\oplus T_2: V_1\oplus V_2 \to W_1\oplus W_2$ such that for any $v\in V_1\oplus V_2$, $$(T_1\oplus T_2)(v_1,v_2):=(T_1(v_1),T_2(v_2))$$ where $v_i\in V_i$.
Based on this definition, it is not hard to see that $$T_{A\oplus B}=T_A \oplus T_B$$ for any matrices $A$ and $B$.

More generally, if $\beta_i$ is an ordered basis for $V_i$, then $\beta:=\beta_1\cup \beta_2$ extending the linear orders on $\beta_i$, such that if $v_i\in \beta_1$ and $v_j\in \beta_2$, then $v_i<v_j$ is an ordered basis for $V_1\oplus V_2$, and $$[T_1\oplus T_2]_{\beta}=[T_1]_{\beta_1}\oplus [T_2]_{\beta_2}.$$
%%%%%
%%%%%
\end{document}
