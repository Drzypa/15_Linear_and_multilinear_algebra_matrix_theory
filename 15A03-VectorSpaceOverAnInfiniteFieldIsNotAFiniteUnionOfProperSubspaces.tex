\documentclass[12pt]{article}
\usepackage{pmmeta}
\pmcanonicalname{VectorSpaceOverAnInfiniteFieldIsNotAFiniteUnionOfProperSubspaces}
\pmcreated{2013-03-22 17:29:43}
\pmmodified{2013-03-22 17:29:43}
\pmowner{loner}{106}
\pmmodifier{loner}{106}
\pmtitle{vector space over an infinite field is not a finite union of proper subspaces}
\pmrecord{9}{39883}
\pmprivacy{1}
\pmauthor{loner}{106}
\pmtype{Theorem}
\pmcomment{trigger rebuild}
\pmclassification{msc}{15A03}

% this is the default PlanetMath preamble.  as your knowledge
% of TeX increases, you will probably want to edit this, but
% it should be fine as is for beginners.
%\documentclass{amsart}
% almost certainly you want these
\usepackage{amssymb}
\usepackage{amsmath}
\usepackage{amsfonts}
\usepackage{amsthm}

\usepackage{mathrsfs}
\usepackage{bbm}
% used for TeXing text within eps files
%\usepackage{psfrag}
% need this for including graphics (\includegraphics)
%\usepackage{graphicx}
% for neatly defining theorems and propositions
%
% making logically defined graphics
%%%\usepackage{xypic}

% there are many more packages, add them here as you need them

% define commands here
\newtheorem{thm}{Theorem}
\begin{document}
\begin{thm}
A vector space $V$ over an infinite field $\mathbbmss{F}$ cannot be
a finite union of proper subspaces of itself.
\end{thm}

\begin{proof}
Let $V = V_{1} \cup V_{2} \cup \ldots \cup V_{n}$ where
each $V_{i}$ is a proper subspace of $V$ and $n > 1$ is minimal.
Because $n$ is minimal, $V_{n} \not\subset V_{1} \cup V_{2}
\cup \ldots \cup V_{n-1}$.

Let $u \not\in V_{n}$ and let $v \in V_{n} \setminus \left(V_{1}
\cup V_{2} \cup \ldots \cup V_{n-1}\right)$.

Define $S = \left\{v + tu : t \in \mathbbmss{F}\right\}$. Since 
$u\not\in V_{n}$ is not the zero vector and the field 
$\mathbbmss{F}$ is infinite, $S$ must be infinite.

Since $S \subset V = V_{1} \cup V_{2} \cup \ldots \cup
V_{n}$ one of the $V_{i}$ must contain infinitely many vectors in
$S$.

However, if $V_{n}$ were to contain a vector, other than $v$, from $S$ there would
exist non-zero $t \in \mathbbmss{F}$ such that $v + tu \in V_{n}$.
But then $tu = v + tu - v \in V_{n}$ and we would have $u \in V_{n}$
contrary to the choice of $u$. Thus $V_{n}$ cannot contain
infinitely many elements in $S$.

If some $V_{i}, 1 \leq i < n$ contained two distinct vectors in $S$,
then there would exist distinct $t_{1}, t_{2} \in \mathbbmss{F}$
such that $v + t_{1}u, v + t_{2}u \in V_{i}$. But then $\left(t_{2}
- t_{1}\right)v = t_{2}\left(v + t_{1}u\right) - t_{1}\left(v +
t_{2}u\right) \in V_{i}$ and we would have $v \in V_{i}$ contrary to
the choice of $v$. Thus for $1 \leq i < n, V_{i}$ cannot contain
infinitely many elements in $S$ either.
\end{proof}
%%%%%
%%%%%
\end{document}
