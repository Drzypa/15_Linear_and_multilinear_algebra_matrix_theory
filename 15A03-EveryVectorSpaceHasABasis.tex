\documentclass[12pt]{article}
\usepackage{pmmeta}
\pmcanonicalname{EveryVectorSpaceHasABasis}
\pmcreated{2013-03-22 13:04:48}
\pmmodified{2013-03-22 13:04:48}
\pmowner{GrafZahl}{9234}
\pmmodifier{GrafZahl}{9234}
\pmtitle{every vector space has a basis}
\pmrecord{14}{33494}
\pmprivacy{1}
\pmauthor{GrafZahl}{9234}
\pmtype{Theorem}
\pmcomment{trigger rebuild}
\pmclassification{msc}{15A03}
\pmsynonym{every vector space has a Hamel basis}{EveryVectorSpaceHasABasis}
\pmrelated{ZornsLemma}
\pmrelated{AxiomOfChoice}
\pmrelated{ZermelosWellOrderingTheorem}
\pmrelated{HaudorffsMaximumPrinciple}
\pmrelated{KuratowskisLemma}

% this is the default PlanetMath preamble.  as your knowledge
% of TeX increases, you will probably want to edit this, but
% it should be fine as is for beginners.

% almost certainly you want these
\usepackage{amssymb}
\usepackage{amsmath}
\usepackage{amsfonts}
\usepackage[latin1]{inputenc}

% used for TeXing text within eps files
%\usepackage{psfrag}
% need this for including graphics (\includegraphics)
%\usepackage{graphicx}
% for neatly defining theorems and propositions
\usepackage{amsthm}
% making logically defined graphics
%%%\usepackage{xypic}

% there are many more packages, add them here as you need them

% define commands here
\newcommand{\<}{\langle}
\renewcommand{\>}{\rangle}
\newcommand{\Bigcup}{\bigcup\limits}
\newcommand{\DirectSum}{\bigoplus\limits}
\newcommand{\Prod}{\prod\limits}
\newcommand{\Sum}{\sum\limits}
\newcommand{\h}{\widehat}
\newcommand{\mbb}{\mathbb}
\newcommand{\mbf}{\mathbf}
\newcommand{\mc}{\mathcal}
\newcommand{\mmm}[9]{\left(\begin{array}{rrr}#1&#2&#3\\#4&#5&#6\\#7&#8&#9\end{array}\right)}
\newcommand{\mf}{\mathfrak}
\newcommand{\ol}{\overline}

% Math Operators/functions
\DeclareMathOperator{\Aut}{Aut}
\DeclareMathOperator{\End}{End}
\DeclareMathOperator{\Frob}{Frob}
\DeclareMathOperator{\cwe}{cwe}
\DeclareMathOperator{\id}{id}
\DeclareMathOperator{\mult}{mult}
\DeclareMathOperator{\we}{we}
\DeclareMathOperator{\wt}{wt}

\begin{document}
This result, trivial in the finite case, is in fact rather surprising
when one thinks of infinite dimensionial vector spaces, and the
definition of a basis: just try to imagine a basis of the vector space
of all continuous mappings $f\colon\mbb{R}\to\mbb{R}$. The theorem is
equivalent to the axiom of choice family of axioms and theorems. Here
we will only prove that Zorn's lemma implies that every vector space
has a basis.

\newtheorem*{thm}{Theorem}
\begin{thm}
Let $X$ be any vector space over any field $F$ and assume Zorn's
lemma. Then if $L$ is a linearly independent subset of $X$, there
exists a basis of $X$ containing $L$. In particular, $X$ does have a
basis at all.
\end{thm}
\begin{proof}
Let $\mc{A}$ be the set of linearly independent subsets of $X$
containing $L$ (in particular, $\mc{A}$ is not empty), then $\mc{A}$
is partially ordered by inclusion. For each chain $C\subseteq\mc{A}$,
define $\h{C}=\bigcup C$. Clearly, $\h{C}$ is an upper bound of $C$. Next we
show that $\h{C}\in\mc{A}$. Let $V:=\{v_1,\ldots,v_n\}\subseteq\h{C}$
be a finite collection of vectors. Then there exist sets $C_1,\ldots,
C_n\in C$ such that $v_i\in C_i$ for all $1\leq i\leq n$. Since $C$ is
a chain, there is a number $k$ with $1\leq k\leq n$ such that
$C_k=\Bigcup_{i=1}^nC_i$ and thus $V\subseteq C_k$, that is $V$ is
linearly independent. Therefore, $\h{C}$ is an element of $\mc{A}$.

According to Zorn's lemma $\mc{A}$ has a maximal element, $M$, which
is linearly independent. We show now that $M$ is a basis. Let $\langle M\rangle$
be the span of $M$. Assume there exists an $x\in X\setminus\langle M\rangle$. Let
$\{x_1,\ldots,x_n\}\subseteq M$ be a finite collection of vectors and
$a_1,\ldots,a_{n+1}\in F$ elements such that
\begin{equation*}
a_1x_1+\cdots+a_nx_n-a_{n+1}x=0.
\end{equation*}
If $a_{n+1}$ was necessarily zero, so would be the other $a_i$, $1\leq i\leq n$,
making $\{x\}\cup M$ linearly independent in contradiction to the
maximality of $M$. If $a_{n+1}\neq 0$, we would have
\begin{equation*}
x=\frac{a_1}{a_{n+1}}x_1+\cdots+\frac{a_n}{a_{n+1}}x_n,
\end{equation*}
contradicting $x\notin\langle M\rangle$. Thus such an $x$ does not exist and
$X=\langle M\rangle$, so $M$ is a generating set and hence a basis.

Taking $L=\emptyset$, we see that $X$ does have a basis at all.
\end{proof}

%%%%%
%%%%%
\end{document}
