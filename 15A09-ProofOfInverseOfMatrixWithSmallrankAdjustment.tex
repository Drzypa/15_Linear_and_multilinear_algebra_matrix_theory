\documentclass[12pt]{article}
\usepackage{pmmeta}
\pmcanonicalname{ProofOfInverseOfMatrixWithSmallrankAdjustment}
\pmcreated{2013-03-22 15:46:08}
\pmmodified{2013-03-22 15:46:08}
\pmowner{kshum}{5987}
\pmmodifier{kshum}{5987}
\pmtitle{proof of inverse of matrix with small-rank adjustment}
\pmrecord{4}{37726}
\pmprivacy{1}
\pmauthor{kshum}{5987}
\pmtype{Proof}
\pmcomment{trigger rebuild}
\pmclassification{msc}{15A09}

\endmetadata

% this is the default PlanetMath preamble.  as your knowledge
% of TeX increases, you will probably want to edit this, but
% it should be fine as is for beginners.

% almost certainly you want these
\usepackage{amssymb}
\usepackage{amsmath}
\usepackage{amsfonts}

% used for TeXing text within eps files
%\usepackage{psfrag}
% need this for including graphics (\includegraphics)
%\usepackage{graphicx}
% for neatly defining theorems and propositions
%\usepackage{amsthm}
% making logically defined graphics
%%%\usepackage{xypic}

% there are many more packages, add them here as you need them

% define commands here
\begin{document}
We will first prove the formula when $A=I$.

Suppose that $R^{-1}+Y^TX$ is invertible. Thus
\[
 (R^{-1}+Y^TX)(R^{-1}+Y^TX)^{-1} = I.
\]
and
\[ R^{-1}(R^{-1}+Y^TX)^{-1}+Y^TX(R^{-1}+Y^TX)^{-1} = I.
\]

Multiply by $XR$ from the left, and multiply by $Y^T$ from the
right, we get
\[ X(R^{-1}+Y^TX)^{-1}Y^T + XR Y^TX(R^{-1}+Y^TX)^{-1}Y^T = XRY^T.
\]

The right hand side is equal to $B-I$, while the left hand side can
be factorized as
\[
 (I+XRY^T)X(R^{-1}+Y^TX)^{-1}Y^T.
\]

So,
\[
 B\cdot (X(R^{-1}+Y^TX)^{-1}Y^T) = B-I.
\]
After rearranging, we obtain
\[
I =  B(I-X(R^{-1}+Y^TX)^{-1}Y^T).
\]

Therefore \begin{equation} (I+XRY^T)^{-1}  =
I-X(R^{-1}+Y^TX)^{-1}Y^T \tag{*}
\end{equation}

For the general case $B=A+XRY^T$, consider
\[
 BA^{-1} = I+XRY^TA^{-1}.
\]
We can apply (*) with $Y^T$ replaced by $Y^TA^{-1}$.
%%%%%
%%%%%
\end{document}
