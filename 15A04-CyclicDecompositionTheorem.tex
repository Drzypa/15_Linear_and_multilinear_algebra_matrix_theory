\documentclass[12pt]{article}
\usepackage{pmmeta}
\pmcanonicalname{CyclicDecompositionTheorem}
\pmcreated{2013-03-22 14:05:10}
\pmmodified{2013-03-22 14:05:10}
\pmowner{CWoo}{3771}
\pmmodifier{CWoo}{3771}
\pmtitle{cyclic decomposition theorem}
\pmrecord{16}{35449}
\pmprivacy{1}
\pmauthor{CWoo}{3771}
\pmtype{Theorem}
\pmcomment{trigger rebuild}
\pmclassification{msc}{15A04}
\pmsynonym{T-admissible}{CyclicDecompositionTheorem}
\pmsynonym{$T$-admissible}{CyclicDecompositionTheorem}
\pmrelated{CyclicSubspace}
\pmdefines{admissible subspace}

% this is the default PlanetMath preamble.  as your knowledge
% of TeX increases, you will probably want to edit this, but
% it should be fine as is for beginners.

% almost certainly you want these
\usepackage{amssymb}
\usepackage{amsmath}
\usepackage{amsfonts}

% used for TeXing text within eps files
%\usepackage{psfrag}
% need this for including graphics (\includegraphics)
%\usepackage{graphicx}
% for neatly defining theorems and propositions
%\usepackage{amsthm}
% making logically defined graphics
%%%\usepackage{xypic}

% there are many more packages, add them here as you need them

% define commands here
\begin{document}
Let $k$ be a field, $V$ a finite dimensional vector space over $k$ and $T$ a linear operator over $V$.  Call a subspace $W\subseteq V$ \emph{$T$-admissible} if $W$ is $T$-invariant and for any polynomial $f(X)\in k[X]$ with $f(T)(v)\in W$ for $v\in V$, there is a $w\in W$ such that $f(T)(v)=f(T)(w)$.  

Let $W_0$ be a proper $T$-admissible subspace of $V$. There are non zero vectors $x_1,...,x_r$ in $V$ with respective annihilator polynomials $p_1,...,p_r$ such that
\begin{enumerate}
\item $V=W_0\oplus Z(x_1,T)\oplus \cdots \oplus Z(x_r,T)$ (See the cyclic subspace definition)
\item $p_k$ divides $p_{k-1}$ for every $k=2,...,r$
\end{enumerate}
Moreover, the integer $r$ and the \PMlinkname{minimal polynomials}{MinimalPolynomialEndomorphism} $p_1,...,p_r$ are uniquely determined by (1),(2) and the fact that none of $x_k$ is zero.

This is ``one of the deepest results in linear algebra'' (Hoffman \& Kunze)
%%%%%
%%%%%
\end{document}
