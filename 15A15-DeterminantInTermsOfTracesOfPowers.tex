\documentclass[12pt]{article}
\usepackage{pmmeta}
\pmcanonicalname{DeterminantInTermsOfTracesOfPowers}
\pmcreated{2013-03-22 15:57:08}
\pmmodified{2013-03-22 15:57:08}
\pmowner{Mathprof}{13753}
\pmmodifier{Mathprof}{13753}
\pmtitle{determinant in terms of traces of powers}
\pmrecord{11}{37964}
\pmprivacy{1}
\pmauthor{Mathprof}{13753}
\pmtype{Theorem}
\pmcomment{trigger rebuild}
\pmclassification{msc}{15A15}

\endmetadata

% this is the default PlanetMath preamble.  as your knowledge
% of TeX increases, you will probably want to edit this, but
% it should be fine as is for beginners.

% almost certainly you want these
\usepackage{amssymb}
\usepackage{amsmath}
\usepackage{amsfonts}

% used for TeXing text within eps files
%\usepackage{psfrag}
% need this for including graphics (\includegraphics)
%\usepackage{graphicx}
% for neatly defining theorems and propositions
%\usepackage{amsthm}
% making logically defined graphics
%%%\usepackage{xypic}

% there are many more packages, add them here as you need them

% define commands here

\begin{document}
It is possible to express the determinant of a matrix in \PMlinkescapetext{terms} of traces of 
powers of a matrix.

The easiest way to derive these expressions is to specialize to the case of
diagonal matrices.  For instance, suppose we have a $2 \times 2$ matrix $M = \operatorname{diag} (u,v)$.  Then 
\begin{eqnarray*}
\operatorname{det} M &=& uv \\
\operatorname{tr} M &=& u + v \\
\operatorname{tr} M^2 &=& u^2 + v^2 \\
\end{eqnarray*} 
From the algebraic identity $(u+v)^2 = u^2 + v^2 + 2uv$, it can be concluded that $\operatorname{det} M = \frac{1}{2} (\operatorname{tr} M)^2 - \frac{1}{2} \operatorname{tr} (M^2)$.

Likewise, one can derive expressions for the determinants of larger matrices from the identities for elementary symmetric polynomials in \PMlinkescapetext{terms} of power sums.  For instance, from the identity 
\[xyz = \frac{1}{6} (x+y+z)^3 - \frac{1}{2} (x^2 + y^2 + z^2) (x + y + z) + \frac{1}{3} (x^3+ y^3 + z^3),\] 
it can be concluded that 
\[\operatorname{det} M = \frac{1}{6} (\operatorname{tr} M)^3 - \frac{1}{2} (\operatorname{tr} M^2)(\operatorname{tr} M) + \frac{1}{3} \operatorname{tr} M^3\]
for a $3\times 3$ matrix $M$.
%%%%%
%%%%%
\end{document}
