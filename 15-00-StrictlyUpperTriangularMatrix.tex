\documentclass[12pt]{article}
\usepackage{pmmeta}
\pmcanonicalname{StrictlyUpperTriangularMatrix}
\pmcreated{2013-03-22 13:42:15}
\pmmodified{2013-03-22 13:42:15}
\pmowner{Daume}{40}
\pmmodifier{Daume}{40}
\pmtitle{strictly upper triangular matrix}
\pmrecord{8}{34381}
\pmprivacy{1}
\pmauthor{Daume}{40}
\pmtype{Definition}
\pmcomment{trigger rebuild}
\pmclassification{msc}{15-00}
\pmsynonym{supertriangular}{StrictlyUpperTriangularMatrix}
\pmdefines{strictly lower triangular matrix}

\endmetadata

% this is the default PlanetMath preamble.  as your knowledge
% of TeX increases, you will probably want to edit this, but
% it should be fine as is for beginners.

% almost certainly you want these
\usepackage{amssymb}
\usepackage{amsmath}
\usepackage{amsfonts}

% used for TeXing text within eps files
%\usepackage{psfrag}
% need this for including graphics (\includegraphics)
%\usepackage{graphicx}
% for neatly defining theorems and propositions
%\usepackage{amsthm}
% making logically defined graphics
%%%\usepackage{xypic} 

% there are many more packages, add them here as you need them

% define commands here
\begin{document}
A \emph{strictly \PMlinkescapetext{upper triangular matrix}} is an upper triangular matrix which has $0$ on the main diagonal.  Similarly a \emph{strictly lower triangular matrix} is a lower triangular matrix which has $0$ on the main diagonal. i.e.

A strictly upper triangular matrix is of the form

$$ \begin{bmatrix}
0 & a_{12} & a_{13} & \cdots & a_{1n} \\
0 & 0 & a_{23} & \cdots & a_{2n} \\
0 & 0 & 0 & \cdots & a_{3n} \\
\vdots & \vdots & \vdots & \ddots & \vdots \\
0 & 0 & 0 & \cdots & 0
\end{bmatrix} $$

A strictly lower triangular matrix is of the form

$$ \begin{bmatrix}
0 & 0 & 0 & \cdots & 0 \\
a_{21} & 0 & 0 & \cdots & 0 \\
a_{31} & a_{32} & 0 & \cdots & 0 \\
\vdots & \vdots & \vdots & \ddots & \vdots \\
a_{n1} & a_{n2} & a_{n3} & \cdots & 0
\end{bmatrix} $$
%%%%%
%%%%%
\end{document}
