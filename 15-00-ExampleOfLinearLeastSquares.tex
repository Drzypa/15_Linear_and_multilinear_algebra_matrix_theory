\documentclass[12pt]{article}
\usepackage{pmmeta}
\pmcanonicalname{ExampleOfLinearLeastSquares}
\pmcreated{2013-03-22 16:51:19}
\pmmodified{2013-03-22 16:51:19}
\pmowner{bloftin}{6104}
\pmmodifier{bloftin}{6104}
\pmtitle{example of linear least squares}
\pmrecord{6}{39102}
\pmprivacy{1}
\pmauthor{bloftin}{6104}
\pmtype{Example}
\pmcomment{trigger rebuild}
\pmclassification{msc}{15-00}

\endmetadata

% this is the default PlanetMath preamble.  as your knowledge
% of TeX increases, you will probably want to edit this, but
% it should be fine as is for beginners.

% almost certainly you want these
\usepackage{amssymb}
\usepackage{amsmath}
\usepackage{amsfonts}

% used for TeXing text within eps files
%\usepackage{psfrag}
% need this for including graphics (\includegraphics)
\usepackage{graphicx}
% for neatly defining theorems and propositions
%\usepackage{amsthm}
% making logically defined graphics
%%%\usepackage{xypic}

% there are many more packages, add them here as you need them

% define commands here

\begin{document}
The assumption of linear least squares is that there is a linear relationship between our measurements $z$ and the variables to be estimated $x$

\begin{equation}
z = Mx + b
\end{equation}

For this example let us assume that our measurements are given in Table 1 and you can see them plotted in Figure 1.

\begin{center}
\begin{tabular}{ccccccccccc}
\hline \\ 
x & -3.0 & -2.5 & -2.0 & -1.5 & -1.0 & -0.5 & 0.0 & 0.5 & 1.0 & 1.5 \\
\PMlinkescapetext{z} & -1.0 & -0.25 & 0.0 & 0.25 & 0.4 & 0.7 & 1.0 & 1.1 & 1.4 & 1.8 \\ [2ex]
\hline 
\end{tabular}
\end{center}
\begin{center}
{\bf Table 1:} Example Data
\end{center}

The linear least squares solution to fit the given data is given by the equation

\begin{equation}
x_{fit} = (A^TA)^{-1}A^Tz
\end{equation}

The only not so obvious step before using a tool like Matlab, is to form the $A$ matrix, which is a \PMlinkescapetext{combination} of an identity vector and $x$ as column vectors, such that

$$ A = [ \,\, 1 \,\, | \,\,x \,\,] $$

This is clarified by looking at the example \PMlinkescapetext{code} in Matlab, \PMlinktofile{LinearLeastSquares.m}{LinearLeastSquares.m}.  A plot of fitting the measurement data with a line such that it minimizes the the mean square of the error is shown in Figure 1.  The equation of the line to fit this data is then

$$z = 0.543 x + 0.947$$

\begin{center}
\includegraphics[scale=.7]{LinearLeastSquares.eps}
\end{center}
\begin{center}
{\bf Figure 1:} Linear Fit of Example Data
\end{center}


%%%%%
%%%%%
\end{document}
