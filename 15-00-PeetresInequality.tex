\documentclass[12pt]{article}
\usepackage{pmmeta}
\pmcanonicalname{PeetresInequality}
\pmcreated{2013-03-22 13:55:26}
\pmmodified{2013-03-22 13:55:26}
\pmowner{Koro}{127}
\pmmodifier{Koro}{127}
\pmtitle{Peetre's inequality}
\pmrecord{10}{34681}
\pmprivacy{1}
\pmauthor{Koro}{127}
\pmtype{Theorem}
\pmcomment{trigger rebuild}
\pmclassification{msc}{15-00}
\pmclassification{msc}{15A39}

\endmetadata

% this is the default PlanetMath preamble.  as your knowledge
% of TeX increases, you will probably want to edit this, but
% it should be fine as is for beginners.

% almost certainly you want these
\usepackage{amssymb}
\usepackage{amsmath}
\usepackage{amsfonts}

% used for TeXing text within eps files
%\usepackage{psfrag}
% need this for including graphics (\includegraphics)
%\usepackage{graphicx}
% for neatly defining theorems and propositions
%\usepackage{amsthm}
% making logically defined graphics
%%%\usepackage{xypic}

% there are many more packages, add them here as you need them

% define commands here

\newcommand{\sR}[0]{\mathbb{R}}
\newcommand{\sC}[0]{\mathbb{C}}
\newcommand{\sN}[0]{\mathbb{N}}
\newcommand{\sZ}[0]{\mathbb{Z}}
\begin{document}
{\bf Theorem\, [Peetre's inequality]}  \cite{barros, trevesI}
If $t$ is a real number and $x,y$ are vectors in
$\sR^n$, then
$$ \Big( \frac{1+|x|^2}{1+|y|^2} \Big)^t \le 2^{|t|} (1+|x-y|^2)^{|t|}.$$

{\bf Proof.} (Following \cite{barros}.)
Suppose $b$ and $c$ are vectors in $\sR^n$. Then, from
$(|b|-|c|)^2\ge 0$, we obtain
$$ 2|b| \cdot |c| \le |b|^2 + |c|^2.$$
Using this inequality and the Cauchy-Schwarz inequality, we obtain
\begin{eqnarray*}
1+ |b-c|^2 &=& 1+ |b|^2 - 2 b\cdot c + |c|^2 \\
&\le & 1+ |b|^2 + 2 |b| |c| + |c|^2 \\
&\le & 1+ 2|b|^2 + 2|c|^2 \\
&\le & 2\big( 1+|b|^2+ |c|^2+|b|^2 |c|^2\big)\\
&= & 2( 1+|b|^2)(1+ |c|^2)
\end{eqnarray*}
Let us define $a=b-c$. Then for any vectors $a$ and $b$, we have
\begin{eqnarray}
\label{AA}
    \frac{1+|a|^2}{1+|b|^2}   \le 2 (1+|a-b|^2).
\end{eqnarray}
Let us now return to the given inequality.
If $t=0$, the claim is trivially true for all $x,y$ in $\sR^n$.
If $t>0$, then raising both sides in inequality \ref{AA} to
the power of $t$, using $t=|t|$, and setting $a=x$, $b=y$ yields the result.
On the other hand, if $t<0$, then raising both sides in inequality
\ref{AA} to the power to $-t$, using $-t=|t|$, and setting
$a=y$, $b=x$ yields the result.
$\Box$


\begin{thebibliography}{9}
 \bibitem{barros} J. Barros-Neta, \emph{An introduction to the theory of distributions},
 Marcel Dekker, Inc., 1973.
 \bibitem{trevesI} F. Treves,
 \emph{Introduction To Pseudodifferential and Fourier Integral Operators},
Vol. I, Plenum Press, 1980.
 \end{thebibliography}
%%%%%
%%%%%
\end{document}
