\documentclass[12pt]{article}
\usepackage{pmmeta}
\pmcanonicalname{Eigenspace}
\pmcreated{2013-03-22 17:23:07}
\pmmodified{2013-03-22 17:23:07}
\pmowner{CWoo}{3771}
\pmmodifier{CWoo}{3771}
\pmtitle{eigenspace}
\pmrecord{9}{39752}
\pmprivacy{1}
\pmauthor{CWoo}{3771}
\pmtype{Definition}
\pmcomment{trigger rebuild}
\pmclassification{msc}{15A18}

\usepackage{amssymb,amscd}
\usepackage{amsmath}
\usepackage{amsfonts}
\usepackage{mathrsfs}

% used for TeXing text within eps files
%\usepackage{psfrag}
% need this for including graphics (\includegraphics)
%\usepackage{graphicx}
% for neatly defining theorems and propositions
\usepackage{amsthm}
% making logically defined graphics
%%\usepackage{xypic}
\usepackage{pst-plot}
\usepackage{psfrag}

% define commands here
\newtheorem{prop}{Proposition}
\newtheorem{thm}{Theorem}
\newtheorem{ex}{Example}
\newcommand{\real}{\mathbb{R}}
\newcommand{\pdiff}[2]{\frac{\partial #1}{\partial #2}}
\newcommand{\mpdiff}[3]{\frac{\partial^#1 #2}{\partial #3^#1}}
\begin{document}
Let $V$ be a vector space over a field $k$.  Fix a linear transformation $T$ on $V$.  Suppose $\lambda$ is an eigenvalue of $T$.  The set $\lbrace v\in V\mid Tv=\lambda v\rbrace$ is called the \emph{eigenspace} (of $T$) corresponding to $\lambda$.  Let us write this set $W_{\lambda}$.

Below are some basic properties of eigenspaces.
\begin{enumerate}
\item $W_{\lambda}$ can be viewed as the kernel of the linear transformation $T-\lambda I$.  As a result, $W_{\lambda}$ is a subspace of $V$.  
\item The dimension of $W_{\lambda}$ is called the geometric multiplicity of $\lambda$.  Let us denote this by $g_{\lambda}$.  It is easy to see that $1\le g_{\lambda}$, since the existence of an eigenvalue means the existence of a non-zero eigenvector corresponding to the eigenvalue.  
\item $W_{\lambda}$ is an invariant subspace under $T$ ($T$-invariant).  
\item $W_{\lambda_1}\cap W_{\lambda_2}=0$ iff $\lambda_1\ne \lambda_2$.
\item In fact, if $W_{\lambda}'$ is the sum of eigenspaces corresponding to eigenvalues of $T$ other than $\lambda$, then $W_{\lambda}\cap W_{\lambda}'=0$.
\end{enumerate}

From now on, we assume $V$ finite-dimensional.  

Let $S_T$ be the set of all eigenvalues of $T$ and let $W=\oplus_{\lambda \in S} W_{\lambda}$.  We have the following properties:

\begin{enumerate}
\item If $m_{\lambda}$ is the algebraic multiplicity of $\lambda$, then $g_{\lambda}\le m_{\lambda}$. 
\item Suppose the characteristic polynomial $p_T(x)$ of $T$ can be factored into linear terms, then $T$ is diagonalizable iff $m_{\lambda}=g_{\lambda}$ for every $\lambda\in S_T$.
\item In other words, if $p_T(x)$ splits over $k$, then $T$ is diagonalizable iff $V=W$.
\end{enumerate}

For example, let $T:\mathbb{R}^2\to \mathbb{R}^2$ be given by $T(x,y)=(x,x+y)$.  Using the standard basis, $T$ is represented by the matrix 
\begin{center}$M_T=
\begin{pmatrix}
1 & 1 \\
0 & 1
\end{pmatrix}.$
\end{center}
From this matrix, it is easy to see that $p_T(x)=(x-1)^2$ is the characteristic polynomial of $T$ and $1$ is the only eigenvalue of $T$ with $m_1=2$.  Also, it is not hard to see that $T(x,y)=(x,y)$ only when $y=0$.  So $W_1$ is a one-dimensional subspace of $\mathbb{R}^2$ generated by $(1,0)$.  As a result, $T$ is not diagonalizable.
%%%%%
%%%%%
\end{document}
