\documentclass[12pt]{article}
\usepackage{pmmeta}
\pmcanonicalname{ProofOfBlockDeterminants}
\pmcreated{2013-03-22 15:27:49}
\pmmodified{2013-03-22 15:27:49}
\pmowner{georgiosl}{7242}
\pmmodifier{georgiosl}{7242}
\pmtitle{proof of block determinants}
\pmrecord{5}{37315}
\pmprivacy{1}
\pmauthor{georgiosl}{7242}
\pmtype{Proof}
\pmcomment{trigger rebuild}
\pmclassification{msc}{15A15}

% this is the default PlanetMath preamble.  as your knowledge
% of TeX increases, you will probably want to edit this, but
% it should be fine as is for beginners.

% almost certainly you want these
\usepackage{amssymb}
\usepackage{amsmath}
\usepackage{amsfonts}

% used for TeXing text within eps files
%\usepackage{psfrag}
% need this for including graphics (\includegraphics)
%\usepackage{graphicx}
% for neatly defining theorems and propositions
%\usepackage{amsthm}
% making logically defined graphics
%%%\usepackage{xypic}

% there are many more packages, add them here as you need them

% define commands here
\begin{document}
If $A^{-1}$ exists, then  
$$\begin{pmatrix} A & B \\ C & D \end{pmatrix}
=\begin{pmatrix} I & O \\ CA^{-1} & I \end{pmatrix}\begin{pmatrix} A & B \\ O & D-CA^{-1}B \end{pmatrix}.$$ So  $$\det \begin{pmatrix} A & B \\ C & D \end{pmatrix}
=\det \begin{pmatrix} I & O \\ CA^{-1} & I \end{pmatrix}\det \begin{pmatrix} A & B \\ O & D-CA^{-1}B \end{pmatrix}. $$
Each of the first matrices in the decompositions are triangular.  Hence their determinants equal $1$.  This means that the determinant of the original matrix equals the determinant of either of the second matrices in the decomposition. Therefore $$\det \begin{pmatrix} A & B \\ C & D \end{pmatrix} =\det(A)\det(D-CA^{-1}B).$$
The second formula follows by using a similar trick.
%%%%%
%%%%%
\end{document}
