\documentclass[12pt]{article}
\usepackage{pmmeta}
\pmcanonicalname{Symmetrizer}
\pmcreated{2013-03-22 16:15:44}
\pmmodified{2013-03-22 16:15:44}
\pmowner{Mathprof}{13753}
\pmmodifier{Mathprof}{13753}
\pmtitle{symmetrizer}
\pmrecord{8}{38370}
\pmprivacy{1}
\pmauthor{Mathprof}{13753}
\pmtype{Definition}
\pmcomment{trigger rebuild}
\pmclassification{msc}{15A04}

% this is the default PlanetMath preamble.  as your knowledge
% of TeX increases, you will probably want to edit this, but
% it should be fine as is for beginners.

% almost certainly you want these
\usepackage{amssymb}
\usepackage{amsmath}
\usepackage{amsfonts}

% used for TeXing text within eps files
%\usepackage{psfrag}
% need this for including graphics (\includegraphics)
%\usepackage{graphicx}
% for neatly defining theorems and propositions
%\usepackage{amsthm}
% making logically defined graphics
%%%\usepackage{xypic}

% there are many more packages, add them here as you need them

% define commands here

\begin{document}
Let $V$ be a vector space over a field $F$. Let $n$ be an integer, where
$n < \mathrm{char}(F)$ if $\mathrm{char}(F) \neq 0$. Let $S_n$ be the symmetric group on
$\{1,\ldots, n\}.$
The linear operator $S: V^{\otimes n} \to V^{\otimes n}$ defined by:

$$
S = \frac{1}{n!} \sum_{\sigma \in S_n} P(\sigma)
$$
is called the \emph{symmetrizer}.
Here $P(\sigma)$ is the permutation operator.
It is clear that $P(\sigma )S = SP(\sigma )= S$ for all $\sigma \in S_n$.

\textbf{\PMlinkescapetext{Theorem}}\\
Let $S$ be the symmetrizer for $V^{\otimes n}$. Then an order-n tensor $A$ is 
\PMlinkname{symmetric}{SymmetricTensor} if and only $S(A) = A$.


\textbf{Proof}\\
If $A$ is  \PMlinkescapetext{symmetric} then 
$$
S(A) = \frac{1}{n!} \sum_{\sigma \in S_n} P(\sigma)A = \frac{1}{n!} \sum_{\sigma \in S_n} A = A.
$$
If $S(A) = A$ then 
$$
P(\sigma)A = P(\sigma)S(A) = P(\sigma)S(A) = S(A) = A
$$
for all $\sigma \in S_n$, so $A$ is  \PMlinkescapetext{symmetric}. 

The theorem says that a \PMlinkescapetext{symmetric tensor} is an eigenvector of the linear operator $S$ corresponding to the eigenvalue 1. It is easy to verify that
$S^2 = S$, so that $S$ is a projection onto $S^n(V)$.


%%%%%
%%%%%
\end{document}
