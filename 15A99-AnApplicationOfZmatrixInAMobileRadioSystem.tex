\documentclass[12pt]{article}
\usepackage{pmmeta}
\pmcanonicalname{AnApplicationOfZmatrixInAMobileRadioSystem}
\pmcreated{2013-03-22 16:14:16}
\pmmodified{2013-03-22 16:14:16}
\pmowner{kshum}{5987}
\pmmodifier{kshum}{5987}
\pmtitle{an application of Z-matrix in a mobile radio system}
\pmrecord{6}{38339}
\pmprivacy{1}
\pmauthor{kshum}{5987}
\pmtype{Application}
\pmcomment{trigger rebuild}
\pmclassification{msc}{15A99}

\endmetadata

% this is the default PlanetMath preamble.  as your knowledge
% of TeX increases, you will probably want to edit this, but
% it should be fine as is for beginners.

% almost certainly you want these
\usepackage{amssymb}
\usepackage{amsmath}
\usepackage{amsfonts}

% used for TeXing text within eps files
%\usepackage{psfrag}
% need this for including graphics (\includegraphics)
%\usepackage{graphicx}
% for neatly defining theorems and propositions
%\usepackage{amsthm}
% making logically defined graphics
%%%\usepackage{xypic}

% there are many more packages, add them here as you need them

% define commands here

\begin{document}
The following is an application of Z-matrix in wireless communication called power balancing problem.

Consider $n$ pairs of mobile users and receiving antennae. For
$i=1,\ldots, n$, mobile user $i$ transmits radio signal to antenna
$i$. Mobile user $i$ transmits at power $P_i$. The radio channel
attenuate the signal and user $i$'s signal is received at antenna
$i$ with power $G_{ii} P_i$, where $G_{ii}$ denote the channel
gain. The radio signals also interfere each other. At antenna $i$,
the interference due to user $j$ has power $G_{ij} P_j$. The
receiver noise power at antenna $i$ is denoted by $n_i$. The
signal to interference plus noise at receiver $i$ is
\[
 \Gamma_i = \frac{G_{ii}P_i}{\sum_{j\neq i} G_{ij} P_j + n_i}
\]
To guarantee the quality of received signal, it is required that
the signal to interference plus noise ratio $\Gamma_i$ is equal to
a predefined constant $\gamma_i$ for all $i$. Given $\gamma_i$,
$i=1,\ldots, n$, we want to find $P_1,\ldots, P_n$ such that the
above equation holds for $i=1,\ldots, n$. Let $A$ be the $n\times
n$ matrix with zero diagonal and $(i,j)$-entry
$(G_{ij}\gamma_i)/G_{ii}$ for $i\neq j$. We want to solve
\[
  (I -A) \mathbf{p} = \mathbf{n}
\]
where $\mathbf{p} = (P_1,\ldots, P_n)^T$ is the power vector and
$\mathbf{n} = (n_i \gamma_i/G_{ii})_{i=1}^n$. The matrix $I -A$ is
a Z-matrix, since all $G_{ij}$ and $\gamma_i$ are positive
constants. The required power vector is $(I-A)^{-1}\mathbf{n}$ if
$I-A$ is invertible. We also required that the components of
$\mathbf{p}$ to be positive as power cannot be negative. The
resulting power vector $(I-A)^{-1} \mathbf{n}$ has positive
components if $(I-A)^{-1}$ is a non-negative matrix. In such case,
$I-A$ is an M-matrix.
%%%%%
%%%%%
\end{document}
