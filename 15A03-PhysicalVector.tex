\documentclass[12pt]{article}
\usepackage{pmmeta}
\pmcanonicalname{PhysicalVector}
\pmcreated{2013-03-22 12:51:53}
\pmmodified{2013-03-22 12:51:53}
\pmowner{rmilson}{146}
\pmmodifier{rmilson}{146}
\pmtitle{physical vector}
\pmrecord{5}{33201}
\pmprivacy{1}
\pmauthor{rmilson}{146}
\pmtype{Definition}
\pmcomment{trigger rebuild}
\pmclassification{msc}{15A03}
\pmclassification{msc}{15A90}

\endmetadata

\newcommand{\vu}{u}
\newcommand{\vv}{v}

\newcommand{\cL}{\mathcal{L}}
\newcommand{\tmat}{\mathcal{M}}
\newcommand{\du}[2]{^{#2}_{\hphantom{#2}\!#1}}
\newcommand{\ud}[2]{_{#2}^{\hphantom{#2}\!#1}}

\usepackage{amsmath}
\usepackage{amsfonts}
\usepackage{amssymb}
\newcommand{\reals}{\mathbb{R}}
\newcommand{\natnums}{\mathbb{N}}
\newcommand{\cnums}{\mathbb{C}}
\newcommand{\znums}{\mathbb{Z}}
\newcommand{\lp}{\left(}
\newcommand{\rp}{\right)}
\newcommand{\lb}{\left[}
\newcommand{\rb}{\right]}
\newcommand{\supth}{^{\text{th}}}
\newtheorem{proposition}{Proposition}
\newtheorem{definition}[proposition]{Definition}
\newcommand{\nl}[1]{{\PMlinkescapetext{{#1}}}}
\newcommand{\pln}[2]{{\PMlinkname{{#1}}{#2}}}
\begin{document}
\paragraph{Definition.}
Let $\cL$ be a collection of labels $\alpha\in\cL$. For each ordered
pair of labels $(\alpha,\beta)\in\cL\times\cL$ let
$\tmat\du{\alpha}{\beta}$ be a non-singular $n\times n$ matrix, the
collection of all such satisfying the following functor-like
consistency conditions:
\begin{itemize}
\item For all $\alpha\in \cL$, the matrix $\tmat\du{\alpha}{\alpha}$ is
  the identity matrix.
\item For all $\alpha,\beta,\gamma \in \cL$ we have
  $$\tmat\du{\alpha}{\gamma} = \tmat\du{\beta}{\gamma}\, \tmat\du{\alpha}{\beta},$$
  where the product in
  the right-hand side is just ordinary matrix multiplication. 
\end{itemize}
We then impose an equivalence relation by stipulating that for all
$\alpha,\beta\in\cL$ and $\vu\in\reals^n$, the pair $(\alpha,\vu)$ is
equivalent to the pair $(\beta,\tmat\du{\alpha}{\beta}\vu)$.  Finally,
we define a physical vector to be an equivalence class of such pairs
relative to the just-defined relation.

The idea behind this definition is that the $\alpha\in \cL$ are labels
of various coordinate systems, and that the matrices
$\tmat\du{\alpha}{\beta}$ encode the corresponding changes of
coordinates. For label $\alpha\in \cL$ and list-vector
$\vu\in\reals^n$ we think of the pair $(\alpha,\vu)$ as the
representation of a physical vector relative to the coordinate system
$\alpha$.

\paragraph{Discussion.}
All scientific disciplines have a need for formalization.  However,
the extent to which rigour is pursued varies from one discipline to
the next.  Physicists and engineers are more likely to regard
mathematics as a tool for modeling and prediction.  As such they are
likely to blur the distinction between list vectors and physical
vectors.  Consider, for example the following excerpt from R.
Feynman's ``Lectures on physics'' \cite{feynman}
\begin{quote}
  All quantities that have a direction, like a step in space, are
  called vectors.  A vector is three numbers.  In order to represent a
  step in space, $\ldots$, we really need three numbers, but we are
  going to invent a single mathematical symbol, $\mathbf{r}$, which is
  unlike any other mathematical symbols we have so far used.  It is
  {\em not} a single number, it represents {\em three} numbers: $x$,
  $y$, and $z$.  It means three numbers, but not only {\em those}
  three numbers, because if we were to use a different coordinate
  system, the three numbers would be changed to $x'$, $y'$, and $z'$.
  However, we want to keep our mathematics simple and so we are going
  to use the {\em same mark} to represent the three numbers $(x,y,z)$
  and the three numbers $(x',y',z')$.  That is, we use the same mark
  to represent the first set of three numbers for one coordinate
  system, but the second set of three numbers if we are using the
  other coordinate system.  This has the advantage that when we change
  the coordinate system, we do not have to change the letters of our
  equations.
\end{quote}
Surely you are joking Mr. Feynman!?  What are we supposed to make of
this definition?  We learn that a vector is both a physical quantity,
and a list of numbers. However we also learn that it is not really a
specific list of numbers, but rather any of a number of possible
lists.  Furthermore, the choice of which list is being used is
dependent on the context (choice of coordinate system), but this is
not really important because we just end up using the same symbol
$\mathbf{r}$ regardless.  

What a muddle!  Even at the informal level one can do better than
Feynman.  The central weakness of his definition is that he is
unwilling to distinguish between physical vectors (quantities) and
their representation (lists of numbers).  Here is an alternative
physical definition from a book by R. Aris on fluid mechanics
\cite{aris}.
\begin{quote}
  There are many physical quantities with which only a single
  magnitude can be associated.  For example, when suitable units of
  mass and length have been adopted the density of a fluid may be
  measured. $\ldots$  There are other quantities associated with a
  point that have not only a magnitude but also a direction.  If a
  force of 1 lb weight is said to act at a certain point, we can still
  ask in what direction the force acts and it is not fully specified
  until this direction is given.  Such a physical quantity is a {\em
  vector}. $\ldots$  We distinguish therefore between the vector as an
  entity and its components which allow us to reconstruct it {\em in a
  particular system of reference.}  The set of components is
  meaningless unless the system of reference is also prescribed, just
  as the magnitude 62.427 is meaningless as a density until the units
  are also prescribed.
  $\ldots$.

  {\em Definition.} A Cartesian vector, $\mathbf{a}$, in three
  dimensions is a quantity with three components $a_1, a_2, a_3$ in
  the frame of reference $O123$, which, under rotation of the
  coordinate frame to $O\bar{1}\bar{2}\bar{3}$, become components
  $\bar{a}_1, \bar{a}_2, \bar{a}_3$, where 
  $$\bar{a}_j = l_{ij} a_i.$$
  The vector $\mathbf{a}$ is to be regarded as an entity, just as the
  physical quantity it represents is an entity.  It is sometimes
  convenient to use the bold face $\mathbf{a}$ to show this. In any
  particular coordinate system it has components $a_1, a_2, a_3$, and
  it is at other times convenient to use the typical component $a_i$.
\end{quote}
Here we see a carefully drawn distinction between physical quantities
and the numerical measurements that represent them.  A system of
measurement, i.e. a choice of units and or coordinate axes, turns
physical quantities into numbers. However the correspondence is not
fixed, but varies according to the choice of measurement system.  This
point of view can be formalized by representing physical vectors as
labeled list vectors, the label specifying a choice of measurement
systems.  The actual vector is then defined to be an equivalence class
of such labeled list vectors.
\begin{thebibliography}{99}
\bibitem{feynman} R. Feynman, R. Leighton, and M. Sands, ``Lectures on
  Physics'', 
  11-4, Vol. I, Addison-Wesley.
\bibitem{aris} R. Aris, ``Vectors, Tensors and the Basic Equations of
  Fluid Mechanics'', Dover.
\end{thebibliography}
%%%%%
%%%%%
\end{document}
