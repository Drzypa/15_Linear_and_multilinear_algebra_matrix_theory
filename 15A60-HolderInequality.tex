\documentclass[12pt]{article}
\usepackage{pmmeta}
\pmcanonicalname{HolderInequality}
\pmcreated{2013-03-22 11:43:06}
\pmmodified{2013-03-22 11:43:06}
\pmowner{PrimeFan}{13766}
\pmmodifier{PrimeFan}{13766}
\pmtitle{H\"older inequality}
\pmrecord{27}{30094}
\pmprivacy{1}
\pmauthor{PrimeFan}{13766}
\pmtype{Theorem}
\pmcomment{trigger rebuild}
\pmclassification{msc}{15A60}
\pmclassification{msc}{55-XX}
\pmclassification{msc}{46E30}
\pmclassification{msc}{42B10}
\pmclassification{msc}{42B05}
\pmsynonym{Holder inequality}{HolderInequality}
\pmsynonym{Hoelder inequality}{HolderInequality}
%\pmkeywords{vector}
%\pmkeywords{norm}
\pmrelated{VectorPnorm}
\pmrelated{CauchySchwartzInequality}
\pmrelated{CauchySchwarzInequality}
\pmrelated{ProofOfMinkowskiInequality}
\pmrelated{ConjugateIndex}
\pmrelated{BoundedLinearFunctionalsOnLpmu}
\pmrelated{ConvolutionsOfComplexFunctionsOnLocallyCompactGroups}
\pmrelated{LpNormIsDualToLq}

\endmetadata

\usepackage{amssymb}
\usepackage{amsmath}
\usepackage{amsfonts}
\begin{document}
The \emph{H\"older inequality} concerns \emph{vector p-norms}: given $1 \leq p$, $q \leq \infty$,

\begin{displaymath}
    \mbox{If }\frac{1}{p}+\frac{1}{q}=1\mbox{ then }|x^Ty| \leq ||\,x\,||_p||\,y\,||_q
\end{displaymath}

An important instance of a H\"older inequality is the \emph{Cauchy-Schwarz inequality}.

There is a version of this result for the \PMlinkname{$L^p$ spaces}{LpSpace}.
If a function $f$ is in $L^p(X)$, then the $L^p$-norm of $f$ is denoted
$||\,f\,||_p$.
Given a measure space $(X,\mathfrak{B},\mu)$, if $f$ is in $L^p(X)$ and $g$ is in $L^q(X)$ (with $1/p + 1/q = 1$), then
the H\"older inequality becomes

\begin{eqnarray*}
\Vert fg\Vert_1 = \int_X \vert fg\vert \mathrm{d}\mu 
                      & \le & 
\left(\int_X|f|^p\mathrm{d}\mu\right)^{\frac{1}{p}}
\left(\int_X|g|^q\mathrm{d}\mu\right)^{\frac{1}{q}}\\
& = & \Vert f\Vert_p\,\Vert g \Vert_q 
\end{eqnarray*}
%%%%%
%%%%%
%%%%%
%%%%%
%%%%%
%%%%%
%%%%%
%%%%%
%%%%%
%%%%%
%%%%%
\end{document}
