\documentclass[12pt]{article}
\usepackage{pmmeta}
\pmcanonicalname{InAVectorSpacelambdaV0IfAndOnlyIflambda0OrVIsTheZeroVector}
\pmcreated{2013-03-22 13:37:34}
\pmmodified{2013-03-22 13:37:34}
\pmowner{aoh45}{5079}
\pmmodifier{aoh45}{5079}
\pmtitle{in a vector space, $\lambda v = 0$ if and only if $\lambda =0$ or $v$ is the zero vector}
\pmrecord{10}{34264}
\pmprivacy{1}
\pmauthor{aoh45}{5079}
\pmtype{Theorem}
\pmcomment{trigger rebuild}
\pmclassification{msc}{15-00}
\pmclassification{msc}{13-00}
\pmclassification{msc}{16-00}

% this is the default PlanetMath preamble.  as your knowledge
% of TeX increases, you will probably want to edit this, but
% it should be fine as is for beginners.

% almost certainly you want these
\usepackage{amssymb}
\usepackage{amsmath}
\usepackage{amsfonts}

% used for TeXing text within eps files
%\usepackage{psfrag}
% need this for including graphics (\includegraphics)
%\usepackage{graphicx}
% for neatly defining theorems and propositions
%\usepackage{amsthm}
% making logically defined graphics
%%%\usepackage{xypic}

% there are many more packages, add them here as you need them

% define commands here
\begin{document}
\PMlinkescapeword{satisfies}
\PMlinkescapeword{satisfy}
\PMlinkescapeword{axiom}

{\bf Theorem} 
Let $V$ be a vector space over the field $F$. 
Further, let $\lambda \in F$ and  $v\in V$.
Then 
$\lambda v = 0$ if and only if $\lambda$ is zero, or
if $v$ is the zero vector, or if both $\lambda$ and $v$ are zero. 

\newcommand{\axiom}[1]{\PMlinkname{axiom #1}{VectorSpace}}

\emph{Proof.} Let us denote by $0_F$ and by $1_F$ the zero and unit
elements in $F$ respectively. Similarly, we denote by $0_V$ the zero vector in $V$. 
Suppose $\lambda = 0_F$. 
Then, by \axiom{8}, we have that
$$  1_Fv + 0_F v = 1_Fv,$$
for all $v\in V$. By \axiom{6}, there is an element in $V$ that cancels
$1_F v$. Adding this element to both \PMlinkescapetext{sides} yields $ 0_F v = 0_V$.
Next, suppose that $v=0_V$. We claim 
that $\lambda 0_V = 0_V$ for all $\lambda\in F$.
This follows from the previous claim if $\lambda = 0$, so
let us assume that $\lambda \neq 0_F$. Then $\lambda^{-1}$ exists, and 
 \axiom{7} implies that
$$ \lambda \lambda^{-1} v +\lambda 0_V = \lambda(\lambda^{-1} v + 0_V)$$
holds for all $v\in V$. Then using
 \axiom{3}, we have that
$$  v + \lambda 0_V = v$$
for all $v\in V$. 
Thus $\lambda 0_V$ satisfies the axiom for the zero vector, and
$\lambda 0_V = 0_V$ for all $\lambda \in F$.

For the other direction, suppose $\lambda v = 0_V$ and $\lambda\neq 0_F$. 
Then, using \axiom{3}, we have that  
$$ v = 1_F v = \lambda^{-1} \lambda v    = \lambda^{-1} 0_V    = 0_V.$$
On the other hand, suppose $\lambda v = 0_V$ and $v \neq 0_V$.
If $\lambda \neq 0$, then the above calculation for $v$ 
is again valid whence
$$ 0_V \neq v = 0_V,$$
which is a contradiction, so $\lambda = 0$. 
$\Box$

This result with proof can be found in \cite{greub}, page 6.

\begin{thebibliography}{9}
\bibitem{greub}
W. Greub,
\emph{Linear Algebra},
Springer-Verlag, Fourth edition, 1975.
\end{thebibliography}
%%%%%
%%%%%
\end{document}
