\documentclass[12pt]{article}
\usepackage{pmmeta}
\pmcanonicalname{DualityWithRespectToANondegenerateBilinearForm}
\pmcreated{2013-03-22 16:23:02}
\pmmodified{2013-03-22 16:23:02}
\pmowner{alozano}{2414}
\pmmodifier{alozano}{2414}
\pmtitle{duality with respect to a non-degenerate bilinear form}
\pmrecord{8}{38528}
\pmprivacy{1}
\pmauthor{alozano}{2414}
\pmtype{Definition}
\pmcomment{trigger rebuild}
\pmclassification{msc}{15A99}
\pmrelated{BilinearForm}
\pmrelated{PolaritiesAndForms}

\endmetadata

% this is the default PlanetMath preamble.  as your knowledge
% of TeX increases, you will probably want to edit this, but
% it should be fine as is for beginners.

% almost certainly you want these
\usepackage{amssymb}
\usepackage{amsmath}
\usepackage{amsthm}
\usepackage{amsfonts}

% used for TeXing text within eps files
%\usepackage{psfrag}
% need this for including graphics (\includegraphics)
%\usepackage{graphicx}
% for neatly defining theorems and propositions
%\usepackage{amsthm}
% making logically defined graphics
%%%\usepackage{xypic}

% there are many more packages, add them here as you need them

% define commands here

\newtheorem{thm}{Theorem}
\newtheorem{defn}{Definition}
\newtheorem{prop}{Proposition}
\newtheorem{lemma}{Lemma}
\newtheorem{cor}{Corollary}

\theoremstyle{definition}
\newtheorem{exa}{Example}

% Some sets
\newcommand{\Nats}{\mathbb{N}}
\newcommand{\Ints}{\mathbb{Z}}
\newcommand{\Reals}{\mathbb{R}}
\newcommand{\Complex}{\mathbb{C}}
\newcommand{\Rats}{\mathbb{Q}}
\newcommand{\Gal}{\operatorname{Gal}}
\newcommand{\Cl}{\operatorname{Cl}}
\begin{document}
\begin{defn}
Let $V$ and $W$ be finite dimensional vector spaces over a field $F$ and let $B:V\times W \to F$ be a non-degenerate bilinear form. Then we say that $V$ and $W$ are dual with respect to $B$.
\end{defn}

\begin{exa}
Let $V$ be a finite dimensional vector space and let $W=V^\ast$ be the dual space of $V$, i.e. $W$ is the vector space formed by all linear transformations $V\to F$. Let $B:V\times V^\ast \to F$ be defined by $B(v,f)=f(v)$ for all $v\in V$ and all $f:V\to F$ in $V^\ast$. Then $B$ is a non-degenerate bilinear form and $V$ and $V^\ast$ are dual with respect to $B$.
\end{exa}

\begin{defn}
Let $f:V\to V$ and $g:W\to W$ be linear transformations. We say that $f$ and $g$ are transposes of each other with respect to $B$ if
$$B(f(v),w)=B(v,g(w))$$
for all $v\in V$ and $w\in W$.
\end{defn}

The reasons why the terms ``dual'' and ``transpose'' are used are explained in the following theorems (here $V^\ast$ denotes the dual vector space of $V$). Notice that for a fixed element $w\in W$ one can define a linear form $V \to F$ which sends $v$ to $B(v,w)$.

\begin{thm}
Let $V,W$ be finite dimensional vector spaces over $F$ which are dual with respect to a non-degenerate bilinear form $B:V\times W \to F$. Then there exist canonical isomorphisms $V \cong W^\ast$ and $W\cong V^\ast$ given by
$$W\to V^\ast,\ w\mapsto (v\mapsto B(v,w));\quad V\to W^\ast,\ v\mapsto (w\mapsto B(v,w)).$$
\end{thm}

\begin{thm}
Let $V,W$ be finite dimensional vector spaces over $F$ which are dual with respect to a non-degenerate bilinear form $B:V\times W \to F$. Moreover, suppose $f:V\to V$ and $g:W\to W$ are transposes of each other with respect to $B$. Let $\mathcal{B}=\{v_1,\ldots,v_n\}$ be a basis of $V$ and let $\mathcal{C}=\{w_1,\ldots,w_n\}$ be the basis of $W$ which maps to the dual basis of $\mathcal{B}$ via the isomporphism $W\cong V^\ast$ defined in the previous theorem. If $A$ is the matrix of $f$ in the basis $\mathcal{B}$ then the matrix of $g$ in the basis $\mathcal{C}$ is $A^T$, the transpose matrix of $A$.
\end{thm}

\begin{proof}[{\bf Proof of Theorem 2.}]
Let $V$ and $W$ be dual with respect to a non-degenerate bilinear
form $B$ and let $f$ and $g$ be transposes of each other, also
with respect to $B$ so that:
$$B(f(v),w)=B(v,g(w))$$
for all $v\in V$ and $w \in W$. By Theorem 1, we have $W\cong V^\ast$. Let
$\mathcal{B}=\{v_1,\ldots,v_n\}$ be a basis for $V$ and let
$\mathcal{C}=\{w_1,\ldots,w_n\}$ be a basis for $W$ which
corresponds to the dual basis of $V^\ast$ via the isomorphism $W\cong V^\ast$. Then $B(v_i,w_j)=1$
for $i=j$ and equal to $0$ otherwise. Let $A=(\alpha_{ij})$ be the
matrix of $f$ with respect to $\mathcal{B}$. Then
$$f(v_j)=\sum_{i=1}^n \alpha_{ij}v_i.$$
Let $A'=(\beta_{ij})$ be the matrix of $g$ with respect to
$\mathcal{C}$ so that $g(w_j)=\sum_i \beta_{ij}w_i$. We will
show that $A'=A^T$, the transpose of $A$. Indeed:
$$B(f(v_j),w_k)=B(\sum_i \alpha_{ij}v_i,w_k)=\alpha_{kj}$$
and also
$$B(f(v_j),w_k)=B(v_j,g(w_k))=B(v_j,\sum_i
\beta_{ik}w_i)=\beta_{jk}.$$ Therefore $\beta_{jk}=\alpha_{kj}$
for all $k$ and $j$, as desired.
\end{proof}
%%%%%
%%%%%
\end{document}
