\documentclass[12pt]{article}
\usepackage{pmmeta}
\pmcanonicalname{ProofOfWeylsInequality}
\pmcreated{2013-03-22 15:33:39}
\pmmodified{2013-03-22 15:33:39}
\pmowner{Andrea Ambrosio}{7332}
\pmmodifier{Andrea Ambrosio}{7332}
\pmtitle{proof of Weyl's inequality}
\pmrecord{9}{37464}
\pmprivacy{1}
\pmauthor{Andrea Ambrosio}{7332}
\pmtype{Proof}
\pmcomment{trigger rebuild}
\pmclassification{msc}{15A42}

% this is the default PlanetMath preamble.  as your knowledge
% of TeX increases, you will probably want to edit this, but
% it should be fine as is for beginners.

% almost certainly you want these
\usepackage{amssymb}
\usepackage{amsmath}
\usepackage{amsfonts}

% used for TeXing text within eps files
%\usepackage{psfrag}
% need this for including graphics (\includegraphics)
%\usepackage{graphicx}
% for neatly defining theorems and propositions
%\usepackage{amsthm}
% making logically defined graphics
%%%\usepackage{xypic}

% there are many more packages, add them here as you need them

% define commands here
\begin{document}
Let $\lambda_i$ be the i-th eigenvalue of $A+E$. Then, by the Courant-Fisher min-max theorem and being $x^HEx \geq 0$ by hypothesis, we have:

$\lambda_i(A+E)=\max\limits_{S, dim S=i}\min\limits_{\|x\|\ne 0}\frac{x^H(A+E)x}{x^H x}=$\\
$=\max\limits_{S,\dim S=i}\min\limits_{\|x\|\ne 0}\left(\frac{x^HAx}{x^Hx}+\frac{x^HEx}{x^Hx}\right)\geq\max\limits_{S,\dim S=i}\min\limits_{\|x\|\ne 0}\frac{x^HAx}{x^Hx}=\lambda_i(A)$.
%%%%%
%%%%%
\end{document}
