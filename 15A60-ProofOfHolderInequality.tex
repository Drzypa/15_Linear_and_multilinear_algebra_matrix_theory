\documentclass[12pt]{article}
\usepackage{pmmeta}
\pmcanonicalname{ProofOfHolderInequality}
\pmcreated{2013-03-22 13:31:16}
\pmmodified{2013-03-22 13:31:16}
\pmowner{paolini}{1187}
\pmmodifier{paolini}{1187}
\pmtitle{proof of H\"{o}lder inequality}
\pmrecord{10}{34110}
\pmprivacy{1}
\pmauthor{paolini}{1187}
\pmtype{Proof}
\pmcomment{trigger rebuild}
\pmclassification{msc}{15A60}
\pmclassification{msc}{46E30}
\pmsynonym{proof of H\"{o}lder inequality}{ProofOfHolderInequality}
\pmsynonym{proof of Holder's inequality}{ProofOfHolderInequality}

\endmetadata

% this is the default PlanetMath preamble.  as your knowledge
% of TeX increases, you will probably want to edit this, but
% it should be fine as is for beginners.

% almost certainly you want these
\usepackage{amssymb}
\usepackage{amsmath}
\usepackage{amsfonts}

% used for TeXing text within eps files
%\usepackage{psfrag}
% need this for including graphics (\includegraphics)
%\usepackage{graphicx}
% for neatly defining theorems and propositions
%\usepackage{amsthm}
% making logically defined graphics
%%%\usepackage{xypic}

% there are many more packages, add them here as you need them

% define commands here
\begin{document}
First we prove the more general form (in measure spaces).

Let $(X,\mu)$ be a measure space and let $f\in L^p(X)$, $g\in L^q(X)$ where $p,q\in [1,+\infty]$ and $\frac 1 p + \frac 1 q = 1$.

The case $p=1$ and $q=\infty$ is obvious since
\[
  \vert f(x) g(x)\vert \le \Vert g\Vert_{L^\infty} \vert f(x)\vert.
\]
Also if $f=0$ or $g=0$ the result is obvious. 
Otherwise notice that (applying \PMlinkid{Young inequality}{YoungInequality}) we have
\[
\frac{\Vert fg\Vert_1}{\Vert f\Vert_p\cdot \Vert g\Vert_q}
= \int_X \frac{\vert f\vert}{\Vert f\Vert_p} \cdot
  \frac{\vert g\vert}{\Vert g\Vert_q} \, d\mu
\le \frac 1 p \int_X \left(\frac{\vert f\vert}{\Vert f\Vert_p}\right)^p\, d\mu
 + \frac 1 q \int_X \left(\frac{\vert g\vert}{\Vert g\Vert_q}\right)^q\, d\mu
= \frac 1 p + \frac 1 q = 1
\]
hence the desired inequality holds
\[
  \int_X \vert f g\vert = \Vert fg\Vert_1 \le \Vert f \Vert_p \cdot
   \Vert g\Vert_q = \left(\int_X \vert f\vert^p\right)^{\frac 1 p}
   \left(\int_X \vert g\vert ^q\right)^{\frac 1 q}.
\]

If $x$ and $y$ are vectors in ${\mathbb R}^n$ or vectors in $\ell^p$ and $\ell^q$-spaces we can specialize the previous result by choosing $\mu$ to be the counting measure on ${\mathbb N}$.  

In this case the proof can also be rewritten, without using measure theory,
as follows.
If we define
\[
  \Vert x \Vert_p = \left(\sum_k \vert x_k\vert^p\right)^{\frac 1 p}
\]
we have
\[
 \frac{\left\vert \sum_k x_k y_k \right\vert}{\Vert x\Vert_p \cdot \Vert y \Vert_q}
\le \frac{\sum_k \vert x_k\vert \vert y_k\vert}{\Vert x\Vert_p\cdot \Vert y\Vert_q}
= \sum_k \frac{\vert x_k\vert}{\Vert x\Vert_p} \frac{\vert y_k\vert}{\Vert y\Vert_q}
\le \frac 1 p \sum_k \frac{\vert x_k\vert^p}{\Vert x\Vert_p^p} +
 \frac 1 q \sum_k \frac{\vert y_k\vert^q}{\Vert y\Vert_q^q}
= \frac 1 p + \frac 1 q = 1.
\]
%%%%%
%%%%%
\end{document}
