\documentclass[12pt]{article}
\usepackage{pmmeta}
\pmcanonicalname{ProofOfTheDeterminantConditionForASequenceOfVectors}
\pmcreated{2013-03-22 14:33:46}
\pmmodified{2013-03-22 14:33:46}
\pmowner{GeraW}{6138}
\pmmodifier{GeraW}{6138}
\pmtitle{proof of the determinant condition for a sequence of vectors}
\pmrecord{5}{36117}
\pmprivacy{1}
\pmauthor{GeraW}{6138}
\pmtype{Proof}
\pmcomment{trigger rebuild}
\pmclassification{msc}{15A15}

\endmetadata

% this is the default PlanetMath preamble.  as your knowledge
% of TeX increases, you will probably want to edit this, but
% it should be fine as is for beginners.

% almost certainly you want these
\usepackage{amssymb}
\usepackage{amsmath}
\usepackage{amsfonts}

% used for TeXing text within eps files
%\usepackage{psfrag}
% need this for including graphics (\includegraphics)
%\usepackage{graphicx}
% for neatly defining theorems and propositions
\usepackage{amsthm}
% making logically defined graphics
%%%\usepackage{xypic}

% there are many more packages, add them here as you need them

% define commands here
\newcommand{\R}{{\mathbb{R}}}
\newcommand{\N}{{\mathbb{N}}}
\newtheorem*{theorem}{Theorem}

\DeclareMathOperator{\spn}{span}
\begin{document}
\begin{theorem} Let $x_1,x_2,...$ be a sequence of $d$ dimensional vectors. Assume that there is $C: \N^d \to \R \smallsetminus \{0\}$
such that
\begin{equation}\label{Eq:assump}
\sum_{\substack{n_1+\cdots+n_d=n \\ 0< n_1 < \cdots < n_d }}C(n_1,...,n_d) \det[x_{n_1},x_{n_2},...,x_{n_d}] = 0
\end{equation} for every $n \in \N$. Then $\det[x_{n_1},x_{n_2},...,x_{n_d}]\!=\!0$ for all $(n_1,...,n_d) \in \N^d$. \end{theorem}
\begin{proof}
Introduce a linear order over the set of ordered tuples: $(n_1,n_2,...,n_d) \prec (\hat{n}_1,\hat{n}_2,...,\hat{n}_d)$ if $\left(\sum_{i=1}^d n_i,\hat{n}_d,\hat{n}_{d-1},...,\hat{n}_1\right)$ precedes $\left(\sum_{i=1}^d\hat{n}_i,n_d,n_{d-1},...,n_1\right)$
lexicographically. Let $(n_1,n_2,...,n_d)$ be the minimal (according to the above order) ordered tuple for which
\begin{equation}\label{Eq:det-is-not-zero}
\det[x_{n_1},x_{n_2},...,x_{n_d}] \ne 0.
\end{equation}
Take another ordered tuple, $(\hat{n}_1,\hat{n}_2,...,\hat{n}_d)$, such that $\sum_{i=1}^d n_i = \sum_{i=1}^d \hat{n}_i$. By minimality, if
$(n_d,n_{d-1},...,n_1)$ precedes $(\hat{n}_d,\hat{n}_{d-1},...,\hat{n}_1)$ lexicographically then
$\det[x_{\hat{n}_1},x_{\hat{n}_2},...,x_{\hat{n}_d}]=0$. %
Otherwise, let $i \in \{0,1,...,d-1\}$ be the first index such that $n_{d-i} \ne \hat{n}_{d-i}$ (more specifically, $n_{d-i} > \hat{n}_{d-i}$). Then,
$\hat{n}_{d-j} = n_{d-j}$ for $j=0,...,i-1$ and $\hat{n}_{d-j} < n_{d-i}$ for $j=i,...,d-1$. Therefore,
\[\det[x_{n_1},...,x_{n_{d-i-1}},x_{\hat{n}_m},x_{n_{d-i+1}},...,x_{n_d}] = 0\] for all $m=1,2,...,d$ (some because of repeated columns and the others
because $\sum_{j=1}^d n_j - n_{d-i} + \hat{n}_m < \sum_{j=1}^d n_j$). Since the vectors $x_{n_1},x_{n_2},...,x_{n_d}$ are linearly independent, we get
that
\[\{x_{\hat{n}_1},x_{\hat{n}_2},...,x_{\hat{n}_d}\} \subset \spn \left(\{x_{n_1},x_{n_2},...,x_{n_d}\} \smallsetminus \{x_{n_{d-i}}\}\right).\]
In particular $\det[x_{\hat{n}_1},x_{\hat{n}_2},...,x_{\hat{n}_d}]=0$. Therefore, \eqref{Eq:assump} reduces to $\det[x_{n_1},x_{n_2},...,x_{n_d}]=0$
which contradicts \eqref{Eq:det-is-not-zero}.

\end{proof}
%%%%%
%%%%%
\end{document}
