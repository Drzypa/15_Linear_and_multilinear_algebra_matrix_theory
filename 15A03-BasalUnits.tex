\documentclass[12pt]{article}
\usepackage{pmmeta}
\pmcanonicalname{BasalUnits}
\pmcreated{2013-03-22 19:17:13}
\pmmodified{2013-03-22 19:17:13}
\pmowner{pahio}{2872}
\pmmodifier{pahio}{2872}
\pmtitle{basal units}
\pmrecord{7}{42220}
\pmprivacy{1}
\pmauthor{pahio}{2872}
\pmtype{Definition}
\pmcomment{trigger rebuild}
\pmclassification{msc}{15A03}
\pmrelated{MatrixRing}
\pmrelated{StandardBasis}

\endmetadata

% this is the default PlanetMath preamble.  as your knowledge
% of TeX increases, you will probably want to edit this, but
% it should be fine as is for beginners.

% almost certainly you want these
\usepackage{amssymb}
\usepackage{amsmath}
\usepackage{amsfonts}

% used for TeXing text within eps files
%\usepackage{psfrag}
% need this for including graphics (\includegraphics)
%\usepackage{graphicx}
% for neatly defining theorems and propositions
 \usepackage{amsthm}
% making logically defined graphics
%%%\usepackage{xypic}

% there are many more packages, add them here as you need them

% define commands here

\theoremstyle{definition}
\newtheorem*{thmplain}{Theorem}

\begin{document}
The set of all $n\!\times\!n$ matrices over a skew field forms an $n^2$-dimensional associative algebra, for the basis of which one can take the \emph{basal units}.\, A basal unit of the algebra has all components zeroes except only one which is 1.\\

E.g.
$$\begin{pmatrix}
0 & 0 & 0 \\
0 & 0 & 1 \\
0 & 0 & 0 
\end{pmatrix}$$
is a basal unit of the algebra of $3\!\times\!3$ matrices.

\begin{thebibliography}{8}
\bibitem{D}{\sc L. E. Dickson}: {\em Algebras and their arithmetics}.\, Dover Publications, Inc.\, New York (1923; second edition 1960).
\end{thebibliography}
%%%%%
%%%%%
\end{document}
