\documentclass[12pt]{article}
\usepackage{pmmeta}
\pmcanonicalname{CorollaryOfSchurDecomposition}
\pmcreated{2013-03-22 13:43:38}
\pmmodified{2013-03-22 13:43:38}
\pmowner{Daume}{40}
\pmmodifier{Daume}{40}
\pmtitle{corollary of Schur decomposition}
\pmrecord{7}{34413}
\pmprivacy{1}
\pmauthor{Daume}{40}
\pmtype{Corollary}
\pmcomment{trigger rebuild}
\pmclassification{msc}{15-00}

\endmetadata

% this is the default PlanetMath preamble.  as your knowledge
% of TeX increases, you will probably want to edit this, but
% it should be fine as is for beginners.

% almost certainly you want these
\usepackage{amssymb}
\usepackage{amsmath}
\usepackage{amsfonts}

% used for TeXing text within eps files
%\usepackage{psfrag}
% need this for including graphics (\includegraphics)
%\usepackage{graphicx}
% for neatly defining theorems and propositions
%\usepackage{amsthm}
% making logically defined graphics
%%%\usepackage{xypic} 

% there are many more packages, add them here as you need them

% define commands here
\begin{document}
\textbf{theorem:}$A\in \mathbb{C}^{n\times n}$ is a normal matrix if and only if there exists a unitary matrix $Q \in \mathbb{C}^{n\times n}$ such that $Q^HAQ = \operatorname{diag}(\lambda_1,\ldots , \lambda_n)$\textit{(the diagonal matrix)} where $^H$ is the conjugate transpose. \cite{1}\\\\
\textbf{proof:} Firstly we show that if there exists a unitary matrix $Q \in \mathbb{C}^{n\times n}$ such that $Q^HAQ = \operatorname{diag}(\lambda_1,\ldots , \lambda_n)$ then $A\in \mathbb{C}^{n\times n}$ is a normal matrix.  Let $D = \operatorname{diag}(\lambda_1,\ldots , \lambda_n)$ then $A$ may be written as $A = QDQ^H$.  Verifying that A is normal follows by the following observation $AA^H = QDQ^HQD^HQ^H = QDD^HQ^H$ and $A^HA = QD^HQ^HQDQ^H = QD^HDQ^H$.  Therefore $A$ is normal matrix because $DD^H = \operatorname{diag}(\lambda_1\bar{\lambda_1},\ldots , \lambda_n \bar{\lambda_n}) = D^HD$.\\
Secondly we show that if $A\in \mathbb{C}^{n\times n}$ is a normal matrix then there exists a unitary matrix $Q \in \mathbb{C}^{n\times n}$ such that $Q^HAQ = \operatorname{diag}(\lambda_1,\ldots , \lambda_n)$.  By Schur decompostion we know that there exists a $Q\in \mathbb{C}^{n \times n}$ such that $Q^HAQ=T$\textit{($T$ is an upper triangular matrix)}.  Since $A$ is a normal matrix then $T$ is also a normal matrix.  The result that $T$ is a diagonal matrix comes from showing that a normal upper triangular matrix is diagonal (see theorem for normal triangular matrices).\\
QED
  
\begin{thebibliography}{1}
\bibitem[GVL]{1} Golub, H. Gene, Van Loan F. Charles: Matrix Computations \textit{(Third Edition)}. The Johns Hopkins University Press, London, 1996.
\end{thebibliography}
%%%%%
%%%%%
\end{document}
