\documentclass[12pt]{article}
\usepackage{pmmeta}
\pmcanonicalname{CompanionMatrix}
\pmcreated{2013-03-22 13:17:12}
\pmmodified{2013-03-22 13:17:12}
\pmowner{aoh45}{5079}
\pmmodifier{aoh45}{5079}
\pmtitle{companion matrix}
\pmrecord{7}{33773}
\pmprivacy{1}
\pmauthor{aoh45}{5079}
\pmtype{Definition}
\pmcomment{trigger rebuild}
\pmclassification{msc}{15A21}

% this is the default PlanetMath preamble.  as your knowledge
% of TeX increases, you will probably want to edit this, but
% it should be fine as is for beginners.

% almost certainly you want these
\usepackage{amssymb}
\usepackage{amsmath}
\usepackage{amsfonts}

% used for TeXing text within eps files
%\usepackage{psfrag}
% need this for including graphics (\includegraphics)
%\usepackage{graphicx}
% for neatly defining theorems and propositions
%\usepackage{amsthm}
% making logically defined graphics
%%%\usepackage{xypic}

% there are many more packages, add them here as you need them

% define commands here
\begin{document}
Given a monic polynomial $p(x)=x^n+a_{n-1}x^{n-1}+\dots+a_1x+a_0$ the \emph{companion matrix} of $p(x)$, denoted $\mathcal{C}_{p(x)}$, is the $n \times n$ matrix with $1$'s down the first subdiagonal and minus the coefficients of $p(x)$ down the last column, or alternatively, as the transpose of this matrix. Adopting the first convention this is simply
\begin{equation*}
\mathcal{C}_{p(x)} = \begin{pmatrix}
0 & 0 & \hdots & \hdots & \hdots & -a_0 \\
1 & 0 & \hdots & \hdots & \hdots & -a_1\\
0 & 1 & \hdots & \hdots & \hdots & -a_2 \\
0 & 0 & \ddots &   &   & \vdots \\
\vdots & \vdots &  & \ddots & &  \vdots \\
0 & 0 & \hdots & \hdots & 1 & -a_{n-1}
\end{pmatrix}.
\end{equation*}

Regardless of which convention is used the \PMlinkname{minimal polynomial}{MinimalPolynomialEndomorphism} of $\mathcal{C}_{p(x)}$ equals $p(x)$, and the characteristic polynomial of $\mathcal{C}_{p(x)}$ is just $(-1)^n p(x)$. The 
$(-1)^n$ is needed because we have defined the characteristic polynomial to be $\det(\mathcal{C}_{p(x)}- xI)$. If we had instead defined the characteristic polynomial to be $\det(xI - \mathcal{C}_{p(x)})$ then this would not be needed.
%%%%%
%%%%%
\end{document}
