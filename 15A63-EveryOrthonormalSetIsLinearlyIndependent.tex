\documentclass[12pt]{article}
\usepackage{pmmeta}
\pmcanonicalname{EveryOrthonormalSetIsLinearlyIndependent}
\pmcreated{2013-03-22 13:33:48}
\pmmodified{2013-03-22 13:33:48}
\pmowner{mathcam}{2727}
\pmmodifier{mathcam}{2727}
\pmtitle{every orthonormal set is linearly independent}
\pmrecord{14}{34169}
\pmprivacy{1}
\pmauthor{mathcam}{2727}
\pmtype{Theorem}
\pmcomment{trigger rebuild}
\pmclassification{msc}{15A63}

\endmetadata

% this is the default PlanetMath preamble.  as your knowledge
% of TeX increases, you will probably want to edit this, but
% it should be fine as is for beginners.

% almost certainly you want these
\usepackage{amssymb}
\usepackage{amsmath}
\usepackage{amsfonts}

% used for TeXing text within eps files
%\usepackage{psfrag}
% need this for including graphics (\includegraphics)
%\usepackage{graphicx}
% for neatly defining theorems and propositions
%\usepackage{amsthm}
% making logically defined graphics
%%%\usepackage{xypic}

% there are many more packages, add them here as you need them

% define commands here
\begin{document}
{\bf Theorem }: An orthonormal set of vectors in an inner product space is linearly independent.

\emph{Proof.} We denote by $\langle \cdot, \cdot \rangle$ the inner product of $L$.  Let $S$ be an orthonormal set of vectors.
Let us first consider the case when $S$ is finite, i.e.,
$S=\{e_1,\ldots, e_n\}$ for some $n$.
Suppose
$$ \lambda_1 e_1 + \cdots + \lambda_n e_n =0$$
for some scalars $\lambda_i$ (belonging to the field on the
underlying vector space of $L$). For a fixed $k$ in $1,\ldots, n$,
we then have
$$0=\langle e_k,0\rangle =  \langle e_k,\lambda_1 e_1 + \cdots + \lambda_n e_n \rangle = \lambda_1 \langle e_k,e_1\rangle + \cdots + \lambda_n \langle e_k,e_n\rangle = \lambda_k,$$

so $\lambda_k=0$, and $S$ is linearly independent.
Next, suppose $S$ is infinite (countable or uncountable). To prove
that $S$ is linearly independent, we need to show that
all finite subsets of $S$ are linearly independent. Since any
subset of an orthonormal set is also orthonormal, the infinite case
follows from the finite case. $\Box$
%%%%%
%%%%%
\end{document}
