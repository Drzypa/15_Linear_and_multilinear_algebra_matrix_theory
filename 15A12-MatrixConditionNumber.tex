\documentclass[12pt]{article}
\usepackage{pmmeta}
\pmcanonicalname{MatrixConditionNumber}
\pmcreated{2013-03-22 13:04:17}
\pmmodified{2013-03-22 13:04:17}
\pmowner{stevecheng}{10074}
\pmmodifier{stevecheng}{10074}
\pmtitle{matrix condition number}
\pmrecord{10}{33480}
\pmprivacy{1}
\pmauthor{stevecheng}{10074}
\pmtype{Definition}
\pmcomment{trigger rebuild}
\pmclassification{msc}{15A12}
\pmclassification{msc}{65F35}
\pmsynonym{matrix condition number}{MatrixConditionNumber}
\pmsynonym{condition number}{MatrixConditionNumber}
\pmrelated{PropertyOfMatrixConditionNumber}
\pmdefines{ill-conditioned}
\pmdefines{well-conditioned}

\endmetadata

\usepackage{amssymb}
\usepackage{amsmath}
\usepackage{amsfonts}

%\usepackage{psfrag}
%\usepackage{graphicx}
%%%\usepackage{xypic}

\providecommand{\norm}[1]{\lVert#1\rVert}
\begin{document}
\section{Matrix Condition Number}

The \emph{condition number for matrix inversion} with respect to a matrix norm
$\norm{\cdot}$ of a square matrix $A$ is defined by
\[
\kappa(A) = \Vert A \Vert \Vert A^{-1} \Vert\,,
\]
if $A$ is non-singular; and $\kappa(A) = +\infty$ if $A$ is singular.

The condition number is a measure of stability or sensitivity of a matrix (or the linear system it represents) to numerical operations. In other words, we may not be able to trust the results of computations on an ill-conditioned matrix.

Matrices with condition numbers near 1 are said to be \emph{well-conditioned}.  Matrices with condition numbers much greater than one (such as around $10^5$ for  a $5 \times 5$ Hilbert matrix) are said to be \emph{ill-conditioned}.  

If $\kappa(A)$ is the condition number of $A$, then $\kappa(A)$ measures 
a sort of inverse distance from $A$ to the set of singular matrices,
normalized by $\norm{A}$.
Precisely, if $A$ is invertible, and $\norm{B - A} < \norm{A^{-1}}^{-1}$,
then $B$ must also be invertible.  On the other hand, in the case of the $2$-norm,
there always exists a singular matrix $B$ such that $\norm{B-A}_2 = \norm{A^{-1}}_2^{-1}$
(so the distance estimate is sharp).

\begin{thebibliography}{3}
\bibitem{Golub} Golub and Van Loan. \emph{Matrix Computations}, 3rd edition. Johns Hopkins University Press, 1996.
\end{thebibliography}
%%%%%
%%%%%
\end{document}
