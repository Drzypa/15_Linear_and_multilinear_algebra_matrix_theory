\documentclass[12pt]{article}
\usepackage{pmmeta}
\pmcanonicalname{KroneckerDelta}
\pmcreated{2013-03-22 12:06:23}
\pmmodified{2013-03-22 12:06:23}
\pmowner{akrowne}{2}
\pmmodifier{akrowne}{2}
\pmtitle{Kronecker delta}
\pmrecord{7}{31221}
\pmprivacy{1}
\pmauthor{akrowne}{2}
\pmtype{Definition}
\pmcomment{trigger rebuild}
\pmclassification{msc}{15A99}
\pmrelated{IdentityMatrix}
\pmrelated{LeviCivitaPermutationSymbol3}

\endmetadata

\usepackage{amssymb}
\usepackage{amsmath}
\usepackage{amsfonts}
\usepackage{graphicx}
%%%\usepackage{xypic}
\begin{document}
The \emph{Kronecker delta} $\delta_{ij}$ is defined as having value 1 when $i=j$ and 0 otherwise ($i$ and $j$ are integers).  It may also be written as $\delta^{ij}$ or  $\delta^i_j$.  It is a special case of the generalized Kronecker delta symbol.

The delta symbol was first used in print by Kronecker in 1868\cite{Higham}. 

{\bf Example.}

The $n \times n$ identity matrix $I$ can be written in terms of the Kronecker delta as simply the matrix of the delta, $I_{ij}=\delta_{ij}$, or simply $I=(\delta_{ij})$.

\begin{thebibliography}{3}
\bibitem{Higham} N. Higham, Handbook of writing for the mathematical sciences, Society for Industrial and Applied Mathematics, 1998. 
\end{thebibliography}
%%%%%
%%%%%
%%%%%
\end{document}
