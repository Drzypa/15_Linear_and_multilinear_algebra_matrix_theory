\documentclass[12pt]{article}
\usepackage{pmmeta}
\pmcanonicalname{DerivativeOfHomogeneousFunction}
\pmcreated{2013-03-22 14:45:05}
\pmmodified{2013-03-22 14:45:05}
\pmowner{matte}{1858}
\pmmodifier{matte}{1858}
\pmtitle{derivative of homogeneous function}
\pmrecord{9}{36390}
\pmprivacy{1}
\pmauthor{matte}{1858}
\pmtype{Theorem}
\pmcomment{trigger rebuild}
\pmclassification{msc}{15-00}

\endmetadata

% this is the default PlanetMath preamble.  as your knowledge
% of TeX increases, you will probably want to edit this, but
% it should be fine as is for beginners.

% almost certainly you want these
\usepackage{amssymb}
\usepackage{amsmath}
\usepackage{amsfonts}
\usepackage{amsthm}

\usepackage{mathrsfs}

% used for TeXing text within eps files
%\usepackage{psfrag}
% need this for including graphics (\includegraphics)
%\usepackage{graphicx}
% for neatly defining theorems and propositions
%
% making logically defined graphics
%%%\usepackage{xypic}

% there are many more packages, add them here as you need them

% define commands here

\newcommand{\sR}[0]{\mathbb{R}}
\newcommand{\sC}[0]{\mathbb{C}}
\newcommand{\sN}[0]{\mathbb{N}}
\newcommand{\sZ}[0]{\mathbb{Z}}

 \usepackage{bbm}
 \newcommand{\Z}{\mathbbmss{Z}}
 \newcommand{\C}{\mathbbmss{C}}
 \newcommand{\R}{\mathbbmss{R}}
 \newcommand{\Q}{\mathbbmss{Q}}



\newcommand*{\norm}[1]{\lVert #1 \rVert}
\newcommand*{\abs}[1]{| #1 |}



\newtheorem{thm}{Theorem}
\newtheorem{defn}{Definition}
\newtheorem{prop}{Proposition}
\newtheorem{lemma}{Lemma}
\newtheorem{cor}{Corollary}
\begin{document}
\begin{thm} Suppose $f\colon \R^n\to \R^m$ is a differentiable
positively homogeneous function of degree $r$. 
Then $\frac{\partial f}{\partial x^i}$ is a 
positively homogeneous function of degree $r-1$.
\end{thm}

%\subsubsection*{Remark}
%No similar claim holds for homogeneous functions as the function  
%$x\mapsto x^2$ on $\R$ shows. 

\begin{proof} By considering component functions if necessary, we can 
assume that $m=1$. 
For $\lambda\in \R$, let $M_\lambda$ be the 
multiplication map, 
\begin{eqnarray*}
M_\lambda\colon \R^n &\to& \R^n \\
                v&\mapsto& \lambda v.
\end{eqnarray*}
For $\lambda>0$ and $v\in \R^n$, we have
\begin{eqnarray*}
  \frac{\partial f}{\partial x^i}(\lambda v) &=&\frac{\partial(f\circ M_\lambda \circ M_{1/\lambda})}{\partial x^i}(\lambda v) \\
  &= &\sum_{l=1}^n\frac{\partial(f\circ M_\lambda)}{\partial x^l} (v)\, \frac{ \partial(x\mapsto x/\lambda)^l}{\partial x^i} (\lambda v) \\
  &= &\frac{\partial(f\circ M_\lambda)}{\partial x^i} (v)\, \frac{1}{\lambda}\\
  &= &\lambda^{r-1}\frac{\partial f}{\partial x^i} (v)
\end{eqnarray*}
as claimed.
\end{proof}
%%%%%
%%%%%
\end{document}
