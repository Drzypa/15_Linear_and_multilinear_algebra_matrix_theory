\documentclass[12pt]{article}
\usepackage{pmmeta}
\pmcanonicalname{ExampleOfNondiagonalizableMatrices}
\pmcreated{2013-03-22 14:14:30}
\pmmodified{2013-03-22 14:14:30}
\pmowner{cvalente}{11260}
\pmmodifier{cvalente}{11260}
\pmtitle{example of non-diagonalizable matrices}
\pmrecord{14}{35686}
\pmprivacy{1}
\pmauthor{cvalente}{11260}
\pmtype{Example}
\pmcomment{trigger rebuild}
\pmclassification{msc}{15-00}

\endmetadata

% this is the default PlanetMath preamble.  as your knowledge
% of TeX increases, you will probably want to edit this, but
% it should be fine as is for beginners.

% almost certainly you want these
\usepackage{amssymb}
\usepackage{amsmath}
\usepackage{amsfonts}

% used for TeXing text within eps files
%\usepackage{psfrag}
% need this for including graphics (\includegraphics)
%\usepackage{graphicx}
% for neatly defining theorems and propositions
%\usepackage{amsthm}
% making logically defined graphics
%%%\usepackage{xypic}

% there are many more packages, add them here as you need them

% define commands here
\begin{document}
Some matrices with real entries which are not diagonalizable over $\mathbb{R}$ \emph{are} diagonalizable over the complex numbers $\mathbb{C}$.

For instance,

\[A = \begin{pmatrix}
          0 &-1
       \\ 1 & 0
       \end{pmatrix} \]

has $\lambda^2+1$ as characteristic polynomial.
This  polynomial  doesn't factor over the reals, but over $\mathbb{C}$ it does. Its roots are $\lambda = \pm i$.

Interpreting the matrix as a linear transformation  $\mathbb{C}^2 \to \mathbb{C}^2$, it has eigenvalues $i$ and $-i$ and linearly independent eigenvectors $(1, -i)$, $(-i, 1)$.  So we can diagonalize $A$:

\[A = \begin{pmatrix}
0 &-1
\\ 1 & 0
\end{pmatrix}=\begin{pmatrix}
1 & -i
\\ -i & 1
\end{pmatrix}\begin{pmatrix}
i&0
\\0&-i
\end{pmatrix}\begin {pmatrix}
.5&.5i
\\.5i&.5 \end{pmatrix}\]

But there exist real matrices which aren't diagonalizable even if complex eigenvectors and eigenvalues are allowed.

For example,

\[B = \begin{pmatrix}
0 & 1
\\ 0 & 0
\end{pmatrix}\]

cannot be written as $U D U^{-1}$ with $D$ diagonal.

In fact, the characteristic polynomial is $\lambda^2$ and it has only one double root $\lambda=0$.
However the eigenspace corresponding to the $0$ (kernel) eigenvalue has dimension 1.

$B \begin{pmatrix}
v_1
\\v_2
\end{pmatrix}
=
\begin{pmatrix}
0
\\0
\end{pmatrix}
\iff v_2 = 0
$ and thus the eigenspace is $ker(B) = span_{\mathbb{C}}\left \{ (1,0)^T \right\}$, with only one dimension.

There isn't a change of basis where $B$ is diagonal.
%%%%%
%%%%%
\end{document}
