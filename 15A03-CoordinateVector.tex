\documentclass[12pt]{article}
\usepackage{pmmeta}
\pmcanonicalname{CoordinateVector}
\pmcreated{2013-03-22 19:02:16}
\pmmodified{2013-03-22 19:02:16}
\pmowner{pahio}{2872}
\pmmodifier{pahio}{2872}
\pmtitle{coordinate vector}
\pmrecord{6}{41914}
\pmprivacy{1}
\pmauthor{pahio}{2872}
\pmtype{Definition}
\pmcomment{trigger rebuild}
\pmclassification{msc}{15A03}
\pmrelated{ListVector}
\pmdefines{coordinates}
\pmdefines{components}

\endmetadata

% this is the default PlanetMath preamble.  as your knowledge
% of TeX increases, you will probably want to edit this, but
% it should be fine as is for beginners.

% almost certainly you want these
\usepackage{amssymb}
\usepackage{amsmath}
\usepackage{amsfonts}

% used for TeXing text within eps files
%\usepackage{psfrag}
% need this for including graphics (\includegraphics)
%\usepackage{graphicx}
% for neatly defining theorems and propositions
 \usepackage{amsthm}
% making logically defined graphics
%%%\usepackage{xypic}

% there are many more packages, add them here as you need them

% define commands here

\theoremstyle{definition}
\newtheorem*{thmplain}{Theorem}

\begin{document}
Let $V$ be a vector space of dimension $n$ over a field $K$.\, If\, $(b_1,\,\ldots,\,b_n)$\, is a basis of $V$, then any element $v$ of $V$ can be uniquely expressed in the form
$$v \;=\; \xi_1b_1\!+\ldots+\!\xi_nb_n$$
with\, $\xi_1,\,\ldots,\,\xi_n \in K$.\, The \PMlinkname{$n$-tuplet}{OrderedTuplet} \,$(\xi_1,\,\ldots,\,\xi_n)$\ is called the \emph{coordinate vector} of $v$ with respect to the basis in question.\, The scalars $\xi_i$ are the \emph{coordinates} (or the \emph{components} of $v$).

It's evident that the correspondence
$$v \;\mapsto\; (\xi_1,\,\ldots,\,\xi_n)$$
provides a linear isomorphism between the vector space $V$ and the vector space formed by the Cartesian product $K^n$.


%%%%%
%%%%%
\end{document}
