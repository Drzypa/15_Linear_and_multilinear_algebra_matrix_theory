\documentclass[12pt]{article}
\usepackage{pmmeta}
\pmcanonicalname{Nullity}
\pmcreated{2013-03-22 12:24:06}
\pmmodified{2013-03-22 12:24:06}
\pmowner{rmilson}{146}
\pmmodifier{rmilson}{146}
\pmtitle{nullity}
\pmrecord{6}{32237}
\pmprivacy{1}
\pmauthor{rmilson}{146}
\pmtype{Definition}
\pmcomment{trigger rebuild}
\pmclassification{msc}{15A03}
\pmrelated{RankLinearMapping}
\pmrelated{RankNullityTheorem}

\usepackage{amsmath}
\usepackage{amsfonts}
\usepackage{amssymb}

\newcommand{\reals}{\mathbb{R}}
\newcommand{\natnums}{\mathbb{N}}
\newcommand{\cnums}{\mathbb{C}}

\newcommand{\lp}{\left(}
\newcommand{\rp}{\right)}
\newcommand{\lb}{\left[}
\newcommand{\rb}{\right]}

\newcommand{\supth}{^{\text{th}}}


\newtheorem{proposition}{Proposition}
\begin{document}
The \emph{nullity} of a linear mapping is the dimension of the mapping's kernel.
For a linear mapping $T:V\rightarrow W$, the nullity of $T$ gives the
number of linearly independent solutions to the equation
$$T(v)=0,\quad v\in V.$$
The nullity is zero if and only if the linear
mapping in question is injective.
%%%%%
%%%%%
\end{document}
