\documentclass[12pt]{article}
\usepackage{pmmeta}
\pmcanonicalname{AnExampleForSchurDecomposition}
\pmcreated{2013-03-22 15:27:02}
\pmmodified{2013-03-22 15:27:02}
\pmowner{georgiosl}{7242}
\pmmodifier{georgiosl}{7242}
\pmtitle{an example for Schur decomposition}
\pmrecord{8}{37299}
\pmprivacy{1}
\pmauthor{georgiosl}{7242}
\pmtype{Application}
\pmcomment{trigger rebuild}
\pmclassification{msc}{15-00}
\pmrelated{SchurDecomposition}
\pmrelated{GramSchmidtOrthogonalization}

\endmetadata

\usepackage{graphicx}
%%%\usepackage{xypic} 
\usepackage{bbm}
\newcommand{\Z}{\mathbbmss{Z}}
\newcommand{\C}{\mathbbmss{C}}
\newcommand{\R}{\mathbbmss{R}}
\newcommand{\Q}{\mathbbmss{Q}}
\newcommand{\mathbb}[1]{\mathbbmss{#1}}
\newcommand{\figura}[1]{\begin{center}\includegraphics{#1}\end{center}}
\newcommand{\figuraex}[2]{\begin{center}\includegraphics[#2]{#1}\end{center}}
\newtheorem{dfn}{Definition}
\usepackage[fleqn]{amsmath}
\begin{document}
Let 
$$
A = \begin{pmatrix} 5 & 7 \\ -2 & -4 \end{pmatrix}.
$$ 
We will find an orthogonal matrix $P$  and an upper triangular matrix $T$ such that $P^tAP=T$ applying the proof of Schur's decomposition.
 We 're following the steps below
\begin{itemize}
\item We find the eigenvalues of $A$ 
\\The eigenvalues of a matrix are precisely the solutions to the equation
$$
\det(\lambda I - A) = 0 \leftrightarrow \lambda^2-\lambda -6=0 
$$
\\Hence the roots of the \PMlinkname{quadratic equation}{QuadraticFormula} are the eigenvalues 
$\lambda_1=-2,\lambda_2=3$
\item We find the eigenvectors
\\For each eigenvalue $\lambda_i$, solving the system
$$ (A-\lambda_i I)X_i=0$$
So we have that
for $\lambda_1=-2$
$$(A+2I)=0  
\leftrightarrow \begin{pmatrix} 7 & 7 \\ -2 & -2 \end{pmatrix}\begin{pmatrix} x_1 \\x_2\end{pmatrix}=\begin{pmatrix} 0\\ 0\end{pmatrix}\rightarrow 
X_1=(1,-1)$$ 
\\Analogously for $\lambda_2=3$ 
the eigenvector $X_2=(7,-2)$
\item We get an orthonormal set of eigenvectors using Gram-Schmidt orthogonalization 
\\Consider the above two eigenvectors which are linearly independent but are not orthogonal 
\begin{align*}
X_1&=(1,-1)\\
X_2&=(7,-2)
\end{align*}
First we take $w_1=X_1=(1,-1)$. Therefore 
\[w_2= X_2 - \frac{w_1\cdot X_2}{\Vert w_1\Vert^2}w_1\] 
that is,
\[
w_2=(\frac{5}{2},\frac{5}{2})
\]
and finally the orthonormal set is 
$\{w_1/\Vert w_1\Vert,w_2/\Vert w_2\Vert\}=\{(\frac{1}{\sqrt{2}},\frac{-1}{\sqrt{2}}),(\frac{1}{\sqrt{2}},\frac{1}{\sqrt{2}})\}$
\\So 
$$P =\frac{1}{\sqrt{2}}\begin{pmatrix} 1 & 1 \\ -1 & 1 \end{pmatrix}.$$
Then $$T=P^tAP=\begin{pmatrix} -2 & 9 \\ 0 & 3 \end{pmatrix}.$$
\end{itemize}
%%%%%
%%%%%
\end{document}
