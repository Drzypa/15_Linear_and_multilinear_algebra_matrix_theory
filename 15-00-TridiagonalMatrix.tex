\documentclass[12pt]{article}
\usepackage{pmmeta}
\pmcanonicalname{TridiagonalMatrix}
\pmcreated{2013-03-22 12:11:37}
\pmmodified{2013-03-22 12:11:37}
\pmowner{akrowne}{2}
\pmmodifier{akrowne}{2}
\pmtitle{tridiagonal matrix}
\pmrecord{5}{31480}
\pmprivacy{1}
\pmauthor{akrowne}{2}
\pmtype{Definition}
\pmcomment{trigger rebuild}
\pmclassification{msc}{15-00}
\pmclassification{msc}{65-00}
\pmsynonym{tridiagonal}{TridiagonalMatrix}
\pmrelated{PentadiagonalMatrix}

\endmetadata

\usepackage{amssymb}
\usepackage{amsmath}
\usepackage{amsfonts}

%\usepackage{psfrag}
%\usepackage{graphicx}
%%%%\usepackage{xypic}
\begin{document}
An $n \times n$ \emph{tridiagonal} matrix is of the form

$$ \begin{bmatrix}
d_1 & u_1 & 0 & 0 & \cdots & 0 \\ 
l_1 & d_2 & u_2 & 0 & \cdots & 0 \\
 0  & l_2 & d_3 & u_3 & \cdots & 0 \\
\vdots & \vdots & \ddots & \ddots & \ddots & \vdots \\ 
 0  & 0 & \cdots &  l_{n-2} & d_{n-1} & u_{n-1} \\ 
 0  & 0 & \cdots & 0 & l_{n-1} & d_{n}
\end{bmatrix} $$
%%%%%
%%%%%
%%%%%
\end{document}
