\documentclass[12pt]{article}
\usepackage{pmmeta}
\pmcanonicalname{ProofOfBauerFikeTheorem}
\pmcreated{2013-03-22 15:33:08}
\pmmodified{2013-03-22 15:33:08}
\pmowner{Andrea Ambrosio}{7332}
\pmmodifier{Andrea Ambrosio}{7332}
\pmtitle{proof of Bauer-Fike theorem}
\pmrecord{9}{37450}
\pmprivacy{1}
\pmauthor{Andrea Ambrosio}{7332}
\pmtype{Proof}
\pmcomment{trigger rebuild}
\pmclassification{msc}{15A42}

% this is the default PlanetMath preamble.  as your knowledge
% of TeX increases, you will probably want to edit this, but
% it should be fine as is for beginners.

% almost certainly you want these
\usepackage{amssymb}
\usepackage{amsmath}
\usepackage{amsfonts}

% used for TeXing text within eps files
%\usepackage{psfrag}
% need this for including graphics (\includegraphics)
%\usepackage{graphicx}
% for neatly defining theorems and propositions
%\usepackage{amsthm}
% making logically defined graphics
%%%\usepackage{xypic}

% there are many more packages, add them here as you need them

% define commands here
\begin{document}
We can assume $\tilde{\lambda}\notin\sigma(A)$ (otherwise, we can choose $\lambda=\tilde{\lambda}$ and theorem is proven, since $\kappa_p(X)>1$).
Then $(A-\tilde{\lambda} I)^{-1}$ exists, so we can write:
\[
\tilde{u}=(A-\tilde{\lambda} I)^{-1}r=X(D-\tilde{\lambda} I)^{-1}X^{-1}r
\]
since $A$ is diagonalizable; taking the \PMlinkname{p-norm}{VectorPnorm} of both sides, we obtain:

\begin{eqnarray*}
1&=&\|\tilde{u}\|_p\\
&=& \|X(D-\tilde{\lambda} I)^{-1}X^{-1}r\|_p \leq \|X\|_p \|(D-\tilde{\lambda} I)^{-1}\|_p \|X^{-1}\|_p \|r\|_p\\
&=& \kappa_p(X)\|(D-\tilde{\lambda} I)^{-1}\|_p \|r\|_p.
\end{eqnarray*}

But, since $(D-\tilde{\lambda} I)^{-1}$ is a diagonal matrix, the p-norm is easily computed, and yields:
\[
\|(D-\tilde{\lambda} I)^{-1}\|_p=\max\limits_{\|x\|_p \ne 0}\frac{\|(D-\tilde{\lambda} I)^{-1}x\|_p}{\|x\|_p}=\max\limits_{\lambda\in\sigma(A)} \frac{1}{|\lambda-\tilde{\lambda}|}=\frac{1}{\min\limits_{\lambda\in\sigma(A)}|\lambda-\tilde{\lambda}|}
\]
whence:
\[
\min\limits_{\lambda\in\sigma(A)}|\lambda-\tilde{\lambda}|\leq\kappa_p(X)||r||_p.
\]
%%%%%
%%%%%
\end{document}
