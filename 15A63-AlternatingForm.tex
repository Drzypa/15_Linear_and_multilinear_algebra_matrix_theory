\documentclass[12pt]{article}
\usepackage{pmmeta}
\pmcanonicalname{AlternatingForm}
\pmcreated{2013-03-22 15:42:17}
\pmmodified{2013-03-22 15:42:17}
\pmowner{CWoo}{3771}
\pmmodifier{CWoo}{3771}
\pmtitle{alternating form}
\pmrecord{7}{37649}
\pmprivacy{1}
\pmauthor{CWoo}{3771}
\pmtype{Definition}
\pmcomment{trigger rebuild}
\pmclassification{msc}{15A63}
\pmsynonym{alternate form}{AlternatingForm}
\pmsynonym{alternating}{AlternatingForm}
\pmsynonym{symplectic hyperbolic plane}{AlternatingForm}
\pmrelated{SymplecticVectorSpace}
\pmrelated{EverySymplecticManifoldHasEvenDimension}
\pmdefines{alternating hyperbolic plane}

\endmetadata

\usepackage{amssymb,amscd}
\usepackage{amsmath}
\usepackage{amsfonts}

% used for TeXing text within eps files
%\usepackage{psfrag}
% need this for including graphics (\includegraphics)
%\usepackage{graphicx}
% for neatly defining theorems and propositions
%\usepackage{amsthm}
% making logically defined graphics
%%%\usepackage{xypic}

% define commands here
\begin{document}
A bilinear form $A$ on a vector space $V$ (over a field $k$) is called an \emph{alternating form} if for all $v\in V$, $A(v,v)=0$.

Since for any $u,v\in V$, $$0=A(u+v,u+v)=A(u,u)+A(u,v)+A(v,u)+A(v,v)=A(u,v)+A(v,u),$$ we see that $A(u,v)=-A(v,u)$.  So an alternating form is automatically a anti-symmetric, or skew symmetric form.  The converse is true if the characteristic of $k$ is not $2$.

Let $V$ be a two dimensional vector space over $k$ with an alternating form $A$. Let $\lbrace e_1,e_2\rbrace$ be a basis for $V$.  The matrix associated with $A$ looks like 

\begin{center}$
\begin{pmatrix}
A(e_1,e_1) & A(e_1,e_2) \\
A(e_2,e_1) & A(e_2,e_2)
\end{pmatrix}=r
\begin{pmatrix}
0 & 1 \\
-1 & 0
\end{pmatrix}=rS,
$\end{center}

where $r=A(e_1,e_2)$.  The skew symmetric matrix $S$ has the property that its diagonal entries are all $0$.  $S$ is called the $2\times 2$ \emph{alternating} or \emph{symplectic matrix}.   

$A$ is called \emph{non-singular} or \emph{non-degenerate} if there exist a vectors $u,v\in V$ such that $A(u,v)\neq 0$.  $u,v$ are necessarily non-zero.  Note that the associated matrix $rS$ is non-singular iff $r\neq 0$ iff $A$ is non-singular.  

In the two dimensional vector space case above, if $A$ is non-singular, we can re-scale the basis elements so that $r=1$.  This means that the matrix associated with $A$ is the alternating matrix.  A two-dimensional vector space which carries a non-singular alternating form is sometimes called an \emph{alternating} or \emph{symplectic hyperbolic plane}.  Some authors also call it simply a hyperbolic plane.  But here on PlanetMath, we will reserve the shorter name for its cousin in the category of quadratic spaces.  Let's denote an alternating hyperbolic plane by $\mathcal{A}$.

\textbf{Remark.}  In general, it can be shown that if $V$ is an $n$-dimensional vector space equipped with a non-singular alternating form $A$, then $V$ can be written as an orthogonal direct sum of the alternating hyperbolic planes $\mathcal{A}$.  In other words, the associated matrix for $A$ has the block form

\begin{center}$
\begin{pmatrix}
S & \boldsymbol{0} & \cdots & \boldsymbol{0} \\
\boldsymbol{0} & S & \cdots & \boldsymbol{0} \\
\vdots & \vdots & \ddots & \vdots \\
\boldsymbol{0} & \boldsymbol{0} & \cdots & S \\
\end{pmatrix},\mbox{ where }\boldsymbol{0}=
\begin{pmatrix}
0 & 0 \\ 0 & 0
\end{pmatrix}.
$\end{center}

Furthermore, $n$ is even.  $V$ is called a symplectic vector space.
%%%%%
%%%%%
\end{document}
