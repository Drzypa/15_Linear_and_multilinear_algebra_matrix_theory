\documentclass[12pt]{article}
\usepackage{pmmeta}
\pmcanonicalname{DeterminantInequalities}
\pmcreated{2013-03-22 15:34:46}
\pmmodified{2013-03-22 15:34:46}
\pmowner{Andrea Ambrosio}{7332}
\pmmodifier{Andrea Ambrosio}{7332}
\pmtitle{determinant inequalities}
\pmrecord{12}{37489}
\pmprivacy{1}
\pmauthor{Andrea Ambrosio}{7332}
\pmtype{Result}
\pmcomment{trigger rebuild}
\pmclassification{msc}{15A15}

\endmetadata

% this is the default PlanetMath preamble.  as your knowledge
% of TeX increases, you will probably want to edit this, but
% it should be fine as is for beginners.

% almost certainly you want these
\usepackage{amssymb}
\usepackage{amsmath}
\usepackage{amsfonts}

% used for TeXing text within eps files
%\usepackage{psfrag}
% need this for including graphics (\includegraphics)
%\usepackage{graphicx}
% for neatly defining theorems and propositions
%\usepackage{amsthm}
% making logically defined graphics
%%%\usepackage{xypic}

% there are many more packages, add them here as you need them

% define commands here
\begin{document}
There are a number of interesting inequalities bounding the determinant of a $n\times n$ complex matrix $A$, where $\rho$ is its spectral radius:

1) $\left |\det(A)\right|\leq\rho^n(A)$\\
2) $\left |\det(A)\right|\leq\prod_{i=1}^n\left (\sum_{j=1}^n\left |a_{ij}\right |\right )=\prod_{i=1}^n \|a_i\|_1$\\
3) $\left |\det(A)\right|\leq\prod_{j=1}^n\left (\sum_{i=1}^n\left |a_{ij}\right |\right )=\prod_{j=1}^n \|a_j\|_1$\\
4) $\left |\det(A)\right|\leq\prod_{i=1}^n\left (\sum_{j=1}^n\left |a_{ij}\right |^2\right )^{\frac{1}{2}}=\prod_{i=1}^n \|a_i\|_2$\\
5) $\left |\det(A)\right|\leq\prod_{j=1}^n\left (\sum_{i=1}^n\left |a_{ij}\right |^2\right )^{\frac{1}{2}}=\prod_{j=1}^n \|a_j\|_2$\\
6) if $A$ is Hermitian positive semidefinite, $\det(A)\leq\prod_{i=1}^n a_{ii}$, with equality if and only if $A$ is diagonal.

Inequalities 4)-6) are known as "Hadamard's inequalities".

(Note that inequalities 2)-5) may suggest the idea that such inequalities could hold: $\left |\det(A)\right|\leq\prod_{i=1}^n\|a_i\|_p$ or $\left |\det(A)\right|\leq\prod_{j=1}^n\|a_j\|_p$ for any $p\in\mathbf{N}$; however, this is not true, as one can easily see with $A=\begin{bmatrix} 1 & 1\\
-1 & 1\end{bmatrix}$ and $p=3$. Actually, inequalities 2)-5) give the best possible estimate of this kind.)

Proofs:

1) $\left|\det(A)\right|=\left|\prod_{i=1}^n \lambda_i\right|=\prod_{i=1}^n\left|\lambda_i\right|\leq\prod_{i=1}^n\rho(A)=\rho^n(A).$

2) If $A$ is singular, the thesis is trivial. Let then $\det(A)\ne 0$. Let's define $B=DA$, $D=diag(d_{11},d_{22},\cdots,d_{nn})$,$d_{ii}=\left(\sum_{j=1}^n \left|a_{ij}\right|\right)^{-1}$. (Note that $d_{ii}$ exist for any $i$, because $\det(A)\ne 0$ implies no all-zero row exists.) So $\|B\|_\infty=\max_i\left(\sum_{j=1}^n \left|b_{ij}\right|\right)=1$ and, since $\rho(B)\leq\|B\|_\infty$, we have:

$\left|\det(B)\right|=\left|\det(D)\right|\left|\det(A)\right|=\left(\prod_{i=1}^n\sum_{j=1}^n\left|a_{ij}\right|\right)^{-1}\left|\det(A)\right|\leq\rho^n(B)\leq\|B\|_\infty^n =1$,

from which:

$\left|\det(A)\right|\leq\prod_{i=1}^n\left(\sum_{j=1}^n\left|a_{ij}\right |\right).$ 

3) Same as 2), but applied to $A^T$.

4)-6) See related proofs attached to "Hadamard's inequalities".
%%%%%
%%%%%
\end{document}
