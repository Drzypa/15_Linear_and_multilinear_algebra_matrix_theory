\documentclass[12pt]{article}
\usepackage{pmmeta}
\pmcanonicalname{ThereExistAdditiveFunctionsWhichAreNotLinear}
\pmcreated{2013-03-22 16:17:50}
\pmmodified{2013-03-22 16:17:50}
\pmowner{paolini}{1187}
\pmmodifier{paolini}{1187}
\pmtitle{there exist additive functions which are not linear}
\pmrecord{5}{38417}
\pmprivacy{1}
\pmauthor{paolini}{1187}
\pmtype{Example}
\pmcomment{trigger rebuild}
\pmclassification{msc}{15A04}

% this is the default PlanetMath preamble.  as your knowledge
% of TeX increases, you will probably want to edit this, but
% it should be fine as is for beginners.

% almost certainly you want these
\usepackage{amssymb}
\usepackage{amsmath}
\usepackage{amsfonts}

% used for TeXing text within eps files
%\usepackage{psfrag}
% need this for including graphics (\includegraphics)
%\usepackage{graphicx}
% for neatly defining theorems and propositions
\usepackage{amsthm}
% making logically defined graphics
%%%\usepackage{xypic}

% there are many more packages, add them here as you need them

% define commands here
\newcommand{\R}{\mathbb R}
\newcommand{\Q}{\mathbb Q}
\newtheorem{theorem}{Theorem}
\newtheorem{definition}{Definition}
\theoremstyle{remark}
\newtheorem{example}{Example}
\begin{document}
\begin{example}
There exists a function $f\colon \R\to\R$ which is additive but not linear.
\end{example}
\begin{proof}
Let $V$ be the infinite dimensional vector space $\R$ over the 
field $\Q$. Since $1$ and $\sqrt 2$ are two independent vectors in $V$, we can extend the set $\{1,\sqrt 2\}$ to a basis $E$ of $V$ (notice that here the axiom of choice is used).

Now we consider a linear function $f\colon V \to \R$ such that $f(1)=1$ while $f(e)=0$ for all $e\in E\setminus\{1\}$. This function is $\Q$-linear (i.e.\ it is additive on $\R$) but it is not $\R$-linear because $f(\sqrt 2)=0\neq \sqrt 2 f(1)$.
\end{proof}
%%%%%
%%%%%
\end{document}
