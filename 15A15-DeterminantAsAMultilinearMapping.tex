\documentclass[12pt]{article}
\usepackage{pmmeta}
\pmcanonicalname{DeterminantAsAMultilinearMapping}
\pmcreated{2013-03-22 13:09:17}
\pmmodified{2013-03-22 13:09:17}
\pmowner{rmilson}{146}
\pmmodifier{rmilson}{146}
\pmtitle{determinant as a multilinear mapping}
\pmrecord{5}{33595}
\pmprivacy{1}
\pmauthor{rmilson}{146}
\pmtype{Theorem}
\pmcomment{trigger rebuild}
\pmclassification{msc}{15A15}
\pmrelated{ExteriorAlgebra}

\usepackage{amsmath}
\usepackage{amsfonts}
\usepackage{amssymb}
\newcommand{\reals}{\mathbb{R}}
\newcommand{\natnums}{\mathbb{N}}
\newcommand{\cnums}{\mathbb{C}}
\newcommand{\znums}{\mathbb{Z}}
\newcommand{\lp}{\left(}
\newcommand{\rp}{\right)}
\newcommand{\lb}{\left[}
\newcommand{\rb}{\right]}
\newcommand{\supth}{^{\text{th}}}
\newtheorem{proposition}{Proposition}
\newtheorem{definition}[proposition]{Definition}

\newtheorem{theorem}[proposition]{Theorem}
\newcommand{\mM}{\mathbf{M}}
\newcommand{\be}{\mathbf{e}}
\begin{document}
Let $\mM = (M_{ij})$ be an $n\times n$ matrix with entries in a field
$K$. The matrix $\mM$ is really the same thing as a list of $n$ column
vectors of size $n$. Consequently, the determinant operation may be
regarded as a mapping
$$\det:\overbrace{K^n\times\ldots\times K^n}^{n \mbox{
    times}}\rightarrow K$$
The determinant of a matrix $\mM$ is then defined to be 
$\det(\mM_1,\ldots,\mM_n),$
where $\mM_j\in K^n$ denotes the $j\supth$ column of $\mM$.

Starting with the definition
\begin{equation}
\label{eq:detdef}
\det(\mM_1,\ldots,\mM_n) = \sum_{\pi\in S_n} \mathrm{sgn}(\pi) M_{1\pi_1}
M_{2\pi_2}\cdots  
M_{n\pi_n}
\end{equation}
the following properties are easily established:
\begin{enumerate}
\item the determinant is multilinear;
\item the determinant is anti-symmetric;
\item the determinant of the identity matrix is $1$.
\end{enumerate}
These three properties uniquely characterize the determinant, and
indeed can --- some would say should --- be used as the definition of
the determinant operation.

Let us prove this.  We  proceed by representing elements of $K^n$
as linear combinations of
$$
\be_1=\begin{pmatrix} 1\\0\\0\\\vdots\\0\end{pmatrix},\quad
\be_2=\begin{pmatrix} 0\\1\\0\\\vdots\\0\end{pmatrix},\quad \ldots\quad
\be_n=\begin{pmatrix} 0\\0\\0\\ \vdots\\1\end{pmatrix},
$$
the standard basis of $K^n$. Let $\mM$ be an $n\times n$ matrix.
The $j\supth$ column is represented as $\sum_i M_{ij}\be_i$; whence
using multilinearity
\begin{align*}
\det(\mM) &= \det\lp \sum_i M_{i1}\be_i\,,\sum_i
M_{i2}\be_i\,,\;\ldots\;,\sum_i M_{in} \be_i\rp\\  
&=\sum_{i_1,\ldots,i_n=1}^n M_{i_11} M_{i_22} \cdots M_{i_n n} 
\det(\be_{i_1},\be_{i_2},\ldots,\be_{i_n})
\end{align*}
The anti-symmetry assumption implies that the expressions
$\det(\be_{i_1},\be_{i_2},\ldots,\be_{i_n})$ vanish if any two of the
indices $i_1,\ldots,i_n$ coincide.  If all $n$ indices are distinct,
$$\det(\be_{i_1},\be_{i_2},\ldots,\be_{i_n}) = \pm
\det(\be_1,\ldots,\be_n),$$
the sign in the above expression being
determined by the  number of transpositions required to rearrange
the list $(i_1,\ldots,i_n)$ into the list $(1,\ldots,n)$. The sign is
therefore the parity of the permutation $(i_1,\ldots,i_n)$.  Since we
also assume that
$$\det(\be_1,\ldots,\be_n)=1,$$ we now recover the original definition 
\eqref{eq:detdef}.
%%%%%
%%%%%
\end{document}
