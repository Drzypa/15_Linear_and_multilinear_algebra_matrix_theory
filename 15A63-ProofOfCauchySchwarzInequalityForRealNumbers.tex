\documentclass[12pt]{article}
\usepackage{pmmeta}
\pmcanonicalname{ProofOfCauchySchwarzInequalityForRealNumbers}
\pmcreated{2013-03-22 14:56:38}
\pmmodified{2013-03-22 14:56:38}
\pmowner{stitch}{17269}
\pmmodifier{stitch}{17269}
\pmtitle{proof of Cauchy-Schwarz inequality for real numbers}
\pmrecord{5}{36635}
\pmprivacy{1}
\pmauthor{stitch}{17269}
\pmtype{Proof}
\pmcomment{trigger rebuild}
\pmclassification{msc}{15A63}

\endmetadata

% this is the default PlanetMath preamble.  as your knowledge
% of TeX increases, you will probably want to edit this, but
% it should be fine as is for beginners.

% almost certainly you want these
\usepackage{amssymb}
\usepackage{amsmath}
\usepackage{amsfonts}

% used for TeXing text within eps files
%\usepackage{psfrag}
% need this for including graphics (\includegraphics)
%\usepackage{graphicx}
% for neatly defining theorems and propositions
%\usepackage{amsthm}
% making logically defined graphics
%%%\usepackage{xypic}

% there are many more packages, add them here as you need them

% define commands here
\def\sse{\subseteq}
\def\bigtimes{\mathop{\mbox{\Huge $\times$}}}
\def\impl{\Rightarrow}
\begin{document}
The version of the Cauchy-Schwartz inequality we want to prove is
\[
  \left(\sum_{k=1}^n a_k b_k \right)^2 \le
    \sum_{k=1}^n a_k^2 \cdot \sum_{k=1}^n b_k^2,
\]
where the $a_k$ and $b_k$ are real numbers, with equality holding only in the
case of proportionality, $a_k=\lambda b_k$ for some real $\lambda$ for all $k$.

The proof is by direct calculation:
\begin{align*}
	\sum_{k=1}^n a_k^2 \cdot \sum_{k=1}^n b_k^2 -
		\left(\sum_{k=1}^n a_k b_k \right)^2
	&= \sum_{k,l=1}^n a_k^2 b_l^2 - a_k b_k a_l b_l \\
	&= \sum_{k,l=1}^n \frac{1}{2}(a_k^2 b_l^2 + a_l^2 b_k^2) - (a_k b_l)(a_l b_k) \\
	&= \frac{1}{2}\sum_{k,l=1}^n (a_k b_l)^2 - 2(a_k b_l)(a_l b_k) + (a_l b_k)^2 \\
	&= \frac{1}{2}\sum_{k,l=1}^n (a_k b_l - a_l b_k)^2 \\
	&\ge 0 .
\end{align*}
The above identity implies that the Cauchy-Schwarz inequality holds.
Moreover, it is an equality only when
\[
  a_k b_l - a_l b_k = 0 \quad \Longleftrightarrow \quad
  \frac{a_k}{b_k} = \frac{a_l}{b_l}  \text{ or }
  \frac{b_k}{a_k} = \frac{b_l}{a_l}  \text{ or }
  a_k = b_k = 0,
\]
for all $k$ and $l$. In other words, equality holds only when $a_k = \lambda b_k$ for all $k$ for some real number $\lambda$.
%%%%%
%%%%%
\end{document}
