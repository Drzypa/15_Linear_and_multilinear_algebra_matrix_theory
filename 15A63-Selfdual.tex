\documentclass[12pt]{article}
\usepackage{pmmeta}
\pmcanonicalname{Selfdual}
\pmcreated{2013-03-22 12:29:40}
\pmmodified{2013-03-22 12:29:40}
\pmowner{rmilson}{146}
\pmmodifier{rmilson}{146}
\pmtitle{self-dual}
\pmrecord{5}{32719}
\pmprivacy{1}
\pmauthor{rmilson}{146}
\pmtype{Definition}
\pmcomment{trigger rebuild}
\pmclassification{msc}{15A63}
\pmclassification{msc}{15A57}
\pmclassification{msc}{15A04}
\pmsynonym{self-adjoint}{Selfdual}
\pmrelated{HermitianMatrix}
\pmrelated{SymmetricMatrix}
\pmrelated{SkewSymmetricMatrix}
\pmdefines{anti self-dual}

\endmetadata

\usepackage{amsmath}
\usepackage{amsfonts}
\usepackage{amssymb}

\newcommand{\Hom}{\mathop{\mathrm{Hom}}\nolimits}
\newcommand{\Mat}{\mathop{\mathrm{Mat}}\nolimits}
\newcommand{\kfield}{\mathbb{K}}
\newcommand{\supt}{^t}
\newcommand{\dual}{^*}
\newcommand{\adj}{^{\displaystyle \star}}

\newcommand{\reals}{\mathbb{R}}
\newcommand{\natnums}{\mathbb{N}}
\newcommand{\cnums}{\mathbb{C}}

\newcommand{\lp}{\left(}
\newcommand{\rp}{\right)}
\newcommand{\lb}{\left[}
\newcommand{\rb}{\right]}

\newcommand{\supth}{^{\text{th}}}


\newtheorem{proposition}{Proposition}
\begin{document}
\paragraph{Definition.}  Let $U$ be a finite-dimensional inner-product space
over a field $\kfield$.  Let $T:U\rightarrow U$ be an endomorphism,
and note that the adjoint endomorphism $T\adj$ is also an endomorphism
of $U$.  It is therefore possible to add, subtract, and compare $T$
and $T\adj$, and we are able to make the following definitions. An
endomorphism $T$ is said to be {\em self-dual} (a.k.a.  {\em
  self-adjoint}) if
$$T=T\adj.$$
By contrast, we say that the endomorphism is {\em anti self-dual} if 
$$T=-T\adj.$$

Exactly the same definitions can be made for an endomorphism of
a complex vector space with a Hermitian inner product.

\paragraph{Relation to the matrix transpose.} All of these definitions have
their counterparts in the matrix setting.  Let $M\in
\Mat_{n,n}(\kfield)$ be the matrix of $T$ relative to an orthogonal
basis of $U$. Then $T$ is self-dual if and only if $M$ is a symmetric matrix,
and anti self-dual if and only if $M$ is a skew-symmetric matrix.

In the case of a Hermitian inner product we must replace the transpose
with the conjugate transpose. Thus $T$ is self dual if and only if $M$ is a Hermitian matrix, i.e.
$$M = \overline{M^t}.$$
It is anti self-dual if and only if
$$M = -\overline{M^t}.$$
%%%%%
%%%%%
\end{document}
