\documentclass[12pt]{article}
\usepackage{pmmeta}
\pmcanonicalname{HomogeneousFunction}
\pmcreated{2013-03-22 14:44:37}
\pmmodified{2013-03-22 14:44:37}
\pmowner{matte}{1858}
\pmmodifier{matte}{1858}
\pmtitle{homogeneous function}
\pmrecord{8}{36381}
\pmprivacy{1}
\pmauthor{matte}{1858}
\pmtype{Definition}
\pmcomment{trigger rebuild}
\pmclassification{msc}{15-00}
\pmsynonym{positively homogeneous function of degree}{HomogeneousFunction}
\pmsynonym{homogeneous function of degree}{HomogeneousFunction}
\pmsynonym{positively homogeneous function}{HomogeneousFunction}
\pmrelated{HomogeneousPolynomial}
\pmrelated{SubLinear}

% this is the default PlanetMath preamble.  as your knowledge
% of TeX increases, you will probably want to edit this, but
% it should be fine as is for beginners.

% almost certainly you want these
\usepackage{amssymb}
\usepackage{amsmath}
\usepackage{amsfonts}
\usepackage{amsthm}

\usepackage{mathrsfs}

% used for TeXing text within eps files
%\usepackage{psfrag}
% need this for including graphics (\includegraphics)
%\usepackage{graphicx}
% for neatly defining theorems and propositions
%
% making logically defined graphics
%%%\usepackage{xypic}

% there are many more packages, add them here as you need them

% define commands here

\newcommand{\sR}[0]{\mathbb{R}}
\newcommand{\sC}[0]{\mathbb{C}}
\newcommand{\sN}[0]{\mathbb{N}}
\newcommand{\sZ}[0]{\mathbb{Z}}

 \usepackage{bbm}
 \newcommand{\Z}{\mathbbmss{Z}}
 \newcommand{\C}{\mathbbmss{C}}
 \newcommand{\R}{\mathbbmss{R}}
 \newcommand{\Q}{\mathbbmss{Q}}



\newcommand*{\norm}[1]{\lVert #1 \rVert}
\newcommand*{\abs}[1]{| #1 |}



\newtheorem{thm}{Theorem}
\newtheorem{defn}{Definition}
\newtheorem{prop}{Proposition}
\newtheorem{lemma}{Lemma}
\newtheorem{cor}{Corollary}
\begin{document}
\begin{defn} Suppose $V,\,W$ are a vector spaces over $\R$, 
and $f\colon V \to W$ is a mapping. 
\begin{itemize}
\item 
   If there exists an $r \in \R$, such that
   $$
     f(\lambda v) = \lambda^r f(v)
   $$
   for all $\lambda \in \R$ and $v\in V$, then $f$ is 
   a \PMlinkescapetext{\emph{homogeneous function of degree $r$}}.
\item 
   If there exists an $r\in \R$, such that
   $$
     f(\lambda v) = |\lambda|^r f(v)
   $$
   for all $\lambda \in \R$ and $v\in V$, 
   then $f$ is 
   \PMlinkescapetext{\emph{absolutely homogeneous function of degree $r$}}.
\item 
   If there exists an $r\in \R$, such that
   $$
     f(\lambda v) = \lambda^r f(v)
   $$
   for all $\lambda \ge 0$ and $v\in V$, then $f$ is 
   a \PMlinkescapetext{\emph{positively homogeneous function of degree $r$}}.
\end{itemize}
\end{defn}

\subsubsection*{Notes}
For any homogeneous function as above, $f(0)=0$.

When the \PMlinkescapetext{type} of homegeneity is clear one simply talks about
$r$-homogeneous functions.
%%%%%
%%%%%
\end{document}
