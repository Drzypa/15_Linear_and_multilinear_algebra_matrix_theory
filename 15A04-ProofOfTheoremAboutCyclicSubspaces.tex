\documentclass[12pt]{article}
\usepackage{pmmeta}
\pmcanonicalname{ProofOfTheoremAboutCyclicSubspaces}
\pmcreated{2013-03-22 17:32:53}
\pmmodified{2013-03-22 17:32:53}
\pmowner{FunctorSalad}{18100}
\pmmodifier{FunctorSalad}{18100}
\pmtitle{proof of theorem about cyclic subspaces}
\pmrecord{7}{39952}
\pmprivacy{1}
\pmauthor{FunctorSalad}{18100}
\pmtype{Proof}
\pmcomment{trigger rebuild}
\pmclassification{msc}{15A04}

\endmetadata

% this is the default PlanetMath preamble.  as your knowledge
% of TeX increases, you will probably want to edit this, but
% it should be fine as is for beginners.

% almost certainly you want these
\usepackage{amssymb}
\usepackage{amsmath}
\usepackage{amsfonts}

% used for TeXing text within eps files
%\usepackage{psfrag}
% need this for including graphics (\includegraphics)
%\usepackage{graphicx}
% for neatly defining theorems and propositions
%\usepackage{amsthm}
% making logically defined graphics
%%%\usepackage{xypic}

% there are many more packages, add them here as you need them

% define commands here

\begin{document}
We first prove the case $r=2$. The $\subseteq$ inclusion is clear, since the right side is a $T$-invariant subspace that contains $v_1+v_2$. 

For the other inclusion, it is sufficient to show that $v_1,v_2 \in Z(v_1+v_2,T)$. The idea is that the action of $T$ on $v_1 + v_2$ can "isolate" the two summands if their annihilator polynomials are coprime. Let's write $m_i$ for $m_{v_i}$.

Since $(m_1,m_2)=1$, there exist polynomials $p$ and $q$ such that 
\begin{equation}pm_1+qm_2=1\end{equation}
this is Bézout's lemma (or the Euclidean algorithm, or the fact that $k[X]$ is a principal ideal domain).


Now $pm_1(T)$ is the projection from $Z(v_1,T) \oplus Z(v_2,T)$ to $Z(v_2,T)$:
\begin{equation}(pm_1)(T)v_1=p(T)m_1(T) v_1 = p(T)0 = 0\end{equation}
(by assumption that $m_1$ is the annihilator polynomial of $v_1$) and 
\begin{equation}(pm_1)(T) = 1-(qm_2)(T)\end{equation} 
(by choice of $p$ and $q$), so
\begin{equation}(pm_1)(T) v_2 = v_2 - q(T)m_2(T) v_2 = v_2 - q(T)0 = v_2\end{equation} 

Any subspace that is invariant under $T$ is also invariant under polynomials of $T$. Therefore, the preceding equations show that $v_2 = (pm_1)(T)(v_1+v_2) \in Z(v_1+v_2,T)$. By symmetry, we also get that $v_1 \in Z(v_1+v_2,T)$.

For the last claim, we note that the annihilator polynomial $m$ of $Z(v_1,T) \oplus Z(v_2,T)$ is the least common multiple of $m_1$ and $m_2$ (that $m$ is a multiple of $m_1$ follows from the fact that $m$ must annihilate $v_1$, and the set of polynomials that annihilate $v_1$ is the ideal generated by $m_1$). Since $m_1$ and $m_2$ are coprime, the lcm is just their product.

That concludes the proof for $r=2$. If $r$ is arbitrary, we can simply apply the $r=2$ case inductively. We only have to check that the coprimality condition is preserved under applying the $r=2$ case to $i=1,2$. But it is well-known that if $p,q,r$ (in $k[X]$ or in any principal ideal domain) are pairwise coprime, then $pq$ and $r$ are also coprime.

%%%%%
%%%%%
\end{document}
