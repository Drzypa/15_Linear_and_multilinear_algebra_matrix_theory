\documentclass[12pt]{article}
\usepackage{pmmeta}
\pmcanonicalname{DecompositionOfOrthogonalOperatorsAsRotationsAndReflections}
\pmcreated{2013-03-22 15:24:11}
\pmmodified{2013-03-22 15:24:11}
\pmowner{stevecheng}{10074}
\pmmodifier{stevecheng}{10074}
\pmtitle{decomposition of orthogonal operators as rotations and reflections}
\pmrecord{15}{37242}
\pmprivacy{1}
\pmauthor{stevecheng}{10074}
\pmtype{Theorem}
\pmcomment{trigger rebuild}
\pmclassification{msc}{15A04}
%\pmkeywords{rotation}
%\pmkeywords{reflection}
%\pmkeywords{orthogonal}
\pmrelated{RotationMatrix}
\pmrelated{OrthogonalMatrices}
\pmrelated{DimensionOfTheSpecialOrthogonalGroup}
\pmrelated{RodriguesRotationFormula}
\pmrelated{DerivationOfRotationMatrixUsingPolarCoordinates}
\pmrelated{DerivationOf2DReflectionMatrix}

\endmetadata

% this is the default PlanetMath preamble.  as your knowledge
% of TeX increases, you will probably want to edit this, but
% it should be fine as is for beginners.

% almost certainly you want these
\usepackage{amssymb}
\usepackage{amsmath}
\usepackage{amsfonts}
\usepackage{amsthm}

% used for TeXing text within eps files
%\usepackage{psfrag}
% need this for including graphics (\includegraphics)
%\usepackage{graphicx}
% for neatly defining theorems and propositions
%\usepackage{amsthm}
% making logically defined graphics
%%%\usepackage{xypic}

% there are many more packages, add them here as you need them
\usepackage{enumerate}

% define commands here
\newcommand{\real}{\mathbb{R}}
\newcommand{\rat}{\mathbb{Q}}
\newcommand{\nat}{\mathbb{N}}


\newcommand{\Vc}{V^\mathbb{C}}
\newcommand{\Tc}{T^\mathbb{C}}

\providecommand{\abs}[1]{\lvert#1\rvert}
\providecommand{\absW}[1]{\left\lvert#1\right\rvert}
\providecommand{\absB}[1]{\Bigl\lvert#1\Bigr\rvert}
\providecommand{\norm}[1]{\lVert#1\rVert}
\providecommand{\normW}[1]{\left\lVert#1\right\rVert}
\providecommand{\normB}[1]{\Bigl\lVert#1\Bigr\rVert}
\providecommand{\defnterm}[1]{\emph{#1}}

\DeclareMathOperator{\linspan}{span}

\newtheorem{thm}{Theorem}
\newtheorem{lem}{Lemma}
\newtheorem{defn}{Definition}
\begin{document}
\begin{thm}
Let $V$ be a $n$-dimensional real inner product space ($0 < n < \infty$). Then every orthogonal operator $T$ on $V$
can be decomposed into a series of two-dimensional rotations and one-dimensional reflections
on mutually orthogonal subspaces of $V$.
\end{thm}

We first explain the general idea behind the proof.  Consider a rotation $R$ of angle $\theta$ in a two-dimensional space.  From its orthonormal basis representation
\[
\begin{bmatrix}
\cos \theta & -\sin \theta \\
\sin \theta & \; \cos \theta
\end{bmatrix}\,,
\]
we find that the characteristic polynomial of $R$ is $t^2 - 2 \cos \theta + 1$,
with (complex) roots $e^{\pm i\theta}$.  
(Not surprising, since multiplication by $e^{i\theta}$ in the complex plane is rotation
by $\theta$.)
Thus given the characteristic polynomial of $R$,
we can almost recover\footnote{Because the complex roots occur in conjugate pairs,
the information about the sign of $\theta$ is lost. This too, is not a surprise,
because the sign of $\theta$, i.e. whether the rotation is clockwise or counterclockwise,
is dependent on the orientation of the basis vectors.} its rotation angle.  

In the case of a reflection $S$, the eigenvalues of $S$ are $-1$ and $1$; so again, the characteristic polynomial of $S$ will provide some information on $S$.

So in $n$ dimensions, we are also going to look at the complex eigenvalues and eigenvectors of $T$ to recover information about the rotations represented by $T$.

But there is one technical point --- $T$ is a transformation on a real vector
space, so it does not really have ``complex eigenvalues and eigenvectors''.
To make this concept rigorous, we must consider the complexification $\Tc$ of $T$,
the operator defined by $\Tc(x+iy) = Tx + iTy$
in the vector space $\Vc$ consisting of elements of the form $x+iy$, for $x, y \in V$.
(For more details, see the entry on \PMlinkname{complexification}{ComplexificationOfVectorSpace}.)

\bigskip

\begin{lem}
For any linear operator $T\colon V \to V$, there exists a one- or two-dimensional subspace $W$ which is invariant under $T$.
\begin{proof}
Consider $\Tc$ and its characteristic polynomial.
By the Fundamental Theorem of Algebra, the characteristic polynomial has a complex root $\lambda = \alpha + i \beta$. Then there is an eigenvector $x+iy \neq 0$ with eigenvalue $\lambda$.  We have
\[
Tx + iTy = \Tc(x+iy) = \lambda(x + iy) = (\alpha x - \beta y) + i(\beta x + \alpha y)\,.
\]
Equating real and imaginary components,
we see that
\begin{align*}
Tx &= \alpha x - \beta y \in W \\
Ty &= \beta x + \alpha y \in W
\end{align*}
where $W = \linspan \{ x, y\}$. $W$ is two-dimensional if $x$ and $y$ are linearly independent; otherwise it is one-dimensional\footnote{
In fact, the space $W$ constructed is two-dimensional if and only if the eigenvalue $\lambda$ is not purely real.  Compare with the remark (i) after
the proof of Theorem 1.

Actually, Lemma 1 has more uses than just proving Theorem 1.
For example, if $\dot{x} = Ax$ is a linear differential equation,
and the constant coefficient matrix $A$ has only simple eigenvalues,
then it is a consequence of Lemma 1, that the differential equation decomposes into a series of disjoint one-variable and two-variable equations.
The solutions are then readily understood: they are always of the form of an
exponential multiplied by a sinusoid, and linear combinations thereof.
The sinusoids will be present whenever there are non-real eigenvalues.
}. And we have $T(W) \subseteq W$ as claimed.
\end{proof}
\end{lem}

\bigskip

\begin{proof}[Proof of Theorem 1]
We recursively factor $T$; formally, the proof will be by induction on the dimension $n$.

The case $n=1$ is trivial.  We have $\det T = \pm 1$; if $\det T = 1$ then $T$ is the identity; otherwise $Tx = -x$ is a reflection.

For larger $n$, by Lemma 1 there exists a $T$-invariant subspace $W$.  The orthogonal complement $W^\perp$ of $W$ is $T$-invariant also, because
for all $x \in W^\perp$ and $y \in W$, 
\begin{align*}
\langle Tx, y \rangle &= \langle x, T^{-1} y \rangle & \textrm{because $T$ preserves inner product} \\
&= 0 & \textrm{because $T^{-1}(W) = W$.}
\end{align*}
Let $T_W$ be the operator that acts as $T$ on $W$ and is the identity on $W^\perp$.
Similarly, let $T_{W^\perp}$ be the operator that acts as $T$ on $W^{\perp}$
and is the identity on $W$.  Then $T = T_W \circ T_{W^\perp}$.
$T_W$ restricted to $W$ is orthogonal, and since $W$ is one- or two-dimensional,
$T_W$ must therefore be a rotation or reflection (or the identity) in a line or plane.

$T_{W^\perp}$ restricted to $W^\perp$ is also orthogonal. $W^\perp$ has dimension $<n$,
so by the induction hypothesis
we can continue to factor it into operators acting on subspaces of $W^\perp$ that are mutually orthogonal. These subspaces will of course also be orthogonal to $W$.

The proof is now complete, except that we did not rule out $T_W$ being a reflection even
when $W$ is two-dimensional.
But of course, if $T_W$ is a reflection in two dimensions,
then it can be factored as a reflection on a one-dimensional subspace $\linspan \{v\}$ 
composed with the identity on $\linspan \{ v \}^\perp$.
\end{proof}

Actually, in the third paragraph of the proof, we implicitly assumed that every orthogonal operator on two-dimensional is either a rotation or reflection.
This is a well-known fact, but it does need to be formally proven:

\begin{thm}
If $V$ is a two-dimensional real inner product space, then every orthogonal operator $T$
is either a rotation or reflection.
\begin{proof}
Fix any orthonormal basis $\{ e_1, e_2 \}$ for $V$.
Since $T$ is orthogonal, $\norm{Te_1} = \norm{e_1} = 1$, i.e. $Te_1$ is a unit vector on the plane, so there exists an angle $\theta$ (unique modulo $2\pi$) 
such that $Te_1 = \cos \theta \, e_1 + \sin \theta \, e_2$.
Similarly $Te_2$ is a unit vector, but since $e_1$ and $e_2$ are orthogonal, so
are $Te_1$ and $Te_2$.  Then it is found that the solution to $Te_2$ must be either $-\sin \theta \,  e_1 + \cos \theta \, e_2$ or $\sin \theta \, e_1 - \cos \theta \, e_2$.  Putting all this together, the matrix for $T$ is:
\[
\begin{bmatrix}
\cos \theta & \mp\sin \theta \\
\sin \theta & \pm \cos \theta
\end{bmatrix}\,.
\]
The first solution for $Te_2$ corresponds to a rotation matrix (and $\det T = 1$);
the second solution for $Te_2$ corresponds to a 
\PMlinkname{reflection matrix}{DerivationOf2DReflectionMatrix} (and $\det T = -1$).
\end{proof}
\end{thm}

\bigskip
\section{Remarks}
\begin{enumerate}[i.]
\item
Observe that the two equations for $Tx$ and $Ty$ appearing in the proof of Lemma 1
do specify a rotation of angle $\pm \theta$ when $\lambda = \alpha + i\beta = e^{i\theta}$
and $x, y$ are orthonormal.
So by examining the complex eigenvales and eigenvectors of $\Tc$,
we can reconstruct the rotation.

This construction can be used to give an alternate proof of Theorem 1; we sketch it below:

The complexified space $\Vc$ has an inner product structure inherited from $V$ (again, see 
\PMlinkname{complexification}{ComplexificationOfVectorSpace} for details).
Let $U = \Tc$.  Since $T$ is orthogonal, $U$ is unitary, and hence normal ($U^* U = UU^*$).
There exists an orthonormal basis of eigenvectors for $U$ (the \PMlinkname{Schur decomposition}{CorollaryOfSchurDecomposition}).
Let $z = x+iy$ is any one of these eigenvectors with a complex, non-real eigenvalue
$\lambda$.  Then $\abs{\lambda} = 1$ because $U$ is unitary,
and $\overline{z} = x-iy$ is another eigenvector with eigenvalue $\overline{\lambda}$.
Using the $\Vc$ inner product formula, the vectors $x/\sqrt{2}$ and $y/\sqrt{2}$ can be shown to be orthonormal.
Then the proof of Lemma 1 shows that $T$ acts as a rotation in the plane $\linspan \{ x, y \}$.
All such planes obtained will be orthogonal to each other.

To summarize, the orthogonal subplanes of rotation are found by grouping conjugate pairs
of complex eigenvectors. If one actually needs to determine the planes of rotation explicitly (for dimensions $n \geq 4$), then probably it is better to work directly with the complexified matrix,
rather than to factor the matrix over the reals.

\item
The decomposition of $T$ is not unique.  However, it is always possible to obtain
a decomposition which contains at most one reflection,
because any two single-dimensional reflections can always be combined into
a two-dimensional rotation.
In any case, the parity of the number of reflections in a decomposition of $T$ is invariant,
because the parity is equal to $\det T$.

\item
In the decomposition, the component rotations and reflections all commute
because they act on orthogonal subspaces.

\item
If we take a basis for $V$ describing the mutually orthogonal subspaces in Theorem 1,
the matrix for $T$ looks like:
\[
\begin{bmatrix}
\cos \theta_1 & -\sin \theta_1 \\
\sin \theta_1 & \; \cos \theta_1 \\
& & \ddots \\
& & & \cos \theta_k & -\sin \theta_k \\
& & & \sin \theta_k & \; \cos \theta_k \\
& & & & & \pm 1 \\
& & & & & & +1 \\
& & & & & & & \ddots \\
& & & & & & & & +1
\end{bmatrix}
\]
where $\theta_1, \dotsc, \theta_k$ are the rotation angles, one for each orthogonal subplane,
and the $\pm 1$ in the middle is the reflective component (if present).  The rest of the entries
in the matrix are zero.

\item
Sometimes any orthogonal operator $T$ with $\det T = 1$
is called a rotation, even though strictly speaking it is actually series of rotations
(each on different ``axes'').  Similarly, when $\det T = -1$,
$T$ may be called a reflection, even though again it is not always a single (one-dimensional)
reflection.

In this language, a rotation composed with a rotation will always be a rotation;
a rotation composed with a reflection is a reflection; and two reflections composed together will always be a rotation.

\item
In $\real^3$, an orthogonal operator with positive determinant is necessarily
a rotation on one axis which is left fixed (except when the operator is the identity).
This follows simply because there is no way to fit more than one orthogonal subplane
into three-dimensional space.

A composition of two rotations in $\real^3$ would then
be a rotation too.
On the other hand it is not at all obvious what relation the axis of rotation
of the composition has with the original two axes of rotation.

For an explicit formula for a rotation matrix in $\real^3$ that does not
require manual calculation of the basis vectors for the rotation subplane, see Rodrigues' rotation formula.

\item
In $\real^n$, reflections can be carried out by first embedding $\real^n$ into $\real^{n+1}$
and then rotating $\real^{n+1}$.  
(Here, the words ``rotation'' and ``reflection'' being taken in their extended sense of (v).)
For example, in the plane, a right hand can be
rotated in $\real^3$ into a left hand. 

To be specific, suppose we embed $\real^n$ in $\real^{n+1}$ as the first $n$ coordinates.
Then we gain an extra degree of freedom in the last coordinate of $\real^{n+1}$
(with coordinate vector $e_{n+1}$).
Given an orthogonal operator $T\colon \real^n \to \real^n$,
we can extend it to an operator on $T'\colon \real^{n+1} \to \real^{n+1}$
by having it act as $T$ on the lower $n$ coordinates,
and setting $T'(e_{n+1}) = -e_{n+1}$.
Since $\det T' = -\det T = 1$,
our new $T'$ will be a rotation (the extra angle of rotation will be by $\pi$)
that reflects sets in the $\real^n$ plane.

\end{enumerate}

\begin{thebibliography}{3}
\bibitem{Friedberg} Friedberg, Insel, Spence. {\it Linear Algebra}. Prentice-Hall, 1997.
\bibitem{Arnold} Vladimir I. Arnol'd (trans. Roger Cooke). {\it Ordinary Differential Equations}. Springer-Verlag, 1992.
\end{thebibliography}
%%%%%
%%%%%
\end{document}
