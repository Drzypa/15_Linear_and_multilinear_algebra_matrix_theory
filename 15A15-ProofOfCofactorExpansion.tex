\documentclass[12pt]{article}
\usepackage{pmmeta}
\pmcanonicalname{ProofOfCofactorExpansion}
\pmcreated{2013-03-22 13:22:08}
\pmmodified{2013-03-22 13:22:08}
\pmowner{Thomas Heye}{1234}
\pmmodifier{Thomas Heye}{1234}
\pmtitle{proof of cofactor expansion}
\pmrecord{13}{33896}
\pmprivacy{1}
\pmauthor{Thomas Heye}{1234}
\pmtype{Proof}
\pmcomment{trigger rebuild}
\pmclassification{msc}{15A15}
\pmsynonym{Laplace expansion}{ProofOfCofactorExpansion}

% this is the default PlanetMath preamble.  as your knowledge
% of TeX increases, you will probably want to edit this, but
% it should be fine as is for beginners.

% almost certainly you want these
\usepackage{amssymb}
\usepackage{amsmath}
\usepackage{amsfonts}

% used for TeXing text within eps files
%\usepackage{psfrag}
% need this for including graphics (\includegraphics)
%\usepackage{graphicx}
% for neatly defining theorems and propositions
\usepackage{amsthm}
% making logically defined graphics
%%%\usepackage{xypic}

% there are many more packages, add them here as you need them

% define commands here
\begin{document}
Let $M \in mat_N(K)$ be a $n \times n$-matrix with entries from a commutative
field $K$. Let $e_1, \ldots, e_n$ denote the vectors of the canonical basis of
$K^n$. For the proof we need the following

\textbf{Lemma:} Let $M_{ij}^*$ be the matrix generated by replacing the $i$-th
row of $M$ by $e_j$. Then
\[\det{M_{ij}^*} =(-1)^{i+j}\det{M_{ij}}\]
where $M_{ij}$ is the $(n -1) \times (n-1)$-matrix obtained from $M$ by removing
its $i$-th row and $j$-th column.

\begin{proof}
By adding appropriate \PMlinkescapeword{multiples} of the $i$-th row of $M_{ij}^*$
to its remaining rows we obtain a matrix with 1 at position $(i,j)$ and 0 at
positions $(k,j)$ ($k \neq i$). Now we apply the permutation
\[(1 2) \circ (2 3) \circ\dots \circ ((i -1) i)\]
to rows and
\[(1 2) \circ (2 3) \circ\dots \circ ((j-1)j)\]
to columns of the matrix. The matrix now looks like this:
\begin{itemize}
\item
Row/column 1 is the vector $e_1$;
\item
under row 1 and right of column 1 is the matrix $M_{ij}$.
\end{itemize}
Since the determinant has changed its sign $i+j-2$ times, we have
\[\det{M_{ij}^*} =(-1)^{i+j}\det{M_{ij}}.\]
Note also that only those permutations
$\pi \in S_n$ are \PMlinkescapeword{effective} for the computation of the determinant of $M_{ij}^*$ where $\pi(i)=j$.
\end{proof}
Now we start out with
\begin{eqnarray*}
\det{M} &=\sum\limits_{\pi \in S_n} \mathrm{sgn} \pi\left(\prod_{j=1}^n
m_{j\pi(j)}\right) \\
&=\sum_{k=1}^n m_{ik}\left(\sum\limits_{\pi \in S_n \mid \pi(i)=k} \mathrm{sgn}
\pi\left(\prod\limits_{1 \le j \le i} m_{j\pi(j)}\right) \cdot 1 \cdot \left(\prod\limits_{i \le j \le n}
m_{j-\pi(j)}\right)\right).
\end{eqnarray*}
From the previous lemma, it follows that the \PMlinkescapeword{inner} \PMlinkescapeword{sum} associated with $M_{ik}$
is the determinant of $M_{ij}^*$. So we have
\[\det{M} =\sum_{k=1}^n M_{ik}\left((-1)^{i+k}\det{M_{ik}}\right).\]
%%%%%
%%%%%
\end{document}
