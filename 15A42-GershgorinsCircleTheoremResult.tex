\documentclass[12pt]{article}
\usepackage{pmmeta}
\pmcanonicalname{GershgorinsCircleTheoremResult}
\pmcreated{2013-03-22 13:48:47}
\pmmodified{2013-03-22 13:48:47}
\pmowner{saki}{2816}
\pmmodifier{saki}{2816}
\pmtitle{Gershgorin's circle theorem result}
\pmrecord{11}{34536}
\pmprivacy{1}
\pmauthor{saki}{2816}
\pmtype{Result}
\pmcomment{trigger rebuild}
\pmclassification{msc}{15A42}

% this is the default PlanetMath preamble.  as your knowledge
% of TeX increases, you will probably want to edit this, but
% it should be fine as is for beginners.

% almost certainly you want these
\usepackage{amssymb}
\usepackage{amsmath}
\usepackage{amsfonts}

% used for TeXing text within eps files
%\usepackage{psfrag}
% need this for including graphics (\includegraphics)
%\usepackage{graphicx}
% for neatly defining theorems and propositions
%\usepackage{amsthm}
% making logically defined graphics
%%%\usepackage{xypic}

% there are many more packages, add them here as you need them

% define commands here
\begin{document}
Since the eigenvalues of $A$ and $A$ transpose are the same, you can get an additional set of discs which has the same centers, $a_{ii}$, but a radius calculated by the column $\sum_{j\neq i}|a_{ji}|$ (instead of the rows). If  a disc is isolated it must contain an eigenvalue. The eigenvalues must lie in the intersection of these circles. Hence, by comparing the row and column discs, the eigenvalues may be located efficiently.
%%%%%
%%%%%
\end{document}
