\documentclass[12pt]{article}
\usepackage{pmmeta}
\pmcanonicalname{GeneralizedInverse}
\pmcreated{2013-03-22 14:31:26}
\pmmodified{2013-03-22 14:31:26}
\pmowner{CWoo}{3771}
\pmmodifier{CWoo}{3771}
\pmtitle{generalized inverse}
\pmrecord{5}{36065}
\pmprivacy{1}
\pmauthor{CWoo}{3771}
\pmtype{Definition}
\pmcomment{trigger rebuild}
\pmclassification{msc}{15A09}
\pmclassification{msc}{62J10}
\pmclassification{msc}{62J12}

\endmetadata

% this is the default PlanetMath preamble.  as your knowledge
% of TeX increases, you will probably want to edit this, but
% it should be fine as is for beginners.

% almost certainly you want these
\usepackage{amssymb,amscd}
\usepackage{amsmath}
\usepackage{amsfonts}

% used for TeXing text within eps files
%\usepackage{psfrag}
% need this for including graphics (\includegraphics)
%\usepackage{graphicx}
% for neatly defining theorems and propositions
%\usepackage{amsthm}
% making logically defined graphics
%%%\usepackage{xypic}

% there are many more packages, add them here as you need them

% define commands here
\begin{document}
Let $A$ be an $m\times n$ matrix with entries in $\mathbb{C}$.  A \emph{generalized inverse}, denoted by $A^{-}$, is an $n\times m$ matrix with entries in $\mathbb{C}$, such that $$AA^{-}A=A.$$

\textbf{Examples}
\begin{enumerate}
\item Let $$A=\begin{pmatrix} 2&3&0 \\ 1&2&0 \\ 0&0&0 \end{pmatrix}.$$
Then any matrix of the form 
$$A^{-}=\begin{pmatrix} 2&-3&a \\ -1&2&b \\ c&d&e \end{pmatrix},$$
where $a,b,c,d$ and $e\in\mathbb{C}$, is a generalized inverse.
\item Using the same example from above, if $a=b=c=d=e=0$, then we have an example of the \emph{Moore-Penrose generalized inverse}, which is a unique matrix.
\item Again, using the example from above, if $a=b=c=d=0$ and $e$ is any complex number, we have an example of a \emph{Drazin inverse}.
\end{enumerate}

\textbf{Remark}
Generalized inverse of a matrix has found many applications in statistics.  For example, in general linear model, one solves the set of normal equations
$$\textbf{X}^{\operatorname{T}}\textbf{X}\boldsymbol{\beta}=\textbf{X}^{\operatorname{T}}\textbf{Y},$$
to get the MLE $\hat{\boldsymbol{\beta}}$ of the parameter vector $\boldsymbol{\beta}$.  If the design matrix $\textbf{X}$ is not of full rank (this occurs often when the model is either an ANOVA or ANCOVA type) and hence $\textbf{X}^{\operatorname{T}}\textbf{X}$ is singular.  Then the MLE can be given by
$$\hat{\boldsymbol{\beta}}=(\textbf{X}^{\operatorname{T}}\textbf{X})^{-}\textbf{X}^{\operatorname{T}}\textbf{Y}.$$
%%%%%
%%%%%
\end{document}
