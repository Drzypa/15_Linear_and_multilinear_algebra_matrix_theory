\documentclass[12pt]{article}
\usepackage{pmmeta}
\pmcanonicalname{RotationMatrix}
\pmcreated{2013-03-22 15:03:57}
\pmmodified{2013-03-22 15:03:57}
\pmowner{matte}{1858}
\pmmodifier{matte}{1858}
\pmtitle{rotation matrix}
\pmrecord{17}{36786}
\pmprivacy{1}
\pmauthor{matte}{1858}
\pmtype{Definition}
\pmcomment{trigger rebuild}
\pmclassification{msc}{15-00}
\pmsynonym{rotational matrix}{RotationMatrix}
\pmrelated{OrthogonalMatrices}
\pmrelated{ExampleOfRotationMatrix}
\pmrelated{DecompositionOfOrthogonalOperatorsAsRotationsAndReflections}
\pmrelated{DerivationOfRotationMatrixUsingPolarCoordinates}
\pmrelated{DerivationOf2DReflectionMatrix}
\pmrelated{TransitionToSkewAngledCoordinates}

% this is the default PlanetMath preamble.  as your knowledge
% of TeX increases, you will probably want to edit this, but
% it should be fine as is for beginners.

% almost certainly you want these
\usepackage{amssymb}
\usepackage{amsmath}
\usepackage{amsfonts}
\usepackage{amsthm}

\usepackage{mathrsfs}

% used for TeXing text within eps files
%\usepackage{psfrag}
% need this for including graphics (\includegraphics)
%\usepackage{graphicx}
% for neatly defining theorems and propositions
%
% making logically defined graphics
%%%\usepackage{xypic}

% there are many more packages, add them here as you need them

% define commands here

\newcommand{\sR}[0]{\mathbb{R}}
\newcommand{\sC}[0]{\mathbb{C}}
\newcommand{\sN}[0]{\mathbb{N}}
\newcommand{\sZ}[0]{\mathbb{Z}}

 \usepackage{bbm}
 \newcommand{\Z}{\mathbbmss{Z}}
 \newcommand{\C}{\mathbbmss{C}}
 \newcommand{\R}{\mathbbmss{R}}
 \newcommand{\Q}{\mathbbmss{Q}}



\newcommand*{\norm}[1]{\lVert #1 \rVert}
\newcommand*{\abs}[1]{| #1 |}



\newtheorem{thm}{Theorem}
\newtheorem{defn}{Definition}
\newtheorem{prop}{Proposition}
\newtheorem{lemma}{Lemma}
\newtheorem{cor}{Corollary}
\begin{document}
\begin{defn} A \emph{rotation matrix} is a 
    (real) orthogonal matrix whose determinant is $+1$. 
All $n\times n$ rotation matrices form a group called
    the \emph{specia{l} orthogona{l} grou{p}} and it is denoted by 
    $\operatorname{SO}(n)$. 
\end{defn}

\subsubsection*{Examples}
\begin{enumerate}
\item The identity matrix in $\R^n$ is a rotation matrix.
\item The most general rotation matrix in $\R^2$ can be written as 
$$
  \begin{pmatrix} 
       \cos \theta & -\sin \theta \\  
      \sin \theta & \cos \theta
  \end{pmatrix},
$$
where $\theta\in \R$. 
Multiplication (from the left) with this matrix 
rotates a vector (in $\R^2$) $\theta$ radians in the anti-clockwise
direction. 
\end{enumerate}

\subsubsection*{Properties}
\begin{enumerate}
\item Suppose $v\in \R^n$ is a unit vector. 
  Then there exists a rotation matrix $R$
  such that $R\cdot v = (1,0,\ldots, 0)$.
\item 
In fact, for $v\in \R^n$, $n\ge 3$, there are many rotation matrices 
$\mathbf{R} \in \operatorname{SO}(n)$ such that 
$R\cdot v = (1,0,\ldots, 0)^T$.
To see this, let $f$ be the mapping   
  $f\colon \operatorname{SO}(n-1)\rightarrow \operatorname{SO}(n)$,
  defined as
$$
 f(Q)=
 \begin{pmatrix}
 1 & 0_{1\times n-1}\\
 0_{n-1\times 1} & Q_{n-1\times n-1}
 \end{pmatrix}.
$$
Then for each $Q\in \operatorname{SO}(n-1)$, $f(Q)$ 
    maps $(1,0,\ldots, 0)^T$ onto itself. Thus, if 
    $R_0 \in \operatorname{SO}(n)$ satisfies $R\cdot v=(1,0,\ldots, 0)^T$, 
    then $f(Q)\cdot R$ satisfies the same property for all 
    $Q\in \operatorname{SO}(n-1)$. 
\end{enumerate}
%%%%%
%%%%%
\end{document}
