\documentclass[12pt]{article}
\usepackage{pmmeta}
\pmcanonicalname{GershgorinsCircleTheorem}
\pmcreated{2013-03-22 13:14:15}
\pmmodified{2013-03-22 13:14:15}
\pmowner{lieven}{1075}
\pmmodifier{lieven}{1075}
\pmtitle{Gershgorin's circle theorem}
\pmrecord{7}{33709}
\pmprivacy{1}
\pmauthor{lieven}{1075}
\pmtype{Theorem}
\pmcomment{trigger rebuild}
\pmclassification{msc}{15A42}
\pmsynonym{Gershgorin's disc theorem}{GershgorinsCircleTheorem}
\pmsynonym{Gerschgorin's circle theorem}{GershgorinsCircleTheorem}
\pmsynonym{Gerschgorin's disc theorem}{GershgorinsCircleTheorem}
\pmrelated{BrauersOvalsTheorem}
\pmdefines{Gershgorin disc}
\pmdefines{Gerschgorin disc}

\endmetadata

% this is the default PlanetMath preamble.  as your knowledge
% of TeX increases, you will probably want to edit this, but
% it should be fine as is for beginners.

% almost certainly you want these
\usepackage{amssymb}
\usepackage{amsmath}
\usepackage{amsfonts}

% used for TeXing text within eps files
%\usepackage{psfrag}
% need this for including graphics (\includegraphics)
%\usepackage{graphicx}
% for neatly defining theorems and propositions
%\usepackage{amsthm}
% making logically defined graphics
%%%\usepackage{xypic}

% there are many more packages, add them here as you need them

% define commands here
\begin{document}
Let $A$ be a square complex matrix. Around every element $a_{ii}$ on the diagonal of the matrix, we draw a circle with radius the sum of the norms of the other elements on the same row $\sum_{j\neq i}|a_{ij}|$. Such circles are called \emph{Gershgorin discs}. 

Theorem: Every eigenvalue of A lies in one of these Gershgorin discs.

Proof: Let $\lambda$ be an eigenvalue of $A$ and $x$ its corresponding eigenvector. Choose $i$ such that $|x_i|={\max}_j |x_j|$. Since $x$ can't be $0$, $|x_i|>0$. Now $Ax=\lambda x$, or looking at the $i$-th component
$$ (\lambda - a_{ii})x_i = \sum_{j\neq i} a_{ij}x_j.$$ Taking the norm on both sides gives $$|\lambda - a_{ii}|=|\sum_{j\neq i} \frac{a_{ij}x_j}{x_i}|\leq \sum_{j\neq i}|a_{ij}|.$$
%%%%%
%%%%%
\end{document}
