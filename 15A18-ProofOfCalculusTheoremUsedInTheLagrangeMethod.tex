\documentclass[12pt]{article}
\usepackage{pmmeta}
\pmcanonicalname{ProofOfCalculusTheoremUsedInTheLagrangeMethod}
\pmcreated{2013-03-22 13:29:51}
\pmmodified{2013-03-22 13:29:51}
\pmowner{mathcam}{2727}
\pmmodifier{mathcam}{2727}
\pmtitle{proof of calculus theorem used in the Lagrange method}
\pmrecord{6}{34073}
\pmprivacy{1}
\pmauthor{mathcam}{2727}
\pmtype{Proof}
\pmcomment{trigger rebuild}
\pmclassification{msc}{15A18}
\pmclassification{msc}{15A42}

\endmetadata

% this is the default PlanetMath preamble.  as your knowledge
% of TeX increases, you will probably want to edit this, but
% it should be fine as is for beginners.

% almost certainly you want these
\usepackage{amssymb}
\usepackage{amsmath}
\usepackage{amsfonts}

% used for TeXing text within eps files
%\usepackage{psfrag}
% need this for including graphics (\includegraphics)
%\usepackage{graphicx}
% for neatly defining theorems and propositions
%\usepackage{amsthm}
% making logically defined graphics
%%%\usepackage{xypic}

% there are many more packages, add them here as you need them

% define commands here
\begin{document}
Let $f(\mathbf{x})$ and $g_i(\mathbf{x}), i=0,{\ldots},m$ 
be differentiable scalar functions; $\mathbf{x} \in R^n$.

We will find local extremes of the function $f(\mathbf{x})$ where
$\nabla f=0$.  This can be proved by contradiction:
\[ \nabla f \neq 0 \]
\[ \Leftrightarrow \exists \epsilon_0 > 0, \forall
\epsilon; 0<\epsilon<\epsilon_0: f(\mathbf{x}-\epsilon \nabla f) < f(\mathbf{x}) < f(\mathbf{x+\epsilon \nabla f})
\]
but then $f(\mathbf{x})$ is not a local extreme.

Now we put up some conditions, such that we should find the $\mathbf{x}
\in S \subset R^n$ that gives a local extreme of $f$.  Let $S=\bigcap_{i=1}^m S_i$, and
let $S_i$ be defined so that $g_i(\mathbf{x})=0 \forall \mathbf{x} \in S_i$. 

Any vector $\mathbf{x} \in R^n$ can have one component perpendicular to
the subset $S_i$ (for visualization, think $n=3$ and let
$S_i$ be a flat surface).  $\nabla g_i$ will be perpendicular to
$S_i$, because:
\[ \exists \epsilon_0>0,  \forall \epsilon; 0<\epsilon<\epsilon_0:
g_i(\mathbf{x}-\epsilon \nabla g_i)<g_i(\mathbf{x})<
g_i(\mathbf{x}+\epsilon \nabla g_i) \]
But $g_i(\mathbf{x})=0$, so any vector $\mathbf{x}+\epsilon \nabla
g_i$ must be outside $S_i$, and also outside $S$.
(todo: I have proved that there might exist a component perpendicular to each subset $S_i$, but not that there exists only one; this should be done)

%We define the local maximum of $f(\mathbf{x})$ within $S$ to be all values such
%that 
%\[ \exists \epsilon_0, \forall \mathbf{x_0} \in S; |\mathbf{x}-\mathbf{x_0}|<\epsilon_0: f(\mathbf{x_0})<f(\mathbf{x}) \]
%and the local minimum similarly.

By the argument above, $\nabla f$ must be zero - but now we can ignore
all components of $\nabla f$ perpendicular to $S$. (todo: this should be expressed more formally and proved)

So we will have a local extreme within $S_i$ if there exists a
$\lambda_i$ such that 
\[ \nabla f = \lambda_i \nabla g_i \] 

We will have local extreme(s) within $S$ where there exists a set
$\lambda_i, i=1,{\ldots},m$
such that 
\[ \nabla f = \sum \lambda_i \nabla g_i \]
%%%%%
%%%%%
\end{document}
