\documentclass[12pt]{article}
\usepackage{pmmeta}
\pmcanonicalname{CompoundMatrix}
\pmcreated{2013-03-22 16:13:39}
\pmmodified{2013-03-22 16:13:39}
\pmowner{Mathprof}{13753}
\pmmodifier{Mathprof}{13753}
\pmtitle{compound matrix}
\pmrecord{9}{38326}
\pmprivacy{1}
\pmauthor{Mathprof}{13753}
\pmtype{Definition}
\pmcomment{trigger rebuild}
\pmclassification{msc}{15-00}
\pmdefines{rth adjugate}
\pmdefines{Sylvester -Franke theorem}

\endmetadata

% this is the default PlanetMath preamble.  as your knowledge
% of TeX increases, you will probably want to edit this, but
% it should be fine as is for beginners.

% almost certainly you want these
\usepackage{amssymb}
\usepackage{amsmath}
\usepackage{amsfonts}

% used for TeXing text within eps files
%\usepackage{psfrag}
% need this for including graphics (\includegraphics)
%\usepackage{graphicx}
% for neatly defining theorems and propositions
%\usepackage{amsthm}
% making logically defined graphics
%%%\usepackage{xypic}

% there are many more packages, add them here as you need them

% define commands here

\begin{document}
Suppose that $A$ is an $m \times n$  matrix with entries from a field $F$ and 
$1 \leq r \leq \min(m,n)$. The $r^{th}$ \emph{compound matrix} or
$r^{th}$ \emph{\PMlinkescapetext{adjugate}} of $A$ is the 
$\binom{m}{r} \times \binom{n}{r}$
matrix whose entries are $\det A[\alpha,\beta])$, 
$\alpha \in Q_{r,m}$ and $\beta \in Q_{r,n}$, arranged in lexicographic order  and we use submatrix notation.
The notation for this matrix is $C_r(A)$.


Properties

\begin{enumerate}
\item
$C_r(AB) = C_r(A)C_r(B)$ when $r$ is less than or equal to the number of rows or columns of $A$ and $B$
\item
If $A$ is nonsingular, the $C_r(A)^{-1} = C_r(A^{-1})$.
\item
If $A$ has complex entries, then $C_r(A^*) = (C_r(A))^*$.
\item
$C_r(A^T) = (C_r(A))^T$
\item
$C_r(\overline{A}) = \overline{C_r(A)}$
\item
For any $k \in F$ $C_r(kA) = k^r C_r(A)$
\item 
$C_r(I_n ) = I_{\binom{n}{r}}$
\item
$\det (C_r(A)) = \det (A)^{\binom{n-1}{r-1}}$ (Sylvester --- Franke theorem)
\end{enumerate}



%%%%%
%%%%%
\end{document}
