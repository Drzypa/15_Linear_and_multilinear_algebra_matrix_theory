\documentclass[12pt]{article}
\usepackage{pmmeta}
\pmcanonicalname{DeterminingRankOfMatrix}
\pmcreated{2013-03-22 17:19:23}
\pmmodified{2013-03-22 17:19:23}
\pmowner{Algeboy}{12884}
\pmmodifier{Algeboy}{12884}
\pmtitle{determining rank of matrix}
\pmrecord{9}{39674}
\pmprivacy{1}
\pmauthor{Algeboy}{12884}
\pmtype{Algorithm}
\pmcomment{trigger rebuild}
\pmclassification{msc}{15A03}
\pmrelated{GaussianElimination}
\pmrelated{ReducedRowEchelonForm}

% this is the default PlanetMath preamble.  as your knowledge
% of TeX increases, you will probably want to edit this, but
% it should be fine as is for beginners.

% almost certainly you want these
\usepackage{amssymb}
\usepackage{amsmath}
\usepackage{amsfonts}

% used for TeXing text within eps files
%\usepackage{psfrag}
% need this for including graphics (\includegraphics)
%\usepackage{graphicx}
% for neatly defining theorems and propositions
%\usepackage{amsthm}
% making logically defined graphics
%%%\usepackage{xypic}

% there are many more packages, add them here as you need them

% define commands here

\begin{document}
One can determine the rank of even large matrices by using row and
column operations to put the matrix in a triangular form.  The method
presented here is a version of row reduction to echelon form,
but some simplifications can be made because we are only interested
in finding the rank of the matrix.

The foundation of this method is that the rank of a matrix is left
invariant by the following operations:
\begin{enumerate}
\item Permuting rows
\item Permuting colums
\item Adding a multiple of a row to another row
\item Multiplying a row by an invertible scalar
\end{enumerate}
The last operation is really not needed, but it can be convenient.
For instance, if some of the numbers in the matrix are huge, one
may want to use this operation to keep the numbers in a reasonable
range or, if one has a matrix of integers, one may want to cancel
denominators so that one does not have to deal with fractions.

These operations can be used to put the matrix in a triangular
form; once this is done, all one has to do to determine the rank
is count how many non-zero rows there are.  A systematic way of
going about this is as follows:
\begin{enumerate}
\item If there are no rows or all the entries of the matrix are
zero, you are done.
\item Permute rows and columns so as to put a non-zero element in
the $(1,1)$ position of the matrix.
\item Subtract multiples of the first row so as to put all the
entries in the first column except the first one zero.
\item Repeat the process starting at step 1 with the submatrix 
gotten by throwing away the first row and the first column.
\end{enumerate}

Let us illustrate this with the following matrix:
\[
\begin{matrix}
0 & 1 & 1 & 0 & 0 \\
1 & 1 & 0 & 0 & 0 \\
1 & 0 & 0 & 0 & 1 \\
0 & 0 & 0 & 0 & 1 \\
1 & 0 & 1 & 0 & 1
\end{matrix}
\]

We interchange the first two rows to put a $1$ in the $(1,1)$ position:
\[
\begin{matrix}
1 & 1 & 0 & 0 & 0 \\
0 & 1 & 1 & 0 & 0 \\
1 & 0 & 0 & 0 & 1 \\
0 & 0 & 0 & 0 & 1 \\
1 & 0 & 1 & 0 & 1
\end{matrix}
\]
We subtract the first row from the third and the fifth rows:
\[
\begin{matrix}
1 & 1 & 0 & 0 & 0 \\
0 & 1 & 1 & 0 & 0 \\
0 & -1 & 0 & 0 & 1 \\
0 & 0 & 0 & 0 & 1 \\
0 & -1 & 1 & 0 & 1
\end{matrix}
\]
We now concentrate on the submatrix gotten by ignoring the 
first column and the first row:
\[
\begin{matrix}
1 & 1 & 0 & 0 & 0 \\
0 & \mathbf{1} & \mathbf{1} & \mathbf{0} & \mathbf{0} \\
0 & \mathbf{-1} & \mathbf{0} & \mathbf{0} & \mathbf{1} \\
0 & \mathbf{0} & \mathbf{0} & \mathbf{0} & \mathbf{1} \\
0 & \mathbf{-1} & \mathbf{1} & \mathbf{0} & \mathbf{1}
\end{matrix}
\]
Since the $(1,1)$ position of this submatrix is not zero,
we do not need to do any permuting.  Instead, we go to the
next step and add the second column to the third and fifth 
columns:
\[
\begin{matrix}
1 & 1 & 0 & 0 & 0 \\
0 & \mathbf{1} & \mathbf{1} & \mathbf{0} & \mathbf{0} \\
0 & \mathbf{0} & \mathbf{1} & \mathbf{0} & \mathbf{1} \\
0 & \mathbf{0} & \mathbf{0} & \mathbf{0} & \mathbf{1} \\
0 & \mathbf{0} & \mathbf{2} & \mathbf{0} & \mathbf{1}
\end{matrix}
\]
We now narrow our focus to the submatrix gotten by throwing
out the first and second rows and columns:
\[
\begin{matrix}
1 & 1 & 0 & 0 & 0 \\
0 & 1 & 1 & 0 & 0 \\
0 & 0 & \mathbf{1} & \mathbf{0} & \mathbf{1} \\
0 & 0 & \mathbf{0} & \mathbf{0} & \mathbf{1} \\
0 & 0 & \mathbf{2} & \mathbf{0} & \mathbf{1}
\end{matrix}
\]
Since the $(1,1)$ entry of this submatrix is again not zero,
we do not need to do any permuting.  Thus, we move to the
next step and subtract twice the third row from the fifth row:
\[
\begin{matrix}
1 & 1 & 0 & 0 & 0 \\
0 & 1 & 1 & 0 & 0 \\
0 & 0 & \mathbf{1} & \mathbf{0} & \mathbf{1} \\
0 & 0 & \mathbf{0} & \mathbf{0} & \mathbf{1} \\
0 & 0 & \mathbf{0} & \mathbf{0} & \mathbf{-2}
\end{matrix}
\]
We now narrow our focus to the submatrix gotten by throwing
out the first, second, and third rows and columns:
\[
\begin{matrix}
1 & 1 & 0 & 0 & 0 \\
0 & 1 & 1 & 0 & 0 \\
0 & 0 & 1 & 0 & 1 \\
0 & 0 & 0 & \mathbf{0} & \mathbf{1} \\
0 & 0 & 0 & \mathbf{0} & \mathbf{-2}
\end{matrix}
\]
Since the $(1,1)$ entry of this submatrix is zero, we must make
a permutation.  We will swap the fourth and the fifth columns:
\[
\begin{matrix}
1 & 1 & 0 & 0 & 0 \\
0 & 1 & 1 & 0 & 0 \\
0 & 0 & 1 & 1 & 0 \\
0 & 0 & 0 & \mathbf{1} & \mathbf{0} \\
0 & 0 & 0 & \mathbf{-2} & \mathbf{0}
\end{matrix}
\]
We add twice the fourth row to the fifth row:
\[
\begin{matrix}
1 & 1 & 0 & 0 & 0 \\
0 & 1 & 1 & 0 & 0 \\
0 & 0 & 1 & 1 & 0 \\
0 & 0 & 0 & \mathbf{1} & \mathbf{0} \\
0 & 0 & 0 & \mathbf{0} & \mathbf{0}
\end{matrix}
\]
We narrow our focus to the submatrix gotten by ignoring 
all but the fifth row and column:
\[
\begin{matrix}
1 & 1 & 0 & 0 & 0 \\
0 & 1 & 1 & 0 & 0 \\
0 & 0 & 1 & 1 & 0 \\
0 & 0 & 0 & 1 & 0 \\
0 & 0 & 0 & 0 & \mathbf{0}
\end{matrix}
\]
Since the sole entry of this submatrix is zero, we are done
and have a triangular matrix.  Since there are four
non-zero rows, the rank is $4$.

In the presentation above, certain entries of the matrix have been
shown in boldface.  When using the method in practice, it is only
necessary to write these entries --- the other entries can be
ignored and need not be copied from step to step.
%%%%%
%%%%%
\end{document}
