\documentclass[12pt]{article}
\usepackage{pmmeta}
\pmcanonicalname{GivensRotation}
\pmcreated{2013-03-22 12:06:10}
\pmmodified{2013-03-22 12:06:10}
\pmowner{akrowne}{2}
\pmmodifier{akrowne}{2}
\pmtitle{Givens rotation}
\pmrecord{8}{31215}
\pmprivacy{1}
\pmauthor{akrowne}{2}
\pmtype{Algorithm}
\pmcomment{trigger rebuild}
\pmclassification{msc}{15A57}
\pmclassification{msc}{65F25}
\pmrelated{GramSchmidtOrthogonalization}

\usepackage{amssymb}
\usepackage{amsmath}
\usepackage{amsfonts}
\usepackage{graphicx}
%%%\usepackage{xypic}
\begin{document}
Let $A$ be an $m \times n$ matrix with $m \ge n$  and full rank (viz. rank $n$). An orthogonal matrix triangularization (QR Decomposition) consists of determining an $m \times m$ orthogonal matrix $Q$ such that

$$ Q^T A = \begin{bmatrix} R \\ 0 \end{bmatrix} $$

with the $n \times n$ upper triangular matrix $R$. One only has then to solve the triangular system $Rx = Py$, where $P$ consists of the first $n$ rows of $Q$.

Householder transformations clear whole columns except for the first element of a vector. If one wants to clear parts of a matrix one element at a time, one can use Givens rotation, which is particularly practical for parallel implementation .

A matrix

$$ G = \begin{bmatrix}
1 & \cdots & 0 & \cdots & 0 & \cdots & 0 \\
\vdots & \ddots & \vdots & \; & \vdots & \; & \vdots \\
0 & \cdots & c & \cdots & s & \cdots & 0 \\
\vdots & \; & \vdots & \ddots & \vdots & \; & \vdots \\
0 & \cdots & -s & \cdots & c & \cdots & 0 \\
\vdots & \; & \vdots & \; & \vdots & \ddots & \vdots \\
0 & \cdots & 0 & \cdots & 0 & \cdots & 1 
\end{bmatrix} $$

with properly chosen $c=\cos(\varphi)$ and $s=\sin(\varphi)$ for some rotation angle $\varphi$ can be used to zero the element $a_{ki}$. The elements can be zeroed column by column from the bottom up in the following order:

$$(m,1),(m,-1,1),\ldots,(2,1),(m,2),\ldots, \\
 (3,2),\ldots,(m,n),\ldots,(n+1,n).$$

$Q$ is then the product of $g = \frac{\left( 2m -n -1 \right)n}{2}$ Givens matrices $Q=G_1G_2\cdots G_g$.

To annihilate the bottom element of a $2 \times 1$ vector:

$$ \begin{pmatrix}c & s \\ -s & c\end{pmatrix}^T \begin{pmatrix} a \\ b \end{pmatrix} = \begin{pmatrix}r \\ 0\end{pmatrix} $$

the conditions $sa + cb = 0$ and $c^2 + s^2 = 1$ give:

$$ c = \frac{a}{\sqrt{a^2+b^2}} , s = \frac{b}{\sqrt{a^2+b^2}} $$

For ``Fast Givens'', see [Golub89].

{\bf References}

\begin{itemize}
\item Originally from The Data Analysis Briefbook
(\PMlinkexternal{http://rkb.home.cern.ch/rkb/titleA.html}{http://rkb.home.cern.ch/rkb/titleA.html})
\end{itemize} 

\begin{itemize}
\item[Golub89] Gene H. Golub and Charles F. van Loan: Matrix Computations, 2nd edn., The John Hopkins University Press, 1989.
\end{itemize}
%%%%%
%%%%%
%%%%%
\end{document}
