\documentclass[12pt]{article}
\usepackage{pmmeta}
\pmcanonicalname{DeterminantsOfSomeMatricesOfSpecialForm}
\pmcreated{2013-03-22 14:03:41}
\pmmodified{2013-03-22 14:03:41}
\pmowner{bwebste}{988}
\pmmodifier{bwebste}{988}
\pmtitle{determinants of some matrices of special form}
\pmrecord{5}{35417}
\pmprivacy{1}
\pmauthor{bwebste}{988}
\pmtype{Result}
\pmcomment{trigger rebuild}
\pmclassification{msc}{15A15}
\pmrelated{BlockDeterminants}

% this is the default PlanetMath preamble.  as your knowledge
% of TeX increases, you will probably want to edit this, but
% it should be fine as is for beginners.

% almost certainly you want these
\usepackage{amssymb}
\usepackage{amsmath}
\usepackage{amsfonts}

% used for TeXing text within eps files
%\usepackage{psfrag}
% need this for including graphics (\includegraphics)
%\usepackage{graphicx}
% for neatly defining theorems and propositions
%\usepackage{amsthm}
% making logically defined graphics
%%%\usepackage{xypic}

% there are many more packages, add them here as you need them

% define commands here

\newcommand{\sR}[0]{\mathbb{R}}
\newcommand{\sC}[0]{\mathbb{C}}
\newcommand{\sN}[0]{\mathbb{N}}
\newcommand{\sZ}[0]{\mathbb{Z}}

% The below lines should work as the command
% \renewcommand{\bibname}{References}
% without creating havoc when rendering an entry in 
% the page-image mode.
\makeatletter
\@ifundefined{bibname}{}{\renewcommand{\bibname}{References}}
\makeatother

\newcommand*{\norm}[1]{\lVert #1 \rVert}
\newcommand*{\abs}[1]{| #1 |}
\begin{document}
Suppose $A$ is $n\times n$ square matrix, $u,v$ are two column  $n$-vectors, and
$\alpha$ is a scalar. Then
\begin{eqnarray*}
\det (A + u v^{\operatorname{T}} ) &=& \det A + v^{\operatorname{T}} \operatorname{adj} A\,\, u, \\
\det \begin{pmatrix} A & u \\ v^{\operatorname{T}} & \alpha \end{pmatrix} &=& \alpha \det A - v^{\operatorname{T}} \operatorname{adj} A\,\, u,
\end{eqnarray*}
where $\operatorname{adj} A$ is the adjugate of $A$.

\begin{thebibliography}{9}
 \bibitem{prasolov} V.V. Prasolov,
 \emph{Problems and Theorems in Linear Algebra},
 American Mathematical Society, 1994.
 \end{thebibliography}
%%%%%
%%%%%
\end{document}
