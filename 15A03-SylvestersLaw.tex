\documentclass[12pt]{article}
\usepackage{pmmeta}
\pmcanonicalname{SylvestersLaw}
\pmcreated{2013-03-22 12:58:58}
\pmmodified{2013-03-22 12:58:58}
\pmowner{rspuzio}{6075}
\pmmodifier{rspuzio}{6075}
\pmtitle{Sylvester's law}
\pmrecord{9}{33356}
\pmprivacy{1}
\pmauthor{rspuzio}{6075}
\pmtype{Theorem}
\pmcomment{trigger rebuild}
\pmclassification{msc}{15A03}
\pmsynonym{Sylvester's theorem}{SylvestersLaw}
\pmrelated{PseudoRiemannianManifold}
\pmdefines{rank}
\pmdefines{signature}

% this is the default PlanetMath preamble.  as your knowledge
% of TeX increases, you will probably want to edit this, but
% it should be fine as is for beginners.

% almost certainly you want these
\usepackage{amssymb}
\usepackage{amsmath}
\usepackage{amsfonts}

% used for TeXing text within eps files
%\usepackage{psfrag}
% need this for including graphics (\includegraphics)
%\usepackage{graphicx}
% for neatly defining theorems and propositions
%\usepackage{amsthm}
% making logically defined graphics
%%%\usepackage{xypic}

% there are many more packages, add them here as you need them

% define commands here
\begin{document}
Let $V$ be an $n$-dimensional real vector space and let $Q \colon V \times V \to \mathbb{R}$ be a symmetric quadratic form.  Then there exists a basis of $V$ such that $Q(u,v) = u^T M v$ where the matrix $M$ is diagonal.

Furthermore, for every choice of basis such that $Q(u,v) = u^T M v$ with $M$ diagonal, the number of positive diagonal entries, the number of negative diagonal entries, and the number of zeros on the diagonal will be the same.  The number of non-zero entries on the diagonal is known as the \emph{rank} of the quadratic form.  

To account for the fact that not all the entries on the diagonal may be positive, one defines a quantity known as the \emph{signature}.  However, there is more than one definition of the signature in use.  Some define the signature as the number of positive diagonal entries minus the number of negative ones. Others define the signature as
the number of strictly positive entries on the diagonal. 

Among some people as, for instance, in pseudo-Riemannian geometry and relativity theory, the term signature is used to refer to a symbolic display like $[++-0]$ which shows the number of positive, negative, and zero diagonal entries or a pair of numbers $(n,m)$ which means that there are $n$ positive entries and $m$ negative ones.

The rank and signature (whichever definition; pick your favorite) of a quadratic form are invariant under change of basis.

This implies that two quadratic forms over the same finite-dimensional real vector space are related by a change of basis if and only if they have the same rank and signature.
%%%%%
%%%%%
\end{document}
