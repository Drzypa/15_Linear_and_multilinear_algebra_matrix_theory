\documentclass[12pt]{article}
\usepackage{pmmeta}
\pmcanonicalname{SimpleTensor}
\pmcreated{2013-03-22 15:26:07}
\pmmodified{2013-03-22 15:26:07}
\pmowner{lars_h}{9802}
\pmmodifier{lars_h}{9802}
\pmtitle{simple tensor}
\pmrecord{5}{37282}
\pmprivacy{1}
\pmauthor{lars_h}{9802}
\pmtype{Definition}
\pmcomment{trigger rebuild}
\pmclassification{msc}{15A69}
\pmsynonym{tensor rank}{SimpleTensor}
\pmsynonym{entangled}{SimpleTensor}
\pmrelated{TensorProduct}
\pmrelated{BasicTensor}
\pmdefines{rank}

\endmetadata

% almost certainly you want these
\usepackage{amssymb}
\usepackage{amsmath}
\usepackage{amsfonts}

% used for TeXing text within eps files
%\usepackage{psfrag}
% need this for including graphics (\includegraphics)
%\usepackage{graphicx}
% for neatly defining theorems and propositions

\usepackage{amsthm}
\newtheorem{theorem}{Theorem}
\theoremstyle{definition}
\newtheorem{definition}[theorem]{Definition}

\newenvironment{warning}{%
  \trivlist\item\relax
  \textbf{Warning.}\hspace*{0.5em}\ignorespaces
}{\endtrivlist}


% making logically defined graphics
%%%\usepackage{xypic}

% there are many more packages, add them here as you need them

% define commands here

\newcommand{\Fpil}{\longrightarrow}
\newcommand{\vek}[1]{\mathbf{#1}}
\newcommand{\K}{\mathcal{K}}
\newcommand{\gobble}[1]{}
\newcommand*{\setOf}[3][\gobble]{%
   \left\{ \, #2 \,\,\vrule\relax#1.\,\, #3 \, \right\}%
}
\newcommand*{\spmatrix}[1]{%
  \left(\begin{smallmatrix}#1\end{smallmatrix}\right)%
}
\begin{document}
The \PMlinkname{tensor product}{TensorProduct} \PMlinkescapephrase{tensor product}
$U \otimes V$ of two vector spaces $U$ and $V$ is 
another vector space which is characterised by being universal 
for bilinear maps on $U \times V$. As part of this package, 
there is an operation $\otimes$ on vectors such that 
\(\vek{u} \otimes \vek{v} \in U \otimes V\) for all 
\(\vek{u} \in U\) and \(\vek{v} \in V\), and the primary subject 
of this article is the image of that operation.

\PMlinkescapeword{component}
\PMlinkescapeword{components}
\PMlinkescapeword{simple}
\PMlinkescapeword{order}
\PMlinkescapeword{generic}
\PMlinkescapeword{degrees}
\PMlinkescapeword{primary}

\begin{definition}
  The element \(\vek{w} \in U \otimes V\) is said to be a 
  \emph{simple tensor} if there exist \(\vek{u} \in U\) and \(\vek{v} 
  \in V\) such that \(\vek{w} = \vek{u} \otimes \vek{v}\).
  
  More generally, the element \(\vek{w} \in W = U_1 \otimes \dotsb 
  \otimes U_k\) is said to be a simple tensor (with respect to the 
  decomposition $U_1 \otimes \dotsb \otimes U_k$ of $W$) if there 
  exist \(\vek{u}_i \in U_i\) for \(i=1,\dotsc,k\) such that 
  \(\vek{w} = \vek{u}_1 \otimes \dotsb \otimes \vek{u}_k\).
\end{definition}

For this definition to be interesting, there must also be tensors 
which are not simple, and indeed most tensors aren't. In order to 
illustrate why, it is convenient to consider the tensor product of 
two finite-dimensional vector spaces \(U = \K^m\) and \(V = \K^n\) 
over some field $\K$. In this case one can let \(U \otimes V = 
\K^{m \times n}\) (the vector space of $m \times n$ matrices), since 
$\K^{m \times n}$ is isomorphic to any generic construction of 
$U \otimes V$ and the tensor product of two spaces is anyway only 
defined up to isomorphism. Furthermore considering elements of $U$ 
and $V$ as column vectors, the tensor product of vectors can be 
defined through
\begin{equation*}
  \vek{u} \otimes \vek{v} = \vek{u} \cdot \vek{v}^{\mathrm{T}}
\end{equation*}
where $\cdot$ denotes the product of two matrices (in this case an 
$m \times 1$ matrix by a $1 \times n$ matrix). As a very concrete 
example of this,
\begin{equation*}
  \begin{pmatrix} u_1 \\ u_2 \\ u_3 \end{pmatrix} \otimes
  \begin{pmatrix} v_1 \\ v_2 \\ v_3 \\ v_4 \end{pmatrix} =
  \begin{pmatrix}
    u_1v_1 & u_1v_2 & u_1v_3 & u_1v_4 \\
    u_2v_1 & u_2v_2 & u_2v_3 & u_2v_4 \\
    u_3v_1 & u_3v_2 & u_3v_3 & u_3v_4 
  \end{pmatrix}
  \text{.}
\end{equation*}
One reason the simple tensors in $U \otimes V$ cannot exhaust this 
space (provded \(m,n \geqslant 2\)) is that there are essentially 
only $m+n-1$ degrees of freedom in the choice of a simple tensor, but 
$mn$ \PMlinkname{dimensions}{Dimension2} in the space $U \otimes V$ 
as a whole. Hence
\begin{equation*}
  \K^m \otimes \K^n \neq 
  \setOf{ \vek{u} \otimes \vek{v} }{ \vek{u} \in \K^m, \vek{v} \in \K^n}
  \qquad\text{when \(m,n \geqslant 2\).}
\end{equation*}

How can one to understand the non-simple tensors, then? In general, 
they are finite sums of simple tensors. One way to see this is from 
the theorem that $U \otimes V$ has a basis consisting of products of 
pairs of basis vectors.

\begin{theorem}[\PMlinkname{tensor product basis}{TensorProductBasis}]
  Let $U$ and $V$ be vector spaces over $\K$ with bases 
  $\{\vek{e}_i\}_{i \in I}$ and $\{\vek{f}_j\}_{j \in J}$ 
  respectively. Then \( \{ \vek{e}_i \otimes \vek{f}_j\}_{(i,j) \in 
  I \times J} \) is a basis for $U \otimes V$.
\end{theorem}

Expressing some arbitrary \(\vek{w} \in U \otimes V\) as a linear 
combination
$$
  \vek{w} = \sum_{r=1}^n \lambda_r(\vek{e}_{i_r} \otimes \vek{f}_{j_r})
$$
with respect to such a basis immediately produces the decomposition
$$
  \vek{w} = \sum_{r=1}^n (\lambda_r\vek{e}_{i_r}) \otimes \vek{f}_{j_r}
$$
as a sum of simple tensors, but this decomposition is often far 
from optimally short. 
Let \(\vek{e}_1 = \spmatrix{1\\0} \in \K^2\) and \(\vek{e}_2 = 
\spmatrix{0\\1} \in \K^2\). The tensor \(\vek{e}_1 \otimes \vek{e}_1 
+ \vek{e}_2 \otimes \vek{e}_2 = \spmatrix{1 & 0 \\ 0 & 1}\) is not 
simple, but as it happens the tensor \(\vek{e}_1 \otimes \vek{e}_1 + 
\vek{e}_1 \otimes \vek{e}_2 + \vek{e}_2 \otimes \vek{e}_1 + 
\vek{e}_2 \otimes \vek{e}_2 = \spmatrix{1&1\\1&1} = 
\spmatrix{1\\1} \otimes \spmatrix{1\\1}\) is simple. In general it 
is not trivial to find the simplest way of expressing a tensor as 
a sum of simple tensors, so there is a name for the length of the 
shortest such sum.


\begin{definition} \label{Def:Rang}
  The \emph{rank} of a tensor $\vek{w}$ is the smallest natural 
  number $n$ such that \(\vek{w} = \vek{w}_1 + \dotsb + \vek{w}_n\) 
  for some set of $n$ simple tensors $\vek{w}_1$, \dots, $\vek{w}_n$.
\end{definition}

In particular, the zero tensor has rank $0$, and all other simple 
tensors have rank $1$.

\begin{warning}
  There is an entirely different concept which is also called 
  `the \PMlinkname{rank of a tensor}{Tensor}', 
  namely the number of components (factors) in the tensor 
  product forming the space in which the tensor lives. This latter 
  `rank' concept does not generalise 
  `\PMlinkname{rank of a matrix}{RankLinearMapping}'. 
  The `rank' of Definition~\ref{Def:Rang} \emph{does} 
  generalise `rank of a matrix'. (It also generalises 
  \PMlinkname{rank of a quadratic form}{Rank5}.)
\end{warning}


\PMlinkescapeword{area}
\PMlinkescapeword{example}
\PMlinkescapeword{theorem}
\PMlinkescapeword{definition}
\PMlinkescapephrase{one way}
\PMlinkescapeword{bases}
\PMlinkescapeword{rank}
\PMlinkescapeword{factor}
\PMlinkescapeword{factors}

One area where the distinction between simple and non-simple tensors 
is particularly important is in Quantum Mechanics, because the state 
space of a pair of quantum systems is in general the tensor product 
of the state spaces of the component systems. When the combined state 
is a simple tensor \(\vek{w} = \vek{u} \otimes \vek{v}\), then that 
state can be understood as though one subsystem has state $\vek{u}$ 
and the other state $\vek{v}$, but when the combined state $\vek{w}$ 
is a non-simple tensor \(\vek{u}_1 \otimes \vek{v}_1 + 
\vek{u}_2 \otimes \vek{v}_2\) then the full system cannot be 
understood by considering the two subsystems in isolation, even if 
there is no apparent interaction between them. This situation is 
often described by saying that the two subsystems are 
\emph{entangled}, or using phrases such as ``either $U$ is in state 
$\vek{u}_1$ and $V$ is in state $\vek{v}_1$, or else $U$ is in state 
$\vek{u}_2$ and $V$ is in state $\vek{v}_2$.'' 
Entanglement is an important part of that which makes quantum systems 
different from probabilistic classical systems. The physical 
interpretations are often mind-boggling, but the mathematical meaning 
is no more mysterious than `non-simple tensor'.

Entanglement can also be a useful concept for understanding pure mathematics. 
One reason that the comultiplication \(\Delta\colon C \Fpil C \otimes C\) 
of a coalgebra $C$ cannot simply be replaced in the definition by two 
maps \(\Delta_L,\Delta_R\colon C \Fpil C\) that compute the `left' 
and `right' parts of $\Delta$ is that value of $\Delta$ may be 
entangled, in which case one left part $\Delta_L(c)$ and one right 
part $\Delta_R(c)$ cannot fully encode $\Delta(c)$.
%%%%%
%%%%%
\end{document}
