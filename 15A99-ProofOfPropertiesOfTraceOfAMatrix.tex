\documentclass[12pt]{article}
\usepackage{pmmeta}
\pmcanonicalname{ProofOfPropertiesOfTraceOfAMatrix}
\pmcreated{2013-03-22 13:42:54}
\pmmodified{2013-03-22 13:42:54}
\pmowner{Daume}{40}
\pmmodifier{Daume}{40}
\pmtitle{proof of properties of trace of a matrix}
\pmrecord{4}{34396}
\pmprivacy{1}
\pmauthor{Daume}{40}
\pmtype{Proof}
\pmcomment{trigger rebuild}
\pmclassification{msc}{15A99}

% this is the default PlanetMath preamble.  as your knowledge
% of TeX increases, you will probably want to edit this, but
% it should be fine as is for beginners.

% almost certainly you want these
\usepackage{amssymb}
\usepackage{amsmath}
\usepackage{amsfonts}

% used for TeXing text within eps files
%\usepackage{psfrag}
% need this for including graphics (\includegraphics)
%\usepackage{graphicx}
% for neatly defining theorems and propositions
%\usepackage{amsthm}
% making logically defined graphics
%%%\usepackage{xypic} 

% there are many more packages, add them here as you need them

% define commands here
\begin{document}
\textbf{Proof of Properties:}
\begin{enumerate}
\item Let us check linearity. For sums we have
\begin{eqnarray*}
\operatorname{trace}(A+B) &=& \sum\limits_{i=1}^{n} (a_{i,i} + b_{i,i})\,\,\,\,\,\,\,\,\,\,\, \mbox{(property of matrix addition)}\\
&=&\sum\limits _{i=1} ^{n} a_{i,i} + \sum\limits _{i=1} ^{n} b_{i,i}\,\,\,\, \mbox{(property of sums)} \\
&=&\operatorname{trace}(A) + \operatorname{trace}(B).
\end{eqnarray*}
Similarly,
% \item Like above the trace is a linear transformation therefore this property
% is inherited. Proving that it is a linear
% transformation.
\begin{eqnarray*}
\operatorname{trace}(cA) &=& \sum\limits _{i=1} ^{n} c\cdot a_{i,i}\,\,\,\,\, \mbox{(property of matrix scalar multiplication)} \\
&=& c\cdot \sum\limits _{i=1} ^{n} a_{i,i}\,\,\,\,\, \mbox{(property of sums)} \\
&=& c\cdot \operatorname{trace}(A).
\end{eqnarray*}
\item The second property follows since the transpose does not alter
the entries on the main diagonal.
\item The proof of the third property follows by exchanging the
summation order. Suppose $A$ is a $n\times m$ matrix and $B$ is a $m\times n$ matrix.
Then
\begin{eqnarray*}
\operatorname{trace} AB &=& \sum\limits_{i=1}^{n} \sum\limits_{j=1}^{m} A_{i,j} B_{j,i} \\
&=& \sum\limits_{j=1}^{m} \sum\limits_{i=1}^{n} B_{j,i} A_{i,j} \,\,\,\,\mbox{(changing summation order)}\\
&=& \operatorname{trace} BA.
\end{eqnarray*}
\item The last property is a consequence of Property 3 and the fact that matrix multiplication is
associative;
\begin{eqnarray*}
\operatorname{trace} (B^{-1} A B) &=& \operatorname{trace} \big((B^{-1} A) B\big) \\
&=& \operatorname{trace} \big(B(B^{-1} A) \big) \\
&=& \operatorname{trace} \big( (BB^{-1}) A \big) \\
&=& \operatorname{trace} (A).
\end{eqnarray*}

\end{enumerate}
%%%%%
%%%%%
\end{document}
