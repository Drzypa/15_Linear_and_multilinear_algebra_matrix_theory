\documentclass[12pt]{article}
\usepackage{pmmeta}
\pmcanonicalname{SchurComplement}
\pmcreated{2013-03-22 15:27:11}
\pmmodified{2013-03-22 15:27:11}
\pmowner{georgiosl}{7242}
\pmmodifier{georgiosl}{7242}
\pmtitle{Schur complement}
\pmrecord{8}{37302}
\pmprivacy{1}
\pmauthor{georgiosl}{7242}
\pmtype{Definition}
\pmcomment{trigger rebuild}
\pmclassification{msc}{15A15}
\pmrelated{BlockDeterminants}
\pmrelated{MatrixInversionLemma}

\endmetadata

% this is the default PlanetMath preamble.  as your knowledge
% of TeX increases, you will probably want to edit this, but
% it should be fine as is for beginners.

% almost certainly you want these
\usepackage{amssymb}
\usepackage{amsmath}
\usepackage{amsfonts}

% used for TeXing text within eps files
%\usepackage{psfrag}
% need this for including graphics (\includegraphics)
%\usepackage{graphicx}
% for neatly defining theorems and propositions
%\usepackage{amsthm}
% making logically defined graphics
%%%\usepackage{xypic}

% there are many more packages, add them here as you need them

% define commands here
\begin{document}
Let A,B,C,D be matrices of sizes $p\times p$, $p\times q$, $q\times p$ and $q\times q$ respectively and suppose that $D$ is invertible. Let 
$$M=\begin{pmatrix} A & B \\ C & D \end{pmatrix}$$
so that $M$ is a $(p+q)\times(p+q)$ matrix.
\\Then the \emph{Schur complement} of the block $D$ of the matrix $M$ is the 
$p\times p$ matrix, $A-BD^{-1}C$. Analogously if $A$ is invertible then the \emph{Schur complement} of the block $A$ of the matrix $M$ is the 
$q\times q$ matrix, $D-CA^{-1}B$.
In the first case, when  $D$ is invertible, the Schur complement arises as the result of performing a partial Gaussian elimination by multiplying the matrix $M$ from the right with the lower triangular block matrix,
$$T=\begin{pmatrix} I & O \\ -D^{-1}C & D^{-1} \end{pmatrix}$$ 
where $I$ is the $p\times p$ identity matrix and $O$ is the $p\times q$ zero matrix. Analogously, in the second case, we take the Schur complement by multiplying the matrix $M$ from the left with the lower triangular block matrix 
$$T=\begin{pmatrix} A^{-1} & O \\ -CA^{-1} & I \end{pmatrix}$$ 
\\\textbf{see also:}
\begin{itemize}
\item Wikipedia, \PMlinkexternal{Schur complement}{http://en.wikipedia.org/wiki/Schur_complement}
\end{itemize}
%%%%%
%%%%%
\end{document}
