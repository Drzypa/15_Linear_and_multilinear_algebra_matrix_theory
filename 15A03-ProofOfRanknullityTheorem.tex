\documentclass[12pt]{article}
\usepackage{pmmeta}
\pmcanonicalname{ProofOfRanknullityTheorem}
\pmcreated{2013-03-22 12:25:13}
\pmmodified{2013-03-22 12:25:13}
\pmowner{rmilson}{146}
\pmmodifier{rmilson}{146}
\pmtitle{proof of rank-nullity theorem}
\pmrecord{4}{32361}
\pmprivacy{1}
\pmauthor{rmilson}{146}
\pmtype{Proof}
\pmcomment{trigger rebuild}
\pmclassification{msc}{15A03}

\usepackage{amsmath}
\usepackage{amsfonts}
\usepackage{amssymb}

\newcommand{\reals}{\mathbb{R}}
\newcommand{\natnums}{\mathbb{N}}
\newcommand{\cnums}{\mathbb{C}}

\newcommand{\lp}{\left(}
\newcommand{\rp}{\right)}
\newcommand{\lb}{\left[}
\newcommand{\rb}{\right]}

\newcommand{\supth}{^{\text{th}}}


\newtheorem{proposition}{Proposition}
\newcommand{\Ker}{\mathop{\mathrm{Ker}}}
\newcommand{\Img}{\mathop{\mathrm{Img}}}
\begin{document}
Let $T:V\rightarrow W$ be a linear mapping, with $V$
finite-dimensional.  We wish to show that
$$\dim V = \dim \Ker T + \dim\Img T$$

The images of a basis of $V$ will span $\Img T$, and hence $\Img T$ is
finite-dimensional.  Choose then a basis $w_1,\ldots,w_n$ of $\Img T$
and choose preimages $v_1,\ldots,v_n\in U$ such that 
$$w_i = T(v_i),\quad i=1\ldots n$$
Choose a basis $u_1,\ldots,u_k$ of $\Ker T$.  The result will follow
once we show that $u_1,\ldots,u_k,v_1,\ldots,v_n$ is a basis of $V$.

Let $v\in V$ be given.  Since $T(v)\in \Img T$, by definition, we can
choose scalars $b_1,\ldots,b_n$ such that
$$T(v)= b_1 w_1 + \ldots b_n w_n.$$
Linearity of $T$ now implies that
$T(b_1 v_1 + \ldots + b_n v_n-v) = 0,$
and hence we can choose scalars $a_1,\ldots, a_k$ such that
$$b_1 v_1 + \ldots + b_n v_n-v = a_1 u_1 + \ldots a_k u_k.$$
Therefore $u_1,\ldots,u_k,v_1,\ldots,v_n$ span $V$.

Next, let $a_1,\ldots, a_k,b_1,\ldots,b_n$ be scalars such that
$$a_1 u_1+\ldots +a_k u_k + b_1 v_1 + \ldots + b_n v_n = 0.$$
By applying $T$ to both sides of this equation it follows that
$$b_1 w_1 + \ldots + b_n w_n =0,$$
and since $w_1,\ldots, w_n$ are linearly independent that
$$b_1= b_2 = \ldots = b_n = 0.$$
Consequently 
$$a_1 u_1+\ldots +a_k u_k = 0$$
as well, and since $u_1,\ldots, u_k$ are also assumed to be linearly
independent we conclude that
$$a_1= a_2 = \ldots = a_k = 0$$
also.  Therefore $u_1,\ldots,u_k,v_1,\ldots,v_n$ are linearly
independent, and are therefore a basis. Q.E.D.
%%%%%
%%%%%
\end{document}
