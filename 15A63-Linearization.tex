\documentclass[12pt]{article}
\usepackage{pmmeta}
\pmcanonicalname{Linearization}
\pmcreated{2013-03-22 14:53:52}
\pmmodified{2013-03-22 14:53:52}
\pmowner{CWoo}{3771}
\pmmodifier{CWoo}{3771}
\pmtitle{linearization}
\pmrecord{5}{36580}
\pmprivacy{1}
\pmauthor{CWoo}{3771}
\pmtype{Definition}
\pmcomment{trigger rebuild}
\pmclassification{msc}{15A63}
\pmclassification{msc}{15A69}
\pmclassification{msc}{16R99}
\pmclassification{msc}{17A99}
\pmsynonym{polarization}{Linearization}
\pmdefines{linearized}

% this is the default PlanetMath preamble.  as your knowledge
% of TeX increases, you will probably want to edit this, but
% it should be fine as is for beginners.

% almost certainly you want these
\usepackage{amssymb,amscd}
\usepackage{amsmath}
\usepackage{amsfonts}

% used for TeXing text within eps files
%\usepackage{psfrag}
% need this for including graphics (\includegraphics)
%\usepackage{graphicx}
% for neatly defining theorems and propositions
%\usepackage{amsthm}
% making logically defined graphics
%%%\usepackage{xypic}

% there are many more packages, add them here as you need them

% define commands here
\begin{document}
\emph{Linearization} is the process of reducing a homogeneous polynomial into a multilinear map over a commutative ring.  There are in general two ways of doing this:

\begin{itemize}
\item \textbf{Method 1.} Given any homogeneous polynomial $f$ of degree $n$ in $m$ indeterminates over a commutative 
scalar ring $R$ (scalar simply means that the elements of $R$ commute with the indeterminates). 
\begin{enumerate}
\item[Step 1]  If all indeterminates are linear in $f$, then we are done.  
\item[Step 2]  Otherwise, pick an indeterminate $x$ such that $x$ is not linear in $f$.  Without loss of generality, 
write $f=f(x,X)$, where $X$ is the set of indeterminates in $f$ excluding $x$.  Define 
$g(x_1,x_2,X):=f(x_1+x_2,X)-f(x_1,X)-f(x_2,X)$.  Then $g$ is a homogeneous polynomial of degree $n$ in $m+1$ 
indeterminates.  However, the highest degree of $x_1,x_2$ is $n-1$, one less that of $x$.
\item[Step 3]  Repeat the process, starting with Step 1, for the homogeneous polynomial $g$.  Continue until the set $X$ 
of indeterminates is enlarged to one $X^{'}$ such that each $x\in X^{'}$ is linear.
\end{enumerate}
\item \textbf{Method 2.} This method applies only to homogeneous polynomials that are also homogeneous in each 
indeterminate, when the other indeterminates are held constant, i.e., $f(tx,X)=t^nf(x,X)$ for some $n\in\mathbb{N}$ and 
any $t\in R$.  Note that if all of the indeterminates in $f$ commute with each other, then $f$ is essentially a 
monomial.  So this method is particularly useful when indeterminates are non-commuting.  If this is the case, then we 
use the following algorithm:
\begin{enumerate}
\item[Step 1]  If $x$ is not linear in $f$ and that $f(tx,X)=t^nf(x,X)$, replace $x$ with a formal linear combination of 
$n$ indeterminates over $R$: $$r_1x_1+\cdots+r_nx_n\mbox{,   where }r_i\in R.$$  
\item[Step 2]  Define a polynomial $g\in R\langle x_1,\ldots,x_n \rangle$, the non-commuting free algebra over $R$ 
(generated by the non-commuting indeterminates $x_i$) by: $$g(x_1,\ldots,x_n):=f(r_1x_1+\cdots+r_nx_n).$$  
\item[Step 3]  Expand $g$ and take the sum of the monomials in $g$ whose coefficent is $r_1\cdots r_n$.  The result is a 
linearization of $f$ for the indeterminate $x$.
\item[Step 4]  Take the next non-linear indeterminate and start over (with Step 1).  Repeat the process until $f$ is 
completely linearized.
\end{enumerate}
\end{itemize}

\textbf{Remarks.}  
\begin{enumerate}
\item The method of linearization is used often in the studies of Lie algebras, Jordan algebras, PI-algebras and 
quadratic forms.
\item If the characteristic of scalar ring $R$ is 0 and $f$ is a monomial in one indeterminate, we can recover $f$ back 
from its linearization by setting all of its indeterminates to a single indeterminate $x$ and dividing the resulting 
polynomial by $n!$: $$f(x)=\frac{1}{n!}\operatorname{linearization}(f)(x,\ldots,x).$$  Please see the first example below.
\item If $f$ is a homogeneous polynomial of degree $n$, then the linearized $f$ is a multilinear map in $n$ 
indeterminates.
\end{enumerate}

\textbf{Examples}.
\begin{itemize}
\item $f(x)=x^2$.  Then $f(x_1+x_2)-f(x_1)-f(x_2)=x_1x_2+x_2x_1$ is a linearization of $x^2$.  In general, if $f(x)=x^n$, 
then the linearization of $f$ is $$\sum_{\sigma\in S_n}x_{\sigma(1)}\cdots x_{\sigma(n)}=
\sum_{\sigma\in S_n}\prod_{i=1}^{n}x_{\sigma(i)},$$ where $S_n$ is the symmetric group on $\lbrace1,\ldots,n\rbrace$.  
If in addition all the indeterminates commute with each other and $n!\neq0$ in the ground ring, then the linearization 
becomes $$n!x_1\cdots x_n=\prod_{i=1}^{n}ix_i.$$
\item $f(x,y)=x^3y^2+xyxyx$.  Since $f(tx,y)=t^3f(x,y)$ and $f(x,ty)=t^2f(x,y)$, $f$ is homogeneous over $x$ and $y$ 
separately, and thus we can linearize $f$.  First, collect all the monomials having coefficient $abc$ in 
$(ax_1+bx_2+cx_3,y)$, we get $$g(x_1,x_2,x_3,y):=\sum x_ix_jx_ky^2+x_iyx_jyx_k,$$ where $i,j,k\in 
{1,2,3}$ and $(i-j)(j-k)(k-i)\neq0$.  Repeat this for $y$ and we have 
$$h(x_1,x_2,x_3,y_1,y_2):=\sum x_ix_jx_k(y_1y_2+y_2y_1)+(x_iy_1x_jy_2x_k+x_iy_2x_jy_1x_k).$$
\end{itemize}
%%%%%
%%%%%
\end{document}
