\documentclass[12pt]{article}
\usepackage{pmmeta}
\pmcanonicalname{DimensionFormulaeForVectorSpaces}
\pmcreated{2013-03-22 16:31:19}
\pmmodified{2013-03-22 16:31:19}
\pmowner{yark}{2760}
\pmmodifier{yark}{2760}
\pmtitle{dimension formulae for vector spaces}
\pmrecord{25}{38701}
\pmprivacy{1}
\pmauthor{yark}{2760}
\pmtype{Feature}
\pmcomment{trigger rebuild}
\pmclassification{msc}{15A03}

\endmetadata

\usepackage{amsmath}

\def\Hom{\operatorname{Hom}}
\def\End{\operatorname{End}}
\def\Im{\operatorname{Im}}
\def\Ker{\operatorname{Ker}}

\def\isomorphic{\cong}

% The following lines should work as the command
% \renewcommand{\bibname}{References}
% without creating havoc when rendering an entry in 
% the page-image mode.
\makeatletter
\@ifundefined{bibname}{}{\renewcommand{\bibname}{References}}
\makeatother

\begin{document}
\PMlinkescapeword{degree}
\PMlinkescapeword{finite}
\PMlinkescapeword{formula}
\PMlinkescapeword{implies}
\PMlinkescapeword{obvious}
\PMlinkescapeword{order}
\PMlinkescapeword{pointwise}
\PMlinkescapeword{section}
\PMlinkescapeword{sum}
\PMlinkescapeword{term}
\PMlinkescapeword{terms}
\PMlinkescapeword{theory}

In this entry we look at various formulae 
involving the dimension of a vector space.

Throughout this entry, $K$ will be a field,
and $V$ and $W$ will be vector spaces over $K$.
The dimension of a vector space $U$ over $K$
will be denoted by $\dim(U)$, 
or by $\dim_K(U)$ if the ground field needs to be emphasized.

All of these formulae potentially involve infinite cardinals,
so the reader should have a basic knowledge of cardinal arithmetic
in order to understand them in full generality.

\section*{Subspaces}

If $S$ and $T$ are subspaces of $V$, then
\[
  \dim(S)+\dim(T)=\dim(S\cap T)+\dim(S+T).
\]

\section*{Rank-nullity theorem}

The rank-nullity theorem states that if $\phi\colon V\to W$ is a linear mapping,
then the dimension of $V$ 
is the sum of the dimensions of the image and kernel of $\phi$:
\[
  \dim(V)=\dim(\Im\phi)+\dim(\Ker\phi).
\]

In particular, if $U$ is a subspace of $V$ then
\[
  \dim(V)=\dim(V/U)+\dim(U).
\]

The rank-nullity theorem can also be stated in terms of short exact sequences:
if
\[
  0\longrightarrow U\longrightarrow V\longrightarrow
  W\longrightarrow 0
\]
is a short exact sequence of vector spaces over $K$,
then
\[
  \dim(V) = \dim(U) + \dim(W).
\]
This can be generalized to infinite exact sequences:
if
\[
  \cdots\longrightarrow V_{n+1}\longrightarrow V_n\longrightarrow 
  V_{n-1}\longrightarrow \cdots
\]
is an exact sequence of vector spaces over $K$,
then
\[
  \sum_{n\rm{~even}}\!\!\dim(V_n)\,=\sum_{n\rm{~odd}}\!\dim(V_n).
\]
(This is indeed a generalization, 
because any finite exact sequence of vectors spaces
terminating with $0$ at both ends
can be considered as an infinite exact sequence
in which all remaining terms are $0$.)

\section*{Direct sums}

If $(V_i)_{i\in I}$ is a family of vector spaces over $K$,
then
\[
  \dim\!\left(\bigoplus_{i\in I}V_i\right) = \sum_{i\in I}\dim(V_i).
\]

\section*{Cardinality of a vector space}

The cardinality of a vector space 
is determined by its dimension and the cardinality of the ground field:
\[
  |V|=\begin{cases}
        |K|^{\dim(V)},&\hbox{if }\dim(V)\hbox{ is finite};\\
        \max\{|K|,\dim(V)\},&\hbox{if }\dim(V)\hbox{ is infinite}.\cr
      \end{cases}
\]

The effect of the above formula is somewhat different
depending on whether $K$ is \PMlinkname{finite}{FiniteField} or infinite.
If $K$ is finite, then it reduces to
\[
  |V|=\begin{cases}
         |K|^{\dim(V)},&\hbox{if }\dim(V)\hbox{ is finite};\\
         \dim(V),&\hbox{if }\dim(V)\hbox{ is infinite}.
      \end{cases}
\]
If $K$ is infinite, then it can be expressed as
\[
  |V|=\begin{cases}
         1,&\hbox{if }\dim(V)=0;\\
         |K|,&\hbox{if }0<\dim(V)\le|K|;\\
         \dim(V),&\hbox{if }\dim(V)\ge|K|.
      \end{cases}
\]

\section*{Change of ground field}

If $F$ is a subfield of $K$,
then $V$ can be considered as a vector space over $F$.
The dimensions of $V$ over $K$ and $F$ are related by the formula
\[
  \dim_F(V)=[K:F]\cdot\dim_K(V).
\]
In this formula, $[K:F]$ is the degree of the field extension $K/F$,
that is, the dimension of $K$ considered as a vector space over $F$.

\section*{Space of functions into a vector space}

If $S$ is any set,
then the set $K^S$ of all functions from $S$ into $K$
becomes a vector space over $K$ if we define the operations pointwise,
that is, $(f+g)(x)=f(x)+g(x)$ and $(\lambda f)(x)=\lambda f(x)$
for all $f,g\in K^S$, all $x\in S$, and all $\lambda\in K$.
The dimension of this vector space is given by
\[
  \dim(K^S)=\begin{cases}
               |S|,&\hbox{if }S\hbox{ is finite};\\
               |K|^{|S|},&\hbox{if }S\hbox{ is infinite}.
            \end{cases}
\]
The case where $S$ is infinite is not straightforward to prove.
Proofs can be found in books by Baer\cite{baer} and Jacobson\cite{jacobson},
among others.

More generally, we can consider the space $V^S$,
which is really just the \PMlinkname{direct product}{DirectProduct} of copies of $V$ indexed by $S$.
We get
\[
  \dim(V^S)=\begin{cases}
               0,&\hbox{if }\dim(V)=0;\\
               |S|\cdot\dim(V),&\hbox{if }S\hbox{ is finite};\\
               |V|^{|S|},&\hbox{otherwise}.
            \end{cases}
\]

\section*{Dual space}

Given any basis $B$ of $V$,
the dual space $V^*$
is isomorphic to $K^B$ via the mapping $f\mapsto f|_B$.
So the formula of the previous section
can be applied to give a formula for the dimension of $V^*$:
\[
  \dim(V^*)=\begin{cases}
               \dim(V),&\hbox{if }\dim(V)\hbox{ is finite};\\
               |K|^{\dim(V)},&\hbox{if }\dim(V)\hbox{ is infinite}.
            \end{cases}
\]
In particular, this formula implies that $V$ is isomorphic to $V^*$
if and only if $V$ is finite-dimensional.
(Students who are familiar with the fact that
an infinite-dimensional Banach space can be isomorphic to its dual
are sometimes surprised to learn that an infinite-dimensional vector space
cannot be isomorphic to its dual,
for a Banach space is surely a vector space.
But the term {\it dual} is used in different senses in these two statements,
so there is no contradiction.
In the theory of Banach spaces
one is usually only interested in the {\it continuous} linear functionals,
and the resulting `continuous' dual
is a subspace of the full dual used in the above formula.)

\section*{Space of linear mappings}

The set $\Hom_K(V,W)$ of all linear mappings from $V$ into $W$
is itself a vector space over $K$,
with the operations defined in the obvious way,
namely $(f+g)(x)=f(x)+g(x)$ and $(\lambda f)(x)=\lambda f(x)$
for all $f,g\in\Hom_K(V,W)$, all $x\in V$, and all $\lambda\in K$.
The dual space $V^*=\Hom_K(V,K)$ considered in the previous section
is a special case of this.
For any basis $B$ of $V$, the mapping $f\mapsto f|_B$
defines an isomorphism between $\Hom_K(V,W)$ and $W^B$,
so that from an earlier section we get
\[
  \dim(\Hom_K(V,W))=\begin{cases}
               0,&\hbox{if }\dim(W)=0;\\
               \dim(V)\cdot\dim(W),&\hbox{if }\dim(V)\hbox{ is finite};\\
               |W|^{\dim(V)},&\hbox{otherwise}.
            \end{cases}
\]

In the special case $W=V$ this can be simplified to
\[
  \dim(\End_K(V))=\begin{cases}
               \dim(V)^2,&\hbox{if }\dim(V)\hbox{ is finite};\\
               |K|^{\dim(V)},&\hbox{otherwise}.
            \end{cases}
\]

\section*{Tensor products}

The dimension of the \PMlinkname{tensor product}{TensorProduct} of $V$ and $W$ is given by
\[
  \dim(V\otimes W)=\dim(V)\cdot\dim(W).
\]

\section*{Banach spaces}

The dimension of a Banach space,
considered as a vector space,
is sometimes called the {\it Hamel dimension},
in order to distinguish it from other concepts of dimension.
For an infinite-dimensional Banach space $B$ we have
\[
  \dim(B)=|B|.
\]
The tricky part of establishing this formula
is to show that the dimension is always
at least the cardinality of the continuum.
A short proof of this is given in a paper by Lacey\cite{lacey}.

The above formula suggests that Hamel dimension 
is not a very useful concept for infinite-dimensional Banach spaces,
which is indeed the case.
Nonetheless, it is interesting to see how Hamel dimension relates to
the usual concept of dimension in Hilbert spaces.
If $H$ is a Hilbert space,
and $d$ is its dimension 
(meaning the cardinality of an orthonormal basis),
then the Hamel dimension $\dim(H)$ is given by
\[
  \dim(H)=\begin{cases}
             d,&\hbox{if }d\hbox{ is finite};\\
             d^{\aleph_0},&\hbox{if }d\hbox{ is infinite}.
          \end{cases}
\]

\begin{thebibliography}{9}
\bibitem{baer}
 Reinhold Baer,
 {\it Linear Algebra and Projective Geometry},
 Academic Press, 1952.
\bibitem{jacobson}
 Nathan Jacobson,
 {\it Lectures in Abstract Algebra},
 Volume II: {\it Linear Algebra},
 D.\ Van Nostrand Company Inc., 1953.
\bibitem{lacey}
 H.\ Elton Lacey,
 {\it The Hamel Dimension of any Infinite Dimensional Separable Banach Space is c},
 Amer.\ Math.\ Mon.\ 80 (1973), 298.
\end{thebibliography}

%%%%%
%%%%%
\end{document}
