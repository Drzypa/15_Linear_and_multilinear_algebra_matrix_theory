\documentclass[12pt]{article}
\usepackage{pmmeta}
\pmcanonicalname{GeneralizedKroneckerDeltaSymbol}
\pmcreated{2013-03-22 13:31:38}
\pmmodified{2013-03-22 13:31:38}
\pmowner{matte}{1858}
\pmmodifier{matte}{1858}
\pmtitle{generalized Kronecker delta symbol}
\pmrecord{5}{34119}
\pmprivacy{1}
\pmauthor{matte}{1858}
\pmtype{Definition}
\pmcomment{trigger rebuild}
\pmclassification{msc}{15A99}
\pmrelated{LeviCivitaPermutationSymbol3}

\endmetadata

% this is the default PlanetMath preamble.  as your knowledge
% of TeX increases, you will probably want to edit this, but
% it should be fine as is for beginners.

% almost certainly you want these
\usepackage{amssymb}
\usepackage{amsmath}
\usepackage{amsfonts}

% used for TeXing text within eps files
%\usepackage{psfrag}
% need this for including graphics (\includegraphics)
%\usepackage{graphicx}
% for neatly defining theorems and propositions
%\usepackage{amsthm}
% making logically defined graphics
%%%\usepackage{xypic}

% there are many more packages, add them here as you need them

% define commands here
\begin{document}
Let $l$ and $n$ be natural numbers such that $1\le l \le n$. 
Further,  let $i_k$  and $j_k$ be natural numbers in $\{1,\cdots, n\}$ 
for all $k$ in $\{1,\cdots, l\}$. 
Then the
\emph{generalized Kronecker delta symbol}, denoted by 
$\delta_{j_1\cdots j_l}\!\!\!\!\!\!\!\!\!\!^{i_1\cdots i_l}$, 
is zero if $i_r=i_s$ 
or $j_r=j_s$ for some $r\neq s$, or if
$\{i_1,\cdots, i_l\} \neq \{j_1,\cdots, j_l\}$ as sets.
If none of the above conditions are met, then 
$\delta_{j_1\cdots j_l}\!\!\!\!\!\!\!\!\!\!^{i_1\cdots i_l}$ 
is defined as the sign of the permutation that maps
$i_1\cdots i_l$ to $j_1\cdots j_l$.

From the definition, it follows that when $l=1$, 
the generalized Kronecker delta symbol reduces to 
the traditional delta symbol $\delta^i_j$. 
Also, for $l=n$, we obtain 
\begin{eqnarray*}
\delta_{j_1\cdots j_n}\!\!\!\!\!\!\!\!\!\!\!\!^{i_1\cdots \,i_n}&=&\varepsilon^{i_1\cdots i_n}\varepsilon_{j_1\cdots j_n},\\
\delta_{j_1\cdots j_n}\!\!\!\!\!\!\!\!\!\!\!\!^{1\cdots \,n}&=&\varepsilon_{j_1\cdots j_n},
\end{eqnarray*}
where $\varepsilon_{j_1\cdots j_n}$ is the Levi-Civita permutation symbol.

For any $l$ we can write the generalized delta function 
as a determinant of traditional delta symbols. Indeed,
if $S(l)$ is the permutation group of $l$ elements, then 
\begin{eqnarray*}
\delta_{j_1\cdots j_l}\!\!\!\!\!\!\!\!\!\!\!^{i_1\cdots i_l} &=& \sum_{\tau\in S(l)} \mbox{sign} \, \tau\, \delta^{i_{\tau(1)}}_{j_1}\cdots \delta^{i_{\tau(l)}}_{j_l} \\
&=& \det \left(  \begin {array}{lll} \delta^{i_1}_{j_1} & \cdots & \delta^{i_l}_{j_1} \\
   \vdots & \ddots & \vdots \\
   \delta^{i_1}_{j_l} &  \cdots & \delta^{i_l}_{j_l} 
    \end{array} \right).
\end{eqnarray*}
The first equality follows since the sum one the first line has only one non-zero term; the term for
which $i_{\tau(k)} = j_k$. The second equality follows from the 
definition of the determinant.
%%%%%
%%%%%
\end{document}
