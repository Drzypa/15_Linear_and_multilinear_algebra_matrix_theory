\documentclass[12pt]{article}
\usepackage{pmmeta}
\pmcanonicalname{Multilinear}
\pmcreated{2013-03-22 12:33:05}
\pmmodified{2013-03-22 12:33:05}
\pmowner{rmilson}{146}
\pmmodifier{rmilson}{146}
\pmtitle{multi-linear}
\pmrecord{10}{32796}
\pmprivacy{1}
\pmauthor{rmilson}{146}
\pmtype{Definition}
\pmcomment{trigger rebuild}
\pmclassification{msc}{15A69}
\pmsynonym{multi-linearity}{Multilinear}
\pmsynonym{multilinear}{Multilinear}
\pmsynonym{multilinearity}{Multilinear}
\pmrelated{BilinearForm}
\pmrelated{BilinearMap}

\endmetadata

\usepackage{amsmath}
\usepackage{amsfonts}
\usepackage{amssymb}

\newcommand{\reals}{\mathbb{R}}
\newcommand{\natnums}{\mathbb{N}}
\newcommand{\cnums}{\mathbb{C}}
\newcommand{\znums}{\mathbb{Z}}

\newcommand{\lp}{\left(}
\newcommand{\rp}{\right)}
\newcommand{\lb}{\left[}
\newcommand{\rb}{\right]}

\newcommand{\supth}{^{\text{th}}}


\newtheorem{proposition}{Proposition}
\begin{document}
Let $V_1, V_2,\ldots, V_n, W$ be vector spaces over a field $K$.  A
mapping $$M: V_1\times V_2\times \cdots \times V_n \rightarrow W$$ is
called {\em multi-linear} or $n$-linear, if $M$ is linear in each of
its arguments.

\paragraph{Notes.}
\begin{itemize}
\item  A bilinear mapping is another name for a $2$-linear mapping.
\item This definition generalizes in an obvious way to rings and
  modules.
\item An excellent example of a multi-linear map is the determinant operation.
 \end{itemize}
%%%%%
%%%%%
\end{document}
