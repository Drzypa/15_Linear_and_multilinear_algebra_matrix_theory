\documentclass[12pt]{article}
\usepackage{pmmeta}
\pmcanonicalname{MinorofAMatrix}
\pmcreated{2013-03-22 14:07:01}
\pmmodified{2013-03-22 14:07:01}
\pmowner{CWoo}{3771}
\pmmodifier{CWoo}{3771}
\pmtitle{minor (of a matrix)}
\pmrecord{7}{35522}
\pmprivacy{1}
\pmauthor{CWoo}{3771}
\pmtype{Definition}
\pmcomment{trigger rebuild}
\pmclassification{msc}{15A15}
\pmrelated{LaplaceExpansion}
\pmrelated{CauchyBinetFormula}
\pmdefines{principal minor}
\pmdefines{cofactor}

% this is the default PlanetMath preamble.  as your knowledge
% of TeX increases, you will probably want to edit this, but
% it should be fine as is for beginners.

% almost certainly you want these
\usepackage{amssymb}
\usepackage{amsmath}
\usepackage{amsfonts}

% used for TeXing text within eps files
%\usepackage{psfrag}
% need this for including graphics (\includegraphics)
%\usepackage{graphicx}
% for neatly defining theorems and propositions
%\usepackage{amsthm}
% making logically defined graphics
%%%\usepackage{xypic}

% there are many more packages, add them here as you need them

% define commands here
\def\sse{\subseteq}
\def\bigtimes{\mathop{\mbox{\Huge $\times$}}}
\def\impl{\Rightarrow}
\begin{document}
\PMlinkescapeword{principal}
\PMlinkescapeword{minor}
\PMlinkescapeword{term}
\PMlinkescapeword{order}

Given an $n\times m$ matrix $A$ with entries $a_{ij}$, a \emph{minor} of $A$ is
the determinant of a smaller matrix formed from its entries by selecting
only some of the rows and columns. Let $K=\{k_1,k_2,\ldots,k_p\}$ and
$L=\{l_1,l_2,\ldots,l_p\}$ be subsets of $\{1,2,\ldots,n\}$ and
$\{1,2,\ldots,m\}$, respectively. The indices are chosen such that
$k_1 < k_2 < \cdots < k_p$ and $l_1 < l_2 < \cdots < l_p$. The $p$-th
order minor defined by $K$ and $L$ is the following determinant
\begin{equation*}
  A\begin{pmatrix} k_1 & k_2 & \cdots & k_p \\
    l_1 & l_2 & \cdots & l_p \end{pmatrix} =
  \begin{vmatrix}
    a_{k_1 l_1} & a_{k_1 l_2} & \cdots & a_{k_1 l_p} \\
    a_{k_2 l_1} & a_{k_2 l_2} & \cdots & a_{k_2 l_p} \\
    \vdots      & \vdots      & \ddots & \vdots      \\
    a_{k_p l_1} & a_{k_p l_2} & \cdots & a_{k_p k_p}
  \end{vmatrix}.
\end{equation*}
If $p$ exceeds either $m$ or $n$, then the minor is
automatically zero. When $p=m=n$, the minor is simply the determinant
of the matrix. If $K=L$, then the minor is called \emph{principal}.
The word \emph{minor} may also refer to just the matrix formed from
the selected rows and columns, not necessarily its determinant. The precise
meaning is usually clear from context.

There does not seem to be a standard notation for matrix minors.
Another possible notation is $[A]_{K,L}$.

Some authors reserve the term \emph{minor} for the case when only one
row and one column are removed. This use is in conjunction with the
concept of a \emph{\PMlinkname{cofactor}{LaplaceExpansion}}.
%%%%%
%%%%%
\end{document}
