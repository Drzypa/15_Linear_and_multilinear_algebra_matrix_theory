\documentclass[12pt]{article}
\usepackage{pmmeta}
\pmcanonicalname{InvertibleMatrix}
\pmcreated{2013-03-22 19:23:09}
\pmmodified{2013-03-22 19:23:09}
\pmowner{CWoo}{3771}
\pmmodifier{CWoo}{3771}
\pmtitle{invertible matrix}
\pmrecord{11}{42340}
\pmprivacy{1}
\pmauthor{CWoo}{3771}
\pmtype{Definition}
\pmcomment{trigger rebuild}
\pmclassification{msc}{15-01}
\pmclassification{msc}{15A09}
\pmclassification{msc}{15A33}

\endmetadata

\usepackage{amssymb,amscd}
\usepackage{amsmath}
\usepackage{amsfonts}
\usepackage{mathrsfs}

% used for TeXing text within eps files
%\usepackage{psfrag}
% need this for including graphics (\includegraphics)
%\usepackage{graphicx}
% for neatly defining theorems and propositions
\usepackage{amsthm}
% making logically defined graphics
%%\usepackage{xypic}
\usepackage{pst-plot}

% define commands here
\newcommand*{\abs}[1]{\left\lvert #1\right\rvert}
\newtheorem{prop}{Proposition}
\newtheorem{thm}{Theorem}
\newtheorem{ex}{Example}
\newcommand{\real}{\mathbb{R}}
\newcommand{\pdiff}[2]{\frac{\partial #1}{\partial #2}}
\newcommand{\mpdiff}[3]{\frac{\partial^#1 #2}{\partial #3^#1}}

\begin{document}
Let $R$ be a ring and $M$ an $m\times n$ matrix over $R$.  $M$ is said to be \emph{left invertible} if there is an $n\times m$ matrix such that $NM=I_n$, where $I_n$ is the $n\times n$ identity matrix.  We call $N$ a \emph{left inverse} of $M$.  Similarly, $M$ is \emph{right invertible} if there is an $n\times m$ matrix $P$, called a \emph{right inverse} of $M$, such that $MP=I_m$, where $I_m$ is the $m\times m$ identity matrix.  If $M$ is both left invertible and right invertible, we say that $M$ is \emph{invertible}.  If $R$ is an associative ring, and $M$ is invertible, then it has a unique left and a unique right inverse, and they are in fact equal, we call this matrix the \emph{inverse} of $M$.

If $R$ is a division ring, then it can be shown that for any matrix $M$ over $R$, $M$ is left invertible iff it is invertible iff it is right invertible.  In addition, when $M$ is invertible, it is a square matrix.  Furthermore, $R$ is a field iff for any square matrix $M$ (over $R$), $M$ is invertible implies that $M^T$, its transpose, is invertible as well.  Invertibility of matrices over a division ring can also be determined by quantities known as ranks and determinants.  It can be shown that a matrix over a division ring is invertible iff its left row rank (or right column rank) is full iff its determinant is non-zero.  For example, the $2\times 2$ matrix
$$\begin{pmatrix}
1 & j \\
i & k
\end{pmatrix}$$
over the Hamiltonian quaternions is not invertible, as its determinant $k-ji=0$.  It is interesting to note that, however, its transpose
$$\begin{pmatrix}
1 & i \\
j & k
\end{pmatrix}$$
is invertible, whose determinant is $2k\ne 0$.  The relationship between determinants and matrix invertibility can also be used to prove the following: preservation of matrix invertibility upon matrix transposition implies commutativity of division ring $D$.  This can be done as follows: given any $a,b\in D$, the $2\times 2$ matrix
$$\begin{pmatrix}
ab & b \\
a & 1
\end{pmatrix}$$
is not invertible because its determinant is $0$.  Therefore, its transpose
$$\begin{pmatrix}
ab & a \\
b & 1
\end{pmatrix}$$
is also not invertible, and its determinant is $0=ab-ba$, whence $D$ is a field.

%%%%%
%%%%%
\end{document}
