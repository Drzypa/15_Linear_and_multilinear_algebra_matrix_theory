\documentclass[12pt]{article}
\usepackage{pmmeta}
\pmcanonicalname{TensorArray}
\pmcreated{2013-03-22 12:40:25}
\pmmodified{2013-03-22 12:40:25}
\pmowner{rmilson}{146}
\pmmodifier{rmilson}{146}
\pmtitle{tensor array}
\pmrecord{10}{32949}
\pmprivacy{1}
\pmauthor{rmilson}{146}
\pmtype{Definition}
\pmcomment{trigger rebuild}
\pmclassification{msc}{15A69}
\pmrelated{Frame}
\pmrelated{Vector2}
\pmrelated{BasicTensor}
\pmrelated{TensorProductClassical}
\pmrelated{Tensor}
\pmdefines{covariant index}
\pmdefines{contravariant index}
\pmdefines{valence}

\usepackage{amsmath}
\usepackage{amsfonts}
\usepackage{amssymb}
\newcommand{\reals}{\mathbb{R}}
\newcommand{\natnums}{\mathbb{N}}
\newcommand{\cnums}{\mathbb{C}}
\newcommand{\znums}{\mathbb{Z}}
\newcommand{\lp}{\left(}
\newcommand{\rp}{\right)}
\newcommand{\lb}{\left[}
\newcommand{\rb}{\right]}
\newcommand{\supth}{^{\text{th}}}
\newtheorem{proposition}{Proposition}
\newtheorem{definition}[proposition]{Definition}
\newcommand{\nl}[1]{{\PMlinkescapetext{{#1}}}}
\newcommand{\pln}[2]{{\PMlinkname{#1}{#2}}}
\newcommand{\kfield}{\mathbb{K}}
\newcommand{\rT}{\mathrm{T}}
\newcommand{\tspace}[1]{\rT^{#1}}
\begin{document}
\paragraph{Introduction.}
\PMlinkescapetext{{\em Tensor arrays}}, or tensors for short\footnote{The term {\em tensor} has other meanings, c.f. the \PMlinkname{tensor entry}{Tensor}.}
are multidimensional arrays with two types of (covariant and
contravariant) indices.  Tensors are widely used in science and
mathematics, because these data structures are the natural choice of
representation for a variety of important physical and geometric
quantities.
  
In this entry we give the definition of a tensor array and establish
some related terminology and notation.  The theory of tensor arrays
incorporates a number of other essential topics: basic tensors, tensor
transformations, outer multiplication, contraction, inner
multiplication, and generalized transposition.  These are fully
described in their separate entries.
%  We warn the reader that the Einstein summation convention
% will be in force throughout.

\paragraph{Valences and the space of tensors arrays.}
Let $\kfield$ be a field\footnote{In physics and differential
  geometry, $\kfield$ is typically $\reals$ or $\cnums$.} and let $I$
be a finite list of indices\footnote{It is advantageous to allow
  general indexing sets, because one can indicate the use of multiple
  frames of reference by employing multiple, disjoint sets of
  indices.}, such as $(1,2,\ldots, n)$.  A tensor array of type
$$(p,q),\quad p,q\in\natnums$$
is a mapping 
$$I^p\times I^q\rightarrow \kfield.$$
The set of all such mappings
will be denoted by $\tspace{p,q}(I,\kfield)$, or when $I$ and
$\kfield$ are clear from the context, simply as $\tspace{p,q}$.  The
numbers $p$ and $q$ are called, respectively, the contravariant and
the covariant valence of the tensor array.

Point-wise addition and scaling give $\tspace{p,q}$ the structure of a
a vector space of dimension $n^{p+q}$, where $n$ is the cardinality of
$I$.  We will interpret $I^0$ as signifying a singleton set.
Consequently $\tspace{p,0}$ and $\tspace{0,q}$ are just the maps from,
respectively, $I^p$ and $I^q$ to $\kfield$.  It is also customary to
identify $\tspace{1,0}$ with $\kfield^I$, the vector space of list
vectors indexed by $I$, and to identify $\tspace{0,1}$ with dual space
$\lp\kfield^I\rp^*$ of linear forms on $\kfield^I$.  Finally,
$\tspace{0,0}$ can be identified with $\kfield$ itself. In other
words, scalars are tensor arrays of zero valence.

Let 
$X:I^p\times I^q\rightarrow \kfield$
be a type $(p,q)$ tensor array.
In writing the values of $X$, it is customary to write contravariant
indices using superscripts, and covariant indices using subscripts.
Thus, for indices $i_1,\ldots, i_p, j_1,\ldots,j_q\in I$ we write 
$$X^{i_1\ldots
  i_p}_{j_1\ldots j_q}$$ instead of \footnote{Curiously, the latter notation is preferred by some
  authors.  See H. Weyl's books and papers, for example.}
$$X(i_1,\ldots,i_p;j_1,\ldots,
j_p).$$

We also mention that it is customary to use columns to represent
contravariant index dimensions, and rows to represent the covariant
index dimensions.  Thus column vectors are type $(1,0)$ tensor arrays,
row vectors are type $(0,1)$ tensor arrays, and matrices, in as much
as they can be regarded either as rows of columns or as columns of
rows, are type $(1,1)$ tensor arrays.\footnote{It is also customary to
  use matrices to also represent type $(2,0)$ and type $(0,2)$ tensor
  arrays (The latter are used to represent quadratic forms.)
  Speaking idealistically, such objects should be typeset,
  respectively, as a column of column vectors and as a row of row
  vectors. However typographical constraints and notational
  convenience dictate that they be displayed as matrices.}

\paragraph{Notes.}
It must be noted that our usage of the term {\em tensor array} is
non-standard.  The traditionally inclined authors simply call these
data structures tensors.  We bother to make the distinction because
the traditional nomenclature is ambiguous and doesn't include the
modern mathematical understanding of the tensor concept. (This is
explained more fully in the \PMlinkname{tensor entry}{Tensor}.)  Precise and
meaningful definitions can only be given by treating the concept of a
tensor array as distinct from the concept of a geometric/abstract
tensor.


We also mention that the term {\em tensor} is often applied to objects
that should more appropriately be termed a {\em tensor field}.  The
latter are tensor-valued functions, or more generally sections of a
tensor bundle.  A tensor is what one gets by evaluating a tensor field
at one point.  Informally, one can also think of a tensor field as a
tensor whose values are functions, rather than constants.
%%%%%
%%%%%
\end{document}
