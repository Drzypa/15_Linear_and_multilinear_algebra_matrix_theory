\documentclass[12pt]{article}
\usepackage{pmmeta}
\pmcanonicalname{Eigenvalue1}
\pmcreated{2013-03-22 14:01:53}
\pmmodified{2013-03-22 14:01:53}
\pmowner{drini}{3}
\pmmodifier{drini}{3}
\pmtitle{eigenvalue}
\pmrecord{8}{35106}
\pmprivacy{1}
\pmauthor{drini}{3}
\pmtype{Definition}
\pmcomment{trigger rebuild}
\pmclassification{msc}{15A18}
\pmrelated{LinearTransformation}
\pmrelated{Scalar}
\pmrelated{Vector}
\pmrelated{Kernel}
\pmrelated{Dimension2}
\pmdefines{eigenvector}

\endmetadata

\usepackage{graphicx}
%%%\usepackage{xypic} 
\usepackage{bbm}
\newcommand{\Z}{\mathbbmss{Z}}
\newcommand{\C}{\mathbbmss{C}}
\newcommand{\R}{\mathbbmss{R}}
\newcommand{\Q}{\mathbbmss{Q}}
\newcommand{\mathbb}[1]{\mathbbmss{#1}}
\newcommand{\figura}[1]{\begin{center}\includegraphics{#1}\end{center}}
\newcommand{\figuraex}[2]{\begin{center}\includegraphics[#2]{#1}\end{center}}
\begin{document}
Let $V$ be a vector space over $k$ and $T$  a linear operator on $V$.
An eigenvalue for $T$ is an scalar $\lambda$ (that is, an element of $k$) such that
$T(z)=\lambda z$ for some nonzero vector $z\in V$.
Is that case, we also say that $z$ is an eigenvector of $T$.

This can also be expressed as follows: $\lambda$ is an eigenvalue for $T$ if the kernel of $A-\lambda I$ is non trivial.

A linear operator can have several eigenvalues (but no more than the dimension of the space). Eigenvectors corresponding to different eigenvalues are linearly independent.
%%%%%
%%%%%
\end{document}
