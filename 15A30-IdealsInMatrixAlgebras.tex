\documentclass[12pt]{article}
\usepackage{pmmeta}
\pmcanonicalname{IdealsInMatrixAlgebras}
\pmcreated{2013-03-22 13:59:28}
\pmmodified{2013-03-22 13:59:28}
\pmowner{mathcam}{2727}
\pmmodifier{mathcam}{2727}
\pmtitle{ideals in matrix algebras}
\pmrecord{10}{34767}
\pmprivacy{1}
\pmauthor{mathcam}{2727}
\pmtype{Topic}
\pmcomment{trigger rebuild}
\pmclassification{msc}{15A30}

% this is the default PlanetMath preamble.  as your knowledge
% of TeX increases, you will probably want to edit this, but
% it should be fine as is for beginners.

% almost certainly you want these
\usepackage{amssymb}
\usepackage{amsmath}
\usepackage{amsfonts}

% used for TeXing text within eps files
%\usepackage{psfrag}
% need this for including graphics (\includegraphics)
%\usepackage{graphicx}
% for neatly defining theorems and propositions
\usepackage{amsthm}
% making logically defined graphics
%%%\usepackage{xypic}

% there are many more packages, add them here as you need them

% define commands here
\newtheorem*{claim}{Claim}
\begin{document}
Let $R$ be a ring with 1. Consider the ring $M_{n \times n}(R)$ of $n \times n$-matrices with entries taken from $R$.

It will be shown that there exists a one-to-one 
correspondence between the (two-sided) ideals of $R$ and the 
(two-sided) ideals of $M_{n \times n}(R)$.

For $1 \le i,j \le n$, let $E_{ij}$ denote the $n 
\times n$-matrix having entry 1 at position 
$(i,j)$ and 0 in all other places. It can be 
easily checked that
\begin{equation}
E_{ij} \cdot E_{kl}=\left\{
\begin{array}{lllll}
0 & \mbox{iff}& k \ne j \\
E_{il} & \mbox{otherwise.}
\end{array}\right.
\end{equation}
Let $\mathfrak{m}$ be an ideal in $M_{n \times 
n}(R)$.
\begin{claim}
The set $\mathfrak{i}\subseteq R$ given by
\[\mathfrak{i}=\{x \in R \mid x\quad\mbox{is an entry 
of } A \in \mathfrak{m}\}\]
is an ideal in $R$, and 
$\mathfrak{m}=M_{n \times n}(\mathfrak{i})$.
\end{claim}
\begin{proof}
$\mathfrak{i} \ne \emptyset$ since $0 \in 
\mathfrak{i}$. Now let $A=(a_{ij})$ and $B=(b_{ij})$ 
be matrices in $\mathfrak{m}$, and $x,y \in R$ be 
entries of $A$ and $B$ respectively, say 
$x=a_{ij}$ and $y=b_{kl}$. Then the matrix $A \cdot 
E_{jl} +E_{ik}\cdot B \in \mathfrak{m}$ has $x+y$ 
at position $(i,l)$, and it follows: If $x,y \in 
\mathfrak{i}$, then $x+y \in \mathfrak{i}$. Since 
$\mathfrak{i}$ is an ideal in $M_{n \times n}(R)$ 
it contains, in particular, the matrices $D_r \cdot A$ and $A \cdot D_r$, where 
\begin{equation*}
D_r :=\sum_{i=1}^n r\cdot E_{ii}, r \in R.
\end{equation*}
thus, $rx, xr \in \mathfrak{i}$. This shows 
that $\mathfrak{i}$ is an ideal in $R$. 
Furthermore, $M_{n \times n}(\mathfrak{i}) 
\subseteq \mathfrak{m}$.

By construction, any matrix $A \in \mathfrak{m}$ 
has entries in $\mathfrak{i}$, so we have
\begin{equation*}
A=\sum\limits_{1 \le i,j \le n} a_{ij}E_{ij}, 
a_{ij} \in \mathfrak{i}
\end{equation*}
so $A \in m_{n \times n}(\mathfrak{i})$. Therefore 
$\mathfrak{m} \subseteq M_{n \times n}(\mathfrak{i})$.
\end{proof}
A consequence of this is: If $F$ is a field, then $M_{n \times n}(F)$ is simple.
%%%%%
%%%%%
\end{document}
