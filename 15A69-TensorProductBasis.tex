\documentclass[12pt]{article}
\usepackage{pmmeta}
\pmcanonicalname{TensorProductBasis}
\pmcreated{2013-03-22 15:24:48}
\pmmodified{2013-03-22 15:24:48}
\pmowner{lars_h}{9802}
\pmmodifier{lars_h}{9802}
\pmtitle{tensor product basis}
\pmrecord{11}{37254}
\pmprivacy{1}
\pmauthor{lars_h}{9802}
\pmtype{Theorem}
\pmcomment{trigger rebuild}
\pmclassification{msc}{15A69}
\pmsynonym{basis construction of tensor product}{TensorProductBasis}
\pmrelated{TensorProduct}
\pmrelated{FreeVectorSpaceOverASet}

\endmetadata

% almost certainly you want these
\usepackage{amssymb}
\usepackage{amsmath}
\usepackage{amsfonts}

% used for TeXing text within eps files
%\usepackage{psfrag}
% need this for including graphics (\includegraphics)
%\usepackage{graphicx}
% for neatly defining theorems and propositions

\usepackage{amsthm}
% \newtheorem*{theorem}{Theorem}
\newenvironment{theorem}{%
   \trivlist\item\relax
   \textbf{Theorem.}\hspace*{0.5em}%
   \ignorespaces
}{\endtrivlist}

% making logically defined graphics
%%%\usepackage{xypic}

% there are many more packages, add them here as you need them

% define commands here

\newcommand{\Fpil}{\longrightarrow}
\newcommand{\vek}[1]{\mathbf{#1}}
\newcommand{\K}{\mathcal{K}}
\newcommand{\gobble}[1]{}
\newcommand*{\setOfBig}[2]{%
   \left\{ \, #1 \,\,\vrule\relax\big.\,\, #2 \, \right\}%
}
\begin{document}
The following theorem describes a basis of the 
\PMlinkname{tensor product}{TensorProduct} 
of two vector spaces, in terms of given bases of the 
\PMlinkescapetext{component} 
spaces. In passing, it also gives a construction of this tensor 
product. The exact same method can be used also for free 
modules over a commutative ring with unit.

% % 2007-09-01: Seems to work now (at least in preview), 
% % but since I'm not overly optimistic about the problem 
% % really having been solved, I'll for now leave this piece 
% % of text as a comment.
% \textbf{Apology:} For some reason the PlanetMath renderer 
% doesn't seem to like this entry---formulas (even single 
% letter ones like $x$) tend to be included in the wrong places 
% in the HTML version of this page, and some are just garbage. 
% If this problem still persists when you try to view this page, 
% then you may get around it by selecting an alternative 
% viewing format.

\PMlinkescapeword{terms}
\PMlinkescapeword{bases}
\PMlinkescapephrase{tensor product}
\PMlinkescapeword{observation}


\begin{theorem}
  Let $U$ and $V$ be vector spaces over a field $\K$ with bases 
  $$
    \{\vek{e}_i\}_{i \in I} \quad\text{and}\quad 
    \{\vek{f}_j\}_{j \in J}
  $$
  respectively. Then
  \begin{equation} \label{Eq:ProdBas}
    \{ \vek{e}_i \otimes \vek{f}_j\}_{(i,j) \in I \times J}
  \end{equation}
  is a basis for the tensor product space $U \otimes V$.
\end{theorem}
\begin{proof}
  Let
  $$
    W = \setOfBig{ \psi\colon I \times J \Fpil \K }{
      f^{-1}\bigl( \K \setminus \{0\} \bigr) \text{ is finite}
    }\text{;}
  $$
  this set is obviously a $\K$-vector-space under pointwise addition 
  and multiplication by scalar (see also 
  \PMlinkname{this}{FreeVectorSpaceOverASet} article). 
  Let \(p\colon U \times V \Fpil W\) be 
  the bilinear map which satisfies
  \begin{equation} \label{Eq:def.p}
    p(\vek{e}_i,\vek{f}_j)(k,l) = \begin{cases}
      1& \text{if \(i=k\) and \(j=l\),}\\
      0& \text{otherwise}
    \end{cases}
  \end{equation}
  for all \(i,k \in I\) and \(j,l \in J\), i.e., 
  \(p(\vek{e}_i,\vek{f}_j) \in W\) is the characteristic function of 
  $\bigl\{(i,j)\bigr\}$. The reasons \eqref{Eq:def.p} uniquely 
  defines $p$ on the whole of $U \times V$ are that 
  $\{\vek{e}_i\}_{i \in I}$ is a basis of $U$, $\{\vek{f}_i\}_{j \in 
  J}$ is a basis of $V$, and $p$ is bilinear.
  
  Observe that
  $$
    \bigl\{ p(\vek{e}_i,\vek{f}_j) \bigr\}_{(i,j) \in I \times J}
  $$
  is a basis of $W$. Since one may always define a linear map 
  by giving its values on the basis elements, this implies that there for every 
  $\K$-vector-space $X$ and every map \(\gamma\colon U \times V \Fpil 
  X\) exists a unique linear map \(\widehat{\gamma}\colon W \Fpil X\) 
  such that
  $$
    \widehat{\gamma}\bigl( p(\vek{e}_i,\vek{f}_j) \bigr) =
    \gamma(\vek{e}_i,\vek{f}_j)
    \quad\text{for all \(i \in I\) and \(j \in J\).}
  $$
  For $\gamma$ that are bilinear it holds for arbitrary \(\vek{u} = 
  \sum_{i \in I'} u_i\vek{e}_i \in U\) and \(\vek{v} = 
  \sum_{j \in J'} v_j\vek{f}_j \in V\) that \(\gamma(\vek{u},\vek{v}) 
  = (\widehat{\gamma} \circ\nobreak p)(\vek{u},\vek{v})\), since
  \begin{multline*}
    \gamma(\vek{u},\vek{v}) =
    \gamma\biggl( \sum_{i \in I'} u_i\vek{e}_i, 
      \sum_{j \in J'} v_j\vek{f}_j \biggr) =
    \sum_{i \in I'} \sum_{j \in J'} u_i v_j 
      \gamma(\vek{e}_i,\vek{f}_j) = \\ =
    \sum_{i \in I'} \sum_{j \in J'} u_i v_j 
      \widehat{\gamma}\bigl( p(\vek{e}_i,\vek{f}_j) \bigr) =
    \widehat{\gamma} \biggl(
      \sum_{i \in I'} \sum_{j \in J'} u_i v_j p(\vek{e}_i,\vek{f}_j)
      \biggr) = \\ =
    \widehat{\gamma} \Biggl(
      p\biggl( \sum_{i \in I'} u_i \vek{e}_i,
      \sum_{j \in J'} v_j \vek{f}_j \biggr)
      \Biggr) =
    \widehat{\gamma}\bigl( p(\vek{u},\vek{v}) \bigr)
    \text{.}
  \end{multline*}
  As this is the defining property of the tensor product $U \otimes 
  V$ however, it follows that $W$ is (an incarnation of) this tensor 
  product, with \(\vek{u} \otimes \vek{v} := p(\vek{u},\vek{v})\). 
  Hence the claim in the theorem is equivalent to the observation 
  about the basis of $W$.
\end{proof}
%%%%%
%%%%%
\end{document}
