\documentclass[12pt]{article}
\usepackage{pmmeta}
\pmcanonicalname{Mmatrix}
\pmcreated{2013-03-22 15:24:54}
\pmmodified{2013-03-22 15:24:54}
\pmowner{kshum}{5987}
\pmmodifier{kshum}{5987}
\pmtitle{M-matrix}
\pmrecord{7}{37257}
\pmprivacy{1}
\pmauthor{kshum}{5987}
\pmtype{Definition}
\pmcomment{trigger rebuild}
\pmclassification{msc}{15A57}

% this is the default PlanetMath preamble.  as your knowledge
% of TeX increases, you will probably want to edit this, but
% it should be fine as is for beginners.

% almost certainly you want these
\usepackage{amssymb}
\usepackage{amsmath}
\usepackage{amsfonts}

% used for TeXing text within eps files
%\usepackage{psfrag}
% need this for including graphics (\includegraphics)
%\usepackage{graphicx}
% for neatly defining theorems and propositions
%\usepackage{amsthm}
% making logically defined graphics
%%%\usepackage{xypic}

% there are many more packages, add them here as you need them

% define commands here
\begin{document}
A Z-matrix $A$ is called an \emph{M-matrix} if it satisfies any one of
the following equivalent conditions.


\begin{enumerate}

\item All principal minors of $A$ are positive.

\item The leading principal minors of $A$ are positive.

\item $A$ can be written in the form $A=kI-B$, where $B$ is a
non-negative matrix whose spectral radius is strictly less than
$k$.

\item All real eigenvalues of $A$ are positive.

\item The real part of any eigenvalue of $A$ is positive.

\item $A$ is non-singular and the inverse of $A$ is non-negative.

\item $Av \geq 0 $ implies $v\geq 0$.

\item There exists a vector $v$ with non-negative entries such
that $Av > 0$.

\item $A+D$ is non-singular for every non-negative diagonal matrix
$D$.

\item $A+kI$ is non-singular for all $k\geq 0$.

\item For each nonzero vector $v$, $v_i (Av)_i>0$ for some $i$.

\item There is a positive diagonal matrix $D$ such that the matrix
$DA + A^TD$ is positive definite.

\item $A$ can be factorized as $LU$, where $L$ is lower
triangular, $U$ is upper triangular, and the diagonal entries of
both $L$ and $U$ are positive.

\item The diagonal entries of $A$ are positive and $AD$ is
strictly diagonally dominant for some positive diagonal matrix
$D$.
\end{enumerate}


{\bf Reference:}


M. Fiedler, {\it Special Matrices and Their Applications in
Numerical Mathematics}, Martinus Nijhoff, Dordrecht, 1986.

R. A. Horn and C. R. Johnson, {\it Topics in Matrix Analysis},
Cambridge University Press, Cambridge, 1991.
%%%%%
%%%%%
\end{document}
