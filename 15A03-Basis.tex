\documentclass[12pt]{article}
\usepackage{pmmeta}
\pmcanonicalname{Basis}
\pmcreated{2013-03-22 12:01:57}
\pmmodified{2013-03-22 12:01:57}
\pmowner{mathcam}{2727}
\pmmodifier{mathcam}{2727}
\pmtitle{basis}
\pmrecord{22}{31041}
\pmprivacy{1}
\pmauthor{mathcam}{2727}
\pmtype{Definition}
\pmcomment{trigger rebuild}
\pmclassification{msc}{15A03}
\pmsynonym{Hamel basis}{Basis}
%\pmkeywords{span}
%\pmkeywords{vector space}
%\pmkeywords{basis}
\pmrelated{Span}
\pmrelated{IntegralBasis}
\pmrelated{BasicTensor}
\pmrelated{Aliasing}
\pmrelated{Subbasis}
\pmrelated{Blade}
\pmrelated{ProofOfGramSchmidtOrthogonalizationProcedure}
\pmrelated{LinearExtension}

\usepackage{amssymb}
\usepackage{amsmath}
\usepackage{amsfonts}
\usepackage{graphicx}
%%%\usepackage{xypic}
\begin{document}
\PMlinkescapeword{basis}
\PMlinkescapeword{bases}

A (Hamel) basis of a vector space is a linearly independent spanning set.

It can be proved that any two bases of the same vector space must have the same cardinality. This introduces the notion of dimension of a vector space, which is precisely the cardinality of the basis, and is denoted by $\operatorname{dim}(V)$, where $V$ is the vector space.

The fact that \PMlinkname{every vector space has a Hamel basis}{EveryVectorSpaceHasABasis} is an important consequence of the axiom of choice (in fact, that proposition is equivalent to the axiom of choice.)

{\bf Examples.}

\begin{itemize}

\item $\beta = \{e_i\}$, $1\le i \le n$, is a basis for $\mathbb{R}^n$ (the $n$-dimensional vector space over the reals).  For $n=4$, 

$$ \beta = \left\{ \begin{pmatrix} 1 \\ 0 \\ 0 \\ 0 \end{pmatrix} ,
\begin{pmatrix} 0 \\ 1 \\ 0 \\ 0 \end{pmatrix} ,
\begin{pmatrix} 0 \\ 0 \\ 1 \\ 0 \end{pmatrix} ,
\begin{pmatrix} 0 \\ 0 \\ 0 \\ 1 \end{pmatrix} \right\}   $$

\item $ \beta = \{ 1, x , x^2 \} $ is a basis for the vector space of polynomials with degree at most 2, over a division ring.

\item The set

$$ \beta =   
\left\{ \begin{bmatrix}1 & 0 \\ 0 & 0 \end{bmatrix} ,
\begin{bmatrix}0 & 1 \\ 0 & 0 \end{bmatrix} ,
\begin{bmatrix}0 & 0 \\ 0 & 1 \end{bmatrix} ,
\begin{bmatrix}0 & 0 \\ 1 & 0 \end{bmatrix} \right\} $$ 

is a basis for the vector space of $2 \times 2$ matrices over a division ring, and assuming that the characteristic of the ring is not 2, then so is

$$ \beta' =   
\left\{ \begin{bmatrix}2 & 0 \\ 0 & 0 \end{bmatrix} ,
\begin{bmatrix}0 & 1 \\ 0 & 0 \end{bmatrix} ,
\begin{bmatrix}0 & 0 \\ 0 & 4 \end{bmatrix} ,
\begin{bmatrix}0 & 0 \\ \frac{1}{2} & 0 \end{bmatrix} \right\}. $$

\item The empty set is a basis for the trivial vector space which consists of the unique element $0$. 

\end{itemize}

\textbf{Remark}.  More generally, for any (left) right module $M$ over a ring $R$, one may define a (left) right basis for $M$ as a subset $B$ of $M$ such that $B$ spans $M$ and is linearly independent.  However, unlike bases for a vector space, bases for a module may not have the same cardinality.
%%%%%
%%%%%
%%%%%
\end{document}
