\documentclass[12pt]{article}
\usepackage{pmmeta}
\pmcanonicalname{BottaPierceWatkinsTheorem}
\pmcreated{2013-03-22 19:20:21}
\pmmodified{2013-03-22 19:20:21}
\pmowner{kammerer}{26336}
\pmmodifier{kammerer}{26336}
\pmtitle{Botta - Pierce - Watkins theorem}
\pmrecord{6}{42287}
\pmprivacy{1}
\pmauthor{kammerer}{26336}
\pmtype{Theorem}
\pmcomment{trigger rebuild}
\pmclassification{msc}{15A04}
%\pmkeywords{linear preserver}
%\pmkeywords{nilpotent matrix}
\pmrelated{FundamentalTheoremOfProjectiveGeometry}
\pmrelated{GerstenhaberSerezhkinTheorem}

\endmetadata

% this is the default PlanetMath preamble.  as your knowledge
% of TeX increases, you will probably want to edit this, but
% it should be fine as is for beginners.

% almost certainly you want these
\usepackage{amssymb}
\usepackage{amsmath}
\usepackage{amsfonts}

% used for TeXing text within eps files
%\usepackage{psfrag}
% need this for including graphics (\includegraphics)
%\usepackage{graphicx}
% for neatly defining theorems and propositions
%\usepackage{amsthm}
% making logically defined graphics
%%%\usepackage{xypic}

% there are many more packages, add them here as you need them

% define commands here
\newtheorem{thm}{Theorem}

\begin{document}
Let $\mathbb{F}$ be an arbitrary field, and let $n$ be a positive integer. Consider $\mathcal{M}_n (\mathbb{F}),$ the vector space of all $n \times n$ matrices over $\mathbb{F}.$ Define
\begin{itemize}
\item
$\mathfrak{sl}_n (\mathbb{F}) = \{A \in \mathcal{M}_n (\mathbb{F}):\, {\rm tr} (A) = 0\},$
\item
$\mathcal{N} = \{A \in \mathcal{M}_n (\mathbb{F}):\, A\, \, \mbox{is nilpotent}\},$
\item
$\mathcal{GL}_n (\mathbb{F}) = \{A \in \mathcal{M}_n (\mathbb{F}):\, \det (A) \neq 0\}.$
\end{itemize}

Notice that $\mathfrak{sl}_n (\mathbb{F})$ is a linear subspace of $\mathcal{M}_n (\mathbb{F})$ and $\mathcal{N} \subseteq \mathfrak{sl}_n (\mathbb{F}).$

The Botta -- Pierce -- Watkins theorem on linear preservers of the nilpotent matrices \cite{BPW} can be formulated as follows.
\begin{thm}
Let $\varphi : \mathfrak{sl}_n (\mathbb{F}) \longrightarrow \mathfrak{sl}_n (\mathbb{F})$ be a linear automorphism. Assume that $\varphi (\mathcal{N}) \subseteq \mathcal{N}.$ Then either $\exists\, P \in \mathcal{GL}_n (\mathbb{F})\, \exists\, c \in \mathbb{F} \setminus \{0\}\, \forall\, A \in \mathfrak{sl}_n (\mathbb{F}) :\, \varphi (A) = c P A P^{-1},$ or $\exists\, P \in \mathcal{GL}_n (\mathbb{F})\, \exists\, c \in \mathbb{F} \setminus \{0\}\, \forall\, A \in \mathfrak{sl}_n (\mathbb{F}) :\, \varphi (A) = c P A^{\rm T} P^{-1}.$
\end{thm}

The original proof is based on the Gerstenhaber - Serezhkin theorem, some elementary algebraic geometry, and the fundamental theorem of projective geometry.
\begin{thebibliography}{99}
\bibitem[BPW]{BPW}
P. Botta, S. Pierce, W. Watkins, Linear transformations that preserve the nilpotent matrices, \emph{Pacific J. Math.} {\bf 104} (No. 1): 39--46 (1983).
\end{thebibliography}
%%%%%
%%%%%
\end{document}
