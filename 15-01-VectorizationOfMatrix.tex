\documentclass[12pt]{article}
\usepackage{pmmeta}
\pmcanonicalname{VectorizationOfMatrix}
\pmcreated{2013-03-22 18:12:20}
\pmmodified{2013-03-22 18:12:20}
\pmowner{pahio}{2872}
\pmmodifier{pahio}{2872}
\pmtitle{vectorization of matrix}
\pmrecord{4}{40785}
\pmprivacy{1}
\pmauthor{pahio}{2872}
\pmtype{Definition}
\pmcomment{trigger rebuild}
\pmclassification{msc}{15-01}
\pmsynonym{vectorization}{VectorizationOfMatrix}
\pmsynonym{vectorisation}{VectorizationOfMatrix}
\pmrelated{transpose}

\endmetadata

% this is the default PlanetMath preamble.  as your knowledge
% of TeX increases, you will probably want to edit this, but
% it should be fine as is for beginners.

% almost certainly you want these
\usepackage{amssymb}
\usepackage{amsmath}
\usepackage{amsfonts}

% used for TeXing text within eps files
%\usepackage{psfrag}
% need this for including graphics (\includegraphics)
%\usepackage{graphicx}
% for neatly defining theorems and propositions
 \usepackage{amsthm}
% making logically defined graphics
%%%\usepackage{xypic}

% there are many more packages, add them here as you need them

% define commands here

\theoremstyle{definition}
\newtheorem*{thmplain}{Theorem}

\begin{document}
The {\em vectorization} of a $m\!\times\!n$ matrix \,$A = (a_{ij})$\, \PMlinkescapetext{transforms} the matrix to a column vector $\mbox{vec}(A)$, which consists of all columns of $A$ stacked in \PMlinkescapetext{sequence}: 
$$\mbox{vec}(A) \;:=\; \left(a_{11}\;\;a_{21}\;\ldots\;a_{m1}\;\;\;a_{12}\;\;a_{22}\;\ldots\;a_{m2}\;\;
\ldots\;\ldots\;\;a_{1n}\;\;a_{2n}\;\ldots\;a_{mn}\right)^\intercal$$
The mapping vec from the vector space formed by the $m\!\times\!n$ matrices to the vector space of the column vectors of the \PMlinkescapetext{length} $mn$ is apparently a linear transformation.

%%%%%
%%%%%
\end{document}
