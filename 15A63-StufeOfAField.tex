\documentclass[12pt]{article}
\usepackage{pmmeta}
\pmcanonicalname{StufeOfAField}
\pmcreated{2013-03-22 15:06:01}
\pmmodified{2013-03-22 15:06:01}
\pmowner{CWoo}{3771}
\pmmodifier{CWoo}{3771}
\pmtitle{stufe of a field}
\pmrecord{5}{36829}
\pmprivacy{1}
\pmauthor{CWoo}{3771}
\pmtype{Definition}
\pmcomment{trigger rebuild}
\pmclassification{msc}{15A63}
\pmclassification{msc}{12D15}
\pmsynonym{level of a field}{StufeOfAField}
\pmrelated{TheoremsOnSumsOfSquares}
\pmdefines{stufe}

% this is the default PlanetMath preamble.  as your knowledge
% of TeX increases, you will probably want to edit this, but
% it should be fine as is for beginners.

% almost certainly you want these
\usepackage{amssymb,amscd}
\usepackage{amsmath}
\usepackage{amsfonts}

% used for TeXing text within eps files
%\usepackage{psfrag}
% need this for including graphics (\includegraphics)
%\usepackage{graphicx}
% for neatly defining theorems and propositions
%\usepackage{amsthm}
% making logically defined graphics
%%%\usepackage{xypic}

% there are many more packages, add them here as you need them

% define commands here
\begin{document}
The \emph{stufe} of a field $F$ is the least number $n$ such that $-1$ can be expressed as a sum of $n$ squares:
$$-1=a_1^2+\cdots+a_n^2,$$
where each $a_i\in F$.  If no such an $n$ exists, then we say that the \emph{stufe} of $F$ is $\infty$.

\textbf{Remarks}.
\begin{itemize}
\item The word ``stufe'', meaning ``level'' in German, is attributed to mathematician Albrecht Pfister.
\item A theorem of Pfister asserts that in a field $F$, if $-1$ can be expressed as a finite sum of squares, then the stufe of $F$ is a power of $2$.
\end{itemize}

\begin{thebibliography}{8}
\bibitem{pfister} A. Pfister, {\em Zur Darstellung definiter Funktionen als Summe von Quadraten}, Inventiones Mathematicae (1967).
\bibitem{rajwade} A. R. Rajwade, {\em Squares}, Cambridge University Press (1993).
\end{thebibliography}
%%%%%
%%%%%
\end{document}
