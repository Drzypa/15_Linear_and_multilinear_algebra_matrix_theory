\documentclass[12pt]{article}
\usepackage{pmmeta}
\pmcanonicalname{EquitableMatrix}
\pmcreated{2013-03-22 14:58:28}
\pmmodified{2013-03-22 14:58:28}
\pmowner{matte}{1858}
\pmmodifier{matte}{1858}
\pmtitle{equitable matrix}
\pmrecord{11}{36674}
\pmprivacy{1}
\pmauthor{matte}{1858}
\pmtype{Definition}
\pmcomment{trigger rebuild}
\pmclassification{msc}{15-00}
\pmrelated{HadamardProduct}

\endmetadata

% this is the default PlanetMath preamble.  as your knowledge
% of TeX increases, you will probably want to edit this, but
% it should be fine as is for beginners.

% almost certainly you want these
\usepackage{amssymb}
\usepackage{amsmath}
\usepackage{amsfonts}
\usepackage{amsthm}

\usepackage{mathrsfs}

% used for TeXing text within eps files
%\usepackage{psfrag}
% need this for including graphics (\includegraphics)
%\usepackage{graphicx}
% for neatly defining theorems and propositions
%
% making logically defined graphics
%%%\usepackage{xypic}

% there are many more packages, add them here as you need them

% define commands here

\newcommand{\sR}[0]{\mathbb{R}}
\newcommand{\sC}[0]{\mathbb{C}}
\newcommand{\sN}[0]{\mathbb{N}}
\newcommand{\sZ}[0]{\mathbb{Z}}

 \usepackage{bbm}
 \newcommand{\Z}{\mathbbmss{Z}}
 \newcommand{\C}{\mathbbmss{C}}
 \newcommand{\R}{\mathbbmss{R}}
 \newcommand{\Q}{\mathbbmss{Q}}



\newcommand*{\norm}[1]{\lVert #1 \rVert}
\newcommand*{\abs}[1]{| #1 |}



\newtheorem{thm}{Theorem}
\newtheorem{defn}{Definition}
\newtheorem{prop}{Proposition}
\newtheorem{lemma}{Lemma}
\newtheorem{cor}{Corollary}
\begin{document}
Equitable matrices have been used in economics and group theory
 \cite{eves}. 

\begin{defn} An $n\times n$ matrix $M=(m_{ij})$ is an 
\emph{equitable matrix} if all $m_{ij}$ are positive, and 
$m_{ij}=m_{ik} m_{kj}$ for all $i,j,k=1,\ldots, n$.
\end{defn}

Setting $i=j=k$ yields $m_{ii}=m_{ii} m_{ii}$ so diagonal elements of
equitable matrices equal $1$. Next, setting $i=j$ yields
$m_{ii}=m_{ik} m_{ki}$, so $m_{ik} =1/m_{ki}$.

\subsubsection*{Examples}
\begin{enumerate}
\item 
An example of an equitable matrix of order $n$ is
$$
  \begin{pmatrix} 1 & \cdots & 1 \\
                  \vdots & \ddots & \vdots \\
                  1 & \cdots & 1 \end{pmatrix}.
$$
This example shows that equitable matrices exist for all $n$. 
\item The most general equitable matrix of orders $2$ and $3$ are
$$
  \begin{pmatrix} 1 & a \\
                  1/a & 1 \end{pmatrix},
$$
and
$$
  \begin{pmatrix} 1 & a & ab  \\
                  1/a & 1 &b \\
                  1/ab & 1/b & 1 \end{pmatrix},
$$
where $a,b,c>0$. 

\end{enumerate}

\subsubsection*{Properties}
\begin{enumerate}
\item A $n \times n$ matrix $M=(m_{ij})$ is equitable if and 
only if it can be expressed in the form
$$
  m_{ij} = \exp (\lambda_i - \lambda_j)
$$
for real numbers $\lambda_1, \lambda_2, \ldots, \lambda_n$ with $\lambda_1 = 0$. (\PMlinkname{proof.}{ParameterizationOfEquitableMatrices})
\item An equitable matrix is completely determined by its first row. 
If $m_{1i}$, $i=1, \ldots, n$ are known, then 
$$
   m_{ij} = \frac{m_{1j}}{m_{1i}}.
$$
\item If $M$ is an $n\times n$ equitable matrix, then 
$$
  \operatorname{exp}(M) = I + \frac{e^n-1}{n} M,
$$
where $\operatorname{exp}$ is the matrix exponential. 
\item Equitable matrices form a group under the Hadamard product \cite{eves}.
\item If $M$ is an $n\times n$ equitable matrix and $s\colon \{1,\ldots, r\}\to \{1, \ldots, n\}$
is a mapping, then 
$$
   K_{ab} = M_{s(a)\, s(b)}, \quad a,b=1,\ldots, r
$$
is an equitable $r\times r$ matrix. In particular, striking the $l$:th row and column in an 
equitable matrix yields a new equitable matrix. 
\end{enumerate}

See \cite{eves} for further properties and references.

\begin{thebibliography}{9}
\bibitem {eves} H. Eves, \emph{Elementary Matrix Theory}, Dover publications, 1980.
\end{thebibliography}
%%%%%
%%%%%
\end{document}
