\documentclass[12pt]{article}
\usepackage{pmmeta}
\pmcanonicalname{PropertiesOfRanksOfMatrices}
\pmcreated{2013-03-22 19:23:23}
\pmmodified{2013-03-22 19:23:23}
\pmowner{CWoo}{3771}
\pmmodifier{CWoo}{3771}
\pmtitle{properties of ranks of matrices}
\pmrecord{8}{42345}
\pmprivacy{1}
\pmauthor{CWoo}{3771}
\pmtype{Definition}
\pmcomment{trigger rebuild}
\pmclassification{msc}{15A03}

\endmetadata

\usepackage{amssymb,amscd}
\usepackage{amsmath}
\usepackage{amsfonts}
\usepackage{mathrsfs}
\usepackage{tabls}

% used for TeXing text within eps files
%\usepackage{psfrag}
% need this for including graphics (\includegraphics)
%\usepackage{graphicx}
% for neatly defining theorems and propositions
\usepackage{amsthm}
% making logically defined graphics
%%\usepackage{xypic}
\usepackage{pst-plot}

% define commands here
\newcommand*{\abs}[1]{\left\lvert #1\right\rvert}
\newtheorem{prop}{Proposition}
\newtheorem{thm}{Theorem}
\newtheorem{ex}{Example}
\newcommand{\real}{\mathbb{R}}
\newcommand{\pdiff}[2]{\frac{\partial #1}{\partial #2}}
\newcommand{\mpdiff}[3]{\frac{\partial^#1 #2}{\partial #3^#1}}

\begin{document}
Let $D$ be a division ring and $M$ a matrix over $D$.  In \PMlinkname{this entry}{RankOfAMatrix}, the ranks of $M$ are defined, and the following properties regarding ranks and elementary matrix operations are shown in the table below:
\begin{center}
preservation of ranks of $M$
\begin{tabular}{|p{8cm}|p{1cm}|p{1cm}|p{1cm}|p{1cm}|}
\hline
operation & left row rank & right row rank & left column rank & right column rank \\
\hline\hline
row exchange & yes & yes & yes & yes \\
\hline
column exchange & yes & yes & yes & yes \\
\hline
row addition & yes & yes & yes & yes \\
\hline
column addition & yes & yes & yes & yes \\
\hline
left non-zero row scalar multiplication & yes & no & no & yes \\
\hline
left non-zero column scalar multiplication & no & yes & yes & no \\
\hline
right non-zero row scalar multiplication & no & yes & yes & no \\
\hline
right non-zero column scalar multiplication & yes & no & no & yes \\
\hline
\end{tabular}
\end{center}
From the properties above, one sees that 
\begin{center}
left column rank &=& right row rank, $\qquad$ right column rank &=& left row rank.
\end{center}
If $D$ is a field, all ranks of $M$ are identical, since left and right multiplications are the same.  Now, back to assuming $D$ a division ring.  Refer the right ranks of $M$ to be either the right row rank or right column rank of $M$.  Left ranks, row ranks, and column ranks are similarly referred.  Ranks of $M$ can be either one of the four ranks of $M$.

The following are additional properties of ranks of matrices of $D$:
\begin{enumerate}
\item if $r\ne 0$, then $rM$ and $Mr$ have the same ranks as those of $M$.
\end{enumerate}

More to come...

%%%%%
%%%%%
\end{document}
