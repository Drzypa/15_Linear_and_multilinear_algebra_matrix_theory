\documentclass[12pt]{article}
\usepackage{pmmeta}
\pmcanonicalname{CharacteristicPolynomialOfAOrthogonalMatrixIsAReciprocalPolynomial}
\pmcreated{2013-03-22 15:33:13}
\pmmodified{2013-03-22 15:33:13}
\pmowner{matte}{1858}
\pmmodifier{matte}{1858}
\pmtitle{characteristic polynomial of a orthogonal matrix is a reciprocal polynomial}
\pmrecord{5}{37452}
\pmprivacy{1}
\pmauthor{matte}{1858}
\pmtype{Theorem}
\pmcomment{trigger rebuild}
\pmclassification{msc}{15-00}
\pmrelated{CharacteristicPolynomialOfASymplecticMatrixIsAReciprocalPolynomial}

\endmetadata

% this is the default PlanetMath preamble.  as your knowledge
% of TeX increases, you will probably want to edit this, but
% it should be fine as is for beginners.

% almost certainly you want these
\usepackage{amssymb}
\usepackage{amsmath}
\usepackage{amsfonts}
\usepackage{amsthm}

\usepackage{mathrsfs}

% used for TeXing text within eps files
%\usepackage{psfrag}
% need this for including graphics (\includegraphics)
%\usepackage{graphicx}
% for neatly defining theorems and propositions
%
% making logically defined graphics
%%%\usepackage{xypic}

% there are many more packages, add them here as you need them

% define commands here

\newcommand{\sR}[0]{\mathbb{R}}
\newcommand{\sC}[0]{\mathbb{C}}
\newcommand{\sN}[0]{\mathbb{N}}
\newcommand{\sZ}[0]{\mathbb{Z}}

 \usepackage{bbm}
 \newcommand{\Z}{\mathbbmss{Z}}
 \newcommand{\C}{\mathbbmss{C}}
 \newcommand{\F}{\mathbbmss{F}}
 \newcommand{\R}{\mathbbmss{R}}
 \newcommand{\Q}{\mathbbmss{Q}}



\newcommand*{\norm}[1]{\lVert #1 \rVert}
\newcommand*{\abs}[1]{| #1 |}



\newtheorem{thm}{Theorem}
\newtheorem{defn}{Definition}
\newtheorem{prop}{Proposition}
\newtheorem{lemma}{Lemma}
\newtheorem{cor}{Corollary}
\newcommand{\ccj}[1]{\overline{#1}}
\begin{document}
\begin{thm}
The characteristic polynomial of a orthogonal matrix is a reciprocal polynomial
\end{thm}

\begin{proof}
Let $A$ be the orthogonal matrix, and let 
$p(\lambda) = \det(A-\lambda I)$ be 
its characteristic polynomial. We wish to prove that 
$$ 
  p(\lambda) = \pm \lambda^n p(1/\lambda).
$$
Since $A^{-1}=A^T$, we have 
$A-\lambda I=-\lambda A (A^T-I/\lambda ).$
Taking the determinant of both sides, and using 
$\det A = \det A^T$ and 
$\det c A = c^n \det A$ ($c\in \mathbb{C}$), 
yields
$$ \det (A-\lambda I) = \pm \lambda^n \det(A-\frac{1}{\lambda} I).$$
\end{proof}

\begin{thebibliography}{9}
\bibitem {eves} H. Eves,
 \emph{Elementary Matrix Theory}, Dover publications, 1980.
\end{thebibliography}
%%%%%
%%%%%
\end{document}
