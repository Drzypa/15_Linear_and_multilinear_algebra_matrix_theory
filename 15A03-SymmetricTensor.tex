\documentclass[12pt]{article}
\usepackage{pmmeta}
\pmcanonicalname{SymmetricTensor}
\pmcreated{2013-03-22 16:15:41}
\pmmodified{2013-03-22 16:15:41}
\pmowner{Mathprof}{13753}
\pmmodifier{Mathprof}{13753}
\pmtitle{symmetric tensor}
\pmrecord{5}{38369}
\pmprivacy{1}
\pmauthor{Mathprof}{13753}
\pmtype{Definition}
\pmcomment{trigger rebuild}
\pmclassification{msc}{15A03}

% this is the default PlanetMath preamble.  as your knowledge
% of TeX increases, you will probably want to edit this, but
% it should be fine as is for beginners.

% almost certainly you want these
\usepackage{amssymb}
\usepackage{amsmath}
\usepackage{amsfonts}

% used for TeXing text within eps files
%\usepackage{psfrag}
% need this for including graphics (\includegraphics)
%\usepackage{graphicx}
% for neatly defining theorems and propositions
%\usepackage{amsthm}
% making logically defined graphics
%%%\usepackage{xypic}

% there are many more packages, add them here as you need them

% define commands here

\begin{document}
Let $V$ be a vector space over a field. Let $S_n$ be the symmetric group on
$\{1, \ldots, n\}$. An order-n \PMlinkname{tensor}{TensorProduct} $A \in V^{\otimes n}$ is said to 
be \emph{\PMlinkescapetext{symmetric}} if $P(\sigma)A = A$ for all $\sigma \in S_n$,
where $P(\sigma)$ is the permutation operator associated to $\sigma$.
The set of symmetric tensors in $V^{\otimes n}$ is denoted by
$S^n(V)$.

%%%%%
%%%%%
\end{document}
