\documentclass[12pt]{article}
\usepackage{pmmeta}
\pmcanonicalname{GoogleMatrix}
\pmcreated{2013-03-22 17:01:05}
\pmmodified{2013-03-22 17:01:05}
\pmowner{Mathprof}{13753}
\pmmodifier{Mathprof}{13753}
\pmtitle{Google matrix}
\pmrecord{6}{39302}
\pmprivacy{1}
\pmauthor{Mathprof}{13753}
\pmtype{Definition}
\pmcomment{trigger rebuild}
\pmclassification{msc}{15A51}

\endmetadata

% this is the default PlanetMath preamble.  as your knowledge
% of TeX increases, you will probably want to edit this, but
% it should be fine as is for beginners.

% almost certainly you want these
\usepackage{amssymb}
\usepackage{amsmath}
\usepackage{amsfonts}

% used for TeXing text within eps files
%\usepackage{psfrag}
% need this for including graphics (\includegraphics)
%\usepackage{graphicx}
% for neatly defining theorems and propositions
%\usepackage{amsthm}
% making logically defined graphics
%%%\usepackage{xypic}

% there are many more packages, add them here as you need them

% define commands here

\begin{document}
Google's PageRank algorithm uses a particular stochastic matrix called the Google matrix. 
The purpose of the PageRank algorithm is to compute a stationary vector of the Google matrix.
The stationary vector is then used to provide a ranking of the pages on the internet.

A directed graph $D$ is constructed whose vertices correspond to web pages and a directed
arc from vertex $i$ to vertex $j$ exists if and only if page $i$ has a link out to 
page $j$. 
Then a stochastic matrix $A=(a_{ij})$ is constructed from $D$: for each $i,j$
set 
$$
a_{ij} = 1/d(i)
$$ 
if the outdegree of vertex $i, d(i)$ is positive and there is an arc from $i$ to $j$ in $D$.
Set 
$$
a_{ij} = 0
$$ 
if $d(i) >0$  but there is no arc from $i$ to $j$ in $D$.

Set
$$
a_{ij} = 1/n
$$ 
if $d(i) = 0$, where $n$ is the order of the matrix. 

Having defined $A$ choose a positive row vector $v^T$ such that $v^T\textbf{1} = 1$
where $\textbf{1}$ is a vector of all ones.
Finally, choose a constant $c \in (0,1)$.
The \emph{Google matrix} $G$
is
$$
G = cA + (1-c)\textbf{1}v^T .
$$
Clearly, $G$ is stochastic. For the actual matrix that Google uses $c$ is about .85.

%%%%%
%%%%%
\end{document}
