\documentclass[12pt]{article}
\usepackage{pmmeta}
\pmcanonicalname{AnnihilatorOfVectorSubspace}
\pmcreated{2013-03-22 15:25:59}
\pmmodified{2013-03-22 15:25:59}
\pmowner{stevecheng}{10074}
\pmmodifier{stevecheng}{10074}
\pmtitle{annihilator of vector subspace}
\pmrecord{5}{37280}
\pmprivacy{1}
\pmauthor{stevecheng}{10074}
\pmtype{Definition}
\pmcomment{trigger rebuild}
\pmclassification{msc}{15A03}
\pmdefines{annihilator}
\pmdefines{annihilated subspace}

\endmetadata

% this is the default PlanetMath preamble.  as your knowledge
% of TeX increases, you will probably want to edit this, but
% it should be fine as is for beginners.

% almost certainly you want these
\usepackage{amssymb}
\usepackage{amsmath}
\usepackage{amsfonts}

% used for TeXing text within eps files
%\usepackage{psfrag}
% need this for including graphics (\includegraphics)
%\usepackage{graphicx}
% for neatly defining theorems and propositions
%\usepackage{amsthm}
% making logically defined graphics
%%%\usepackage{xypic}

% there are many more packages, add them here as you need them
\usepackage{enumerate}

% define commands here
\newcommand{\real}{\mathbb{R}}
\newcommand{\rat}{\mathbb{Q}}
\newcommand{\nat}{\mathbb{N}}

\providecommand{\abs}[1]{\lvert#1\rvert}
\providecommand{\absW}[1]{\left\lvert#1\right\rvert}
\providecommand{\absB}[1]{\Bigl\lvert#1\Bigr\rvert}
\providecommand{\norm}[1]{\lVert#1\rVert}
\providecommand{\normW}[1]{\left\lVert#1\right\rVert}
\providecommand{\normB}[1]{\Bigl\lVert#1\Bigr\rVert}
\providecommand{\defnterm}[1]{\emph{#1}}

\DeclareMathOperator{\linspan}{span}
\begin{document}
If $V$ is a vector space, and $S$ is any subset of $V$,
the \defnterm{annihilator} of $S$, denoted by $S^0$,
is the subspace of the dual space $V^*$
that kills every vector in $S$:
\[
S^0 = \{ \phi \in V^* : \phi(v) = 0 \textrm{ for all } v \in S \}\,.
\]

Similarly, if $\Lambda$ is any subset of $V^*$, the \defnterm{annihilated subspace}
of $\Lambda$ is
\[
\Lambda^{-0} = \{ v \in V : \phi(v) = 0 \textrm{ for all } \phi \in \Lambda \}
= \bigcap_{\phi \in \Lambda} \ker \phi\,.
\]
(Note: this may not be the standard notation.)

\section{Properties}
Assume $V$ is finite-dimensional.
Let $W$ and $\Phi$ denote subspaces of $V$ and $V^*$, respectively,
and let $\widehat{\:}$ denote the natural isomorphism from $V$ to its double dual $V^{**}$.
\begin{enumerate}[i.]
\item
$S^0 = \left(\linspan S\right)^0$
\item
$\Lambda^{-0} = \left(\linspan \Lambda \right)^{-0}$
\item
$W^{00} = \widehat{W}$
\item
$\left(\Phi^{-0}\right)^0 = \Phi$
\item
$\left(W^0\right)^{-0} = W$
\item
$\dim W + \dim W^0 = \dim V$ (a dimension theorem)
\item
$\dim \Phi + \dim \Phi^{-0} = \dim V^* = \dim V$
\item
$(W_1 + W_2)^0 = W_1^0 \cap W_2^0$, where $W_1 + W_2$ denotes
the sum of two subspaces of $V$.
\item
If $T: V \to V$ is a linear operator, and $W = \ker T$,
then the image of the pullback $T^*: V^* \to V^*$ is $W^0$.
\end{enumerate}

\begin{thebibliography}{3}
\bibitem{Friedberg} Friedberg, Insel, Spence. {\it Linear Algebra}. Prentice-Hall, 1997.
\end{thebibliography}
%%%%%
%%%%%
\end{document}
