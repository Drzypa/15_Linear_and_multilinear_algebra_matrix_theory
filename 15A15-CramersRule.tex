\documentclass[12pt]{article}
\usepackage{pmmeta}
\pmcanonicalname{CramersRule}
\pmcreated{2013-03-22 11:55:27}
\pmmodified{2013-03-22 11:55:27}
\pmowner{akrowne}{2}
\pmmodifier{akrowne}{2}
\pmtitle{Cramer's rule}
\pmrecord{10}{30626}
\pmprivacy{1}
\pmauthor{akrowne}{2}
\pmtype{Theorem}
\pmcomment{trigger rebuild}
\pmclassification{msc}{15A15}

\usepackage{amssymb}
\usepackage{amsmath}
\usepackage{amsfonts}
\usepackage{graphicx}
%%%%\usepackage{xypic}
\begin{document}
Let $Ax=b$ be the matrix form of a system of $n$ linear equations in $n$ unknowns, with $x$ and $b$ as $n\times 1$ column vectors and $A$ an $n \times n$ matrix. If $\det(A)\ne 0$, then this system has a unique solution, and for each $i$ ($1\le i \le n$) ,

$$ x_i = \frac{\det(M_i)}{\det(A)} $$

where $M_i$ is $A$ with column $i$ replaced by $b$.
%%%%%
%%%%%
%%%%%
%%%%%
\end{document}
