\documentclass[12pt]{article}
\usepackage{pmmeta}
\pmcanonicalname{PropertiesOfLinearIndependence}
\pmcreated{2013-03-22 18:05:37}
\pmmodified{2013-03-22 18:05:37}
\pmowner{CWoo}{3771}
\pmmodifier{CWoo}{3771}
\pmtitle{properties of linear independence}
\pmrecord{5}{40633}
\pmprivacy{1}
\pmauthor{CWoo}{3771}
\pmtype{Result}
\pmcomment{trigger rebuild}
\pmclassification{msc}{15A03}

\usepackage{amssymb,amscd}
\usepackage{amsmath}
\usepackage{amsfonts}
\usepackage{mathrsfs}

% used for TeXing text within eps files
%\usepackage{psfrag}
% need this for including graphics (\includegraphics)
%\usepackage{graphicx}
% for neatly defining theorems and propositions
\usepackage{amsthm}
% making logically defined graphics
%%\usepackage{xypic}
\usepackage{pst-plot}

% define commands here
\newcommand*{\abs}[1]{\left\lvert #1\right\rvert}
\newtheorem{prop}{Proposition}
\newtheorem{thm}{Theorem}
\newtheorem{ex}{Example}
\newcommand{\real}{\mathbb{R}}
\newcommand{\pdiff}[2]{\frac{\partial #1}{\partial #2}}
\newcommand{\mpdiff}[3]{\frac{\partial^#1 #2}{\partial #3^#1}}
\begin{document}
Let $V$ be a vector space over a field $k$.  Below are some basic properties of linear independence.

\begin{enumerate}
\item $S\subseteq V$ is never linearly independent if $0\in S$.
\begin{proof} Since $1\cdot 0=0$. \end{proof}
\item If $S$ is linearly independent, so is any subset of $S$.  As a result, if $S$ and $T$ are linearly independent, so is $S\cap T$.  In addition, $\varnothing$ is linearly independent, its spanning set being the singleton consisting of the zero vector $0$.
\begin{proof} If $r_1v_1+\cdots r_nv_n=0$, where $v_i\in T$, then $v_i\in S$, so $r_i=0$ for all $i=1,\ldots, n$. \end{proof}
\item If $S_1\subseteq S_2\subseteq \cdots$ is a chain of linearly independent subsets of $V$, so is their union.
\begin{proof} Let $S$ be the union.  If $r_1v_1+\cdots r_nv_n=0$, then $v_i\in S_{a(i)}$, for each $i$.  Pick the largest $S_{a(i)}$ so that all $v_i$'s are in it.  Since this set is linearly independent, $r_i=0$ for all $i$.  
\end{proof}
\item $S$ is a basis for $V$ iff $S$ is a maximal linear independent subset of $V$.  Here, maximal means that any proper superset of $S$ is linearly dependent.
\begin{proof} If $S$ is a basis for $V$, then it is linearly independent and spans $V$.  If we take any vector $v\notin S$, then $v$ can be expressed as a linear combination of elements in $S$, so that $S\cup \lbrace v\rbrace$ is no longer linearly independent, for the coefficient in front of $v$ is non-zero.  Therefore, $S$ is maximal.  

Conversely, suppose $S$ is a maximal linearly independent set in $V$.  Let $W$ be the span of $S$.  If $W\ne V$, pick an element $v\in V-W$.  Suppose $0=r_1v_1+\cdots r_nv_n+rv$, where $v_i\in S$, then $-rv = r_1v_1+\cdots +r_nv_n$.  If $r\ne 0$, then $v$ would be in the span of $S$, contradicting the assumption.  So $r=0$, and as a result, $r_i=0$, since $S$ is linearly independent.  This shows that $S\cup \lbrace v\rbrace$ is linearly independent, which is impossible since $S$ is assumed to be maximal.  Therefore, $W=V$.
\end{proof}
\end{enumerate}

\textbf{Remark}.  All of the properties above can be generalized to modules over rings, except the last one, where the implication is only one-sided: basis implying maximal linear independence.
%%%%%
%%%%%
\end{document}
