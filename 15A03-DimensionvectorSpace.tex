\documentclass[12pt]{article}
\usepackage{pmmeta}
\pmcanonicalname{DimensionvectorSpace}
\pmcreated{2013-03-22 12:42:31}
\pmmodified{2013-03-22 12:42:31}
\pmowner{rmilson}{146}
\pmmodifier{rmilson}{146}
\pmtitle{dimension (vector space)}
\pmrecord{13}{32993}
\pmprivacy{1}
\pmauthor{rmilson}{146}
\pmtype{Definition}
\pmcomment{trigger rebuild}
\pmclassification{msc}{15A03}
\pmrelated{dimension3}
\pmdefines{dimension}
\pmdefines{codimension}
\pmdefines{finite-dimensional}
\pmdefines{infinite-dimensional}

\usepackage{amsmath}
\usepackage{amsfonts}
\usepackage{amssymb}
\newcommand{\reals}{\mathbb{R}}
\newcommand{\natnums}{\mathbb{N}}
\newcommand{\cnums}{\mathbb{C}}
\newcommand{\znums}{\mathbb{Z}}
\newcommand{\lp}{\left(}
\newcommand{\rp}{\right)}
\newcommand{\lb}{\left[}
\newcommand{\rb}{\right]}
\newcommand{\supth}{^{\text{th}}}
\newtheorem{proposition}{Proposition}
\newtheorem{definition}[proposition]{Definition}
\newcommand{\nl}[1]{\PMlinkescapetext{{#1}}}
\newcommand{\pln}[2]{\PMlinkname{#1}{#2}}
\begin{document}
Let $V$ be a vector space over a field $K$.  We say that $V$ is
\emph{finite-dimensional} if there exists a finite basis of $V$. Otherwise we
call $V$ \emph{infinite-dimensional}.

It can be shown that every basis of $V$ has the same cardinality. We call this cardinality the \emph{dimension} of $V$. In particular, if
$V$ is finite-dimensional,  then every basis of $V$ will consist of a finite set $v_1,\ldots, v_n$.  We then call the natural number $n$ the \emph{dimension} of $V$. 

Next, let $U\subset V$ a subspace.  The dimension of the quotient
vector space $V/U$ is called the \emph{codimension} of $U$ relative to $V$.

In circumstances where the choice of field is ambiguous, the
dimension of a vector space depends on the choice of field.  For
example, every complex vector space is also a real vector space, and
therefore has a real dimension, double its complex dimension.
%%%%%
%%%%%
\end{document}
