\documentclass[12pt]{article}
\usepackage{pmmeta}
\pmcanonicalname{DerivationOfRotationMatrixUsingPolarCoordinates}
\pmcreated{2013-03-22 15:25:02}
\pmmodified{2013-03-22 15:25:02}
\pmowner{stevecheng}{10074}
\pmmodifier{stevecheng}{10074}
\pmtitle{derivation of rotation matrix using polar coordinates}
\pmrecord{9}{37261}
\pmprivacy{1}
\pmauthor{stevecheng}{10074}
\pmtype{Derivation}
\pmcomment{trigger rebuild}
\pmclassification{msc}{15-00}
%\pmkeywords{rotation matrix}
\pmrelated{RotationMatrix}
\pmrelated{PolarCoordinates}
\pmrelated{DecompositionOfOrthogonalOperatorsAsRotationsAndReflections}
\pmrelated{DerivationOf2DReflectionMatrix}

% this is the default PlanetMath preamble.  as your knowledge
% of TeX increases, you will probably want to edit this, but
% it should be fine as is for beginners.

% almost certainly you want these
\usepackage{amssymb}
\usepackage{amsmath}
\usepackage{amsfonts}

% used for TeXing text within eps files
%\usepackage{psfrag}
% need this for including graphics (\includegraphics)
\usepackage{graphicx}
% for neatly defining theorems and propositions
%\usepackage{amsthm}
% making logically defined graphics
%%%\usepackage{xypic}

% there are many more packages, add them here as you need them
\usepackage{enumerate}

% define commands here
\newcommand{\real}{\mathbb{R}}
\newcommand{\rat}{\mathbb{Q}}
\newcommand{\nat}{\mathbb{N}}

\newcommand{\vx}{\boldsymbol{x}}
\newcommand{\vy}{\boldsymbol{y}}
\newcommand{\vv}{\boldsymbol{v}}

\providecommand{\abs}[1]{\lvert#1\rvert}
\providecommand{\absW}[1]{\left\lvert#1\right\rvert}
\providecommand{\absB}[1]{\Bigl\lvert#1\Bigr\rvert}
\providecommand{\norm}[1]{\lVert#1\rVert}
\providecommand{\normW}[1]{\left\lVert#1\right\rVert}
\providecommand{\normB}[1]{\Bigl\lVert#1\Bigr\rVert}
\providecommand{\defnterm}[1]{\emph{#1}}
\begin{document}
We derive formally the expression for the rotation of a two-dimensional vector
$\vv = a\vx + b\vy$ by an angle $\phi$ counter-clockwise.  Here 
$\vx$ and $\vy$ are perpendicular unit vectors that are oriented counter-clockwise
(the usual orientation).

\begin{center}
\includegraphics{rotation-matrix.eps}
\end{center}

In terms of polar coordinates, $\vv$ may be rewritten:
\begin{align*}
\vv &= r(\cos \theta \, \vx + \sin \theta \, \vy) \,, \quad a = r\cos \theta\,; b = r\sin\theta\,, \\
\intertext{for some angle $\theta$ and radius $r \geq 0$.
To rotate a vector $\vv$ by $\phi$ really means to shift its
polar angle by a constant amount $\phi$ but leave its polar radius fixed.
Therefore, the result of the rotation must be:}
\vv' &= r\bigl(\cos(\theta + \phi) \, \vx + \sin(\theta + \phi) \, \vy\bigr) \\
\intertext{Expanding using the angle addition formulae, we obtain}
\vv' &= r\bigl(\cos \theta \cos \phi - \sin \theta \sin \phi) \, \vx + (\sin \theta \cos \phi + \cos \theta \sin \phi) \, \vy\bigr)  \\
&= (a \cos \phi - b \sin \phi) \, \vx + (b \cos \phi + a \sin \phi) \, \vy\,.
\end{align*}
When this transformation is written out in $[\vx, \vy]$-coordinates, we obtain the formula for the rotation matrix:
\[
\vv' = \begin{bmatrix}
\cos \phi & -\sin \phi \\
\sin \phi & \cos \phi
\end{bmatrix} \begin{bmatrix} a \\ b \end{bmatrix} \,.
\]
%%%%%
%%%%%
\end{document}
