\documentclass[12pt]{article}
\usepackage{pmmeta}
\pmcanonicalname{ClosureOfAVectorSubspaceInANormedSpaceIsAVectorSubspace}
\pmcreated{2013-03-22 15:00:16}
\pmmodified{2013-03-22 15:00:16}
\pmowner{gumau}{3545}
\pmmodifier{gumau}{3545}
\pmtitle{closure of a vector subspace in a normed space is a vector subspace}
\pmrecord{7}{36709}
\pmprivacy{1}
\pmauthor{gumau}{3545}
\pmtype{Result}
\pmcomment{trigger rebuild}
\pmclassification{msc}{15A03}
\pmclassification{msc}{46B99}
\pmclassification{msc}{54A05}
\pmrelated{ClosureOfAVectorSubspaceIsAVectorSubspace2}
\pmrelated{ClosureOfSetsClosedUnderAFinitaryOperation}

\endmetadata

% this is the default PlanetMath preamble.  as your knowledge
% of TeX increases, you will probably want to edit this, but
% it should be fine as is for beginners.

% almost certainly you want these
\usepackage{amssymb}
\usepackage{amsmath}
\usepackage{amsfonts}

% used for TeXing text within eps files
%\usepackage{psfrag}
% need this for including graphics (\includegraphics)
%\usepackage{graphicx}
% for neatly defining theorems and propositions
%\usepackage{amsthm}
% making logically defined graphics
%%%\usepackage{xypic}

% there are many more packages, add them here as you need them

% define commands here
\begin{document}
Let $(V, \| \cdot \|)$ be a normed space, and $S \subset V$ a vector subspace. Then $\overline{S}$ is a vector subspace in $V$.

\textbf{Proof}

First of all, $0 \in \overline{S}$ because $0 \in S$. Now, let $x, y \in \overline{S}$, and $\lambda \in K$ (where $K$ is the ground field of the vector space $V$). Then there are two sequences in $S$, say $(x_n)_{n \in \mathbb{N}}$ and $(y_n)_{n \in \mathbb{N}}$ which converge to $x$ and $y$ respectively.

Then, the sequence $(x_n + \lambda \cdot y_n)_{n \in \mathbb{N}}$ is a sequence in $S$ (because $S$ is a vector subspace), and it's trivial (use properties of the norm) that this sequence converges to $x + \lambda \cdot y$, and so this sum is a vector which lies in $\overline{S}$. 


We have proved that $\overline{S}$ is a vector subspace. QED.
%%%%%
%%%%%
\end{document}
