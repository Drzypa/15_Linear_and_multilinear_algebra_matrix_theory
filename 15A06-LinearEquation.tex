\documentclass[12pt]{article}
\usepackage{pmmeta}
\pmcanonicalname{LinearEquation}
\pmcreated{2013-03-22 12:25:59}
\pmmodified{2013-03-22 12:25:59}
\pmowner{rmilson}{146}
\pmmodifier{rmilson}{146}
\pmtitle{linear equation}
\pmrecord{8}{32496}
\pmprivacy{1}
\pmauthor{rmilson}{146}
\pmtype{Definition}
\pmcomment{trigger rebuild}
\pmclassification{msc}{15A06}
\pmsynonym{linear problem}{LinearEquation}
\pmsynonym{linear system}{LinearEquation}
\pmrelated{HomogeneousLinearProblem}
\pmrelated{FiniteDimensionalLinearProblem}
\pmdefines{consistent}
\pmdefines{inconsistent}
\pmdefines{particular solution}

\endmetadata

\usepackage{amsmath}
\usepackage{amsfonts}
\usepackage{amssymb}
\newcommand{\reals}{\mathbb{R}}
\newcommand{\natnums}{\mathbb{N}}
\newcommand{\cnums}{\mathbb{C}}
\newcommand{\znums}{\mathbb{Z}}
\newcommand{\lp}{\left(}
\newcommand{\rp}{\right)}
\newcommand{\lb}{\left[}
\newcommand{\rb}{\right]}
\newcommand{\supth}{^{\text{th}}}
\newtheorem{proposition}{Proposition}
\newtheorem{definition}[proposition]{Definition}

\newtheorem{theorem}[proposition]{Theorem}
\begin{document}
Let $L:U\rightarrow V$ be a linear mapping, and $v\in V$ an element of
the codomain.  A {\em linear equation}  is
a relation of the form,
$$L(u)=v,$$
where $u\in U$ is to be considered as the unknown.  The
solution set of a linear equation is the set of $u\in U$ that satisfy the
above constraint, or to be more precise, the pre-image $L^{-1}(v)$.  The equation is
called inconsistent if no solutions exist, that is, if the pre-image is
the empty set. Otherwise, the equation is called \emph{consistent}.

The general solution of
a linear equation has the form
$$u=u_p + u_h,\quad u_p,u_h\in U,$$
where 
$$L(u_p)=v$$
is a \emph{particular solution} and where
$$L(u_h)=0$$
is any  solution of the corresponding homogeneous problem, i.e. an
element of the kernel of $L$.

{\bf Notes.} Elementary treatments of linear algebra focus almost
exclusively on finite-dimensional linear problems. They neglect to
mention the underlying mapping, preferring to focus instead on
``variables and equations.'' However, the scope of the general concept
is considerably wider, e.g.  linear differential equations such as
$$y''+y = 0.$$
%%%%%
%%%%%
\end{document}
