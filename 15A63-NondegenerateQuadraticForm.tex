\documentclass[12pt]{article}
\usepackage{pmmeta}
\pmcanonicalname{NondegenerateQuadraticForm}
\pmcreated{2013-03-22 15:05:58}
\pmmodified{2013-03-22 15:05:58}
\pmowner{CWoo}{3771}
\pmmodifier{CWoo}{3771}
\pmtitle{non-degenerate quadratic form}
\pmrecord{6}{36828}
\pmprivacy{1}
\pmauthor{CWoo}{3771}
\pmtype{Definition}
\pmcomment{trigger rebuild}
\pmclassification{msc}{15A63}
\pmclassification{msc}{11E39}
\pmclassification{msc}{47A07}
\pmsynonym{non degenerate quadratic form}{NondegenerateQuadraticForm}
\pmsynonym{non singular quadratic form}{NondegenerateQuadraticForm}
\pmdefines{non-degenerate quadratic form}
\pmdefines{non-singular quadratic form}
\pmdefines{regular quadratic form}

% this is the default PlanetMath preamble.  as your knowledge
% of TeX increases, you will probably want to edit this, but
% it should be fine as is for beginners.

% almost certainly you want these
\usepackage{amssymb,amscd}
\usepackage{amsmath}
\usepackage{amsfonts}

% used for TeXing text within eps files
%\usepackage{psfrag}
% need this for including graphics (\includegraphics)
%\usepackage{graphicx}
% for neatly defining theorems and propositions
%\usepackage{amsthm}
% making logically defined graphics
%%%\usepackage{xypic}

% there are many more packages, add them here as you need them

% define commands here
\begin{document}
Let $k$ be a field of characteristic not 2.  Then a quadratic form $Q$ over a vector space $V$ (over a field $k$) is said to be \PMlinkescapetext{\emph{non-degenerate}}, if its associated bilinear form:
$$B(x,y)=\frac{1}{2}(Q(x+y)-Q(x)-Q(y))$$
is non-degenerate.

If we fix a basis $\boldsymbol{b}$ for $V$, then $Q(x)$ can be written as 

$$Q(x)=x^TAx$$

for some symmetric matrix $A$ over $k$.  Then it's not hard to see that $Q$ is non-degenerate iff $A$ is non-singular.  Because of this, a non-degenerate quadratic form is also known as a \emph{non-singular} quadratic form.  A third name for a non-degenerate quadratic form is that of a \emph{regular quadratic form}.
%%%%%
%%%%%
\end{document}
