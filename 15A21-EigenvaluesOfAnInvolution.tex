\documentclass[12pt]{article}
\usepackage{pmmeta}
\pmcanonicalname{EigenvaluesOfAnInvolution}
\pmcreated{2013-03-22 13:38:57}
\pmmodified{2013-03-22 13:38:57}
\pmowner{Koro}{127}
\pmmodifier{Koro}{127}
\pmtitle{eigenvalues of an involution}
\pmrecord{4}{34301}
\pmprivacy{1}
\pmauthor{Koro}{127}
\pmtype{Proof}
\pmcomment{trigger rebuild}
\pmclassification{msc}{15A21}

\endmetadata

% this is the default PlanetMath preamble.  as your knowledge
% of TeX increases, you will probably want to edit this, but
% it should be fine as is for beginners.

% almost certainly you want these
\usepackage{amssymb}
\usepackage{amsmath}
\usepackage{amsfonts}

% used for TeXing text within eps files
%\usepackage{psfrag}
% need this for including graphics (\includegraphics)
%\usepackage{graphicx}
% for neatly defining theorems and propositions
%\usepackage{amsthm}
% making logically defined graphics
%%%\usepackage{xypic}

% there are many more packages, add them here as you need them

% define commands here

\newcommand{\sR}[0]{\mathbb{R}}
\newcommand{\sC}[0]{\mathbb{C}}
\newcommand{\sN}[0]{\mathbb{N}}
\newcommand{\sZ}[0]{\mathbb{Z}}
\begin{document}
\emph{Proof.} For the first claim suppose $\lambda$ is an eigenvalue
corresponding to an eigenvector $x$ of $A$. That is, $Ax = \lambda x$.
Then $A^2x = \lambda Ax$, so $x=\lambda^2x$. As an eigenvector, $x$ is non-zero, and
 $\lambda = \pm 1$. Now property (1) follows since the determinant is
the product of the eigenvalues. For property (2), suppose that
$A-\lambda I = -\lambda A(A-1/\lambda I)$, where $A$ and $\lambda$ are as above.
Taking the determinant of both
sides, and using part (1), and the properties of the determinant, yields
 $$ \det (A-\lambda I) = \pm \lambda^n \det(A-\frac{1}{\lambda} I).$$
Property (2) follows.
$\Box$
%%%%%
%%%%%
\end{document}
