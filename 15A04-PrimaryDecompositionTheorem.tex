\documentclass[12pt]{article}
\usepackage{pmmeta}
\pmcanonicalname{PrimaryDecompositionTheorem}
\pmcreated{2013-03-22 14:15:30}
\pmmodified{2013-03-22 14:15:30}
\pmowner{gumau}{3545}
\pmmodifier{gumau}{3545}
\pmtitle{primary decomposition theorem}
\pmrecord{8}{35707}
\pmprivacy{1}
\pmauthor{gumau}{3545}
\pmtype{Theorem}
\pmcomment{trigger rebuild}
\pmclassification{msc}{15A04}

\endmetadata

% this is the default PlanetMath preamble.  as your knowledge
% of TeX increases, you will probably want to edit this, but
% it should be fine as is for beginners.

% almost certainly you want these
\usepackage{amssymb}
\usepackage{amsmath}
\usepackage{amsfonts}

% used for TeXing text within eps files
%\usepackage{psfrag}
% need this for including graphics (\includegraphics)
%\usepackage{graphicx}
% for neatly defining theorems and propositions
%\usepackage{amsthm}
% making logically defined graphics
%%%\usepackage{xypic}

% there are many more packages, add them here as you need them

% define commands here
\begin{document}
This is an important theorem in linear algebra. It states the following:
Let $k$ be a field, $V$ a vector space over $k$, $\dim V =n$, and $T\colon V\to V$ a linear operator, such that its minimal polynomial (or its annihilator polynomial) is $m_{T}$, which decomposes in $k[X]$ into irreducible factors as $m_{T}=p_{1}^{\alpha_{1}} \ldots p_{r}^{\alpha_{r}}$. Then,
\begin{enumerate}
\item  $V=\bigoplus_{i=1}^{r}\ker (p_{i}^{\alpha_{i}}(T))$
\item $\ker (p_{i}^{\alpha_{i}}(T))$ is $T$-invariant for every $i$
\item If $T_{i}$ is the restriction of $T$ to $\ker (p_{i}^{\alpha_{i}}(T))$, then $m_{T_{i}}=p_{i}^{\alpha_{i}}$
\end{enumerate}

This is a consequence of a more general theorem:
Let $V$, $T$ be as above, and $f \in k[X]$ such that $f(T)=0$, with $f=f_{1} \ldots f_{r}$ and $(f_{i},f_{j})=1$ if $i \neq j$, then
\begin{enumerate}
\item $V=\bigoplus _{i=1}^{r}\ker(f_{i}(T))$
\item $\ker(f_{i}(T))$ is $T$-invariant for every $i$
\end{enumerate}

To illustrate its importance, the primary decomposition theorem, together with the cyclic decomposition theorem, imply the existence and uniqueness of the Jordan canonical form.
%%%%%
%%%%%
\end{document}
