\documentclass[12pt]{article}
\usepackage{pmmeta}
\pmcanonicalname{BlockDeterminants}
\pmcreated{2013-03-22 15:25:57}
\pmmodified{2013-03-22 15:25:57}
\pmowner{georgiosl}{7242}
\pmmodifier{georgiosl}{7242}
\pmtitle{block determinants}
\pmrecord{20}{37279}
\pmprivacy{1}
\pmauthor{georgiosl}{7242}
\pmtype{Theorem}
\pmcomment{trigger rebuild}
\pmclassification{msc}{15A15}
\pmrelated{SchurComplement}
\pmrelated{DeterminantsOfSomeMatricesOfSpecialForm}

\endmetadata

% this is the default PlanetMath preamble.  as your knowledge
% of TeX increases, you will probably want to edit this, but
% it should be fine as is for beginners.

% almost certainly you want these
\usepackage{amssymb}
\usepackage{amsmath}
\usepackage{amsfonts}

% used for TeXing text within eps files
%\usepackage{psfrag}
% need this for including graphics (\includegraphics)
%\usepackage{graphicx}
% for neatly defining theorems and propositions
%\usepackage{amsthm}
% making logically defined graphics
%%%\usepackage{xypic}

% there are many more packages, add them here as you need them

% define commands here
\begin{document}
If $A$ and $D$ are square matrices
\begin{itemize}
\item If $A^{-1}$ exists, then $$ \det \begin{pmatrix} A & B \\ C & D \end{pmatrix}=\det(A)\det(D-CA^{-1}B)\\$$  
\item If $D^{-1}$ exists, then $$\det \begin{pmatrix} A & B \\ C & D \end{pmatrix}=\det(D)\det(A-BD^{-1}C)$$
\end{itemize}
The matrices $D-CA^{-1}B$ and $A-BD^{-1}C$ are called the Schur complements of $A$ and $D$, respectively.
\\Mention that 
\begin{itemize}
\item If $A$, $D$ are square matrices, then  
$$ \det \begin{pmatrix} A & B \\ O & D \end{pmatrix}=\det(A) \det(D)$$, where $O$ is a zero matrix.
\item Also we have that $$\det \begin{pmatrix} A & O \\ O & B \end{pmatrix}=\det(A)\det(B).$$ 
\item Another useful result for block determinants is the following.
\\As $J=\begin{pmatrix} O & I \\ -I & O \end{pmatrix}$ is a symplectic matrix, 
we have that $\det J=1$. Using now the fact that $\det MN= \det(M) \det(N)$ for any $M$, $N$ square matrices, we have that
$$ \det \begin{pmatrix} O & A \\ B & O \end{pmatrix}
=\det \begin{pmatrix} O & A \\ B & O \end{pmatrix}\det J
=-\det(A)\det(B)$$
\end{itemize}
This holds for any square matrices $A$,\,$B$ and for the last point $A$,\,$B$ have also the same order. They do not need to be
invertible.
%%%%%
%%%%%
\end{document}
