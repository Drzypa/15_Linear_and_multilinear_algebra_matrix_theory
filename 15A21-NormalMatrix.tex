\documentclass[12pt]{article}
\usepackage{pmmeta}
\pmcanonicalname{NormalMatrix}
\pmcreated{2013-03-22 13:41:10}
\pmmodified{2013-03-22 13:41:10}
\pmowner{Daume}{40}
\pmmodifier{Daume}{40}
\pmtitle{normal matrix}
\pmrecord{12}{34358}
\pmprivacy{1}
\pmauthor{Daume}{40}
\pmtype{Definition}
\pmcomment{trigger rebuild}
\pmclassification{msc}{15A21}
\pmsynonym{normal}{NormalMatrix}
\pmrelated{TheoremForNormalTriangularMatrices}

\endmetadata

% this is the default PlanetMath preamble.  as your knowledge
% of TeX increases, you will probably want to edit this, but
% it should be fine as is for beginners.

% almost certainly you want these
\usepackage{amssymb}
\usepackage{amsmath}
\usepackage{amsfonts}

% used for TeXing text within eps files
%\usepackage{psfrag}
% need this for including graphics (\includegraphics)
%\usepackage{graphicx}
% for neatly defining theorems and propositions
%\usepackage{amsthm}
% making logically defined graphics
%%%\usepackage{xypic} 

% there are many more packages, add them here as you need them

% define commands here
\begin{document}
\PMlinkescapeword{properties}

A complex matrix $A \in \mathbb{C}^{n\times n}$ is said to be \emph{normal} if $A^\ast A = AA^\ast$ where $^\ast$ denotes the conjugate transpose.\\
Similarly for a real matrix $A \in \mathbb{R}^{n\times n}$ is said to be \emph{normal} if $A^TA=AA^T$ where $T$ denotes the transpose.\\\\
\textbf{properties:}
\begin{itemize}
\item Equivalently a complex matrix $A \in \mathbb{C}^{n\times n}$ is said to be \emph{normal} if it satisfies $[A,A^\ast]=0$ where $[,]$ is the commutator bracket.

\item Equivalently a real matrix $A \in \mathbb{R}^{n\times n}$ is said to be \emph{normal} if it satisfies $[A,A^T]=0$ where $[,]$ is the commutator bracket.

\item Let $A$ be a square complex matrix of order $n$. It follows from Schur's inequality that if $A$ is a normal matrix then $\sum_{i=1}^n |\lambda_i|^2 = \operatorname{trace} A^\ast A$ where $^\ast$ is the conjugate transpose and $\lambda_i$ are the eigenvalues of $A$.

\item   A complex square matrix is diagonal if and only if it is normal,  triangular.(see theorem for normal triangular matrices).
\end{itemize}

\textbf{examples:}
\begin{itemize}
\item $\begin{pmatrix}
a & b \\
-b & a \\\end{pmatrix}$ where $a,b \in \mathbb{R}$
\item $\begin{pmatrix}
1 & i \\
-i & 1 \\\end{pmatrix}$
\end{itemize}

\textbf{see also:}
\begin{itemize}
\item Wikipedia, \PMlinkexternal{normal matrix}{http://www.wikipedia.org/wiki/Normal_matrix}
\end{itemize}
%%%%%
%%%%%
\end{document}
