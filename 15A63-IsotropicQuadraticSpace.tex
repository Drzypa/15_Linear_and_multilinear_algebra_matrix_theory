\documentclass[12pt]{article}
\usepackage{pmmeta}
\pmcanonicalname{IsotropicQuadraticSpace}
\pmcreated{2013-03-22 15:41:57}
\pmmodified{2013-03-22 15:41:57}
\pmowner{CWoo}{3771}
\pmmodifier{CWoo}{3771}
\pmtitle{isotropic quadratic space}
\pmrecord{10}{37643}
\pmprivacy{1}
\pmauthor{CWoo}{3771}
\pmtype{Definition}
\pmcomment{trigger rebuild}
\pmclassification{msc}{15A63}
\pmclassification{msc}{11E81}
\pmrelated{QuadraticMap2}
\pmrelated{QuadraticForm}
\pmdefines{isotropic vector}
\pmdefines{isotropic quadratic form}
\pmdefines{anisotropic vector}
\pmdefines{anisotropic quadratic form}
\pmdefines{anisotropic quadratic space}
\pmdefines{totally isotropic quadratic space}
\pmdefines{totally isotropic quadratic form}

\endmetadata

\usepackage{amssymb,amscd}
\usepackage{amsmath}
\usepackage{amsfonts}

% used for TeXing text within eps files
%\usepackage{psfrag}
% need this for including graphics (\includegraphics)
%\usepackage{graphicx}
% for neatly defining theorems and propositions
%\usepackage{amsthm}
% making logically defined graphics
%%%\usepackage{xypic}

% define commands here
\begin{document}
A vector $v$ (an element of $V$) in a quadratic space $(V,Q)$ is \emph{isotropic} if
\begin{enumerate}
\item $v\ne0$ and
\item $Q(v)=0$.
\end{enumerate}
Otherwise, it is called \emph{anisotropic}.  A quadratic space $(V,Q)$ is isotropic if it contains an isotropic vector.  Otherwise, it is anisotropic.  A quadratic space $(V,Q)$ is \emph{totally isotropic} if every one of its non-zero vector is isotropic, or that $Q(V)=0$.

Similarly, an \emph{isotropic quadratic form} is one which has a non-trivial kernel, or that there exists a vector $v$ such that $Q(v)=0$.  The definitions for that of an \emph{anisotropic quadratic form} and that of a \emph{totally isotropic quadratic form} should now be clear from the above discussion (anisotropic: $\operatorname{ker}(Q)=0$; totally isotropic: $\operatorname{ker}(Q)=V$).

\textbf{Examples.}
\begin{itemize}
\item Consider the quadratic form $Q(x,y)=x^2+y^2$ in the vector space $\mathbb{R}^2$ over the reals.  It is clearly anisotropic since there are no real numbers $a,b$ not both $0$, such that $a^2+b^2=0$.
\item However, the same form is isotropic in $\mathbb{C}^2$ over $\mathbb{C}$, since $1^2+i^2=0$; the complex numbers are algebraically closed.
\item Again, using the same form $x^2+y^2$, but in $\mathbb{R}^3$ over the reals , we see that it is isotropic since the $z$ term is missing, so that $Q(0,0,1)=0^2+0^2=0$.  
\item If we restrict $Q$ to the subspace consisting of the $z$-axis ($x=y=0$) and call it $Q_z$, then $Q_z$ is totally isotropic, and the $z$-axis is a totally isotropic subspace.
\item The quadratic form $Q(x,y)=x^2-y^2$ is clearly isotropic in any vector space over any field.  In general, this is true if the coefficients of a diagonal quadratic form $Q$ consist of $1, -1, 0$ ($0$ is optional) and nothing else.
\end{itemize}
%%%%%
%%%%%
\end{document}
