\documentclass[12pt]{article}
\usepackage{pmmeta}
\pmcanonicalname{FrobeniusProduct}
\pmcreated{2013-03-22 18:11:34}
\pmmodified{2013-03-22 18:11:34}
\pmowner{pahio}{2872}
\pmmodifier{pahio}{2872}
\pmtitle{Frobenius product}
\pmrecord{8}{40769}
\pmprivacy{1}
\pmauthor{pahio}{2872}
\pmtype{Definition}
\pmcomment{trigger rebuild}
\pmclassification{msc}{15A60}
\pmclassification{msc}{15A63}
\pmsynonym{Frobenius inner product}{FrobeniusProduct}
\pmrelated{NormedVectorSpace}
\pmrelated{FrobeniusMatrixNorm}
\pmrelated{Product}
\pmdefines{Frobenius norm}

\endmetadata

% this is the default PlanetMath preamble.  as your knowledge
% of TeX increases, you will probably want to edit this, but
% it should be fine as is for beginners.

% almost certainly you want these
\usepackage{amssymb}
\usepackage{amsmath}
\usepackage{amsfonts}

% used for TeXing text within eps files
%\usepackage{psfrag}
% need this for including graphics (\includegraphics)
%\usepackage{graphicx}
% for neatly defining theorems and propositions
 \usepackage{amsthm}
% making logically defined graphics
%%%\usepackage{xypic}

% there are many more packages, add them here as you need them

% define commands here

\theoremstyle{definition}
\newtheorem*{thmplain}{Theorem}

\begin{document}
If\, $A = (a_{ij})$\, and\, $B = (b_{ij})$\, are real $m\!\times\!n$ matrices, their {\em Frobenius product} is defined as 
$$\langle A,\,B \rangle_F \;:=\; \sum_{i,\,j}a_{ij}b_{ij}.$$
It is easily seen that\, $\langle A,\,B \rangle_F$\, is equal to the trace of the matrix $A^\intercal B$ and $AB^\intercal $, and that the Frobenius product is an inner product of the vector space formed by the $m\!\times\!n$ matrices; it \PMlinkescapetext{induces} the {\em Frobenius norm} of this vector space.
%%%%%
%%%%%
\end{document}
