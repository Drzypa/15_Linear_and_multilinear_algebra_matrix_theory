\documentclass[12pt]{article}
\usepackage{pmmeta}
\pmcanonicalname{TraceOfAMatrix}
\pmcreated{2013-03-22 11:59:56}
\pmmodified{2013-03-22 11:59:56}
\pmowner{Daume}{40}
\pmmodifier{Daume}{40}
\pmtitle{trace of a matrix}
\pmrecord{20}{30930}
\pmprivacy{1}
\pmauthor{Daume}{40}
\pmtype{Definition}
\pmcomment{trigger rebuild}
\pmclassification{msc}{15A99}
\pmrelated{ShursInequality}

\usepackage{amssymb}
\usepackage{amsmath}
\usepackage{amsfonts}
\usepackage{graphicx}
%%%\usepackage{xypic}
\begin{document}
\textbf{Definition} \\
Let $A=(a_{i,j})$ be a square matrix of
order $n$. 
The trace of the matrix is the sum of the main diagonal:
\begin{center} $\operatorname{trace}(A)= \sum\limits _{i=1} ^{n}
 a_{i,i}$
\end{center}

\textbf{Notation:}\\
The trace of a matrix $A$ is also commonly denoted as $\operatorname{Tr}(A)$ 
or $\operatorname{Tr}A$.
 
\textbf{Properties:}
 \begin{enumerate}
\item  The trace is a linear transformation from the space of square matrices to
 the real numbers. In other words, if $A$ and $B$ are square matrices with real (or complex) entries, 
of same order and $c$ is a scalar, then 
\begin{eqnarray*}
  \operatorname{trace}(A+B) &=& \operatorname{trace}(A)+ \operatorname{trace}(B), \\
  \operatorname{trace}(cA) &=& c\cdot \operatorname{trace}(A).
\end{eqnarray*}
\item For  the transpose and conjugate transpose, we have for any 
square matrix $A$ with real (or complex) entries, 
\begin{eqnarray*} 
\operatorname{trace} (A^t) &=& \operatorname{trace} (A), \\
\operatorname{trace} (A^\ast) &=& \overline{\operatorname{trace} (A)}.
\end{eqnarray*}
\item If $A$ and $B$ are matrices such that $AB$ is a square matrix, then 
$$ \operatorname{trace} (AB) = \operatorname{trace} (BA).$$

For this reason it is possible to define the trace of a linear transformation, as the choice of basis does not affect the trace.
Thus, if $A,B,C$ are matrices such that $ABC$ is a square matrix, then
$$ \operatorname{trace} (ABC) = \operatorname{trace} (CAB) = \operatorname{trace} (BCA).$$
\item If $B$ is in invertible square matrix of same order as $A$, then 
$$ \operatorname{trace} (A) = \operatorname{trace} (B^{-1}A B).$$
In other words, the trace of similar matrices are equal. 
\item Let $A$ be a square matrix of order $n$ with real (or complex)
entries $a_{ij}$. Then
\begin{eqnarray*}
\operatorname{trace} A^\ast A &=& \operatorname{trace} A A^\ast \\
                &=& \sum_{i,j=1}^n |a_{ij}|^2.
\end{eqnarray*}
Here $^\ast$ is the complex conjugate, and $|\cdot|$ is the complex modulus.
In particular, $\operatorname{trace} A^\ast A\ge 0$ with equality if and only if $A=0$.
(See the Frobenius matrix norm.)
\item Various inequalities for $\operatorname{trace}$ are given in 
\cite{yang}.
\end{enumerate}
 
See the proof of properties of trace of a matrix.

\begin{thebibliography}{9}
\bibitem{ehrlich}
The Trace of a Square Matrix. Paul Ehrlich, [online] 
\PMlinkexternal{http://www.math.ufl.edu/~ehrlich/trace.html}{http://www.math.ufl.edu/~ehrlich/trace.html}
\bibitem{yang}
Z.P. Yang,  X.X. Feng, \emph{A note on the trace inequality for 
products of Hermitian matrix power},
Journal of Inequalities in Pure and Applied Mathematics,
Volume 3,  Issue 5, 2002, Article 78, 
\PMlinkexternal{online}{http://www.emis.de/journals/JIPAM/v3n5/082_02.html}.
\end{thebibliography}
%%%%%
%%%%%
%%%%%
\end{document}
