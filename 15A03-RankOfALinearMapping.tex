\documentclass[12pt]{article}
\usepackage{pmmeta}
\pmcanonicalname{RankOfALinearMapping}
\pmcreated{2013-03-22 12:24:03}
\pmmodified{2013-03-22 12:24:03}
\pmowner{yark}{2760}
\pmmodifier{yark}{2760}
\pmtitle{rank of a linear mapping}
\pmrecord{13}{32236}
\pmprivacy{1}
\pmauthor{yark}{2760}
\pmtype{Definition}
\pmcomment{trigger rebuild}
\pmclassification{msc}{15A03}
\pmsynonym{rank}{RankOfALinearMapping}
\pmrelated{Nullity}
\pmrelated{RankNullityTheorem2}

\usepackage{amsmath}
\usepackage{amsfonts}
\usepackage{amssymb}

\newcommand{\rank}{\operatorname{rank}}
\begin{document}
The \emph{rank} of a linear mapping $L\colon U\to V$ is defined to be
the $\dim L(U)$, the dimension of the mapping's image.  Speaking less
formally, the rank gives the number of independent linear constraints
on $u\in U$ imposed by the equation
\[ L(u)=0. \]

\subsubsection*{Properties}
\begin{enumerate}
\item If $V$ is finite-dimensional, then $\rank L=\dim V$ if and only
  if $L$ is surjective.
\item If $U$ is finite-dimensional, then $\rank L=\dim U$ if and only
  if $L$ is injective.
\item Composition of linear mappings does not increase rank.  If
  $M\colon V\to W$ is another linear mapping, then \[\rank ML \le
  \rank L\] and
  \[\rank ML \le \rank M.\] Equality holds in the first case if
  and only if $M$ is an isomorphism, and in the second case if and
  only if $L$ is an isomorphism.
\end{enumerate}
%%%%%
%%%%%
\end{document}
