\documentclass[12pt]{article}
\usepackage{pmmeta}
\pmcanonicalname{HermitianFormOverADivisionRing}
\pmcreated{2013-03-22 15:41:04}
\pmmodified{2013-03-22 15:41:04}
\pmowner{CWoo}{3771}
\pmmodifier{CWoo}{3771}
\pmtitle{Hermitian form over a division ring}
\pmrecord{12}{37626}
\pmprivacy{1}
\pmauthor{CWoo}{3771}
\pmtype{Definition}
\pmcomment{trigger rebuild}
\pmclassification{msc}{15A63}
\pmdefines{Hermitian form}
\pmdefines{skew Hermitian form}

\endmetadata

\usepackage{amssymb,amscd}
\usepackage{amsmath}
\usepackage{amsfonts}

% used for TeXing text within eps files
%\usepackage{psfrag}
% need this for including graphics (\includegraphics)
%\usepackage{graphicx}
% for neatly defining theorems and propositions
%\usepackage{amsthm}
% making logically defined graphics
%%%\usepackage{xypic}

% define commands here
\begin{document}
Let $D$ be a division ring admitting an \PMlinkname{involution}{Involution2} $*$.  Let $V$ be a vector space over $D$.  A \emph{Hermitian form} over $D$ is a function from $V\times V$ to $D$, denoted by $(\cdot,\cdot)$ with the following properties, for any $v,w\in V$ and $d\in D$:

\begin{enumerate}
\item $(\cdot,\cdot)$ is additive in each of its arguments,
\item $(du,v)=d(u,v)$,
\item $(u,dv)=(u,v)d^*$,
\item $(u,v)=(v,u)^*$.
\end{enumerate}

Note that if the Hermitian form $(\cdot,\cdot)$ is non-trivial and if $*$ is the identity on $D$, then $D$ is a field and $(\cdot,\cdot)$ is just a symmetric bilinear form.

If we replace the last condition by $(u,v)=-(v,u)^*$, then $(\cdot,\cdot)$ over $D$ is called a \emph{skew Hermitian form}.

\textbf{Remark.}  Every skew Hermitian form over a division ring induces a Hermitian form and vice versa.
%%%%%
%%%%%
\end{document}
