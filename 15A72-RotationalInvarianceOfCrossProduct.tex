\documentclass[12pt]{article}
\usepackage{pmmeta}
\pmcanonicalname{RotationalInvarianceOfCrossProduct}
\pmcreated{2013-03-22 13:33:53}
\pmmodified{2013-03-22 13:33:53}
\pmowner{matte}{1858}
\pmmodifier{matte}{1858}
\pmtitle{rotational invariance of cross product}
\pmrecord{6}{34171}
\pmprivacy{1}
\pmauthor{matte}{1858}
\pmtype{Theorem}
\pmcomment{trigger rebuild}
\pmclassification{msc}{15A72}
\pmclassification{msc}{15A90}

% this is the default PlanetMath preamble.  as your knowledge
% of TeX increases, you will probably want to edit this, but
% it should be fine as is for beginners.

% almost certainly you want these
\usepackage{amssymb}
\usepackage{amsmath}
\usepackage{amsfonts}

% used for TeXing text within eps files
%\usepackage{psfrag}
% need this for including graphics (\includegraphics)
%\usepackage{graphicx}
% for neatly defining theorems and propositions
%\usepackage{amsthm}
% making logically defined graphics
%%%\usepackage{xypic}

% there are many more packages, add them here as you need them

% define commands here

\newcommand{\sR}[0]{\mathbb{R}}
\newcommand{\sC}[0]{\mathbb{C}}
\newcommand{\sN}[0]{\mathbb{N}}
\newcommand{\sZ}[0]{\mathbb{Z}}
\begin{document}
\newcommand{\vu}[0]{\textbf{u}}
\newcommand{\vv}[0]{\textbf{v}}
\newcommand{\vR}[0]{\textbf{R}}

{\bf Theorem} \\
Let $\vR$ be a rotational $3\times 3$ matrix, i.e., a real
matrix with $\det \vR = 1$ and $\vR^{-1} = \vR^T$.
Then for all vectors $\vu,\vv$ in $\mathbb{R}^3$,
$$ \vR \cdot (\vu\times \vv) = (\vR\cdot \vu)\times (\vR\cdot \vv).$$

\emph{Proof.}
Let us first fix some right hand oriented orthonormal basis in $\mathbb{R}^3$.
Further, let $\{u^1,u^2,u^3\}$ and $\{v^1,v^2,v^3\}$ be the components
of $\vu$ and
$\vv$ in that basis. Also, in the chosen basis, we denote the entries
of $\vR$ by $R_{ij}$. Since  $\vR$ is rotational, we have
$R_{ij} R_{kj} = \delta_{ik}$ where $\delta_{ik}$ is the
Kronecker delta symbol. Here we use the Einstein summation convention.
Thus, in the previous expression, on the left hand side, $j$ should be summed
over $1,2,3$. We shall use the
Levi-Civita permutation symbol $\varepsilon$ to write the cross product.
Then the $i$:th coordinate of $\vu\times \vv$ equals
$(\vu\times \vv)^i = \varepsilon^{ijk} u^j v^k$.
For the  $k$th component of $(\vR\cdot \vu)\times (\vR\cdot \vv)$ we
then have
\begin{eqnarray*}
((\vR\cdot \vu)\times (\vR\cdot \vv))^k &=& \varepsilon^{imk} R_{ij} R_{mn} u^j v^n \\
&=& \varepsilon^{iml} \delta_{kl} R_{ij} R_{mn} u^j v^n \\
&=& \varepsilon^{iml} R_{kr} R_{lr} R_{ij} R_{mn} u^j v^n \\
&=& \varepsilon^{jnr} \det \vR\, R_{kr} u^j v^n.
\end{eqnarray*}
The last line follows since
$\varepsilon^{ijk} R_{im} R_{jn} R_{kr} = \varepsilon^{mnr}\varepsilon^{ijk} R_{i1} R_{j2} R_{k3} = \varepsilon^{mnr} \det \vR$.
Since $\det \vR = 1$, it follows that
\begin{eqnarray*}
((\vR\cdot \vu)\times (\vR\cdot \vv))^k &=& R_{kr} \varepsilon^{jnr} u^j v^n\\
        &=& R_{kr} (\vu\times \vv)^r \\
        &=& (\vR\cdot \vu\times \vv)^k
\end{eqnarray*}
as claimed. $\Box$
%%%%%
%%%%%
\end{document}
