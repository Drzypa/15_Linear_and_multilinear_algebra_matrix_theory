\documentclass[12pt]{article}
\usepackage{pmmeta}
\pmcanonicalname{PositionVector}
\pmcreated{2013-03-22 15:25:15}
\pmmodified{2013-03-22 15:25:15}
\pmowner{pahio}{2872}
\pmmodifier{pahio}{2872}
\pmtitle{position vector}
\pmrecord{18}{37265}
\pmprivacy{1}
\pmauthor{pahio}{2872}
\pmtype{Definition}
\pmcomment{trigger rebuild}
\pmclassification{msc}{15A72}
\pmsynonym{radius vector}{PositionVector}
\pmrelated{ExampleOfCurvatureSpaceCurve}
\pmrelated{DyadProduct}
\pmrelated{TiltCurve}

% this is the default PlanetMath preamble.  as your knowledge
% of TeX increases, you will probably want to edit this, but
% it should be fine as is for beginners.

% almost certainly you want these
\usepackage{amssymb}
\usepackage{amsmath}
\usepackage{amsfonts}

% used for TeXing text within eps files
%\usepackage{psfrag}
% need this for including graphics (\includegraphics)
%\usepackage{graphicx}
% for neatly defining theorems and propositions
 \usepackage{amsthm}
% making logically defined graphics
%%%\usepackage{xypic}

% there are many more packages, add them here as you need them

% define commands here

\theoremstyle{definition}
\newtheorem*{thmplain}{Theorem}
\begin{document}
In the space $\mathbb{R}^3$, the vector 
$$\vec{r} \;:=\; (x,\,y,\,z) \;=\; x\vec{i}+y\vec{j}+z\vec{k}$$
directed from the origin to a point \,$(x,\,y,\,z)$\, is the {\em position vector} of this point.  When the point is \PMlinkescapetext{variable}, $\vec{r}$ \PMlinkescapetext{represents} a vector field and its \PMlinkescapetext{length}
$$r \;:=\; \sqrt{x^2\!+\!y^2\!+\!z^2}$$
a scalar \PMlinkescapetext{field}.

The \PMlinkescapetext{simple formulae}
\begin{itemize}
\item $\nabla\!\cdot\vec{r} \;=\; 3$
\item $\nabla\!\times\!\vec{r} \;=\; \vec{0}$
\item $\displaystyle\nabla r \;=\; \frac{\vec{r}}{r} \;=\; \vec{r}^{\,0}$
\item $\displaystyle\nabla\frac{1}{r} \;=\; -\frac{\vec{r}}{r^3} \;=\; -\frac{\vec{r}^{\,0}}{r^2}$
\item $\displaystyle\nabla^2\frac{1}{r} \;=\; 0$
\end{itemize}
are valid, where $\vec{r}^{\,0}$ is the unit vector having the direction of $\vec{r}$.

If\, $\vec{c}$\, is a \PMlinkescapetext{constant} vector,\, $\vec{U}\!\!:\mathbb{R}^3\to\mathbb{R}^3$\, a vector function and\, $f\!\!:\mathbb{R}\to\mathbb{R}$\, is a twice differentiable function, then the formulae
\begin{itemize}
\item $\nabla(\vec{c}\cdot\!\vec{r})\; \;= \vec{c}$
\item $\nabla\cdot(\vec{c}\times\vec{r}) \;=\; 0$
\item $(\vec{U}\!\cdot\!\nabla)\vec{r} \;=\; \vec{U}$
\item $(\vec{U}\!\times\!\nabla)\!\cdot\!\vec{r} \;=\; 0$
\item $(\vec{U}\!\times\!\nabla)\!\times\!\vec{r} \;=\; -2\vec{U}$
\item $\nabla f(r) \;=\; f'(r)\,\vec{r}^{\,0}$
\item $\displaystyle\nabla^2f(r) \;=\; f''(r)\!+\frac{2}{r}f'(r)$
\end{itemize}
hold.

\begin{thebibliography}{9}
\bibitem{VV}{\sc K. V\"ais\"al\"a:} {\em Vektorianalyysi}. \,Werner S\"oderstr\"om Osakeyhti\"o, Helsinki (1961).
\end{thebibliography}
%%%%%
%%%%%
\end{document}
