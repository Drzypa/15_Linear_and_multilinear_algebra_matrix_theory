\documentclass[12pt]{article}
\usepackage{pmmeta}
\pmcanonicalname{HyperbolicPairsAndBasis}
\pmcreated{2013-03-22 15:50:59}
\pmmodified{2013-03-22 15:50:59}
\pmowner{Algeboy}{12884}
\pmmodifier{Algeboy}{12884}
\pmtitle{hyperbolic pairs and basis}
\pmrecord{13}{37833}
\pmprivacy{1}
\pmauthor{Algeboy}{12884}
\pmtype{Definition}
\pmcomment{trigger rebuild}
\pmclassification{msc}{15A99}
\pmclassification{msc}{15A63}
\pmsynonym{symplectic pair}{HyperbolicPairsAndBasis}
\pmdefines{hyperbolic pair}
\pmdefines{hyperbolic basis}

\usepackage{latexsym}
\usepackage{amssymb}
\usepackage{amsmath}
\usepackage{amsfonts}
\usepackage{amsthm}

%%\usepackage{xypic}

%-----------------------------------------------------

%       Standard theoremlike environments.

%       Stolen directly from AMSLaTeX sample

%-----------------------------------------------------

%% \theoremstyle{plain} %% This is the default

\newtheorem{thm}{Theorem}

\newtheorem{coro}[thm]{Corollary}

\newtheorem{lem}[thm]{Lemma}

\newtheorem{lemma}[thm]{Lemma}

\newtheorem{prop}[thm]{Proposition}

\newtheorem{conjecture}[thm]{Conjecture}

\newtheorem{conj}[thm]{Conjecture}

\newtheorem{defn}[thm]{Definition}

\newtheorem{remark}[thm]{Remark}

\newtheorem{ex}[thm]{Example}



%\countstyle[equation]{thm}



%--------------------------------------------------

%       Item references.

%--------------------------------------------------


\newcommand{\exref}[1]{Example-\ref{#1}}

\newcommand{\thmref}[1]{Theorem-\ref{#1}}

\newcommand{\defref}[1]{Definition-\ref{#1}}

\newcommand{\eqnref}[1]{(\ref{#1})}

\newcommand{\secref}[1]{Section-\ref{#1}}

\newcommand{\lemref}[1]{Lemma-\ref{#1}}

\newcommand{\propref}[1]{Prop\-o\-si\-tion-\ref{#1}}

\newcommand{\corref}[1]{Cor\-ol\-lary-\ref{#1}}

\newcommand{\figref}[1]{Fig\-ure-\ref{#1}}

\newcommand{\conjref}[1]{Conjecture-\ref{#1}}


% Normal subgroup or equal.

\providecommand{\normaleq}{\unlhd}

% Normal subgroup.

\providecommand{\normal}{\lhd}

\providecommand{\rnormal}{\rhd}
% Divides, does not divide.

\providecommand{\divides}{\mid}

\providecommand{\ndivides}{\nmid}


\providecommand{\union}{\cup}

\providecommand{\bigunion}{\bigcup}

\providecommand{\intersect}{\cap}

\providecommand{\bigintersect}{\bigcap}
\begin{document}
\begin{defn}
Given a reflexive non-degenerate sesquilinear form $b:V\times V\rightarrow k$,
a \emph{hyperbolic pair} is a pair $e,f\in V$ such that 
\[b(e,e)=0=b(f,f)\textnormal{ and } b(e,f)=1.\]
\end{defn}

The span of a hyperbolic pair is a hyperbolic line (recall that a line refers to the projective dimension thus we have a 2-dimensional subspace but a 1-dimensional projective space).

\begin{defn}
A \emph{hyperbolic basis} for a vector space $V$ with respect to a reflexive 
non-degenerate sesquilinear form $b$ is a basis
$\{e_1,f_1,\dots,e_m,f_m\}$ where 
\[b(e_i,e_j)=0=b(f_i,f_j)\textnormal{ and } b(e_i,f_j)=\delta_{ij}.\]
\end{defn}

Thus a hyperbolic basis is a basis composed of hyperbolic pairs.  Furthermore,
if $V$ has a hyperbolic basis then setting $H_i=\langle e_i,f_i\rangle$ shows
\[V=H_1\perp H_2\perp\cdots \perp H_m\]
where $X\perp Y=X\oplus Y$ with the added condition $b(X,Y)=0$.

Hyperbolic bases are the foundation of a ``standard basis'' for a vector spaces $V$ equipped with a reflexive non-degenerate sesquilinear form.

\section{Symmetric pairs}

A symmetric hyperbolic pair is a hyperbolic pair $e,f$ for which $b$ restricted to $L=\langle e,f\rangle$ is a symmetric bilinear form.  This requires the additional condition that $b(e,f)=1=b(f,e)$.

This means that the form restricted the \emph{hyperbolic line} $L=\langle e,f\rangle$ can be represented by the matrix
\[\begin{bmatrix} 0 & 1\\ 1 & 0\end{bmatrix}.\]
When $1/2\in k$ we can consider the associated quadratic form
\[q(v)=\frac{1}{2}b(v,v)\]
so if $v=x e+y f$ we arrive at the polynomial 
\[q(xe+yf)=\frac{1}{2}\left([x, y]\begin{bmatrix} 0 & 1\\ 1 & 0\end{bmatrix}\begin{bmatrix} x\\ y\end{bmatrix}\right)=x y.\]

Suppose the field is $\mathbb{R}$.  Then we can associate a graph to the
equations $c=q(xe+yf)=xy$ for any fixed $c\in \mathbb{R}$.  If $c=0$ then
$x=0$ or $y=0$ so the graph is the $x$ and $y$-axis -- also called the
degenerate hyperbola.  If $c\neq 0$ then $x\neq 0$ and so $y=\frac{c}{x}$.
This is the graph of a a hyperbola, hence the title of a hyperbolic pair.

Symmetric bilinear maps are often preferred to be presented as diagonal
matrices so that they reflect the content of Sylvester's Law of Inertia.
When $1/2\in k$ (characteristic of $k$ is not 2) we can 
diagonalize any symmetric hyperbolic pair as follows:
\[\begin{bmatrix} \frac{1}{2} & 1 \\ -\frac{1}{2} & 1\end{bmatrix}
\begin{bmatrix} 0 & 1\\ 1 & 0 \end{bmatrix}
\begin{bmatrix} \frac{1}{2} & -\frac{1}{2}\\ 1 & 1\end{bmatrix}
=\begin{bmatrix} 1 & 0\\ 0 & -1\end{bmatrix}.\]
That is, we can change the basis to
\[e\mapsto u:=\frac{1}{2}e+f,\qquad f\mapsto v:=-\frac{1}{2}e+f.\]
Then $b(u,u)=1$, $b(v,v)=-1$, and $b(u,v)=b(v,u)=0$.  Alternatively we find 
under this basis we have the quadratic form $q(xu+yv)=x^2-y^2$ which is also
seen as the standard equation of a hyperbola.

If we think of a quadratic form as generalizing norms -- that is length, then
we are observing that on a hyperbolic line length is not Euclidean, in fact, 
as the usual Euclidean length of $(x,y)$, $x^2+y^2$, gets large, the associated hyperbolic length get small: $x^2-y^2$ may get small, even 0 or negative.  Thus the curvature of this space is negative (consider the graph of $z=x^2-y^2$ which is a saddle.)

All symmetric hyperbolic pairs are isometric so decomposing a bilinear form into
the radical plus hyperbolic pairs plus any left over anisotropic complement produces 
a standard basis which allows for easy comparison of one symmetric bilinear form to another.

\section{Alternating pairs}

An alternating hyperbolic pair is a hyperbolic pair $e,f$ for which $b$ restricted 
to $L=\langle e,f\rangle$ is an alternating bilinear form.  This requires the 
additional condition that $b(e,f)=1=-b(f,e)$.

This means that the form restricted the hyperbolic line $L=\langle e,f\rangle$ 
can be represented by the matrix
\[\begin{bmatrix} 0 & 1\\ -1 & 0\end{bmatrix}.\]
Although we do not associate a quadratic form with an alternating bilinear
(since $b(v,v)=0$ for all $v\in V$) we can still derive the equations of
a hyperbola.  Specifically
\[b(xe,yf)=xy.\]
So again setting $c=b(xe,yf)=xy$ we observe the various hyperbola graphs.

Alternating hyperbolic pairs cannot be diagonalized as every element 
$v\in V$ has $b(v,v)=0$.

All alternating bilinear forms decompose into hyperbolic lines and the radical and 
any two alternating hyperbolic lines are isometric and thus simply indicating the number 
of hyperbolic pairs in an alternating bilinear form specifies the form uniquely.  If we 
further insist the form is non-degenerate then the dimension of the vector space
specifies the form completely.

\section{Hermitian pairs}

A \emph{\PMlinkescapetext{Hermitian hyperbolic pair}} is a hyperbolic pair $e,f$ for which $b$ restricted to 
$L=\langle e,f\rangle$ is an Hermitian bilinear form.  This requires the 
additional condition that $b(e,f)=1=b(f,e)$.  The associated matrix does not reveal
 much difference from the symmetric as we still obtain
\[\begin{bmatrix} 0 & 1\\ 1 & 0\end{bmatrix}.\]
What is different is how the matrix is used to compute the bilinear products:
\[b(x_1e+y_1 f,x_2 e+y_2 f)=
[x_1,y_1]\begin{bmatrix} 0 & 1\\ 1 & 0\end{bmatrix}
\begin{bmatrix} \bar{x_2}\\ \bar{y_2}\end{bmatrix}.\]
So if we compute $b(v,v)$ we find:
\[b(xe+y f,x e+y f)=
[x,y]\begin{bmatrix} 0 & 1\\ 1 & 0\end{bmatrix}
\begin{bmatrix} \bar{x}\\ \bar{y}\end{bmatrix}
=x\bar{y}+y\bar{x}=(x\bar{y})+\overline{(x\bar{y})}.\]
We see from this that two hyperbolic pairs of Hermitian type need not be isometric 
unless we further consider the automorphism of the two forms.


\section{Characteristic 2}

Hyperbolic pairs over fields of characteristic 2 are a special breed because they 
are at the same time symmetric and alternating.  That is, the form
is the matrix:
\[\begin{bmatrix} 0 & 1\\ 1 & 0\end{bmatrix}=\begin{bmatrix} 0 & 1 \\ -1 & 0
\end{bmatrix}.\]
Thus the form cannot be diagonalized as it is alternating.  Here it is generally 
more useful to use a quadratic form then the bilinear form.  Unfortunately
because we cannot recover the quadratic form from the bilinear form
on account that $b(v,v)=0$, such a quadratic form must be provided externally
from some other method.  Thus it is not always feasible.

%%%%%
%%%%%
\end{document}
