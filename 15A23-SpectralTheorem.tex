\documentclass[12pt]{article}
\usepackage{pmmeta}
\pmcanonicalname{SpectralTheorem}
\pmcreated{2013-03-22 12:45:49}
\pmmodified{2013-03-22 12:45:49}
\pmowner{rmilson}{146}
\pmmodifier{rmilson}{146}
\pmtitle{spectral theorem}
\pmrecord{9}{33072}
\pmprivacy{1}
\pmauthor{rmilson}{146}
\pmtype{Theorem}
\pmcomment{trigger rebuild}
\pmclassification{msc}{15A23}
\pmclassification{msc}{15A63}
\pmclassification{msc}{15A18}
\pmrelated{DiagonalizableOperator}
\pmdefines{normal operator}

\usepackage{amsmath}
\usepackage{amsfonts}
\usepackage{amssymb}
\newcommand{\reals}{\mathbb{R}}
\newcommand{\natnums}{\mathbb{N}}
\newcommand{\cnums}{\mathbb{C}}
\newcommand{\znums}{\mathbb{Z}}
\newcommand{\lp}{\left(}
\newcommand{\rp}{\right)}
\newcommand{\lb}{\left[}
\newcommand{\rb}{\right]}
\newcommand{\supth}{^{\text{th}}}
\newtheorem{proposition}{Proposition}
\newtheorem{definition}[proposition]{Definition}
\newcommand{\nl}[1]{{\PMlinkescapetext{{#1}}}}
\newcommand{\pln}[2]{{\PMlinkname{{#1}}{#2}}}
\newcommand{\adj}{^{\displaystyle \star}}
\begin{document}
Let $U$ be a finite-dimensional, unitary space and let $M:U\rightarrow
U$ be an endomorphism.  We say that $M$ is normal if it commutes with
its Hermitian adjoint, i.e.
$$MM\adj= M\adj M.$$

\paragraph{Spectral Theorem}
Let $M:U\rightarrow U$ be a linear transformation of a unitary space.
TFAE
\begin{enumerate}
\item The transformation $M$ is normal.
\item Letting
  $$\Lambda=\{\lambda\in\cnums \mid M-\lambda E \mbox{ is singular}\},$$
  where $E$ is the identity operator,
  denote the spectrum (set of eigenvalues) of $M$, the corresponding
  eigenspaces
  $$E_\lambda=\ker(M-\lambda E),\quad \lambda\in\Lambda$$
  give an orthogonal, direct sum
  decomposition of $U$, i.e. 
  $$U=\bigoplus_{\lambda\in\Lambda} E_\lambda,$$
  and $E_{\lambda_1}
  \perp E_{\lambda_2}$ for distinct eigenvalues $\lambda_1\neq
  \lambda_2$.
\item We can decompose $M$ as the sum
  $$M = \sum_{\lambda\in \Lambda} \lambda P_\lambda,$$
  where  $\Lambda\in\cnums$ is  a finite subset of complex numbers
  indexing a family of commuting orthogonal projections
  $P_\lambda:U\rightarrow U$, i.e.
  $$
    P_\lambda{}\adj = P_\lambda \qquad
    P_\lambda P_\mu =
    \begin{cases}
      P_\lambda & \lambda = \mu \\
      0 & \lambda\neq \mu,
    \end{cases}
    $$
    and where WLOG
  $$\sum_{\lambda\in\Lambda} P_\lambda = 1_U.$$
\item There exists an orthonormal basis of $U$ that diagonalizes $M$.
\end{enumerate}

\paragraph{Remarks.}
\begin{enumerate}
\item Here are some important classes of normal operators,
  distinguished by the nature of their eigenvalues.
  \begin{itemize}
  \item Hermitian operators.  Eigenvalues are real.
  \item Unitary transformations.  Eigenvalues lie on the unit circle,
    i.e. the set of complex numbers of modulus 1.
  \item Orthogonal projections.  Eigenvalues are either 0 or 1.
  \end{itemize}
\item There is a well-known version of the spectral theorem for
  $\reals$, namely that a self-adjoint (symmetric) transformation of a
  real inner product spaces can diagonalized and that eigenvectors
  corresponding to different eigenvalues are orthogonal.  An even more
  down-to-earth version of this theorem says that a symmetric, real
  matrix can always be diagonalized by an orthonormal basis of
  eigenvectors. 
\item There are several versions of increasing sophistication of the
  spectral theorem that hold in infinite-dimensional, Hilbert space
  setting.  In such a context one must distinguish between the
  so-called discrete and continuous (no corresponding eigenspace)
  spectrums, and replace the representing sum for the operator with
  some kind of an integral. The definition of self-adjointness is also
  quite tricky for unbounded operators.  Finally, there are versions
  of the spectral theorem, of importance in theoretical quantum
  mechanics, that can be applied to continuous 1-parameter groups of
  commuting, self-adjoint operators.
\end{enumerate}
%%%%%
%%%%%
\end{document}
