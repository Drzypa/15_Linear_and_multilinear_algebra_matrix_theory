\documentclass[12pt]{article}
\usepackage{pmmeta}
\pmcanonicalname{InverseOfMatrixWithSmallrankAdjustment}
\pmcreated{2013-03-22 15:46:06}
\pmmodified{2013-03-22 15:46:06}
\pmowner{kshum}{5987}
\pmmodifier{kshum}{5987}
\pmtitle{inverse of matrix with small-rank adjustment}
\pmrecord{8}{37724}
\pmprivacy{1}
\pmauthor{kshum}{5987}
\pmtype{Theorem}
\pmcomment{trigger rebuild}
\pmclassification{msc}{15A09}

% this is the default PlanetMath preamble.  as your knowledge
% of TeX increases, you will probably want to edit this, but
% it should be fine as is for beginners.

% almost certainly you want these
\usepackage{amssymb}
\usepackage{amsmath}
\usepackage{amsfonts}

% used for TeXing text within eps files
%\usepackage{psfrag}
% need this for including graphics (\includegraphics)
%\usepackage{graphicx}
% for neatly defining theorems and propositions
%\usepackage{amsthm}
% making logically defined graphics
%%%\usepackage{xypic}

% there are many more packages, add them here as you need them

% define commands here
\begin{document}
Suppose that an $n\times n$ matrix $B$ is obtained by adding a
small-rank adjustment $XRY^T$ to matrix $A$,
\[
 B = A+XRY^T,
\]
where $X$ and $Y$ are $n\times r$ matrices, and $R$ is an $r\times
r$ matrix. Assume that the inverse of $A$ is known and $r$ is much
smaller than $n$. The following formula for $B^{-1}$ is often
useful,
\[
 B^{-1} = A^{-1} - A^{-1}X (R^{-1}+Y^TA^{-1}X)^{-1} Y^T A^{-1}
\]

provided that all inverses in the formula exist.

In particular, when $r=1$ and $A=I$, we have
\[
 (I+xy^T)^{-1} = I- \frac{xy^T}{1+y^Tx}
\]
for any $n\times 1$ column vectors $x$ and $y$ such that $1+y^Tx\neq
0$.
%%%%%
%%%%%
\end{document}
