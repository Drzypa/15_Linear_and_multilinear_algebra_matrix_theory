\documentclass[12pt]{article}
\usepackage{pmmeta}
\pmcanonicalname{SemilinearTransformation}
\pmcreated{2013-03-22 15:51:06}
\pmmodified{2013-03-22 15:51:06}
\pmowner{Algeboy}{12884}
\pmmodifier{Algeboy}{12884}
\pmtitle{semilinear transformation}
\pmrecord{20}{37835}
\pmprivacy{1}
\pmauthor{Algeboy}{12884}
\pmtype{Definition}
\pmcomment{trigger rebuild}
\pmclassification{msc}{15A04}
\pmsynonym{semilinear map}{SemilinearTransformation}
\pmsynonym{semilinear transform}{SemilinearTransformation}
\pmsynonym{semi-linear transformation}{SemilinearTransformation}
\pmsynonym{semi-linear map}{SemilinearTransformation}
%\pmkeywords{field automorphism}
%\pmkeywords{linear}
%\pmkeywords{Gamma L}
\pmrelated{ClassicalGroups}
\pmrelated{ProjectiveSpace}
\pmdefines{semilinear transform}
\pmdefines{Gamma L}

\endmetadata

\usepackage{latexsym}
\usepackage{amssymb}
\usepackage{amsmath}
\usepackage{amsfonts}
\usepackage{amsthm}

%%\usepackage{xypic}

%-----------------------------------------------------

%       Standard theoremlike environments.

%       Stolen directly from AMSLaTeX sample

%-----------------------------------------------------

%% \theoremstyle{plain} %% This is the default

\newtheorem{thm}{Theorem}

\newtheorem{coro}[thm]{Corollary}

\newtheorem{lem}[thm]{Lemma}

\newtheorem{lemma}[thm]{Lemma}

\newtheorem{prop}[thm]{Proposition}

\newtheorem{conjecture}[thm]{Conjecture}

\newtheorem{conj}[thm]{Conjecture}

\newtheorem{defn}[thm]{Definition}

\newtheorem{remark}[thm]{Remark}

\newtheorem{ex}[thm]{Example}



%\countstyle[equation]{thm}



%--------------------------------------------------

%       Item references.

%--------------------------------------------------


\newcommand{\exref}[1]{Example-\ref{#1}}

\newcommand{\thmref}[1]{Theorem-\ref{#1}}

\newcommand{\defref}[1]{Definition-\ref{#1}}

\newcommand{\eqnref}[1]{(\ref{#1})}

\newcommand{\secref}[1]{Section-\ref{#1}}

\newcommand{\lemref}[1]{Lemma-\ref{#1}}

\newcommand{\propref}[1]{Prop\-o\-si\-tion-\ref{#1}}

\newcommand{\corref}[1]{Cor\-ol\-lary-\ref{#1}}

\newcommand{\figref}[1]{Fig\-ure-\ref{#1}}

\newcommand{\conjref}[1]{Conjecture-\ref{#1}}


% Normal subgroup or equal.

\providecommand{\normaleq}{\unlhd}

% Normal subgroup.

\providecommand{\normal}{\lhd}

\providecommand{\rnormal}{\rhd}
% Divides, does not divide.

\providecommand{\divides}{\mid}

\providecommand{\ndivides}{\nmid}


\providecommand{\union}{\cup}

\providecommand{\bigunion}{\bigcup}

\providecommand{\intersect}{\cap}

\providecommand{\bigintersect}{\bigcap}
\begin{document}
Let $K$ be a field and $k$ its prime subfield.  For example, if $K$ is $\mathbb{C}$ then $k$ is $\mathbb{Q}$, and if $K$ is the finite field of order $q=p^i$, then $k$ is $\mathbb{Z}_p$.

\begin{defn}
Given a field automorphism $\theta$ of $K$, a function
$f:V\rightarrow W$ between two $K$ vector spaces $V$ and $W$ is 
$\theta$-semilinear, or simply semilinear, if for all $x,y\in V$ and $l\in K$ 
it follows:  (shown here first in left hand notation and then in the preferred right hand notation.)
\begin{enumerate}
\item $f(x+y)=f(x)+f(y)$,  (in right hand notation: $(x+y)f=xf+yf$.)
\item $f(lx)=l^\theta f(x)$, (in right hand notation: $(lx)f=l^\theta xf$.)
\end{enumerate}
where $l^\theta$ denotes the image of $l$ under $\theta$.
\end{defn}

\begin{remark}
$\theta$ must be a field automorphism for $f$ to remain additive, for example,
$\theta$ must fix the prime subfield as
\[n^\theta xf=(nx)f=(x+\cdots +x)f=n(xf).\]
Also 
\[(l_1+l_2)^\theta xf=((l_1+l_2)x)f=(l_1 x)f+(l_2 x)f=(l_1^\theta + l_2^\theta)xf\]
so $(l_1+l_2)^\theta=l_1^\theta+l_2^\theta$.  Finally,
\[(l_1 l_2)^\theta xf=((l_1 l_2 x)f=l_1^\theta (l_2 x)f=l_1^\theta l_2^\theta xf.\]
\end{remark}

Every linear transformation is semilinear, but the converse is generally not true.  If we treat $V$ and $W$ as vector spaces over $k$, (by considering $K$ as  vector space over $k$ first) then every $\theta$-semilinear map is a $k$-linear map, where $k$ is the prime subfield of $K$.  

\textbf{Example}
\begin{itemize}
\item
Let $K=\mathbb{C}$, $V=\mathbb{C}^n$ with standard basis $e_1,\dots,e_n$.  Define the map $f:V\rightarrow V$ by 
      \[f\left(\sum_{i=1}z_i e_i\right)=\sum_{i=1}^n \bar{z}_i e_i.\]
$f$ is semilinear (with respect to the complex conjugation field automorphism)
but not linear. 
\item
Let $K=GF(q)$ -- the Galois field of order $q=p^i$, $p$ the characteristic.
Let $l^\theta=l^p$, for $l\in K$.  By the Freshman's dream it is known that this is a field automorphism.  To every linear map $f:V\rightarrow W$ between 
vector spaces $V$ and $W$ over $K$ we can establish a $\theta$-semilinear map
      \[\left(\sum_{i=1}l_i e_i\right)\tilde{f}=\sum_{i=1}^n l_i^\theta e_i f.\]
\end{itemize}
$\Box$


Indeed every linear map can be converted into a semilinear map in such a way.
This is part of a general observation collected into the following result.

\begin{defn}
Given a vector space $V$, the set of all invertible semilinear maps (over all field automorphisms) is the group $\Gamma L(V)$.
\end{defn}

\begin{prop}
Given a vector space $V$ over $K$, and $k$ the prime subfield of $K$, then
$\Gamma L(V)$ decomposes as the semidirect product
\[\Gamma L(V)=GL(V)\rtimes Gal(K/k)\]
where $Gal(K/k)$ is the Galois group of $K/k$.
\end{prop}
\begin{remark}
We identify $Gal(K/k)$ with a subgroup of $\Gamma L(V)$ by fixing a basis
$B$ for $V$ and defining the semilinear maps:
\[\sum_{b\in B} l_b b\mapsto \sum_{b\in B} l_b^\sigma b\]
for any $\sigma\in Gal(K/k)$.  We shall denoted this subgroup by $Gal(K/k)_B$.  We also see these complements to $GL(V)$ in $\Gamma L(V)$ are acted on regularly by $GL(V)$ as they correspond to a change of basis.
\end{remark}
\begin{proof}
Every linear map is semilinear thus $GL(V)\leq \Gamma L(V)$.  Fix a basis
$B$ of $V$.  Now given any semilinear map $f$ with respect to a field automorphism $\sigma\in Gal(K/k)$, then define $g:V\rightarrow V$ by \[\left(\sum_{b\in B} l_b b\right)g=\sum_{b\in B} (l_b^{\sigma^{-1}} b)f
=\sum_{b\in B} l_b (b)f.\]
As $(B)f$ is also a basis of $V$, it follows $g$ is simply a basis exchange
of $V$ and so linear and invertible: $g\in GL(V)$.

Set $h:=g^{-1}f$.  For every $v=\sum_{b\in B}l_b\neq 0$ in $V$,
\[vh=vg^{-1}f=\sum_{b\in B} l_b^{\sigma} b\]
thus $h$ is in the $Gal(K/k)$ subgroup relative to the fixed basis $B$.
This factorization is unique to the fixed basis $B$.  Furthermore, $GL(V)$ is normalized by the action of $Gal(K/k)_B$, so $\Gamma L(V)=GL(V)\rtimes Gal(K/k)$.
\end{proof}

The $\Gamma L(V)$ groups extend the typical classical groups in $GL(V)$.  The importance in considering such maps follows from the consideration of projective geometry.

The projective geometry of a vector space $V$, denoted $PG(V)$, is the lattice of all subspaces of $V$.  Although the typical semilinear map is not a linear map, it does follow that every semilinear map $f:V\rightarrow W$ induces an order-preserving map $f:PG(V)\rightarrow PG(W)$.  That is, every semilinear map induces a projectivity.  The converse of this observation is the Fundamental Theorem of Projective Geometry.  Thus semilinear maps are useful because they define the automorphism group of the projective geometry of a vector space.


\bibliographystyle{amsplain}
\providecommand{\bysame}{\leavevmode\hbox to3em{\hrulefill}\thinspace}
\providecommand{\MR}{\relax\ifhmode\unskip\space\fi MR }
% \MRhref is called by the amsart/book/proc definition of \MR.
\providecommand{\MRhref}[2]{%
\href{http://www.ams.org/mathscinet-getitem?mr=#1}{#2}
}
\providecommand{\href}[2]{#2}
\begin{thebibliography}{10}


\bibitem{HP}
Gruenberg, K. W. and Weir, A.J.
\emph{Linear Geometry 2nd Ed.} (English)
[B] Graduate Texts in Mathematics. 49. New York - Heidelberg - Berlin: Springer-Verlag. X, 198 p. DM 29.10; \$ 12.80 (1977).

\end{thebibliography}

%%%%%
%%%%%
\end{document}
