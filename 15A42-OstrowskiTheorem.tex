\documentclass[12pt]{article}
\usepackage{pmmeta}
\pmcanonicalname{OstrowskiTheorem}
\pmcreated{2013-03-22 15:36:29}
\pmmodified{2013-03-22 15:36:29}
\pmowner{Andrea Ambrosio}{7332}
\pmmodifier{Andrea Ambrosio}{7332}
\pmtitle{Ostrowski theorem}
\pmrecord{22}{37528}
\pmprivacy{1}
\pmauthor{Andrea Ambrosio}{7332}
\pmtype{Theorem}
\pmcomment{trigger rebuild}
\pmclassification{msc}{15A42}

\endmetadata

% this is the default PlanetMath preamble.  as your knowledge
% of TeX increases, you will probably want to edit this, but
% it should be fine as is for beginners.

% almost certainly you want these
\usepackage{amssymb}
\usepackage{amsmath}
\usepackage{amsfonts}

% used for TeXing text within eps files
%\usepackage{psfrag}
% need this for including graphics (\includegraphics)
%\usepackage{graphicx}
% for neatly defining theorems and propositions
\usepackage{amsthm}
% making logically defined graphics
%%%\usepackage{xypic}

% there are many more packages, add them here as you need them

% define commands here
\begin{document}
Let $A$ be a complex $n\times n$ matrix, $R_i=\sum_{j\ne i}\left|a_{ij}\right|, C_j=\sum_{i\ne j}\left|a_{ij}\right|\quad 1\leq i\leq n, 1\leq j\leq n$. Let's consider, for any $\alpha\in (0,1)$, the circles of this kind: $O_i=\left\{z\in\mathbf{C}:\left|z-a_{ii}\right|\leq R_i^\alpha C_i^{1-\alpha}\right\}\quad 1\leq i\leq n$.

Theorem (A. Ostrowski): For any $\alpha\in (0,1)$, all the eigenvalues of $A$ lie in the union of these $n$ circles:$\sigma(A)\subseteq\bigcup_{i}O_i$.

\begin{proof}
If $R_i=0$, the theorem says $a_{ii}$ is an eigenvalue, which
is obviously true. Let's then concentrate on the $R_i\neq 0$. By eigenvalue definition, we have:

\[
(\lambda -a_{ii})x_{i}=\sum_{j\neq i}a_{ij}x_{j} 
\]

so that, recalling H\"older's inequality with $p=1/\alpha $ and $q=1/(1-\alpha
)$ (to have $p,q>1$, we must have $\alpha \in (0,1)$)

\begin{eqnarray*}
\left\vert \lambda -a_{ii}\right\vert \left\vert x_{i}\right\vert &\leq
&\sum_{j\neq i}\left\vert a_{ij}\right\vert \left\vert x_{j}\right\vert \\
&=&\sum_{j\neq i}\left\vert a_{ij}\right\vert ^{\alpha }\left\vert
a_{ij}\right\vert ^{1-\alpha }\left\vert x_{j}\right\vert \\
&\leq &\left( \sum_{j\neq i}(\left\vert a_{ij}\right\vert ^{\alpha
})^{1/\alpha }\right) ^{\alpha }\left( \sum_{j\neq i}(\left\vert
a_{ij}\right\vert ^{1-\alpha }\left\vert x_{j}\right\vert )^{1/(1-\alpha
)}\right) ^{1-\alpha } \\
&=&\left( \sum_{j\neq i}\left\vert a_{ij}\right\vert \right) ^{\alpha
}\left( \sum_{j\neq i}\left\vert a_{ij}\right\vert \left\vert
x_{j}\right\vert ^{1/(1-\alpha )}\right) ^{1-\alpha } \\
 &=&R_{i}^{\alpha }\left(
\sum_{j\neq i}\left\vert a_{ij}\right\vert \left\vert x_{j}\right\vert
^{1/(1-\alpha )}\right) ^{1-\alpha }
\end{eqnarray*}
which means%
\[
\frac{\left\vert \lambda -a_{ii}\right\vert ^{1/(1-\alpha )}}{R_{i}^{\alpha
/(1-\alpha )}}\left\vert x_{i}\right\vert ^{1/(1-\alpha )}\leq \sum_{j\neq
i}\left\vert a_{ij}\right\vert \left\vert x_{j}\right\vert ^{1/(1-\alpha )} 
\]

Summing over all $i$, one obtains
\[
\sum_{i=1}^{n}\frac{\left\vert \lambda -a_{ii}\right\vert ^{1/(1-\alpha )}}{%
R_{i}^{\alpha /(1-\alpha )}}\left\vert x_{i}\right\vert ^{1/(1-\alpha )}\leq
\sum_{i=1}^{n}\sum_{j\neq i}\left\vert a_{ij}\right\vert \left\vert x_{j}\right\vert ^{1/(1-\alpha)}=\sum_{j=1}^{n}C_{j}\left\vert x_{j}\right\vert ^{1/(1-\alpha )} 
\]

If, for each $i$, the coefficient of $\left\vert x_{i}\right\vert ^{1/(1-\alpha )}$ in the first
sum would be greater than the coefficient of the same term in the right-hand
side, inequality couldn't hold. So we can conclude that at least one index $p$ exists
such as

\[
\frac{\left\vert \lambda -a_{pp}\right\vert ^{1/(1-\alpha )}}{R_{p}^{\alpha
/(1-\alpha )}}\leq C_{p} 
\]

that is

\[
\left\vert \lambda -a_{pp}\right\vert \leq R_{p}^{\alpha }C_{p}^{1-\alpha } 
\]

which is the thesis.
\end{proof}

\bigskip

Remarks:

The Gershgorin theorem is obtained as a limit for $\alpha \rightarrow 0$ or
for $\alpha \rightarrow 1$; in other words, Ostrowski's theorem represents a kind of "continuous deformation" between the two Gershgorin rows and columns sets.

\begin{thebibliography}{6}
\bibitem{Horn} R. A. Horn, C. R. Johnson,
\emph{Matrix Analysis}, Cambridge University Press, 1985
\end{thebibliography}
%%%%%
%%%%%
\end{document}
