\documentclass[12pt]{article}
\usepackage{pmmeta}
\pmcanonicalname{RanknullityTheorem}
\pmcreated{2013-03-22 12:24:09}
\pmmodified{2013-03-22 12:24:09}
\pmowner{rmilson}{146}
\pmmodifier{rmilson}{146}
\pmtitle{rank-nullity theorem}
\pmrecord{8}{32238}
\pmprivacy{1}
\pmauthor{rmilson}{146}
\pmtype{Theorem}
\pmcomment{trigger rebuild}
\pmclassification{msc}{15A03}
\pmclassification{msc}{15A06}
\pmrelated{Overdetermined}
\pmrelated{Underdetermined}
\pmrelated{RankLinearMapping}
\pmrelated{Nullity}
\pmrelated{UnderDetermined}
\pmrelated{FiniteDimensionalLinearProblem}

\usepackage{amsmath}
\usepackage{amsfonts}
\usepackage{amssymb}

\newcommand{\reals}{\mathbb{R}}
\newcommand{\natnums}{\mathbb{N}}
\newcommand{\cnums}{\mathbb{C}}

\newcommand{\lp}{\left(}
\newcommand{\rp}{\right)}
\newcommand{\lb}{\left[}
\newcommand{\rb}{\right]}

\newcommand{\supth}{^{\text{th}}}


\newtheorem{proposition}{Proposition}
\begin{document}
The sum of the rank and the nullity of a linear mapping gives
the dimension of the mapping's domain. More precisely, let
$T:V\rightarrow W$ be a linear mapping. If $V$ is a
finite-dimensional, then
$$\dim V = \dim \mathop{\mathrm{Ker}} T + \dim \mathop{\mathrm{Img}}
T.$$

The intuitive content of the Rank-Nullity theorem is the principle that
\begin{quote}\em
  Every independent linear constraint takes away one degree of freedom.
\end{quote}
The rank is just the number of independent linear constraints on $v\in
V$ imposed
by the equation
$$T(v)=0.$$
The dimension of $V$ is the number of unconstrained degrees of freedom.
The nullity is the degrees of freedom in the resulting space of
solutions.
To put it yet another way:
\begin{quote} \em
  The number of variables {\bf minus} the number of independent linear
  constraints {\bf equals} 
 the number of linearly independent solutions.
\end{quote}
%%%%%
%%%%%
\end{document}
