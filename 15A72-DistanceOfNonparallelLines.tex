\documentclass[12pt]{article}
\usepackage{pmmeta}
\pmcanonicalname{DistanceOfNonparallelLines}
\pmcreated{2013-03-22 15:27:16}
\pmmodified{2013-03-22 15:27:16}
\pmowner{pahio}{2872}
\pmmodifier{pahio}{2872}
\pmtitle{distance of non-parallel lines}
\pmrecord{15}{37304}
\pmprivacy{1}
\pmauthor{pahio}{2872}
\pmtype{Derivation}
\pmcomment{trigger rebuild}
\pmclassification{msc}{15A72}
\pmsynonym{distance of lines}{DistanceOfNonparallelLines}
\pmrelated{LineInSpace}
\pmrelated{DistanceFromPointToALine}
\pmrelated{EuclideanDistance}
\pmrelated{AngleBetweenTwoLines}
\pmdefines{agonic lines}
\pmdefines{skew lines}

% this is the default PlanetMath preamble.  as your knowledge
% of TeX increases, you will probably want to edit this, but
% it should be fine as is for beginners.

% almost certainly you want these
\usepackage{amssymb}
\usepackage{amsmath}
\usepackage{amsfonts}

% used for TeXing text within eps files
%\usepackage{psfrag}
% need this for including graphics (\includegraphics)
%\usepackage{graphicx}
% for neatly defining theorems and propositions
 \usepackage{amsthm}
% making logically defined graphics
%%%\usepackage{xypic}

% there are many more packages, add them here as you need them

% define commands here

\theoremstyle{definition}
\newtheorem*{thmplain}{Theorem}
\begin{document}
As an application of the \PMlinkname{vector product}{CrossProduct} we derive the expression of the \PMlinkescapetext{distance} $d$ between two non-parallel \PMlinkescapetext{straight} lines in $\mathbb{R}^3$.

Suppose that the position vectors of the points of the two non-parallel lines are expressed in parametric forms
           $$\vec{r} = \vec{a}\!+\!s\vec{u}$$
and
           $$\vec{r} = \vec{b}\!+\!t\vec{v},$$
where $s$ and $t$ are parameters.\, A common \PMlinkescapetext{normal} vector of the lines is the cross product $\vec{u}\times\vec{v}$ of the direction vectors of the lines, and it may be normed to a unit vector
      $$\vec{n} := \frac{\vec{u}\!\times\!\vec{v}}{|\vec{u}\!\times\!\vec{v}|}$$
by dividing it by its \PMlinkescapetext{length}, which is distinct from 0 because of the non-parallelity.\, The vectors $\vec{a}$ and $\vec{b}$ are the position vectors of certain points $A$ and $B$ on the lines, and thus their difference $\vec{a}\!-\!\vec{b}$ is the vector from $B$ to $A$.\, If we project $\vec{a}\!-\!\vec{b}$ on the unit normal $\vec{n}$, the obtained vector
       $$\vec{d} := [(\vec{a}\!-\!\vec{b})\!\cdot\!\vec{n}]\,\vec{n}$$
has the sought \PMlinkescapetext{length}\, 
$d = |(\vec{a}\!-\!\vec{b})\!\cdot\!\vec{n}|$,\, i.e.
$$d = \frac{|(\vec{a}\!-\!\vec{b})\cdot(\vec{u}\!\times\!\vec{v})|}{|\vec{u}\!\times\!\vec{v}|}.
$$
For illustrating that $d$ is the minimal distance between points of the two lines we underline, that the construction of $d$ guarantees that it connects two points on the lines and is perpendicular to both lines --- thus any displacement of its end point makes it longer.

\textbf{Notes.}\; The numerator is the absolute value of a triple scalar product.\, If the lines intersect each other, then the connecting vector $\vec{a}\!-\!\vec{b}$ is at right angles to the common normal vector $\vec{n}$ of their plane and thus the dot product of these vectors vanishes, i.e. also\, $d = 0$.\, If the lines do not intersect, they are called {\em agonic lines} or {\em skew lines};\, then\, $d > 0$.
%%%%%
%%%%%
\end{document}
