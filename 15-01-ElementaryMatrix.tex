\documentclass[12pt]{article}
\usepackage{pmmeta}
\pmcanonicalname{ElementaryMatrix}
\pmcreated{2013-03-22 18:30:38}
\pmmodified{2013-03-22 18:30:38}
\pmowner{CWoo}{3771}
\pmmodifier{CWoo}{3771}
\pmtitle{elementary matrix}
\pmrecord{13}{41195}
\pmprivacy{1}
\pmauthor{CWoo}{3771}
\pmtype{Definition}
\pmcomment{trigger rebuild}
\pmclassification{msc}{15-01}
\pmrelated{MatrixUnit}
\pmrelated{GaussianElimination}
\pmdefines{elementary operation}
\pmdefines{elementary column operation}
\pmdefines{elementary row operation}
\pmdefines{basic diagonal matrix}
\pmdefines{transposition matrix}
\pmdefines{row replacement matrix}

\usepackage{amssymb,amscd}
\usepackage{amsmath}
\usepackage{amsfonts}
\usepackage{mathrsfs}

% used for TeXing text within eps files
%\usepackage{psfrag}
% need this for including graphics (\includegraphics)
%\usepackage{graphicx}
% for neatly defining theorems and propositions
\usepackage{amsthm}
% making logically defined graphics
%%\usepackage{xypic}
\usepackage{pst-plot}

% define commands here
\newcommand*{\abs}[1]{\left\lvert #1\right\rvert}
\newtheorem{prop}{Proposition}
\newtheorem{thm}{Theorem}
\newtheorem{ex}{Example}
\newcommand{\real}{\mathbb{R}}
\newcommand{\pdiff}[2]{\frac{\partial #1}{\partial #2}}
\newcommand{\mpdiff}[3]{\frac{\partial^#1 #2}{\partial #3^#1}}

\begin{document}
\subsubsection*{Elementary Operations on Matrices}

Let $\mathbb{M}$ be the set of all $m\times n$ matrices (over some commutative ring $R$).  An operation on $\mathbb{M}$ is called an \emph{elementary row operation} if it takes a matrix $M\in \mathbb{M}$, and does one of the following:
\begin{enumerate}
\item interchanges of two rows of $M$,
\item multiply a row of $M$ by a non-zero element of $R$,
\item add a (constant) multiple of a row of $M$ to another row of $M$.
\end{enumerate}
An \emph{elementary column operation} is defined similarly.  An operation on $\mathbb{M}$ is an \emph{elementary operation} if it is either an elementary row operation or elementary column operation.

For example, if $M=\begin{pmatrix} a & b \\ c & d \\ e & f \end{pmatrix}$, then the following operations correspond respectively to the three types of elementary row operations described above
\begin{enumerate}
\item $\begin{pmatrix} a & b \\ e & f \\ c & d \end{pmatrix}$ is obtained by interchanging rows 2 and 3 of $M$,
\item $\begin{pmatrix} a & b \\ rc & rd \\ e & f \end{pmatrix}$ is obtained by multiplying $r\ne 0$ to the second row of $M$,
\item $\begin{pmatrix} a & b \\ c & d \\ sa+e & sb+f \end{pmatrix}$ is obtained by adding to row 1 multiplied by $s$ to row 3 of $M$.
\end{enumerate}

Some immediate observation: elementary operations of type 1 and 3 are always invertible.  The inverse of type 1 elementary operation is itself, as interchanging of rows twice gets you back the original matrix.  The inverse of type 3 elementary operation is to add the negative of the multiple of the first row to the second row, thus returning the second row back to its original form.  Type 2 is invertible provided that the multiplier has an inverse.

Some notation: for each type $k$ (where $k=1,2,3$) of elementary operations, let $E_c^k(A)$ be the set of all matrices obtained from $A$ via an elementary column operation of type $k$, and $E_r^k(A)$ the set of all matrices obtained from $A$ via an elementary row operation of type $k$.

\subsubsection*{Elementary Matrices}

Now, assume $R$ has $1$.  An $n\times n$ \emph{elementary matrix} is a (square) matrix obtained from the identity matrix $I_n$ by performing an elementary operation.  As a result, we have three types of elementary matrices, each corresponding to a type of elementary operations:
\begin{enumerate}
\item \emph{transposition matrix} $T_{ij}$: an matrix obtained from $I_n$ with rows $i$ and $j$ switched,
\item \emph{basic diagonal matrix} $D_i(r)$: a diagonal matrix whose entries are $1$ except in cell $(i,i)$, whose entry is a non-zero element $r$ of $R$
\item \emph{row replacement matrix} $E_{ij}(s)$: $I_n + s U_{ij}$, where $s\in R$ and $U_{ij}$ is a matrix unit with $i\ne j$.
\end{enumerate}

For example, among the $3\times 3$ matrices, we have 
$$T_{12} = \begin{pmatrix} 0 & 1 & 0 \\ 1 & 0 & 0 \\ 0 & 0 & 1 \end{pmatrix}, \quad
D_3(r) = \begin{pmatrix} 1 & 0 & 0 \\ 0 & 1 & 0 \\ 0 & 0 & r \end{pmatrix},\quad\mbox{and}\quad
E_{32}(s) = \begin{pmatrix} 1 & 0 & 0 \\ 0 & 1 & 0 \\ 0 & s & 1 \end{pmatrix}$$

For each positive integer $n$, let $\mathbb{E}^k(n)$ be the collection of all $n\times n$ elementary matrices of type $k$, where $k=1,2,3$.

Below are some basic properties of elementary matrices:
\begin{itemize}
\item $T_{ij}=T_{ji}$, and $T_{ij}^2=I_n$.
\item $D_i(r)D_i(r^{-1})=I_n$, provided that $r^{-1}$ exists.
\item $E_{ij}(s) E_{ij}(-s) = I_n$.
\item $\det(T_{ij})=-1$, $\det(D_i(r))=r$, and $\det(E_{ij}(s))=1$.
\item If $A$ is an $m\times n$ matrix, then $$E_c^k(A)=\lbrace AE \mid E \in \mathbb{E}^k(n) \rbrace \qquad \mbox{and} \qquad E_r^k(A)=\lbrace EA \mid E \in \mathbb{E}^k(m) \rbrace.$$
\item Every non-singular matrix can be written as a product of elementary matrices.  This is the same as saying: given a non-singular matrix, one can perform a finite number of elementary row (column) operations on it to obtain the identity matrix.
\end{itemize}

\textbf{Remarks}.  
\begin{itemize}
\item One can also define elementary matrix operations on matrices over general rings.  However, care must be taken to consider left scalar multiplications and right scalar multiplications as separate operations.
\item The discussion above pertains to elementary linear algebra.  In algebraic K-theory, an elementary matrix is defined only as a row replacement matrix (type 3) above.
\end{itemize}

%%%%%
%%%%%
\end{document}
