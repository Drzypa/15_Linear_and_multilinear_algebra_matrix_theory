\documentclass[12pt]{article}
\usepackage{pmmeta}
\pmcanonicalname{GraphOfEquationxyConstant}
\pmcreated{2013-03-22 17:30:12}
\pmmodified{2013-03-22 17:30:12}
\pmowner{pahio}{2872}
\pmmodifier{pahio}{2872}
\pmtitle{graph of equation $\,xy =$ constant}
\pmrecord{9}{39892}
\pmprivacy{1}
\pmauthor{pahio}{2872}
\pmtype{Derivation}
\pmcomment{trigger rebuild}
\pmclassification{msc}{15-00}
\pmclassification{msc}{51N20}
\pmsynonym{equation $xy =$ constant}{GraphOfEquationxyConstant}
%\pmkeywords{rectangular hyperbola}
\pmrelated{Variation}
\pmrelated{RuledSurface}
\pmrelated{ExactTrigonometryTables}
\pmrelated{Hyperbola2}
\pmrelated{UncertaintyPrinciple}
\pmrelated{Polytrope}

\endmetadata

% this is the default PlanetMath preamble.  as your knowledge
% of TeX increases, you will probably want to edit this, but
% it should be fine as is for beginners.

% almost certainly you want these
\usepackage{amssymb}
\usepackage{amsmath}
\usepackage{amsfonts}

% used for TeXing text within eps files
%\usepackage{psfrag}
% need this for including graphics (\includegraphics)
%\usepackage{graphicx}
% for neatly defining theorems and propositions
 \usepackage{amsthm}
% making logically defined graphics
%%%\usepackage{xypic}

% there are many more packages, add them here as you need them

% define commands here

\theoremstyle{definition}
\newtheorem*{thmplain}{Theorem}

\begin{document}
Consider the equation \,$xy = c$,\, i.e.
\begin{align}
    y = \frac{c}{x},
\end{align}
where $c$ is a non-zero real constant.  Such a dependence between the real variables $x$ and $y$ is called an \PMlinkname{inverse proportionality}{Variation}.\\  

The graph of (1) may be inferred to be a \PMlinkname{hyperbola}{Hyperbola2}, because the curve has two asymptotes (see asymptotes of graph of rational function) and because the form 
\begin{align}
    xy-c = 0
\end{align}
of the equation is of second \PMlinkname{degree}{PolynomialRing} (see conic, tangent of conic section).\\

One can also see the graph of the equation (2) in such a coordinate system ($x',\,y'$) where the equation takes a canonical form of the \PMlinkname{hyperbola}{Hyperbola2}.  The symmetry of (2) with respect to the variables $x$ and $y$ suggests to take for the new coordinate axes the axis angle bisectors\, $y = \pm{x}$.  Therefore one has to rotate the old coordinate axes $45^\circ$, i.e.
\begin{align}
\begin{cases}
\displaystyle x = x'\cos45^\circ-y'\sin45^\circ = \frac{x'-y'}{\sqrt{2}}\\
\displaystyle y = x'\sin45^\circ+y'\cos45^\circ = \frac{x'+y'}{\sqrt{2}}
\end{cases}
\end{align}
($\sin45^\circ = \cos45^\circ = \frac{1}{\sqrt{2}}$).  Substituting (3) into (2) yields
$$\frac{x'^2-y'^2}{2}-c = 0,$$
i.e.
\begin{align}
   \frac{x'^2}{2c}-\frac{y'^2}{2c} = 1.
\end{align}
This is recognised to be the equation of a rectangular hyperbola with the transversal axis and the \PMlinkname{conjugate axis}{Hyperbola2} on the coordinate axes.
%%%%%
%%%%%
\end{document}
