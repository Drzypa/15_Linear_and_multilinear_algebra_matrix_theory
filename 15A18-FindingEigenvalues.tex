\documentclass[12pt]{article}
\usepackage{pmmeta}
\pmcanonicalname{FindingEigenvalues}
\pmcreated{2013-03-22 15:52:35}
\pmmodified{2013-03-22 15:52:35}
\pmowner{PrimeFan}{13766}
\pmmodifier{PrimeFan}{13766}
\pmtitle{finding eigenvalues}
\pmrecord{6}{37873}
\pmprivacy{1}
\pmauthor{PrimeFan}{13766}
\pmtype{Example}
\pmcomment{trigger rebuild}
\pmclassification{msc}{15A18}
%\pmkeywords{eigenvalue}
%\pmkeywords{eigenvector}

\endmetadata

% this is the default PlanetMath preamble.  as your knowledge
% of TeX increases, you will probably want to edit this, but
% it should be fine as is for beginners.

% almost certainly you want these
\usepackage{amssymb}
\usepackage{amsmath}
\usepackage{amsfonts}

% used for TeXing text within eps files
%\usepackage{psfrag}
% need this for including graphics (\includegraphics)
%\usepackage{graphicx}
% for neatly defining theorems and propositions
%\usepackage{amsthm}
% making logically defined graphics
%%%\usepackage{xypic}

% there are many more packages, add them here as you need them

% define commands here
\begin{document}
This example investigates eigenvalues and the similarity transformation used
to diagonalize matrices.  We seek the eigenvalues of the matrix $A$ below.  Afterward,
we can transform this matrix into a diagonal matrix which has many useful applications.\\
$$A=\left( \begin{array}{cc}
2 & 1 \\
1 & 2
\end{array} \right )$$
Here, we need to solve the corresponding matrix equation;\\
$$\left( \begin{array}{cc}
2 & 1 \\
1 & 2
\end{array} \right )
\left( \begin{array}{c}
x_1 \\
x_2
\end{array} \right )=\lambda
\left( \begin{array}{c}
x_1 \\
x_2
\end{array} \right )$$ or\\
$$AX=\lambda X$$ 
rearranging gives
$$AX-\lambda X=0$$ or\\
$$(A-\lambda I) X=0$$\\
We seek the values for $\lambda$ and $X$.
First, we need to solve the characteristic equation of $A$.  We do this by finding
$det(A-\lambda I)$.  First, calculating $A-\lambda I$ gives;\\
$$A -\lambda I=
\left( \begin{array}{cc}
2-\lambda & 1 \\
1 & 2-\lambda
\end{array} \right )$$
Next, calculating $det(A-\lambda I)$ yields\\
$$det(A-\lambda I)=(2-\lambda)^2-1=\lambda^2-4\lambda+3=(\lambda-1)(\lambda-3)=0$$\\
Substituting $\lambda=1$ into $(A-\lambda I)X$ gives...\\
$$\left\{ \begin{array}{c}
x_1+x_2=0 \\
x_1+x_2=0
\end{array} \right.$$
so that $x_2=-x_1$ and the corresponding eigenvector is
$$\left( \begin{array}{c}
t \\
-t
\end{array} \right )=t
\left( \begin{array}{c}
1 \\
-1
\end{array} \right )$$  where $t\ne 0.$\\
Substituting $\lambda=3$ gives...\\
$$\left\{ \begin{array}{c}
-x_1+x_2=0 \\
x_1-x_2=0
\end{array} \right.$$
so that $x_2=x_1$ and the corresponding eigenvector is
$$\left( \begin{array}{c}
t \\
t
\end{array} \right )=t
\left( \begin{array}{c}
1 \\
1
\end{array} \right )$$  where $t\ne 0.$\\
Finally, to diagonalize $A$ we let the eigenvectors be the columns of a new matrix\\
$$P=\left( \begin{array}{cc}
1 & -1 \\
1 & 1
\end{array} \right )$$
and then since our eigenvectors are linearly independent we can also find;\\
$$P^{-1}=\frac{1}{2} \left( \begin{array}{cc}
1 & 1 \\
-1 & 1
\end{array} \right )$$
then we create a diagonal matrix as follows...\\
$$D=P^{-1}AP=\left( \begin{array}{cc}
1 & 0 \\
0 & 3
\end{array} \right )$$
Computing powers of $A$ is a very useful application of $D$.  Solving for $A$ lets us compute powers of $A$\\
$$A=PDP^{-1}$$\\
so that\\
$$A^n=PD^nP^{-1}$$\\  or
$$A^n=P\left( \begin{array}{cc}
1^n & 0 \\
0 & 3^n
\end{array} \right )P^{-1}$$\\
%%%%%
%%%%%
\end{document}
