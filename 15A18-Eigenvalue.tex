\documentclass[12pt]{article}
\usepackage{pmmeta}
\pmcanonicalname{Eigenvalue}
\pmcreated{2013-03-22 12:11:52}
\pmmodified{2013-03-22 12:11:52}
\pmowner{Koro}{127}
\pmmodifier{Koro}{127}
\pmtitle{eigenvalue}
\pmrecord{15}{31496}
\pmprivacy{1}
\pmauthor{Koro}{127}
\pmtype{Definition}
\pmcomment{trigger rebuild}
\pmclassification{msc}{15A18}
\pmrelated{EigenvalueProblem}
\pmrelated{SimilarMatrix}
\pmrelated{Eigenvector}
\pmrelated{SingularValueDecomposition}
\pmdefines{eigenvalue}
\pmdefines{spectral value}

\endmetadata

\usepackage{amssymb}
\usepackage{amsmath}
\usepackage{amsfonts}
\begin{document}
\newcommand{\Z}{\mathbb{Z}}
\newcommand{\R}{\mathbb{R}}
\newcommand{\C}{\mathbb{C}}
\PMlinkescapeword{clearly}
\PMlinkescapeword{information}
\PMlinkescapeword{scheme}
\PMlinkescapeword{connection}
Let $V$ be a vector space over a field $k$, and let $A$ be an
endomorphism of $V$ (meaning a linear mapping of $V$ into itself).
A scalar $\lambda\in k$ is said to be an
\emph{eigenvalue} of $A$ if there is a nonzero $x \in V$ for which
\begin{equation}
Ax = \lambda x\;.
\end{equation}
Geometrically, one thinks of a vector whose direction is unchanged
by the action of $A$, but whose magnitude is multiplied by $\lambda$.

If $V$ is finite dimensional, elementary linear algebra shows that
there are several equivalent definitions of an eigenvalue:

\noindent
(2) The linear mapping $$B=\lambda I - A$$
i.e. $B:x\mapsto \lambda x-Ax$, has no inverse.

\noindent
(3) $B$ is not injective.

\noindent
(4) $B$ is not surjective.

\noindent
(5) $\det(B)=0$, i.e. $\det(\lambda I-A)=0$.

But if $V$ is of infinite dimension, (5) has no meaning and the
conditions (2) and (4) are not equivalent to (1).
A scalar $\lambda$ satisfying (2) (called a \emph{spectral value} of
$A$) need not be an eigenvalue. Consider for example the complex
vector space $V$ of all sequences $(x_n)_{n=1}^{\infty}$ of complex
numbers with the obvious operations, and the map $A: V \to V$ given by
$$ A(x_1, x_2, x_3, \dots) = (0, x_1, x_2, x_3, \dots) \;. $$
Zero is a spectral value of $A$, but clearly not an eigenvalue.

Now suppose again that $V$ is of finite dimension, say $n$.
The function $$\chi(\lambda)=\det(B)$$
is a polynomial of degree $n$ over $k$ in the
variable $\lambda$, called the \emph{characteristic polynomial} of the
endomorphism $A$. (Note that some writers define the characteristic
polynomial as $\det(A-\lambda I)$ rather than $\det(\lambda I-A)$, but the
two have the same zeros.)

If $k$ is $\C$ or any other algebraically closed field, or if $k=\R$
and $n$ is odd, then $\chi$ has at least one zero, meaning that $A$
has at least one eigenvalue. In no case does $A$ have more than $n$
eigenvalues.

Although we didn't need to do so here, one can compute the coefficients
of $\chi$ by introducing a basis of $V$ and the corresponding matrix for
$B$. Unfortunately, computing $n \times n$ determinants and finding roots
of polynomials of degree $n$ are computationally messy procedures
for even moderately large $n$, so for most practical purposes
variations on this naive scheme are needed.  See the eigenvalue
problem for more information.

If $k=\C$ but the coefficients of $\chi$ are real (and in particular if
$V$ has a basis for which the matrix of $A$ has only real entries), then
the non-real eigenvalues of $A$ appear in conjugate pairs. For example,
if $n=2$ and, for some basis, $A$ has the matrix
$$ A = \begin{pmatrix} 0 & -1 \\ 1 & 0 \end{pmatrix} $$
then $\chi(\lambda)=\lambda^2+1$, with the two zeros $\pm i$.

Eigenvalues are of relatively little importance in connection with
an infinite-dimensional vector space, unless that space is endowed with
some additional structure, typically that of a Banach space or Hilbert space. But in those cases the notion is of great value in
physics, engineering, and mathematics proper. Look for ``spectral theory''
for more on that subject.
%%%%%
%%%%%
%%%%%
\end{document}
