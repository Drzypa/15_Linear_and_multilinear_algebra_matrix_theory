\documentclass[12pt]{article}
\usepackage{pmmeta}
\pmcanonicalname{ToeplitzMatrix}
\pmcreated{2013-03-22 13:04:22}
\pmmodified{2013-03-22 13:04:22}
\pmowner{akrowne}{2}
\pmmodifier{akrowne}{2}
\pmtitle{Toeplitz matrix}
\pmrecord{8}{33482}
\pmprivacy{1}
\pmauthor{akrowne}{2}
\pmtype{Definition}
\pmcomment{trigger rebuild}
\pmclassification{msc}{15A09}
\pmclassification{msc}{65F35}
\pmclassification{msc}{15A57}

\usepackage{amssymb}
\usepackage{amsmath}
\usepackage{amsfonts}

%\usepackage{psfrag}
%\usepackage{graphicx}
%%%\usepackage{xypic}
\begin{document}
\section{Toeplitz Matrix}

A \emph{Toeplitz matrix} is any $n\times n$ matrix with values constant along each (top-left to lower-right) diagonal.  That is, a Toeplitz matrix has the form

$$ \begin{bmatrix}
 a_0 & a_1 & a_2 & \cdots & a_{n-1} \\
 a_{-1} & a_0 & a_1 & \ddots & \vdots \\
 a_{-2} & a_{-1} & a_0 & \ddots & a_2 \\
 \vdots & \ddots & \ddots & \ddots & a_1 \\
 a_{-(n-1)} & \cdots & a_{-2} & a_{-1} & a_0
\end{bmatrix} $$

Numerical problems involving Toeplitz matrices typically have fast solutions (only $2n-1$ distinct elements need to be solved for, as opposed to $n^2$).  For example, the inverse of a symmetric, positive-definite $n\times n$ Toeplitz matrix can be found in $\mathcal{O}(n^2)$ \PMlinkname{time}{TimeComplexity}.

\begin{thebibliography}{3}
\bibitem{Golub} Golub and Van Loan, \emph{Matrix Computations}, Johns Hopkins University Press 1993
\end{thebibliography}
%%%%%
%%%%%
\end{document}
