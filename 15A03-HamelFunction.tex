\documentclass[12pt]{article}
\usepackage{pmmeta}
\pmcanonicalname{HamelFunction}
\pmcreated{2013-03-22 14:15:19}
\pmmodified{2013-03-22 14:15:19}
\pmowner{mathcam}{2727}
\pmmodifier{mathcam}{2727}
\pmtitle{Hamel function}
\pmrecord{7}{35703}
\pmprivacy{1}
\pmauthor{mathcam}{2727}
\pmtype{Definition}
\pmcomment{trigger rebuild}
\pmclassification{msc}{15A03}
\pmclassification{msc}{54C40}

\endmetadata

% this is the default PlanetMath preamble.  as your knowledge
% of TeX increases, you will probably want to edit this, but
% it should be fine as is for beginners.

% almost certainly you want these
\usepackage{amssymb}
\usepackage{url}
\usepackage{amsmath}
\usepackage{amsfonts}

% used for TeXing text within eps files
%\usepackage{psfrag}
% need this for including graphics (\includegraphics)
%\usepackage{graphicx}
% for neatly defining theorems and propositions
%\usepackage{amsthm}
% making logically defined graphics
%%%\usepackage{xypic}

% there are many more packages, add them here as you need them

% define commands here
\begin{document}
A function $h : \mathbb{R}^n \to \mathbb{R}$ is said to be a \emph{Hamel function} if $h$, considered as a subset $\{(x,h(x)\}\subset \mathbb{R}^{n+1}$, is a Hamel basis for $\mathbb{R}^{n+1}$ over $\mathbb{Q}$.  We denote the set of $n$-dimensional Hamel function by $HF(\mathbb{R}^n)$.

{\bf References}
\begin{itemize}
\item Poltka, K.  \emph{On Functions Whose Graph is a Hamel Basis}.  Unpublised Ph.D. work.  Online at \url{http://academic.scranton.edu/faculty/PLOTKAK2/publications/ham_0911.pdf}
\end{itemize}
%%%%%
%%%%%
\end{document}
