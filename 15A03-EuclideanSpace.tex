\documentclass[12pt]{article}
\usepackage{pmmeta}
\pmcanonicalname{EuclideanSpace}
\pmcreated{2013-03-22 14:17:19}
\pmmodified{2013-03-22 14:17:19}
\pmowner{rmilson}{146}
\pmmodifier{rmilson}{146}
\pmtitle{Euclidean space}
\pmrecord{16}{35743}
\pmprivacy{1}
\pmauthor{rmilson}{146}
\pmtype{Definition}
\pmcomment{trigger rebuild}
\pmclassification{msc}{15A03}
\pmclassification{msc}{51M05}
\pmrelated{EuclideanVectorProperties}
\pmrelated{InnerProduct}
\pmrelated{PositiveDefinite}
\pmrelated{EuclideanDistance}
\pmrelated{Vector}
\pmdefines{Euclidean plane}

\endmetadata

\usepackage{amsmath}
\usepackage{amsfonts}
\usepackage{amssymb}
\begin{document}
\section{Definition}
\emph{Euclidean $n$-space} is a metric space $(E,d)$
with the property that the group of isometries is transitive and is
isomorphic to an $n$-dimensional Euclidean vector space.  To be more
precise, we are saying that there exists an $n$-dimensional Euclidean
vector space $V$ with inner product $\langle \cdot,\cdot\rangle$ and a
mapping
\[ +: E\times V\to E \]
such that the following hold:
\begin{enumerate}
\item For all $x,y\in E$ there exists a unique $u\in V$ satisfying
\[ y=x+u,\quad d(x,y)^2=\langle u,u\rangle,\]
\item For all $x,y\in E$ and all $u\in V$ we have
\[ d(x+u,y+u)=d(x,y).\]
\item For all $x\in E$ and all $u,v\in V$ we have
\[ (x+u)+v=x+(u+v).\]
\end{enumerate}
Putting it more succinctly: $V$ acts transitively and effectively on
$E$ by isometries.

\paragraph{Remarks.}
\begin{itemize}
\item
The difference between Euclidean space and a Euclidean vector space is
one of loss of structure.  Euclidean space is a Euclidean vector space
that has ``forgotten'' its origin.
\item A 2-dimensional Euclidean space is often called a
  \emph{Euclidean plane}.
\end{itemize}
%%%%%
%%%%%
\end{document}
