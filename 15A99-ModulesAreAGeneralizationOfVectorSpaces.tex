\documentclass[12pt]{article}
\usepackage{pmmeta}
\pmcanonicalname{ModulesAreAGeneralizationOfVectorSpaces}
\pmcreated{2013-03-22 13:38:18}
\pmmodified{2013-03-22 13:38:18}
\pmowner{jgade}{861}
\pmmodifier{jgade}{861}
\pmtitle{modules are a generalization of vector spaces}
\pmrecord{7}{34288}
\pmprivacy{1}
\pmauthor{jgade}{861}
\pmtype{Example}
\pmcomment{trigger rebuild}
\pmclassification{msc}{15A99}
%\pmkeywords{Vector Space}
%\pmkeywords{Generalization}
%\pmkeywords{Module}

\endmetadata

% this is the default PlanetMath preamble.  as your knowledge
% of TeX increases, you will probably want to edit this, but
% it should be fine as is for beginners.

% almost certainly you want these
\usepackage{amssymb}
\usepackage{amsmath}
\usepackage{amsfonts}

% used for TeXing text within eps files
%\usepackage{psfrag}
% need this for including graphics (\includegraphics)
%\usepackage{graphicx}
% for neatly defining theorems and propositions



% making logically defined graphics
%%%\usepackage{xypic}

% there are many more packages, add them here as you need them

% define commands here
\begin{document}
A \PMlinkid{module}{1022} is the natural generalization of a vector space, in fact, when working over a field it is just another word for a vector space.

If $M$ and $N$ are $R$-modules then a mapping $f: M\to N$ is called an $R$-morphism (or homomorphism) if:
\[
\forall x,y\in M: f(x+y) = f(x) + f(y) \quad \mathrm{ and } \quad
\forall x\in M \forall \lambda \in R: f(\lambda x) = \lambda f(x)
\] Note as mentioned in the beginning, if $R$ is a field, these properties are the defining properties for a linear transformation.

Similarly in vector space terminology the image $\mathrm{Im} f := \{f(x): x\in M\}$ and kernel
$\mathrm{Ker} f := \{x\in M : f(x) = 0_N \}$ are called the range and null-space respectively.
%%%%%
%%%%%
\end{document}
