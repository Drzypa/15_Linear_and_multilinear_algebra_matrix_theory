\documentclass[12pt]{article}
\usepackage{pmmeta}
\pmcanonicalname{PentadiagonalMatrix}
\pmcreated{2013-03-22 13:23:23}
\pmmodified{2013-03-22 13:23:23}
\pmowner{drini}{3}
\pmmodifier{drini}{3}
\pmtitle{pentadiagonal matrix}
\pmrecord{6}{33927}
\pmprivacy{1}
\pmauthor{drini}{3}
\pmtype{Definition}
\pmcomment{trigger rebuild}
\pmclassification{msc}{15-00}
\pmclassification{msc}{65-00}
\pmsynonym{penta-diagonal matrix}{PentadiagonalMatrix}
%\pmkeywords{pentadiagonal penta-diagonal}
\pmrelated{TridiagonalMatrix}

\endmetadata

% this is the default PlanetMath preamble.  as your knowledge
% of TeX increases, you will probably want to edit this, but
% it should be fine as is for beginners.

% almost certainly you want these
\usepackage{amssymb}
\usepackage{amsmath}
\usepackage{amsfonts}

% used for TeXing text within eps files
%\usepackage{psfrag}
% need this for including graphics (\includegraphics)
%\usepackage{graphicx}
% for neatly defining theorems and propositions
%\usepackage{amsthm}
% making logically defined graphics
%%%\usepackage{xypic}

% there are many more packages, add them here as you need them

% define commands here
\begin{document}
An $n \times n$ {\bf pentadiagonal} matrix (with $n\ge 3$) is 
a matrix of the form
\[
\begin{pmatrix}
c_1    &   d_1  &   e_1     &   0       &  \cdots  & \cdots  &  0       \\ 
b_1    &   c_2  &   d_2     &   e_2     &  \ddots  &         &  \vdots  \\
a_1    &   b_2  &   \ddots  &   \ddots  &  \ddots  & \ddots  &  \vdots  \\
  0    &   a_2  &  \ddots   &  \ddots   & \ddots   & e_{n-3} &  0       \\ 
\vdots & \ddots &   \ddots   &   \ddots  &  \ddots  & d_{n-2} &  e_{n-2} \\
\vdots &        &   \ddots  &   a_{n-3} &  b_{n-2} & c_{n-1} &  d_{n-1} \\
0      & \cdots &   \cdots  &   0       &  a_{n-2} & b_{n-1} &  c_n
\end{pmatrix}.
\]
It follows that a pentadiagonal matrix is determined by five vectors:
one $n$-vector $c=(c_1,\ldots, c_n)$, 
two $(n-1)$-vectors $b=(b_1,\ldots, b_{n-1})$ 
               and  $d=(d_1,\ldots, d_{n-1})$,
and two $(n-2)$-vectors $a=(a_1,\ldots, a_{n-2})$ 
               and  $e=(e_1,\ldots, e_{n-2})$.
It follows that a pentadiagonal matrix is completely determined
by $n+2(n-1)+2(n-2)=5n-6$ scalars.
%%%%%
%%%%%
\end{document}
