\documentclass[12pt]{article}
\usepackage{pmmeta}
\pmcanonicalname{TensorDensity}
\pmcreated{2013-03-22 14:55:18}
\pmmodified{2013-03-22 14:55:18}
\pmowner{rspuzio}{6075}
\pmmodifier{rspuzio}{6075}
\pmtitle{tensor density}
\pmrecord{12}{36608}
\pmprivacy{1}
\pmauthor{rspuzio}{6075}
\pmtype{Definition}
\pmcomment{trigger rebuild}
\pmclassification{msc}{15A72}
\pmsynonym{density}{TensorDensity}
\pmrelated{tensor}

\endmetadata

% this is the default PlanetMath preamble.  as your knowledge
% of TeX increases, you will probably want to edit this, but
% it should be fine as is for beginners.

% almost certainly you want these
\usepackage{amssymb}
\usepackage{amsmath}
\usepackage{amsfonts}

% used for TeXing text within eps files
%\usepackage{psfrag}
% need this for including graphics (\includegraphics)
%\usepackage{graphicx}
% for neatly defining theorems and propositions
%\usepackage{amsthm}
% making logically defined graphics
%%%\usepackage{xypic}

% there are many more packages, add them here as you need them

% define commands here
\begin{document}
\subsection{Heuristic definition}

A tensor density is a quantity whose transformation law under change of basis involves the determinant of the transformation matrix (as opposed to a tensor, whose transformation law does not involve the determinant).  

\subsection{Linear Theory}

For any real number $p$, we may define a representation $\rho_p$ of the group $GL(\mathbb{R}^k)$ on the vector space of tensor arrays of rank $m,n$ as follows:
 $$(\rho_p (M) T)^{i_1, \ldots, i_n}_{j_1, \ldots j_m} = (\mathop{\rm det}(M))^p M^{i_1}_{l_1} \cdots M^{i_n}_{l_n} (M^{-1})_{k_1}^{j_1} \cdots (M^{-1})_{k_m}^{j_m} T^{i_1, \ldots, i_n}_{j_1, \ldots j_m}$$

A \emph{tensor density} $T$ of rank $m,n$ and weight $p$ is an element of the vector space on which this representation acts.

Note that if the weight equals zero, the concept of tensor density reduces to that of a tensor.

\subsection{Examples}

The simplest example of such a quantity is a scalar density.  Under a change of basis $y^i = M^i_j x^j$, a scalar density transforms as follows:
 $$\rho_p (S) = (\mathop{\rm det}(M))^p S$$

An important example of a tensor density is the Levi-Civita permutation symbol.  It is a density of weight $1$ because, under a change of coordinates,
 $$(\rho_1 \epsilon)_{j_1, \ldots j_m} = (\mathop{\rm det}(M))  (M^{-1})_{k_1}^{j_1} \cdots (M^{-1})_{k_m}^{j_m} \epsilon^{i_1, \ldots, i_n}_{j_1, \ldots j_m} = \epsilon_{k_1, \ldots k_m}$$

\subsection{Tensor Densities on Manifolds}

As with tensors, it is possible to define tensor density fields on manifolds.  On each coordinate neighborhood, the density field is given by a tensor array of functions.  When two neighborhoods overlap, the tensor arrays are related by the change of variable formula
 $$T^{i_1, \ldots, i_n}_{j_1, \ldots j_m} (x) = (\mathop{\rm det}(M))^p M^{i_1}_{l_1} \cdots M^{i_n}_{l_n} (M^{-1})_{k_1}^{j_1} \cdots (M^{-1})_{k_m}^{j_m} T^{i_1, \ldots, i_n}_{j_1, \ldots j_m} (y)$$
where $M$ is the Jacobian matrix of the change of variables.
%%%%%
%%%%%
\end{document}
