\documentclass[12pt]{article}
\usepackage{pmmeta}
\pmcanonicalname{ExampleOfBoundedOperatorWithNoEigenvalues}
\pmcreated{2013-03-22 17:57:53}
\pmmodified{2013-03-22 17:57:53}
\pmowner{asteroid}{17536}
\pmmodifier{asteroid}{17536}
\pmtitle{example of bounded operator with no eigenvalues}
\pmrecord{7}{40470}
\pmprivacy{1}
\pmauthor{asteroid}{17536}
\pmtype{Example}
\pmcomment{trigger rebuild}
\pmclassification{msc}{15A18}
\pmclassification{msc}{47A10}

\endmetadata

% this is the default PlanetMath preamble.  as your knowledge
% of TeX increases, you will probably want to edit this, but
% it should be fine as is for beginners.

% almost certainly you want these
\usepackage{amssymb}
\usepackage{amsmath}
\usepackage{amsfonts}

% used for TeXing text within eps files
%\usepackage{psfrag}
% need this for including graphics (\includegraphics)
%\usepackage{graphicx}
% for neatly defining theorems and propositions
%\usepackage{amsthm}
% making logically defined graphics
%%%\usepackage{xypic}

% there are many more packages, add them here as you need them

% define commands here

\begin{document}
In this entry we show that there are operators with no eigenvalues. Moreover, we exhibit an operator $T$ in a Hilbert space which is bounded, self-adjoint, has a non-empty spectrum but no eigenvalues. 

Consider the Hilbert space \PMlinkname{$L^2([0,1])$}{L2SpacesAreHilbertSpaces} and let $f:[0,1] \longrightarrow \mathbb{C}$ be the function $f(t) = t$.

Let $T:L^2([0,1]) \longrightarrow L^2([0,1])$ be the \PMlinkname{operator of multiplication}{MultiplicationOperatorOnMathbbL22} by $f$
\begin{displaymath}
T(\varphi) = f\varphi \,, \qquad\qquad \varphi \in L^2([0,1])
\end{displaymath}

Thus, $T$ is a bounded operator, since it is a multiplication operator  (see \PMlinkname{this entry}{OperatorNormOfMultiplicationOperatorOnL2}). Also, it is easily seen that $T$ is self-adjoint.

We now prove that $T$ has no eigenvalues: suppose $\lambda \in \mathbb{C}$ is an eigenvalue of $T$ and $\varphi$ is an eigenvector. Then,
\begin{displaymath}
T\varphi = \lambda \varphi
\end{displaymath}
This means that $(f-\lambda)\varphi =0$, but this is impossible for $\varphi \neq 0$ since $f-\lambda$ has at most one zero. Hence, $T$ has no eigenvalues.

Of course, since the Hilbert space is complex, the spectrum of $T$ is non-empty (see \PMlinkname{this entry}{SpectrumIsANonEmptyCompactSet}). Moreover, the spectrum of $T$ can be easily computed and seen to be the whole interval $[0,1]$, as we explain now:

It is known that an operator of multiplication by a continuous function $g$ is invertible if and only if $g$ is invertible. Thus, for every $\lambda \in \mathbb{C}$, $T-\lambda I$ is easily seen to be the operator of multiplication by $(f - \lambda)$. Hence, $T-\lambda I $ is not invertible if and only if $\lambda \in [0,1]$, i.e. $\sigma(T) = [0,1]$.
%%%%%
%%%%%
\end{document}
