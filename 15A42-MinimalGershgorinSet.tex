\documentclass[12pt]{article}
\usepackage{pmmeta}
\pmcanonicalname{MinimalGershgorinSet}
\pmcreated{2013-03-22 15:35:57}
\pmmodified{2013-03-22 15:35:57}
\pmowner{Andrea Ambrosio}{7332}
\pmmodifier{Andrea Ambrosio}{7332}
\pmtitle{minimal Gershgorin set}
\pmrecord{11}{37514}
\pmprivacy{1}
\pmauthor{Andrea Ambrosio}{7332}
\pmtype{Definition}
\pmcomment{trigger rebuild}
\pmclassification{msc}{15A42}

% this is the default PlanetMath preamble.  as your knowledge
% of TeX increases, you will probably want to edit this, but
% it should be fine as is for beginners.

% almost certainly you want these
\usepackage{amssymb}
\usepackage{amsmath}
\usepackage{amsfonts}

% used for TeXing text within eps files
%\usepackage{psfrag}
% need this for including graphics (\includegraphics)
%\usepackage{graphicx}
% for neatly defining theorems and propositions
\usepackage{amsthm}
% making logically defined graphics
%%%\usepackage{xypic}

% there are many more packages, add them here as you need them

% define commands here
\begin{document}
\PMlinkescapetext{Strictly} related to Gershgorin's theorem is the so called "minimal Gershgorin set".
For every $A\in\mathbf{C}^{n,n}$, $\mathbf{x}>0$ meaning $x_i>0\quad \forall i$, let's define its \emph{minimal Gershgorin set} $G(A)$ as:
\[
G(A)=\bigcap_{\mathbf{x}>0} G_{\mathbf{x}}(A),
\]
where
\[
G_{\mathbf{x}}(A)=\bigcup_{i=1}^n \left\{z\in\mathbf{C}:\left|z-a_{ii}\right|\leq\frac{1}{x_i}\sum_{j\ne i}\left|a_{ij}\right|x_j\right\}.
\]

Theorem: Let $A\in\mathbf{C}^{n,n}$, let $\sigma(A)$ be the spectrum of $A$ and let $G(A)$ be its minimal Gershgorin set defined as above. Then
\[
\sigma(A)\subseteq G(A).
\]

\begin{proof}

Given $\mathbf{x}>0$, let $X=diag\left\{x_1, x_2, \ldots, x_n\right\}$and let $B_X=X^{-1}AX$. Then $A$ and $B_X$ share the same spectrum, being similar. Due to definition, and keeping in mind that $X^{-1}=diag\left\{x_1^{-1}, x_2^{-1}, \ldots, x_n^{-1}\right\}$, we have $b_{ij}^{(X)}=a_{ij}\frac{x_j}{x_i}$
and, applying Gershgorin theorem to $B_X$, we get:

$\sigma(A)=\sigma(B_X)\subseteq\bigcup_{i=1}^n \left\{z\in\mathbf{C}:\left|z-a_{ii}\right|\leq\frac{1}{x_i}\sum_{j\ne i}\left|a_{ij}\right|x_j\right\}$

and, since this is true for any $\mathbf{x}>0$, we finally get the thesis.
\end{proof}
%%%%%
%%%%%
\end{document}
