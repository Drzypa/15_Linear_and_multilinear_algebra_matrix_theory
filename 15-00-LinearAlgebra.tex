\documentclass[12pt]{article}
\usepackage{pmmeta}
\pmcanonicalname{LinearAlgebra}
\pmcreated{2013-03-22 12:26:16}
\pmmodified{2013-03-22 12:26:16}
\pmowner{rmilson}{146}
\pmmodifier{rmilson}{146}
\pmtitle{linear algebra}
\pmrecord{10}{32530}
\pmprivacy{1}
\pmauthor{rmilson}{146}
\pmtype{Topic}
\pmcomment{trigger rebuild}
\pmclassification{msc}{15-00}
\pmclassification{msc}{15-01}

\usepackage{amsmath}
\usepackage{amsfonts}
\usepackage{amssymb}

\newcommand{\reals}{\mathbb{R}}
\newcommand{\natnums}{\mathbb{N}}
\newcommand{\cnums}{\mathbb{C}}

\newcommand{\lp}{\left(}
\newcommand{\rp}{\right)}
\newcommand{\lb}{\left[}
\newcommand{\rb}{\right]}

\newcommand{\supth}{^{\text{th}}}


\newtheorem{proposition}{Proposition}
\begin{document}
Linear algebra is the branch of mathematics devoted to the theory of
linear structure.  The axiomatic treatment of linear structure is
based on the notions of a {\em linear space} (more commonly known as a
{\em vector space}), and a {\em linear mapping}.  Broadly speaking,
there are two fundamental questions considered by linear algebra:
\begin{itemize}
\item the solution of a linear equation, and
\item diagonalization, a.k.a. the eigenvalue problem.
\end{itemize}
From the geometric point of view, ``linear'' is synonymous with
``straight'', and consequently linear algebra can be regarded as the
branch of mathematics dealing with lines and planes, as well as with
transformations of space that preserve ``straightness'', e.g.
rotations and reflections. The two fundamental questions, in geometric
terms, deal with
\begin{itemize}
\item the intersection of hyperplanes, and
\item the principal axes of an ellipsoid.
\end{itemize}

Linearity is a very basic notion, and consequently linear algebra has
applications in numerous areas of mathematics, science, and
engineering.  Diverse disciplines, such as differential equations,
differential geometry, the theory of relativity, quantum mechanics,
electrical circuits, computer graphics, and information theory benefit
from the notions and techniques of linear algebra.

Euclidean geometry is related to a specialized branch of linear
algebra that deals with linear measurement.  Here the relevant notions
are length and angle.  A typical question is the determination of
lines perpendicular to a given plane.  A somewhat less specialized
branch deals with affine structure, where the key notion is that of
area and volume. Here   determinants play an essential role.

Yet another branch of linear algebra is concerned with computation,
algorithms, and numerical approximation.  Important examples of such
techniques include: Gaussian elimination, the method of least squares,
LU factorization, QR decomposition, Gram-Schmidt orthogonalization,
singular value decomposition, and a number of iterative algorithms for
the calculation of eigenvalues and eigenvectors.


The following subject outline surveys 
key topics in linear algebra.

\begin{enumerate}
\item {\bf Linear structure.}
  \begin{enumerate}
  \item {\bf Introduction:} systems of linear equations, Gaussian
    elimination, matrices, matrix operations.
  \item {\bf Foundations:} fields and vector spaces, subspace, linear
    independence, basis, ordered basis, dimension, direct sum decomposition.
  \item {\bf Linear mappings:} linearity axioms, kernels and images,
    injectivity, surjectivity, bijections, compositions, inverses,
    matrix representations, change of bases, conjugation, similarity.
  \end{enumerate}
\item {\bf Affine structure.}
  \begin{enumerate}
  \item {\bf Determinants:} characterizing properties, cofactor
    expansion, permutations, Cramer's rule, classical adjoint.
  \item {\bf Geometric aspects:} Euclidean volume, orientation,
    equiaffine transformations, determinants as geometric invariants
    of linear transformations.
  \end{enumerate}
\item {\bf Diagonalization and Decomposition.}
  \begin{enumerate}
  \item {\bf Basic notions:} eigenvector, eigenvalue, eigenspace,
    characteristic polynomial.
  \item {\bf Obstructions:} imaginary eigenvalues, nilpotent
    transformations, classification of 2-dimensional real
    transformations.
  \item {\bf Structure theory:} invariant subspaces, Cayley-Hamilton
    theorem, Jordan canonical form, rational canonical form.
  \end{enumerate}
\item {\bf Multi-linearity.}
  \begin{enumerate}
  \item {\bf Foundations:} vector space dual, bilinearity, bilinear
    transpose, Gram-Schmidt orthogonalization.
  \item {\bf Bilinearity:} bilinear forms, symmetric bilinear forms,
    quadratic forms, signature and Sylvester's theorem, orthogonal
    transformations, skew-symmetric bilinear forms, symplectic
    transformations.
  \item {\bf Tensor algebra:} tensor product, contraction, invariants
    of linear transformations, symmetry operations.
  \end{enumerate}
\item {\bf Euclidean and Hermitian structure.}
  \begin{enumerate}
  \item{\bf Foundations:} inner product axioms, the adjoint operation,
    symmetric transformations, skew-symmetric transformations,
    self-adjoint transformations, normal transformations.
  \item{\bf Spectral theorem:} diagonalization of self-adjoint
    transformations, diagonalization of quadratic forms.
  \end{enumerate}
\item {\bf Computational and numerical methods.}
  \begin{enumerate}
  \item {\bf Linear problems:}  LU-factorization, QR decomposition, least
    squares, Householder transformations.
  \item {\bf Eigenvalue problems:} singular value decomposition, Gauss and
    Jacobi-Siedel iterative algorithms.
  \end{enumerate}
\end{enumerate}
%%%%%
%%%%%
\end{document}
