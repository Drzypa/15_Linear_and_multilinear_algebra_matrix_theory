\documentclass[12pt]{article}
\usepackage{pmmeta}
\pmcanonicalname{ProofThatCommutingMatricesAreSimultaneouslyTriangularizable}
\pmcreated{2013-03-22 15:27:08}
\pmmodified{2013-03-22 15:27:08}
\pmowner{georgiosl}{7242}
\pmmodifier{georgiosl}{7242}
\pmtitle{proof that commuting matrices are simultaneously triangularizable}
\pmrecord{10}{37301}
\pmprivacy{1}
\pmauthor{georgiosl}{7242}
\pmtype{Proof}
\pmcomment{trigger rebuild}
\pmclassification{msc}{15A23}

\endmetadata

\usepackage{graphicx}
%%%\usepackage{xypic} 
\usepackage{bbm}
\newcommand{\Z}{\mathbbmss{Z}}
\newcommand{\C}{\mathbbmss{C}}
\newcommand{\R}{\mathbbmss{R}}
\newcommand{\Q}{\mathbbmss{Q}}
\newcommand{\mathbb}[1]{\mathbbmss{#1}}
\newcommand{\figura}[1]{\begin{center}\includegraphics{#1}\end{center}}
\newcommand{\figuraex}[2]{\begin{center}\includegraphics[#2]{#1}\end{center}}
\newtheorem{dfn}{Definition}
\usepackage[fleqn]{amsmath}
\begin{document}
Proof by induction on $n$, order of matrix.
\\For $n=1$ we can simply take $Q=1$.
We assume that there exists a common unitary matrix $S$ that triangularizes simultaneously commuting matrices ,$(n-1)\times (n-1)$.\\So we have to show that the statement is valid for commuting matrices, $n\times n$.
From hypothesis $A$ and $B$ are commuting matrices $n\times n$ so these matrices have a common eigenvector.\\Let
$Ax=\lambda x$,  $Bx=\mu x$ where $x$ be the common eigenvector of unit length and $\lambda$, $\mu$ are the eigenvalues of $A$ and $B$ respectively. Consider the matrix, $R=\begin{pmatrix} x & X \end{pmatrix}$ 
where $X$ be orthogonal complement of $x$ and $R^HR=I$, then we have that 
$$R^HAR = \begin{pmatrix} \lambda & x^HAX \\ 0 & X^HAX \end{pmatrix}$$
$$R^HBR = \begin{pmatrix} \mu & x^HBX \\ 0 & X^HBX \end{pmatrix}$$
It is obvious that the above matrices and also 
$X^HBX$, $X^HAX$ ,$(n-1)\times(n-1)$ matrices are commuting matrices. Let $B_1=X^HBX$ and $A_1=X^HAX$ then 
there exists unitary matrix $S$ such that $S^HB_1S=\bar T_2,\, S^HA_1S=\bar T_1.$ Now $Q=R\begin{pmatrix} 1 & 0 \\ 0 & S \end{pmatrix}$ is a unitary matrix,
$Q^HQ=I$ and we have 
$$Q^HAQ=\begin{pmatrix} 1 & 0 \\ 0 & S^H \end{pmatrix}R^HAR\begin{pmatrix} 1 & 0 \\ 0 & S \end{pmatrix}=\begin{pmatrix} \lambda & x^HAXS \\ 0 & \bar T_1 \end{pmatrix}=T_1.$$
Analogously we have that $$Q^HBQ=T_2.$$
%%%%%
%%%%%
\end{document}
