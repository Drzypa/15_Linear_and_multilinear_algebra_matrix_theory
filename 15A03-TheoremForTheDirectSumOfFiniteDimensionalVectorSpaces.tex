\documentclass[12pt]{article}
\usepackage{pmmeta}
\pmcanonicalname{TheoremForTheDirectSumOfFiniteDimensionalVectorSpaces}
\pmcreated{2013-03-22 13:36:17}
\pmmodified{2013-03-22 13:36:17}
\pmowner{matte}{1858}
\pmmodifier{matte}{1858}
\pmtitle{theorem for the direct sum of finite dimensional vector spaces}
\pmrecord{8}{34233}
\pmprivacy{1}
\pmauthor{matte}{1858}
\pmtype{Theorem}
\pmcomment{trigger rebuild}
\pmclassification{msc}{15A03}

% this is the default PlanetMath preamble.  as your knowledge
% of TeX increases, you will probably want to edit this, but
% it should be fine as is for beginners.

% almost certainly you want these
\usepackage{amssymb}
\usepackage{amsmath}
\usepackage{amsfonts}

% used for TeXing text within eps files
%\usepackage{psfrag}
% need this for including graphics (\includegraphics)
%\usepackage{graphicx}
% for neatly defining theorems and propositions
%\usepackage{amsthm}
% making logically defined graphics
%%%\usepackage{xypic}

% there are many more packages, add them here as you need them

% define commands here
\begin{document}
{\bf Theorem} 
Let $S$ and $T$ be subspaces of a finite dimensional vector space
$V$. Then $V$ is the direct sum of $S$ and $T$, i.e., $V=S\oplus T$,
if and only if $\dim V = \dim S + \dim T$ and $S\cap T = \{0\}$. 

\emph{Proof.} Suppose that $V=S\oplus T$. Then, by definition, 
$V=S+T$ and $S\cap T=\{0\}$.  
The dimension theorem for subspaces states that 
$$\dim (S+T) + \dim S\cap T = \dim S + \dim T.$$
Since the dimension of the zero vector space $\{0\}$
is zero, we have that 
$$\dim V = \dim S + \dim T,$$
and the first direction of the claim follows. 

For the other direction, suppose $\dim V = \dim S + \dim T$ 
and $S\cap T = \{0\}$. Then the 
dimension theorem theorem for subspaces implies that 
$$\dim (S+T) = \dim V.$$
Now $S+T$ is a subspace of $V$ with the same dimension
as $V$ so, 
\PMlinkname{by Theorem 1 on this page}{VectorSubspace}, 
$V=S+T$. This proves the second direction. $\Box$
%%%%%
%%%%%
\end{document}
