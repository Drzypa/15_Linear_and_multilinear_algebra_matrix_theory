\documentclass[12pt]{article}
\usepackage{pmmeta}
\pmcanonicalname{ABIsConjugateToBA}
\pmcreated{2013-03-22 16:00:40}
\pmmodified{2013-03-22 16:00:40}
\pmowner{Algeboy}{12884}
\pmmodifier{Algeboy}{12884}
\pmtitle{AB is conjugate to BA}
\pmrecord{4}{38045}
\pmprivacy{1}
\pmauthor{Algeboy}{12884}
\pmtype{Theorem}
\pmcomment{trigger rebuild}
\pmclassification{msc}{15A04}

\endmetadata

\usepackage{latexsym}
\usepackage{amssymb}
\usepackage{amsmath}
\usepackage{amsfonts}
\usepackage{amsthm}

%%\usepackage{xypic}

%-----------------------------------------------------

%       Standard theoremlike environments.

%       Stolen directly from AMSLaTeX sample

%-----------------------------------------------------

%% \theoremstyle{plain} %% This is the default

\newtheorem{thm}{Theorem}

\newtheorem{coro}[thm]{Corollary}

\newtheorem{lem}[thm]{Lemma}

\newtheorem{lemma}[thm]{Lemma}

\newtheorem{prop}[thm]{Proposition}

\newtheorem{conjecture}[thm]{Conjecture}

\newtheorem{conj}[thm]{Conjecture}

\newtheorem{defn}[thm]{Definition}

\newtheorem{remark}[thm]{Remark}

\newtheorem{ex}[thm]{Example}



%\countstyle[equation]{thm}



%--------------------------------------------------

%       Item references.

%--------------------------------------------------


\newcommand{\exref}[1]{Example-\ref{#1}}

\newcommand{\thmref}[1]{Theorem-\ref{#1}}

\newcommand{\defref}[1]{Definition-\ref{#1}}

\newcommand{\eqnref}[1]{(\ref{#1})}

\newcommand{\secref}[1]{Section-\ref{#1}}

\newcommand{\lemref}[1]{Lemma-\ref{#1}}

\newcommand{\propref}[1]{Prop\-o\-si\-tion-\ref{#1}}

\newcommand{\corref}[1]{Cor\-ol\-lary-\ref{#1}}

\newcommand{\figref}[1]{Fig\-ure-\ref{#1}}

\newcommand{\conjref}[1]{Conjecture-\ref{#1}}


% Normal subgroup or equal.

\providecommand{\normaleq}{\unlhd}

% Normal subgroup.

\providecommand{\normal}{\lhd}

\providecommand{\rnormal}{\rhd}
% Divides, does not divide.

\providecommand{\divides}{\mid}

\providecommand{\ndivides}{\nmid}


\providecommand{\union}{\cup}

\providecommand{\bigunion}{\bigcup}

\providecommand{\intersect}{\cap}

\providecommand{\bigintersect}{\bigcap}










\begin{document}
\begin{prop}
Given square matrices $A$ and $B$ where one is invertible then $AB$ is conjugate to $BA$.
\end{prop}
\begin{proof}
If $A$ is invertible then $A^{-1} ABA=BA$.  Similarly if $B$ is invertible then
$B$ serves to conjugate $BA$ to $AB$.
\end{proof}


The result of course applies to any ring elements $a$ and $b$ where one is invertible.  It also holds for all group elements.

\begin{remark}
This is a partial generalization to the observation that the Cayley table of an abelian group is symmetric about the main diagonal.  In abelian groups this follows because $AB=BA$.  But in non-abelian groups $AB$ is only conjugate to $BA$.  Thus the conjugacy class of a group are symmetric about the main diagonal.
\end{remark}
 
\begin{coro}
If $A$ or $B$ is invertible then $AB$ and $BA$ have the same eigenvalues.
\end{coro}

This leads to an alternate proof of \PMlinkname{$AB$ and $BA$ being almost isospectral.}{ABAndBAAreAlmostIsospectral}  If $A$ and $B$ are both non-invertible, then we restrict to the non-zero eigenspaces $E$ of $A$ so that $A$ is invertible on $E$.  Thus $(AB)|_E$ is conjugate to $(BA)|_E$ and so indeed the two transforms have identical non-zero eigenvalues.
%%%%%
%%%%%
\end{document}
