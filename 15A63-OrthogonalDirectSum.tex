\documentclass[12pt]{article}
\usepackage{pmmeta}
\pmcanonicalname{OrthogonalDirectSum}
\pmcreated{2013-03-22 15:42:02}
\pmmodified{2013-03-22 15:42:02}
\pmowner{CWoo}{3771}
\pmmodifier{CWoo}{3771}
\pmtitle{orthogonal direct sum}
\pmrecord{9}{37645}
\pmprivacy{1}
\pmauthor{CWoo}{3771}
\pmtype{Definition}
\pmcomment{trigger rebuild}
\pmclassification{msc}{15A63}
\pmsynonym{orthogonal sum}{OrthogonalDirectSum}

\usepackage{amssymb,amscd}
\usepackage{amsmath}
\usepackage{amsfonts}

% used for TeXing text within eps files
%\usepackage{psfrag}
% need this for including graphics (\includegraphics)
%\usepackage{graphicx}
% for neatly defining theorems and propositions
%\usepackage{amsthm}
% making logically defined graphics
%%%\usepackage{xypic}

% define commands here
\begin{document}
Let $(V_1,B_1)$ and $(V_2,B_2)$ be two vector spaces, each equipped with a symmetric bilinear form.  Form the direct sum of the two vector spaces $V:=V_1\oplus V_2$.  Next define a symmetric bilinear form $B$ on $V$ by
$$B((u_1,u_2),(v_1,v_2)):=B_1(u_1,v_1)+B_2(u_2,v_2),$$
where $u_1,v_1\in V_1$ and $u_2,v_2\in V_2$.  Since $B((u_1,0),(u_2,0))=B_1(u_1,u_2)$, we see that $B=B_1$ when the domain of $B$ is restricted to $V_1$.  Therefore, $V_1$ can be viewed as a subspace of $V$ with respect to $B$.  The same holds for $V_2$. 
 
Now suppose $(u,0)\in V_1$ and $(0,v)\in V_2$ are two arbitrary vectors.  Then $B((u,0),(0,v))=B_1(u,0)+B_2(0,v)=0+0=0$.  In other words, $V_1$ and $V_2$ are ``orthogonal'' to one another with respect to $B$.

From the above discussion, we say that $(V,B)$ is the \emph{orthogonal direct sum} of $(V_1,B_1)$ and $(V_2,B_2)$.  Clearly the above construction is unique (up to linear isomorphisms respecting the bilinear forms).  As vectors from $V_1$ and $V_2$ can be seen as being ``perpendicular'' to each other, we appropriately write $V$ as $$V_1\bot V_2.$$

\textbf{Orthogonal Direct Sums of Quadratic Spaces.}  Since a symmetric biliner form induces a quadratic form (on the same space), we can speak of orthogonal direct sums of quadratic spaces.  If $(V_1,Q_1)$ and $(V_2,Q_2)$ are two quadratic spaces, then the orthogonal direct sum of $V_1$ and $V_2$ is the direct sum of $V_1$ and $V_2$ with the corresponding quadratic form defined by 
$$Q((u,v)):=Q_1(u)+Q_2(v).$$
It may be shown that any $n$-dimensional quadratic space $(V,Q)$ is an orthogonal direct sum of $n$ one-dimensional quadratic subspaces.  The quadratic form associated with a one-dimensional quadratic space is nothing more than $ax^2$ (the form is uniquely determined by the single coefficient $a$), and the space associated with this form is generally written as $\langle a\rangle$.  A finite dimensional quadratic space $V$ is commonly written as $$\langle a_1\rangle \bot \cdots \bot \langle a_n \rangle,\mbox{ or simply } \langle a_1,\ldots,a_n\rangle,$$ where $n$ is the dimension of $V$.

\textbf{Remark.}  The orthogonal direct sum may also be defined for vector spaces associated with bilinear forms that are \PMlinkname{alternating}{AlternatingForm}, skew symmetric or Hermitian.  The construction is similar to the one discussed above.
%%%%%
%%%%%
\end{document}
