\documentclass[12pt]{article}
\usepackage{pmmeta}
\pmcanonicalname{MatrixLogarithm}
\pmcreated{2013-03-22 15:31:22}
\pmmodified{2013-03-22 15:31:22}
\pmowner{Andrea Ambrosio}{7332}
\pmmodifier{Andrea Ambrosio}{7332}
\pmtitle{matrix logarithm}
\pmrecord{11}{37394}
\pmprivacy{1}
\pmauthor{Andrea Ambrosio}{7332}
\pmtype{Definition}
\pmcomment{trigger rebuild}
\pmclassification{msc}{15A90}
\pmclassification{msc}{15A99}
\pmrelated{NaturalLogarithm2}
\pmrelated{MatrixFNorm}
\pmrelated{FrobeniusMatrixNorm}
\pmdefines{principal logarithm}

\endmetadata

% this is the default PlanetMath preamble.  as your knowledge
% of TeX increases, you will probably want to edit this, but
% it should be fine as is for beginners.

% almost certainly you want these
\usepackage{amssymb}
\usepackage{amsmath}
\usepackage{amsfonts}

% used for TeXing text within eps files
%\usepackage{psfrag}
% need this for including graphics (\includegraphics)
%\usepackage{graphicx}
% for neatly defining theorems and propositions
%\usepackage{amsthm}
% making logically defined graphics
%%%\usepackage{xypic}

% there are many more packages, add them here as you need them

% define commands here
\newcommand{\norm}[1]{\ensuremath{||#1||_F}}
\begin{document}
Unlike the scalar logarithm, there are no naturally-defined bases for the matrix logarithm; therefore, the matrix logarithm is always taken to be the natural logarithm.  In general, there may be an infinite number of matrices $B$ satisfying $\exp(B)=A$; these are known as the logarithms of $A$.

As for the scalar natural logarithm, the matrix logarithm can be defined as a power series when $A$ is a square matrix and $\norm{I-A} < 1$, where $\norm{\cdot}$ is the Frobenius matrix norm.  The logarithm this formula produces is known as the \emph{principal logarithm} of $A$.
\begin{equation*}
\log(A) = -\sum_{k=1}^\infty \frac{(I-A)^k}{k} =\log(I+X) = \sum_{k=1}^\infty \frac{(-1)^{k+1}}{k} X^k
\end{equation*}

Since this series expansion does not converge for all $A$, it is not a global inverse function for the matrix exponential.  In particular, $\exp^{\log A}=A$ only holds for $\norm{I-A} < 1$, and $\log(\exp^A)=A$ only holds for $\norm{A} < 2$.

There are other, more general methods of calculating the matrix logarithm.  For example, see \htmladdnormallink{An Explicit Formula for the Matrix Logarithm}{http://arxiv.org/abs/math/0410556}.
%%%%%
%%%%%
\end{document}
