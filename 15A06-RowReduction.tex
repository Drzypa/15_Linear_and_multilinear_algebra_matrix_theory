\documentclass[12pt]{article}
\usepackage{pmmeta}
\pmcanonicalname{RowReduction}
\pmcreated{2013-03-22 12:06:48}
\pmmodified{2013-03-22 12:06:48}
\pmowner{rmilson}{146}
\pmmodifier{rmilson}{146}
\pmtitle{row reduction}
\pmrecord{18}{31238}
\pmprivacy{1}
\pmauthor{rmilson}{146}
\pmtype{Algorithm}
\pmcomment{trigger rebuild}
\pmclassification{msc}{15A06}
\pmsynonym{Gaussian elimination}{RowReduction}
\pmsynonym{Gauss-Jordan elimination}{RowReduction}
\pmrelated{RowEchelonForm}
\pmrelated{ReducedRowEchelonForm}
\pmrelated{ElementaryMatrix}
\pmdefines{pivot}
\pmdefines{row operation}
\pmdefines{row exchange}
\pmdefines{row replacement}
\pmdefines{row scaling}

\endmetadata

\usepackage{amsmath}
\usepackage{amsfonts}
\usepackage{amssymb}

\newcommand{\reals}{\mathbb{R}}
\newcommand{\natnums}{\mathbb{N}}
\newcommand{\cnums}{\mathbb{C}}
\newcommand{\znums}{\mathbb{Z}}

\newcommand{\lp}{\left(}
\newcommand{\rp}{\right)}
\newcommand{\lb}{\left[}
\newcommand{\rb}{\right]}

\newcommand{\supth}{^{\text{th}}}


\newtheorem{proposition}{Proposition}
\newtheorem{definition}[proposition]{Definition}
\begin{document}
\emph{Row reduction}, also known as \emph{Gaussian elimination}, is an algorithm for solving a system of linear equations
\[
\begin{array}{ccccccccl}
a_{11} x_1 &+ &a_{12} x_2 &+ &\ldots &+ &a_{1m} x_m &=& b_1 \\
a_{21} x_1 &+ &a_{22} x_2 &+ &\ldots &+ &a_{2m} x_m &=& b_2 \\
\vdots & & \vdots & & \ddots && \vdots && \vdots\\
a_{n1} x_1 &+ &a_{n2} x_2 &+ &\ldots &+ &a_{nm} x_m &=& b_n \\
\end{array}
\]
To describe row reduction, it is convenient to formulate a linear
system as a single matrix-vector equation $Ax = b,$ where
\[
A = \begin{bmatrix}
a_{11} & a_{12} & \cdots & a_{1m} \\
a_{21} & a_{22} & \cdots & a_{2m} \\
\vdots & \vdots & \ddots & \vdots \\
a_{n1} & a_{n2} & \cdots & a_{nm}
\end{bmatrix}, \qquad
b = \begin{bmatrix}
b_1 \\ b_2 \\ \vdots \\ b_n
\end{bmatrix},  \qquad
x = \begin{bmatrix}
x_1 \\ x_2 \\ \vdots \\ x_m
\end{bmatrix}
\]
are, respectively, the $n\times m$ matrix of coefficients of the
linear system, the $n$-place column vector of the scalars from the
right-hand of the equations, and the $m$-place column vector of
unknowns.

The method consists of combining the coefficient matrix $A$ with the
right hand vector $b$ to form the ``augmented'' $n \times (m + 1)$ matrix

$$ \begin{bmatrix} A & b \end{bmatrix} =
\begin{bmatrix}
a_{11} & a_{12} & \cdots & a_{1m} & b_1 \\
a_{21} & a_{22} & \cdots & a_{2m} & b_2 \\
\vdots & \vdots & \ddots & \vdots & \vdots \\
a_{n1} & a_{n2} & \cdots & a_{nm} & b_n
\end{bmatrix} =
\begin{bmatrix}
  R_1 \\ R_2 \\ \vdots \\ R_n,
\end{bmatrix}
$$
where each $R_i$ is the $m+1$-place row vector corresponding to row
$i$ of the augmented matrix.

A sequence of elementary row operations is then applied to this matrix
so as to transform it to row echelon form. The elementary operations are:
\begin{itemize}
\item \textbf{row scaling:} the multiplication a row by a nonzero
  scalar;
\[ R_i \mapsto c R_i,\quad c\neq 0;\]
\item \textbf{row exchange:} the  exchanges of two rows;
  \[ R_i \leftrightarrow R_j;\]
\item \textbf{row replacement:} the  addition of a multiple of one row to
  another row;
  \[ R_i \mapsto R_i + c R_j,\quad c\neq 0,\; i\neq j.\]
\end{itemize}
Note that these operations are ``legal'' because $x$ is a solution of
the transformed system if and only if it is a solution of the initial system.

If the number of equations equals the number of variables ($m=n$), and
if the coefficient matrix $A$ is \PMlinkname{non-singular}{singular}, then the algorithm will
terminate when the augmented matrix has the following form:

$$ \begin{bmatrix}
a'_{11} & a'_{12} & \cdots & a'_{1n} & b_1 \\
0 & a_{22}' & \cdots & a_{2n}' & b_2' \\
\vdots & \vdots & \ddots & \vdots & \vdots \\
0 & 0 & \cdots & a_{nn}' & b_n'
\end{bmatrix} $$
With these assumptions, there exists a unique solution, which can be
obtained from the above matrix by back substitution.

% Assume we have transformed the first column, and we want to continue the elimination with the
% following matrix

% $$ \begin{bmatrix}
% a_{11} & a_{12} & \cdots & a_{1n} & b_1 \\
% 0 & a_{22}' & \cdots & a_{2n}' & b_2' \\
% 0 & a_{32}' & \cdots & a_{3n}' & b_3' \\
% \vdots & \vdots & \ddots & \vdots & \vdots \\
% 0 & a_{n2}' & \cdots & a_{nn}' & b_n'
% \end{bmatrix} $$

% To zero $a_{32}'$, we want to divide the second row by the ``pivot''
% $a_{22}'$ , multiply it with $a_{32}'$, and subtract it from the third
% row. If the pivot is zero, we have to swap two rows. This procedure
% frequently breaks down, not only for ill-conditioned matrices.
% Therefore, most programs perform partial pivoting or complete pivoting
% (normally not necessary.)

For the general case, the termination procedure is somewhat more
complicated. First recall that a matrix is in echelon form if each
row has more leading zeros than the rows above it. A pivot is the
leading non-zero entry of some row. We then have
\begin{itemize}
\item If there is a pivot in the last column, the system is
inconsistent ; there will be no solutions.
\item If that is not the case, then the general solution will have $d$
degrees of freedom, where $d$ is the number of columns from $1$ to
$m$ that have no pivot. To be more precise, the general solution
will have the form of one particular solution plus an arbitrary linear
combination of $d$ linearly independent $n$-vectors.

In even more prosaic language, the variables in the non-pivot
columns are to be considered ``free variables'' and should be
``moved'' to the right-hand side of the equation. The general
solution is then obtained by arbitrarily choosing values of the free
variables, and then solving for the remaining ``non-free'' variables
that reside in the pivot columns.
\end{itemize}

A variant of Gaussian elimination is Gauss-Jordan elimination. In
this variation we reduce to echelon form, and then if the system
proves to be consistent, continue to apply the elementary row
operations until the augmented matrix is in {\em reduced echelon
form}. This means that not only does each pivot have all zeroes
below it, but that each pivot also has all zeroes above it.

In essence, Gauss-Jordan elimination performs the back substitution;
the values of the unknowns can be read off directly from the terminal
augmented matrix. Not surprisingly, Gauss-Jordan elimination is
slower than Gaussian elimination. It is useful, however, for solving
systems on paper.
%%%%%
%%%%%
%%%%%
\end{document}
