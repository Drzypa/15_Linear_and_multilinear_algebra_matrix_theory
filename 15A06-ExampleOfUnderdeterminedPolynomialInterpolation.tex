\documentclass[12pt]{article}
\usepackage{pmmeta}
\pmcanonicalname{ExampleOfUnderdeterminedPolynomialInterpolation}
\pmcreated{2013-03-22 12:35:22}
\pmmodified{2013-03-22 12:35:22}
\pmowner{rmilson}{146}
\pmmodifier{rmilson}{146}
\pmtitle{example of under-determined polynomial interpolation}
\pmrecord{5}{32840}
\pmprivacy{1}
\pmauthor{rmilson}{146}
\pmtype{Example}
\pmcomment{trigger rebuild}
\pmclassification{msc}{15A06}

\usepackage{amsmath}
\usepackage{amsfonts}
\usepackage{amssymb}
\newcommand{\cP}{\mathcal{P}}

\newcommand{\reals}{\mathbb{R}}
\newcommand{\natnums}{\mathbb{N}}
\newcommand{\cnums}{\mathbb{C}}
\newcommand{\znums}{\mathbb{Z}}

\newcommand{\lp}{\left(}
\newcommand{\rp}{\right)}
\newcommand{\lb}{\left[}
\newcommand{\rb}{\right]}

\newcommand{\supth}{^{\text{th}}}


\newtheorem{proposition}{Proposition}
\newtheorem{definition}[proposition]{Definition}
\begin{document}
Consider the following
interpolation problem:
\begin{quote}
  \em
  Given $x_1,y_1,x_2,y_2\in \reals$ with $x_1\neq x_2$ to determine
  all cubic polynomials
  $$p(x)= ax^3+bx^2+cx+d,\quad x,a,b,c,d\in \reals$$
  such that 
  $$p(x_1) = y_1,\quad p(x_2) = y_2.$$  
\end{quote}
This is a linear problem.  Let $\cP_3$ denote the vector space of
cubic polynomials. The underlying linear mapping is the
multi-evaluation mapping
$$E:\cP_3\rightarrow\reals^2,$$
given by
$$ p\mapsto 
\begin{pmatrix}
  p(x_1)\\ p(x_2)
\end{pmatrix},\quad p\in \cP_3
$$
The interpolation problem in question is represented by the equation
$$E(p) = 
\begin{pmatrix}
  y_1 \\ y_2
\end{pmatrix}
$$
where $p\in \cP_3$ is the unknown.  One can recast the problem into the
traditional form by taking standard bases of  $\cP_3$ and $\reals^2$ and
then seeking all possible $a,b,c,d\in\reals$ such that
$$
\begin{pmatrix}
   \lp x_1\rp^3 &    \lp x_1\rp^2 & x_1 & 1 \\
   \lp x_2\rp^3 &    \lp x_2\rp^2 & x_2 & 1 \\
\end{pmatrix}
\begin{pmatrix}
  a\\b\\c\\d
\end{pmatrix} 
=
\begin{pmatrix}
  y_1 \\ y_2
\end{pmatrix}
$$
However, it is best to treat this problem at an abstract level,
rather than mucking about with row reduction.  The Lagrange
interpolation formula gives us a particular solution, namely the linear
polynomial 
$$p_0(x) = \frac{x-x_1}{x_2-x_1}y_1 + \frac{x-x_2}{x_1-x_2} y_2,\quad
x\in \reals$$
The general solution of our interpolation problem is therefore given
as
$p_0 + q$, where $q\in \cP_3$ is a solution of the homogeneous
problem
$$E(q)=0.$$
A basis of solutions for the latter is, evidently, 
$$q_1(x) = (x-x_1)(x-x_2),\quad q_2(x) = x q_1(x),\qquad x\in \reals$$
The general solution to our interpolation problem is therefore
$$p(x) =  \frac{x-x_1}{x_2-x_1}y_1 + \frac{x-x_2}{x_1-x_2} y_2 +
(ax+b)(x-x_1)(x-x_2),\quad x\in\reals,$$
with $a,b\in \reals$ arbitrary.  The general under-determined
interpolation problem is treated in an entirely analogous manner.
%%%%%
%%%%%
\end{document}
