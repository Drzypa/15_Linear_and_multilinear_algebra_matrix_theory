\documentclass[12pt]{article}
\usepackage{pmmeta}
\pmcanonicalname{LinearTransformationIsContinuousIfItsDomainIsFiniteDimensional}
\pmcreated{2013-03-22 15:17:59}
\pmmodified{2013-03-22 15:17:59}
\pmowner{matte}{1858}
\pmmodifier{matte}{1858}
\pmtitle{linear transformation is continuous if its domain is finite dimensional}
\pmrecord{7}{37098}
\pmprivacy{1}
\pmauthor{matte}{1858}
\pmtype{Theorem}
\pmcomment{trigger rebuild}
\pmclassification{msc}{15A04}

\endmetadata

% this is the default PlanetMath preamble.  as your knowledge
% of TeX increases, you will probably want to edit this, but
% it should be fine as is for beginners.

% almost certainly you want these
\usepackage{amssymb}
\usepackage{amsmath}
\usepackage{amsfonts}
\usepackage{amsthm}

\usepackage{mathrsfs}

% used for TeXing text within eps files
%\usepackage{psfrag}
% need this for including graphics (\includegraphics)
%\usepackage{graphicx}
% for neatly defining theorems and propositions
%
% making logically defined graphics
%%%\usepackage{xypic}

% there are many more packages, add them here as you need them

% define commands here

\newcommand{\sR}[0]{\mathbb{R}}
\newcommand{\sC}[0]{\mathbb{C}}
\newcommand{\sN}[0]{\mathbb{N}}
\newcommand{\sZ}[0]{\mathbb{Z}}

 \usepackage{bbm}
 \newcommand{\Z}{\mathbbmss{Z}}
 \newcommand{\C}{\mathbbmss{C}}
 \newcommand{\F}{\mathbbmss{F}}
 \newcommand{\R}{\mathbbmss{R}}
 \newcommand{\Q}{\mathbbmss{Q}}



\newcommand*{\norm}[1]{\lVert #1 \rVert}
\newcommand*{\abs}[1]{| #1 |}



\newtheorem{thm}{Theorem}
\newtheorem{defn}{Definition}
\newtheorem{prop}{Proposition}
\newtheorem{lemma}{Lemma}
\newtheorem{cor}{Corollary}
\begin{document}
\begin{thm}
A linear transformation is continuous if the domain is finite dimensional.
\end{thm}

\begin{proof}
Suppose $L\colon X\to Y$ is the transformation, $\dim X = n$,
  and  $\Vert \cdot \Vert_X$, $\Vert \cdot \Vert_Y$ are the norms
  on $X$, $Y$, respectively.
By \PMlinkname{this result}{ContinuityIsPreservedWhenCodomainIsExtended}
and \PMlinkname{this result}{SubspaceTopologyInAMetricSpace},
it suffices to prove that $L\colon X\to L(X)$ is continuous
when $L(X)$ is equipped with the topology given by $\Vert \cdot \Vert_Y$
restricted onto $L(X)$. 
Also, since continuity and boundedness are equivalent, it suffices to
  prove that $L$ is bounded. 
Let $e_1,\ldots, e_n$ be a basis for $X$ such that 
$L$ is invertible on $\operatorname{span} \{e_{1}, \ldots, e_k\}$ and
$\operatorname{ker} L = \operatorname{span} \{e_{k+1}, \ldots, e_n\}$ for 
$k=1,\ldots, n$. (The zero map is always continuous.)
Let $f_i=L(e_i)$ for $i=1,\ldots, k$, so that 
$\operatorname{span}\{f_1, \ldots, f_k\}=L(X)$. 
Let us define new norms on $X$ and $L(X)$,
\begin{eqnarray*}
  \Vert x \Vert'_X &=& \sqrt{\sum_{i=1}^n \alpha_i^2},\\
  \Vert y \Vert'_X &=& \sqrt{\sum_{i=1}^k \beta_i^2},
\end{eqnarray*}
for $x=\sum_{i=1}^n \alpha_i e_i\in X$ and 
    $y=\sum_{i=1}^k \beta_i f_i \in Y$.
Since norms on finite dimensional vector spaces are equivalent, it follows
that 
\begin{eqnarray*}
    1/C \Vert x \Vert'_X \le \Vert x \Vert_X \le C \Vert x \Vert'_X, \quad x\in X \\
    1/D \Vert y \Vert'_Y \le \Vert y \Vert_Y \le D \Vert y \Vert'_Y, \quad y\in L(X)
\end{eqnarray*}
for some constants $C,D>0$.
For $x=\sum_{i=1}^n \alpha_i e_i\in X$,
\begin{eqnarray*}
\Vert L(x)\Vert_Y &\le & D \Vert \sum_{i=1}^k \alpha_i f_i \Vert'_Y \\
   &=& D \sqrt{\sum_{i=1}^k \alpha_i^2} \\
   &\le& D \sqrt{\sum_{i=1}^n \alpha_i^2} \\
   &=& D \Vert x \Vert'_X \\
   &=& CD \Vert x \Vert_X.
\end{eqnarray*}
Thus $L\colon X\to L(X)$ is bounded.
\end{proof}
%%%%%
%%%%%
\end{document}
